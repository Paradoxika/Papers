\section{Introduction}

Proofs are the central object of this work.
In this chapter we define the resolution calculus, which is extended to reason about equality in Section \ref{sec:calculus}.
We define what a proof in this calculus is, measures of proofs and what it means to process a proof.
Our method is stated so it compresses resolution proofs.
However, the method is in its core independent of the underlying proof system.
As long as a proof system is able to express congruence reasoning in a systematic way, we can easily adapt our compression algorithm to that system.

In this chapter we present a method to compress proofs in length.
The method manipulates SMT proofs of the theory of equality.
To this end, in Section \ref{sec:calculus} we extend the resolution calculus presented in 
Section \ref{sec:resolution} to handle equality and its axioms.
The proof compression method is based on the idea of replacing long explanations for the equality of two terms by shorter ones.
In Section \ref{sec:npcomplete} we show that finding the shortest explanation is NP-complete.
In Section \ref{sec:algorithm} we present our explanation producing congruence closure algorithm, which is applied in the proof compression algorithm presented in Section \ref{sec:proofproduction}.
Closing this chapter, we give an outlook of possible future work.

