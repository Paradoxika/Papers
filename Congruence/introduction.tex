\section{Introduction}

ToDo:
SMT... Equality... Congruence Closure... Interpolation... Applications... Motivation... Proof Compression (Post-Processing)...

The Propositional Satisfiability (SAT) Problem and the Satisfiability
Modulo Theories (SMT) Problem are among the key decision problems.
Many other problems are solved by reducing them either to SAT or SMT
instances. On top of simple SAT/UNSAT answers, many modern solvers
can provide additional information: satisfying models in case of
satisfiability, and refutation proofs in case of unsatisfiability.
Refutation proofs allow an easy computation of Craig interpolants
\cite{todo}\todo{Cite}, which in turn have widespread applications.
They are used in quantifier elimination \cite{todo}\todo{Cite Jiang
CAV}, symbolic model-checking \cite{todo}\todo{cite McMillan, IC3,
etc.}, logic synthesis \cite{todo}\todo{Cite stuff by Jiang},
property synthesis \cite{todo}\todo{Koenighofer FMCAD14}, and
controller synthesis \cite{todo}\todo{Hofferek et al. FMCAD13}, to
name just a few. In all these applications it is often desirable to
work with small proofs, which in turn lead to small interpolants.
Thus, compressing proofs can have significant effects for many
different applications. 


The main contribution of this paper is a novel method for compressing
the length of SMT proofs containing equality reasoning based on
congruence closure. Previous works \cite{Todo}\todo{Cite all Skeptik
publications and others} on the compression of SMT proofs have
focused exclusively on the propositional resolution parts of the
proofs. Other works \cite{Todo}\todo{Cite Cut-introduction works} on
proof compression beyond the propositional level have targeted
sequent calculus proofs without special treatment of equality. This
is, to the best of our knowledge, the first time that an algorithm
for compressing congruence closure proofs is proposed.

The proof compression algorithm proposed here is based on reproving
the congruence closure lemmas with a more sophisticated congruence
closure algorithm, which improves previous explanation-producing
\cite{Nieuwenhuis2005,Nieuwenhuis2007} and proof-producing
\cite{Fontaine2004} congruence closure algorithms by using a
Dijkstra-inspired algorithm for searching for short paths in the
congruence graph. Although this improved algorithm is computationally
more expensive, the extra cost is affordable in a post-processing
stage, where the number of lemmas in the proof is small in comparison
to the number of lemmas considered during proof search.

Another major contribution of this paper is a proof (in Section
\ref{sec:npcomplete}) that deciding whether a shorter congruence
closure explanation exists is an NP-complete problem. This implies
that our Dijkstra-inspired algorithm finds only good explanations,
though not optimal ones.

A secondary contribution of this paper is a detailed description of
the production of a resolution proof with equality axioms (as
described in Sections \ref{sec:resolution} and \ref{sec:calculus})
from the congruence graph. Although this is probably folklore among
developers of fine-grained proof-producing SMT-solvers (e.g. VeriT
\cite{ToDo}\todo{Cite}), such a description (in Section
\ref{sec:proofproduction}) seemed to be missing in the literature up
to now. In \cite{Nieuwenhuis2005,Nieuwenhuis2007}, only the
production of lemmas (explanations) is described, and in
\cite{Fontaine2004} the described production of proofs uses a special
purpose calculus different from resolution.

Resolution is a desirable proof calculus, because it is already
commonly used for the propositional parts of SMT proofs and also for
SAT proofs \cite{ToDo}\todo{Cite TraceCheckFormatArminBiere}. Simple
and efficient proof checkers for resolution are easily implementable.
Nevertheless, the methods described here are essentially independent
of the underlying proof system. As long as a proof system is able to
express congruence reasoning in a systematic way, these methods could
be easily adapted to it.
