\documentclass[svgnames,table,onlymath,notheorems]{beamer}

%\usefonttheme{sserif}


\usepackage{helvet}
\usepackage[utf8]{inputenc}
\usepackage[english]{babel}
\usepackage{ragged2e}

\usefonttheme[onlymath]{serif}
\renewcommand{\familydefault}{\sfdefault}

%\usepackage[no-math]{fontspec}
%\usepackage{unicode-math}
%\setsansfont[
  %BoldFont=HelveticaNeueLTCom-Md,
  %ItalicFont=HelveticaNeueLTCom-LtIt, 
  %SmallCapsFont=TeXGyre Heros,
  %SmallCapsFeatures={Letters=SmallCaps},
%]{HelveticaNeueLTCom-Lt}
%\setmathfont{HelveticaNeueLTCom-Lt}
%\setmathfont[range=\mathit/{greek}]{TeXGyre Heros Italic}

\usecolortheme{orchid}
\setbeamertemplate{blocks}[rounded][shadow=true]
\setbeamertemplate{navigation symbols}{
  \usebeamerfont{footline}%
  \usebeamercolor[fg]{footline}%
  \hspace{1em}%
  \insertframenumber/\inserttotalframenumber
}%remove navigation symbols

\setbeamerfont{itemize/enumerate body}{family=\sffamily}                                  

%\usepackage{fontspec}
%\setsansfont{\sfdefault}

%\setlist[itemize]{%
  %font={\familydefault} % set the label font
%%  font={\bfseries\sffamily\color{red}}, % if colour is needed
%}

\usepackage{amsmath,amssymb}
\usepackage{tikz}
%\usepackage{siunitx}
\usetikzlibrary{calc}
\usetikzlibrary{arrows,automata,positioning}
\usetikzlibrary{fit}
\usetikzlibrary{positioning}
\usetikzlibrary{arrows}
\usepackage{drawproof}
\newenvironment<>{subpart}[1]
{ \begin{block}#2{#1}
  \begin{itemize}
}{
  \end{itemize}
  \end{block}
}

%\renewcommand{\myitem}[2]{\item[#1] \textcolor{blue}{\emph{#2}}}

\setbeamerfont{institute}{size=\tiny,family=\sffamily}

\title{
  NP-completeness of small conflict set generation for congruence closure
}

\date{SMT Workshop\\ San Francisco, $19^{th}$ of July 2015}

\author{\textbf{Andreas Fellner}\inst{1}$^{,}$\inst{2}
   \and Pascal Fontaine\inst{3}
   \and\\ Georg Hofferek\inst{4}
   \and Bruno Woltzenlogel Paleo\inst{2}$^{,}$\inst{5}
	}
\institute[shortinst]{$^1$ IST-Austria, Klosterneuburg (Austria)\and%
 $^2$ Vienna University of Technology (Austria)\and%
 $^3$ Inria, Loria, U. of Lorraine (France)\and%
 $^4$ IAIK, Graz University of Technology (Austria)\and%
 $^5$ Australian National University (Australia)
}

%-----------------------------------------------------------------------

\begin{document}
%\rm % Use ROMAN fonts

%\section{}

\begin{frame}

\maketitle

\end{frame}

\begin{frame}

\center \textit{Sorry for giving a 25 minute presentation, \\but I did not have time to prepare a 5 minute one.}

\end{frame}

\begin{frame}
\frametitle{Conflict Set}

\begin{itemize}
		\item Unsatisfiable set of equations and negated equations
		
	\end{itemize}
	\uncover<2->{
	\begin{subpart}{Example}
		\item $\{$\visible<2>{$g(c_1,\ldots,c_n) = d,$} \visible<2-4>{$f(a) = a,$} \uncover<2->{$a = b,$} \visible<2-6>{$b = f(b),$}  \visible<2-7>{$f(a) \neq f(b)$}$\}$
		\uncover<4->{\item Transitivity}%: $t_1 = t_2$ and $t_2 = t_3$ then $t_1 = t_3$}
		\uncover<6->{\item Congruence: $t_1 = s_1$ and $\ldots$ $t_n = s_n$ implies \\\makebox[2.4cm]{}$f(t_1, \dots, t_n) = f(s_1, \dots, s_n) $}
	\end{subpart}
	}

\end{frame}

\begin{frame}
\frametitle{Why do we want small conflict sets? (1)}

\begin{itemize}
	\item Speed up SMT decision procedures
\end{itemize}
\resizebox{\linewidth}{!}{%\documentclass{standalone}
%\usepackage{tikz}
%\usetikzlibrary{positioning}
%\usetikzlibrary{arrows}
%\input{arrowsnew}
%\def\checkmark{\tikz\fill[scale=0.4](0,.35) -- (.25,0) -- (1,.7) -- (.25,.15) -- cycle;} 

 %\begin{document}

\begin{tikzpicture}[node distance = 1.2cm]
	\uncover<2->{\node (input) [draw,rectangle,rounded corners=2pt] {Input SMT problem $\Psi$};}
	\uncover<3->{
		\node (abstr) [draw,rectangle,rounded corners=2pt,below = of input] {Propositional logic abstraction $\phi$};
		}
	\uncover<4->{
		\node (checksat) [draw,rectangle,rounded corners=2pt, below = of abstr,align=center] {Check satisfiability of $\phi$\\with SAT-solver};
		}
	\uncover<5->{
		\node (unsat) [draw,rectangle,rounded corners=2pt, right=of checksat,xshift=1.6cm] {Report unsatisfiability of $\Psi$};
		}
	\uncover<6->{
		\node (sat) [draw,rectangle,rounded corners=2pt,below=of checksat,align=center] {Check consistency of $\phi$\\with $\mathcal{T}$-solver (congruence closure)};
		}
	\uncover<7->{
		\node (theoryconsistent) [draw,rectangle,rounded corners=2pt,right=of sat,xshift=1cm] {Report satisfiability of $\Psi$};
		}
	\uncover<8->{
		\node (incon) [draw,rectangle,rounded corners=2pt,left=of checksat,align=center,yshift=1cm] {Add $C$ to $\phi$ as clause};
		}
	
	\uncover<3->{
		\draw [arrows={-latex},auto] (input) to node {Treat equations as propositional variables} (abstr);
		}
	
	\uncover<4->{\draw [arrows={-latex},auto] (abstr) to node {} (checksat);}
	
	\uncover<5->{\draw [arrows={-latex},auto] (checksat) to node {UNSAT} (unsat);}
	
	\uncover<6->{\draw [arrows={-latex},auto] (checksat) to node {SAT} (sat);}
	
	\uncover<7->{\draw [arrows={-latex},auto] (sat) to node {consistent} (theoryconsistent);}
	
	\uncover<8->{\draw [arrows={-latex},midway,bend=45,out=180,in=270,align=center] (sat) to node[fill=white,yshift=.5cm,xshift=-.5cm] {inconsistent\\counterexample (conflict set) $C$} (incon);}
	
	\uncover<9->{\draw [arrows={-latex},auto,bend=45,out=90,in=180,align=center] (incon.north) to (abstr.west);}
	
\end{tikzpicture}

 %\end{document}}
\uncover<10->{
\begin{subpart}{Smaller conflict set}
	%\begin{itemize}
	 \item Eliminate more spurious counterexamples at once
	 \item Fewer loops
	%\end{itemize}
\end{subpart}
}
\end{frame}

%\begin{frame}
%\frametitle{Why do we want small conflict sets? (2)}
%
%\begin{itemize}
	%\item Smaller Proofs
	%\item Proof corresponding to transitivity 
%\end{itemize}
%\resizebox{.8\linewidth}{!}{
\centering
\begin{tikzpicture}[node distance=2.5cm]
	%tikzstyle{every node}=[font=\small]
	
	\rootnode;
	
	\withchildren{root}{n8}{$f(a) \neq f(b)$}{n7}{$f(a) = f(b)$};
	
	\withchildren{n7}{n5}{$b \neq f(b), f(a) = f(b)$}{n6}{$b = f(b)$};
	
	\withchildren{n5}{n3}{$a \neq b, b \neq f(b), f(a) = f(b)$}{n4}{$a = b$};
	
	\proofnode[above left of=n3]{n2}{$f(a) \neq a,a \neq b, b \neq f(b), f(a) = f(b)$};
	\proofnode[above right of=n3, xshift=1cm]{n1}{$f(a) = a$};
	\drawchildren{n3}{n1}{n2};
\end{tikzpicture}

}
%%
\centering
\begin{tikzpicture}[node distance=2.5cm]
	%tikzstyle{every node}=[font=\small]
	
	\rootnode;
	
	\withchildren{root}{n8}{$f(a) \neq f(b)$}{n7}{$f(a) = f(b)$};
	
	\withchildren{n7}{n5}{$b \neq f(b), f(a) = f(b)$}{n6}{$b = f(b)$};
	
	\withchildren{n5}{n3}{$a \neq b, b \neq f(b), f(a) = f(b)$}{n4}{$a = b$};
	
	\proofnode[above left of=n3]{n2}{$f(a) \neq a,a \neq b, b \neq f(b), f(a) = f(b)$};
	\proofnode[above right of=n3, xshift=1cm]{n1}{$f(a) = a$};
	\drawchildren{n3}{n1}{n2};
\end{tikzpicture}


%
%\end{frame}

\begin{frame}
\frametitle{Why do we want small conflict sets? (2)}

\begin{itemize}
	\item Smaller proofs
	\item Proof corresponding to \alt<2>{congruence}{transitivity}
\end{itemize}
%\resizebox{.75}{!}{
\centering
\begin{tikzpicture}[node distance=2.5cm]
	%tikzstyle{every node}=[font=\small]
	
	\rootnode;
	
	\withchildren{root}{n4}{$f(a) \neq f(b)$}{n3}{$f(a) = f(b)$};
	
	\withchildren{n3}{n1}{$a \neq b, f(a) = f(b)$}{n2}{$a = b$};
	
	%\withchildren{n7}{n5}{$b \neq f(b), f(a) = f(b)$}{n6}{$b = f(b)$};
	%
	%\withchildren{n5}{n3}{$a \neq b, b \neq f(b), f(a) = f(b)$}{n4}{$a = b$};
	%
	%\proofnode[above left of=n3]{n2}{$f(a) \neq a,a \neq b, b \neq f(b), f(a) = f(b)$}
	%\proofnode[above right of=n3, xshift=1cm]{n1}{$f(a) = a$};
	%\drawchildren{n3}{n1}{n2};
\end{tikzpicture}

}
\alt<2>{
\centering
\begin{tikzpicture}[node distance=2.5cm]
	%tikzstyle{every node}=[font=\small]
	
	\rootnode;
	
	\withchildren{root}{n4}{$f(a) \neq f(b)$}{n3}{$f(a) = f(b)$};
	
	\withchildren{n3}{n1}{$a \neq b, f(a) = f(b)$}{n2}{$a = b$};
	
	%\withchildren{n7}{n5}{$b \neq f(b), f(a) = f(b)$}{n6}{$b = f(b)$};
	%
	%\withchildren{n5}{n3}{$a \neq b, b \neq f(b), f(a) = f(b)$}{n4}{$a = b$};
	%
	%\proofnode[above left of=n3]{n2}{$f(a) \neq a,a \neq b, b \neq f(b), f(a) = f(b)$}
	%\proofnode[above right of=n3, xshift=1cm]{n1}{$f(a) = a$};
	%\drawchildren{n3}{n1}{n2};
\end{tikzpicture}

}{\resizebox{.8\linewidth}{!}{
\centering
\begin{tikzpicture}[node distance=2.5cm]
	%tikzstyle{every node}=[font=\small]
	
	\rootnode;
	
	\withchildren{root}{n8}{$f(a) \neq f(b)$}{n7}{$f(a) = f(b)$};
	
	\withchildren{n7}{n5}{$b \neq f(b), f(a) = f(b)$}{n6}{$b = f(b)$};
	
	\withchildren{n5}{n3}{$a \neq b, b \neq f(b), f(a) = f(b)$}{n4}{$a = b$};
	
	\proofnode[above left of=n3]{n2}{$f(a) \neq a,a \neq b, b \neq f(b), f(a) = f(b)$};
	\proofnode[above right of=n3, xshift=1cm]{n1}{$f(a) = a$};
	\drawchildren{n3}{n1}{n2};
\end{tikzpicture}

}}

\end{frame}

\begin{frame}

\frametitle{Conflict Set vs Explanation}

\begin{subpart}{Explanation for $s = t$}
	\item Set of equations $E$, such that $E \models s = t$
	\item $E \cup \{s \neq t\}$ is a conflict set
\end{subpart}

\begin{subpart}{Conflict set $C$}
	\item There is $s \neq t \in C$, such that
	\item $C \setminus \{s \neq t\}$ is an explanation for $s=t$
\end{subpart}

\end{frame}

\begin{frame}

\frametitle{Small Explanation Decision Problem}

\centering Given a set of equations $E$, a target equation $s = t$ and $k \in \mathbb{N}$, \\does there exist an explanation $E' \subseteq E$ of $s = t$ with $|E'| \leq k$?
%\centering Given a set of input equations $E$, a target equation $s = t$ and $k \in \mathbb{N}$, does there exist a set $E' \subseteq E$ with $|E'| \leq k$ and $E'$ is an explanation for $s=t$?

\uncover<2->{\vspace{1cm} \centering\alert{\textbf{NP-complete}}}

\uncover<3->{
\vspace{1cm} 
\begin{block}{Small Explanation is in NP} 
\begin{enumerate}
\item Guess $E' \subseteq E$, which is polynomial in input size.
\item Check $E' \models s = t$ with congruence closure algorithm in polynomial time.
\end{enumerate}	
\end{block}

}

%\uncover<3->{Reduction of SAT to the short explanation decision problem}

\end{frame}

\begin{frame}

\frametitle{NP-hardness}

\begin{subpart}{Reduction of SAT}

\item Given a propositional logic formula in CNF\\\hspace{.5cm} $\phi = C_1 \wedge \dots \wedge C_n$ 
\item Using variables $x_1,\dots, x_m$
\item Construct a set of equations $E$ and a target equation $s = t$, such that
\end{subpart}
$$\phi \text{ is satisfiable}$$
$$\textit{if and only if} $$
$$\text{There exists an explanation } E' \subseteq E \text{ of } s=t \text{ with } |E'| \leq 3n + 4m - 1$$

\end{frame}

\begin{frame}

\frametitle{Example of Reduction}

\begin{equation*}
\phi = (x_1 \vee x_2 \vee \neg x_3) \wedge (\neg x_2 \vee x_3) \wedge (\neg x_1 \vee \neg x_3)
\end{equation*}
%$$n = 3, m = 3$$
\centering \uncover<2->{Equations $E$ $( a - b \leadsto a = b \in E)$}\\
\centering \uncover<13->{Target Equation $c_1 = c'_3$}
\resizebox{\linewidth}{!}{
\centering
\begin{tikzpicture}[node distance=.8cm]
\newcommand\myadd[2]{\the\numexpr(#1)+(#2)\relax}
\newcommand\s{3} %c1,c2,c3 and x1,x2,x3 (new: c's also)
\newcommand\x{\myadd{\s}{3}} %c1,c2,c3 and x1,x2,x3 have been shown
\newcommand\y{\myadd{\x}{4}}  %Their connections have been shown
\uncover<\s->{\node(c1){$c_1\phantom{'}$};}

\uncover<\myadd{\x}{1}->{
\node[right =.25cm of c1] (t1x2) {$t_1(\hat{x}_2)$};
\node[above =.8cm of t1x2] (t1x1) {$t_1(\hat{x}_1)$};
\node[below =.8cm of t1x2] (t1x3) {$t_1(\hat{x}_3)$};
\draw [-] (c1) to (t1x2);
\draw [-] (c1) to (t1x1);
\draw [-] (c1) to (t1x3);
}

\uncover<\myadd{\s}{1}->{\node[right = 3.1cm of c1](c1p){$c_1'$};}

\uncover<\myadd{\x}{2}->{
\node[left =.25cm of c1p] (t12) {$t_1(\top_2)$};
\node[above=.8cm of t12] (t11) {$t_1(\top_1)$};
\node[below=.8cm of t12] (t13) {$t_1(\bot_3)$};
\draw [-] (c1p) to (t11);
\draw [-] (c1p) to (t12);
\draw [-] (c1p) to (t13);
}

\uncover<\s->{\node[right = .3cm of c1p](c2){$c_2\phantom{'}$};}
\uncover<\myadd{\s}{1}->{\draw [-] (c1p) to (c2);}

\uncover<\myadd{\x}{3}->{
\node[right=.25cm of c2, yshift=.8cm]  (t2x2) {$t_2(\hat{x}_2)$};
\node[right=.25cm of c2, yshift=-.8cm] (t2x3) {$t_2(\hat{x}_3)$};
\draw [-] (c2) to (t2x2);
\draw [-] (c2) to (t2x3);
}

\uncover<\myadd{\s}{1}->{\node[right = 3.1cm of c2](c2p){$c_2'$};}

\uncover<\myadd{\x}{4}->{
\node[left =.25cm of c2p, yshift=.8cm]  (t22) {$t_2(\bot_2)$};
\node[left =.25cm of c2p, yshift=-.8cm] (t23) {$t_2(\top_3)$};
\draw [-] (c2p) to (t23);
\draw [-] (c2p) to (t22);
}

\uncover<\s->{\node[right = .3cm of c2p](c3){$c_3\phantom{'}$};}
\uncover<\myadd{\s}{1}->{\draw [-] (c2p) to (c3);}

\uncover<\myadd{\x}{5}->{
\node[right =.25cm of c3, yshift=.8cm] (t3x1) {$t_3(\hat{x}_1)$};
\node[right =.25cm of c3, yshift=-.8cm] (t3x3) {$t_3(\hat{x}_3)$};
\draw [-] (c3) to (t3x1);
\draw [-] (c3) to (t3x3);
}

\uncover<\myadd{\s}{1}->{\node[right = 3.1cm of c3](c3p){$c_3'$};}

\uncover<\myadd{\x}{6}->{
\node[left =.25cm of c3p, yshift=.8cm] (t31) {$t_3(\bot_1)$};
\node[left =.25cm of c3p, yshift=-.8cm] (t33) {$t_3(\bot_3)$};
\draw [-] (c3p) to (t31);
\draw [-] (c3p) to (t33);
}

\uncover<\myadd{\s}{2}->{
\node [below =2cm of c2] (x1) {$\hat{x}_1$};
\node [below =.5cm of x1] (x2) {$\hat{x}_2$};
\node [below =.5cm of x2] (x3) {$\hat{x}_3$};
}

\uncover<\myadd{\s}{3}->{
\node [right = of x1] (t1) {$\top_1$};
\node [left = of x1] (f1) {$\bot_1$};

\draw [-] (x1) to (t1);
\draw [-] (x1) to (f1);

\node [right = of x2] (t2) {$\top_2$};
\node [left = of x2] (f2) {$\bot_2$};

\draw [-] (x2) to (t2);
\draw [-] (x2) to (f2);

\node [right = of x3] (t3) {$\top_3$};
\node [left = of x3] (f3) {$\bot_3$};

\draw [-] (x3) to (t3);
\draw [-] (x3) to (f3);
}

%\draw [dotted] (t1x1) to (t11);
%\draw [dotted] (t1x3) to (t13);
%\draw [dotted] (t2x2) to (t22);
%\draw [dotted] (t2x3) to (t23);
%\draw [dotted] (t3x2) to (t32);

\end{tikzpicture}

}

\end{frame}

\begin{frame}

\frametitle{Proof Arguments}

\begin{itemize}
\setlength\itemsep{1cm}
\uncover<2->{
\item Translate assignment $\mathcal{I}$ to subset of equations $E'$:
	\begin{align*} 
		x_i \in \mathcal{I} &\Leftrightarrow \hat{x_i} = \top_i \in E'\\
		\neg x_i \in \mathcal{I} &\Leftrightarrow \hat{x_i} = \bot_i \in E'
	\end{align*}
}
\par
\noindent\phantom{\parbox{\linewidth}{%
\item Every short explanation contains the translation of an assignment
\item Satisfying assignments translate to short explanations
%\begin{itemize}
	%\item $\phi$ satisfiable then there is a short explanation
%\end{itemize}
\item Non satisfying assignments do not translate to explanations
%\begin{itemize}
	%\item $\phi$ unsatisfiable then there is no (short) explanation
%\end{itemize}
}}\par
\end{itemize}

\end{frame}

\begin{frame}

\frametitle{Example of Reduction}

\begin{equation*}
%\big\{C_1 = x_1 \vee x_2 \vee \neg x_3, C_2 = \neg x_2 \vee x_3, C_3 = \neg x_1 \vee \neg x_3\big\}.
\phi = (x_1 \vee x_2 \vee \neg x_3) \wedge (\neg x_2 \vee x_3) \wedge (\neg x_1 \vee \neg x_3)
\end{equation*}
%$$n = 3, m = 3$$
\centering Equations $E$ $( a - b \leadsto a = b \in E)$\\
\centering Target Equation $c_1 = c'_3$
\resizebox{\linewidth}{!}{
\centering
\begin{tikzpicture}[node distance=.8cm]
\node(c1){$c_1\phantom{'}$};


\node[right =.25cm of c1] (t1x2) {$t_1(\hat{x}_2)$};
\node[above =.8cm of t1x2] (t1x1) {$t_1(\hat{x}_1)$};
\node[below =.8cm of t1x2] (t1x3) {$t_1(\hat{x}_3)$};
\draw [-] (c1) to (t1x1);
\draw [-] (c1) to (t1x2);
\draw [-] (c1) to (t1x3);

\node[right = 3.1cm of c1](c1p){$c_1'$};

\node[left =.25cm of c1p] (t12) {$t_1(\top_2)$};
\node[above=.8cm of t12] (t11) {$t_1(\top_1)$};
\node[below=.8cm of t12] (t13) {$t_1(\bot_3)$};

\draw [-] (c1p) to (t11);
\draw [-] (c1p) to (t12);
\draw [-] (c1p) to (t13);

\node[right = .3cm of c1p](c2){$c_2\phantom{'}$};
\draw [-] (c1p) to (c2);

\node[right=.25cm of c2, yshift=.8cm]  (t2x2) {$t_2(\hat{x}_2)$};
\node[right=.25cm of c2, yshift=-.8cm] (t2x3) {$t_2(\hat{x}_3)$};
\draw [-] (c2) to (t2x2);
\draw [-] (c2) to (t2x3);

\node[right = 3.1cm of c2](c2p){$c_2'$};

\node[left =.25cm of c2p, yshift=.8cm]  (t22) {$t_2(\bot_2)$};
\node[left =.25cm of c2p, yshift=-.8cm] (t23) {$t_2(\top_3)$};
\draw [-] (c2p) to (t23);
\draw [-] (c2p) to (t22);

\node[right = .3cm of c2p](c3){$c_3\phantom{'}$};
\draw [-] (c2p) to (c3);


\node[right =.25cm of c3, yshift=.8cm] (t3x1) {$t_3(\hat{x}_1)$};
\node[right =.25cm of c3, yshift=-.8cm] (t3x3) {$t_3(\hat{x}_3)$};
\draw [-] (c3) to (t3x1);
\draw [-] (t3x3) to (c3);

\node[right = 3.1cm of c3](c3p){$c_3'$};

\node[left =.25cm of c3p, yshift=.8cm] (t31) {$t_3(\bot_1)$};
\node[left =.25cm of c3p, yshift=-.8cm] (t33) {$t_3(\bot_3)$};
\draw [-] (c3p) to (t31);
\draw [-] (c3p) to (t33);

\node [below =2cm of c2] (x1) {$\hat{x}_1$};
\node [below =.5cm of x1] (x2) {$\hat{x}_2$};
\node [below =.5cm of x2] (x3) {$\hat{x}_3$};

%\alt<8-9>{\draw [-,line width = 2pt,red] (x3) to (t3);}{\draw [-] (x3) to (t3);}

\node [right = of x1] (t1) {$\top_1$};
\node [left = of x1] (f1) {$\bot_1$};

\alt<4>{\draw [-,line width = 2pt,red] (x1) to (t1);}{\draw [-] (x1) to (t1);}
\alt<6->{\draw [-,line width = 2pt,red] (x1) to (f1);}{\draw [-] (x1) to (f1);}


\node [right = of x2] (t2) {$\top_2$};
\node [left = of x2] (f2) {$\bot_2$};


\draw [-] (x2) to (t2);
\alt<4,6->{\draw [-,line width = 2pt,red] (x2) to (f2);}{\draw [-] (x2) to (f2);}

\node [right = of x3] (t3) {$\top_3$};
\node [left = of x3] (f3) {$\bot_3$};


\alt<6>{\draw [-,line width = 2pt,red] (x3) to (t3);}{\draw [-] (x3) to (t3);}
\alt<4>{\draw [-,line width = 2pt,red] (x3) to (f3);}{\draw [-] (x3) to (f3);}

%\uncover<5-6,9,12-14>{\draw [dotted,red,line width = 2pt] ($(t1x1.east) -(3pt,0)$) to ($(t11.west)+(3pt,0)$);}
%\uncover<12-13>  {\draw [dotted,red,line width = 2pt] ($(t3x1.east) -(3pt,0)$) to ($(t31.west)+(3pt,0)$);}
%\uncover<5-6,9,12-14>{\draw [dotted,red,line width = 2pt] ($(t2x2.east) -(3pt,0)$) to ($(t22.west)+(3pt,0)$);}
%\uncover<5-6,14>{\draw [dotted,red,line width = 2pt] ($(t1x3.east) -(3pt,0)$) to ($(t13.west)+(3pt,0)$);}
%\uncover<5-6,14>{\draw [dotted,red,line width = 2pt] ($(t3x3.east) -(3pt,0)$) to ($(t33.west)+(3pt,0)$);}

\uncover<2->{
\node [right =.3cm of t1, align=left,xshift=1.5cm] (assignments) {Assignment:};
}

\uncover<3->{
\node [below =.05cm of assignments] (i1) {
\alt<5->{
		\begin{tabular}{l}
	  $\mathcal{I}_2 = \{\neg x_1,\neg x_2, x_3\}$\\
		%$\mathcal{I}_2 \not\models \phi$\\
		%\color{red}{No explanation}
		\end{tabular}
	}{
	\begin{tabular}{l}
		$\mathcal{I}_1 = \{x_1,\neg x_2, \neg x_3\}$\\
		%$\mathcal{I}_1 \models \phi$\\
		%\color{red}{Explanation $E_1$}
		\end{tabular}
	}
};
}
%
%\uncover<13->{
	%\node [below =.1cm of i1,xshift=1.5cm] (solution) {
		%\alt<14>{
			%\begin{tabular}{l}
				%$n=3,m=3$\\
				%$|E_1| = 11 = 3n + 4m - 1 (-3m)$
			%\end{tabular}
		%}{
		%\begin{tabular}{l}
			%Include clauses \\
			%$(x_1 \vee \neg x_1)$,$(x_2 \vee \neg x_2)$,$(x_3 \vee \neg x_3)$
			%\end{tabular}
		%}
	%};
%}

\end{tikzpicture}

}

%\centering \color{red}{Explanation $E'$ for $c_1 = c'_3$\only<15->{\\Size $11 = 3n + 4m - 1 (-3m)$\\$(-3m)$ for clauses $(x_1 \vee \neg x_1), (x_2 \vee \neg x_2), (x_3 \vee \neg x_3)$}}

\end{frame}

\begin{frame}

\frametitle{Proof Arguments}

\begin{itemize}
\setlength\itemsep{1cm}

\item Translate assignment $\mathcal{I}$ to subset of equations $E'$:
	\begin{align*} 
		x_i \in \mathcal{I} &\Leftrightarrow \hat{x_i} = \top_i \in E'\\
		\neg x_i \in \mathcal{I} &\Leftrightarrow \hat{x_i} = \bot_i \in E'
	\end{align*}

\uncover<2->{\item Every short explanation contains the translation of an assignment}
\par
\noindent\phantom{\parbox{\linewidth}{%
\item Satisfying assignments translate to short explanations
%\begin{itemize}
	%\item $\phi$ satisfiable then there is a short explanation
%\end{itemize}
\item Non satisfying assignments are no explanations
%\begin{itemize}
	%\item $\phi$ unsatisfiable then there is no (short) explanation
%\end{itemize}
}}\par
\end{itemize}

\end{frame}

\begin{frame}

\frametitle{Example of Reduction}

\begin{equation*}
%\big\{C_1 = x_1 \vee x_2 \vee \neg x_3, C_2 = \neg x_2 \vee x_3, C_3 = \neg x_1 \vee \neg x_3\big\}.
\phi = (x_1 \vee x_2 \vee \neg x_3) \wedge (\neg x_2 \vee x_3) \wedge (\neg x_1 \vee \neg x_3)
\end{equation*}
%$$n = 3, m = 3$$
\centering Equations $E$ $( a - b \leadsto a = b \in E)$\\
\centering Target Equation $c_1 = c'_3$
\resizebox{\linewidth}{!}{
\centering
\begin{tikzpicture}[node distance=.8cm]
\node(c1){$c_1\phantom{'}$};


\node[right =.25cm of c1] (t1x2) {$t_1(\hat{x}_2)$};
\node[above =.8cm of t1x2] (t1x1) {$t_1(\hat{x}_1)$};
\node[below =.8cm of t1x2] (t1x3) {$t_1(\hat{x}_3)$};
\alt<2-5>{\draw [-,line width = 2pt,red] (c1) to (t1x1);}{\draw [-] (c1) to (t1x1);}
\draw [-] (c1) to (t1x2);
\alt<6->{\draw [-,line width = 2pt,red] (c1) to (t1x3);}{\draw [-] (c1) to (t1x3);}

\node[right = 3.1cm of c1](c1p){$c_1'$};

\node[left =.25cm of c1p] (t12) {$t_1(\top_2)$};
\node[above=.8cm of t12] (t11) {$t_1(\top_1)$};
\node[below=.8cm of t12] (t13) {$t_1(\bot_3)$};

\alt<2-5>{\draw [-,line width = 2pt,red] (c1p) to (t11);}{\draw [-] (c1p) to (t11);}
\draw [-] (c1p) to (t12);
\alt<6->{\draw [-,line width = 2pt,red] (c1p) to (t13);}{\draw [-] (c1p) to (t13);}

\node[right = .3cm of c1p](c2){$c_2\phantom{'}$};
\alt<2->{\draw [-,line width = 2pt,red] (c1p) to (c2);}{\draw [-] (c1p) to (c2);}

\node[right=.25cm of c2, yshift=.8cm]  (t2x2) {$t_2(\hat{x}_2)$};
\node[right=.25cm of c2, yshift=-.8cm] (t2x3) {$t_2(\hat{x}_3)$};
\alt<2-6>{\draw [-,line width = 2pt,red] (c2) to (t2x2);}{\draw [-] (c2) to (t2x2);}
\alt<7->{\draw [-,line width = 2pt,red] (c2) to (t2x3);}{\draw [-] (c2) to (t2x3);}

\node[right = 3.1cm of c2](c2p){$c_2'$};

\node[left =.25cm of c2p, yshift=.8cm]  (t22) {$t_2(\bot_2)$};
\node[left =.25cm of c2p, yshift=-.8cm] (t23) {$t_2(\top_3)$};
\alt<7->{\draw [-,line width = 2pt,red] (c2p) to (t23);}{\draw [-] (c2p) to (t23);}
\alt<2-6>{\draw [-,line width = 2pt,red] (c2p) to (t22);}{\draw [-] (c2p) to (t22);}

\node[right = .3cm of c2p](c3){$c_3\phantom{'}$};
\alt<2->{\draw [-,line width = 2pt,red] (c2p) to (c3);}{\draw [-] (c2p) to (c3);}


\node[right =.25cm of c3, yshift=.8cm] (t3x1) {$t_3(\hat{x}_1)$};
\node[right =.25cm of c3, yshift=-.8cm] (t3x3) {$t_3(\hat{x}_3)$};
\draw [-] (c3) to (t3x1);
\alt<2->{\draw [-,line width = 2pt,red] (t3x3) to (c3);}{\draw [-] (t3x3) to (c3);}

\node[right = 3.1cm of c3](c3p){$c_3'$};

\node[left =.25cm of c3p, yshift=.8cm] (t31) {$t_3(\bot_1)$};
\node[left =.25cm of c3p, yshift=-.8cm] (t33) {$t_3(\bot_3)$};
\draw [-] (c3p) to (t31);
\alt<2->{\draw [-,line width = 2pt,red] (c3p) to (t33);}{\draw [-] (c3p) to (t33);}

\node [below =2cm of c2] (x1) {$\hat{x}_1$};
\node [below =.5cm of x1] (x2) {$\hat{x}_2$};
\node [below =.5cm of x2] (x3) {$\hat{x}_3$};

\node [right = of x1] (t1) {$\top_1$};
\node [left = of x1] (f1) {$\bot_1$};

\alt<4-5>{\draw [-,line width = 2pt,red] (x1) to (t1);}{\draw [-] (x1) to (t1);}
\draw [-] (x1) to (f1);


\node [right = of x2] (t2) {$\top_2$};
\node [left = of x2] (f2) {$\bot_2$};


\draw [-] (x2) to (t2);
\alt<4-6>{\draw [-,line width = 2pt,red] (x2) to (f2);}{\draw [-] (x2) to (f2);}

\node [right = of x3] (t3) {$\top_3$};
\node [left = of x3] (f3) {$\bot_3$};

\alt<7->{\draw [-,line width = 2pt,red] (x3) to (t3);}{\draw [-] (x3) to (t3);}
\alt<4->{\draw [-,line width = 2pt,red] (x3) to (f3);}{\draw [-] (x3) to (f3);}

\uncover<5>{\draw [dotted,red,line width = 2pt] ($(t1x1.east) -(3pt,0)$) to ($(t11.west)+(3pt,0)$);}
%\uncover<12-13>  {\draw [dotted,red,line width = 2pt] ($(t3x1.east) -(3pt,0)$) to ($(t31.west)+(3pt,0)$);}
\uncover<5-6>{\draw [dotted,red,line width = 2pt] ($(t2x2.east) -(3pt,0)$) to ($(t22.west)+(3pt,0)$);}
\uncover<5->{\draw [dotted,red,line width = 2pt] ($(t1x3.east) -(3pt,0)$) to ($(t13.west)+(3pt,0)$);}
\uncover<5->{\draw [dotted,red,line width = 2pt] ($(t3x3.east) -(3pt,0)$) to ($(t33.west)+(3pt,0)$);}
\uncover<7->{\draw [dotted,red,line width = 2pt] ($(t2x3.east) -(3pt,0)$) to ($(t23.west)+(3pt,0)$);}

\uncover<2->{
\node [right =.15cm of t1, align=left,xshift=1.5cm] (short) {$3n-1$ equations for clauses};
}
\uncover<3->{
\node [below =.15cm of short, align=left] (short2) {$m$ equations for assignment};
}
\uncover<8->{
\node [below =.15cm of short2, align=center] (solution) {Force total assignment by adding clauses\\$(x_1 \vee \neg x_1)$,$(x_2 \vee \neg x_2)$,$(x_3 \vee \neg x_3)$\\[.15cm]$3m$ equations for extra clauses};
}
%
%\uncover<3->{
%\node [below =.05cm of assignments] (i1) {
%\alt<7-13>{
	%\alt<10-13>{
		%\begin{tabular}{l}
			%$\mathcal{I}_3 = \{x_1, \neg x_2\}$\\
			%$\mathcal{I}_3 \not \models \phi$\\
			%\color{red}{Problem}
		%\end{tabular}
		%}{
		%\begin{tabular}{l}
	  %$\mathcal{I}_2 = \{x_1,\neg x_2, x_3\}$\\
		%$\mathcal{I}_2 \not\models \phi$\\
		%\color{red}{No explanation}
		%\end{tabular}
	%}
	%}{
	%\begin{tabular}{l}
		%$\mathcal{I}_1 = \{x_1,\neg x_2, \neg x_3\}$\\
		%$\mathcal{I}_1 \models \phi$\\
		%\color{red}{Explanation $E_1$}
		%\end{tabular}
	%}
%};
%}
%
%\uncover<13->{
	%\node [below =.1cm of i1,xshift=1.5cm] (solution) {
		%\alt<14>{
			%\begin{tabular}{l}
				%$n=3,m=3$\\
				%$|E_1| = 11 = 3n + 4m - 1 (-3m)$
			%\end{tabular}
		%}{
		%\begin{tabular}{l}
			%Include clauses \\
			%$(x_1 \vee \neg x_1)$,$(x_2 \vee \neg x_2)$,$(x_3 \vee \neg x_3)$
			%\end{tabular}
		%}
	%};
%}

\end{tikzpicture}

}

%\centering \color{red}{Explanation $E'$ for $c_1 = c'_3$\only<15->{\\Size $11 = 3n + 4m - 1 (-3m)$\\$(-3m)$ for clauses $(x_1 \vee \neg x_1), (x_2 \vee \neg x_2), (x_3 \vee \neg x_3)$}}

\end{frame}

\begin{frame}

\frametitle{Proof Arguments}

\begin{itemize}
\setlength\itemsep{1cm}

\item Translate assignment $\mathcal{I}$ to subset of equations $E'$:
	\begin{align*} 
		x_i \in \mathcal{I} &\Leftrightarrow \hat{x_i} = \top_i \in E'\\
		\neg x_i \in \mathcal{I} &\Leftrightarrow \hat{x_i} = \bot_i \in E'
	\end{align*}

\item Every short explanation contains the translation of an assignment

\uncover<2->{\item Satisfying assignments translate to short explanations}
%\begin{itemize}
	%\item \uncover<2->{$\phi$ satisfiable then there is a short explanation}
%\end{itemize}

\uncover<3->{
\item Non satisfying assignments do not translate to explanations
}
%\begin{itemize}
	%\uncover<3->{\item $\phi$ unsatisfiable then there is no (short) explanation}
%\end{itemize}

\end{itemize}

\end{frame}

\begin{frame}

\frametitle{Example of Reduction}

\begin{equation*}
%\big\{C_1 = x_1 \vee x_2 \vee \neg x_3, C_2 = \neg x_2 \vee x_3, C_3 = \neg x_1 \vee \neg x_3\big\}.
\phi = (x_1 \vee x_2 \vee \neg x_3) \wedge (\neg x_2 \vee x_3) \wedge (\neg x_1 \vee \neg x_3)
\end{equation*}
%$$n = 3, m = 3$$
\centering Equations $E$ $( a - b \leadsto a = b \in E)$\\
\centering Target Equation $c_1 = c'_3$
\resizebox{\linewidth}{!}{
\centering
\begin{tikzpicture}[node distance=.8cm]
\node(c1){$c_1\phantom{'}$};


\node[right =.25cm of c1] (t1x2) {$t_1(\hat{x}_2)$};
\node[above =.8cm of t1x2] (t1x1) {$t_1(\hat{x}_1)$};
\node[below =.8cm of t1x2] (t1x3) {$t_1(\hat{x}_3)$};
\alt<6>{\draw [-,line width = 2pt,red] (c1) to (t1x1);}{\draw [-] (c1) to (t1x1);}
\draw [-] (c1) to (t1x2);
\draw [-] (c1) to (t1x3);

\node[right = 3.1cm of c1](c1p){$c_1'$};

\node[left =.25cm of c1p] (t12) {$t_1(\top_2)$};
\node[above=.8cm of t12] (t11) {$t_1(\top_1)$};
\node[below=.8cm of t12] (t13) {$t_1(\bot_3)$};

\alt<6>{\draw [-,line width = 2pt,red] (c1p) to (t11);}{\draw [-] (c1p) to (t11);}
\draw [-] (c1p) to (t12);
\draw [-] (c1p) to (t13);

\node[right = .3cm of c1p](c2){$c_2\phantom{'}$};
\alt<6>{\draw [-,line width = 2pt,red] (c1p) to (c2);}{\draw [-] (c1p) to (c2);}

\node[right=.25cm of c2, yshift=.8cm]  (t2x2) {$t_2(\hat{x}_2)$};
\node[right=.25cm of c2, yshift=-.8cm] (t2x3) {$t_2(\hat{x}_3)$};
\alt<6>{\draw [-,line width = 2pt,red] (c2) to (t2x2);}{\draw [-] (c2) to (t2x2);}
\draw [-] (c2) to (t2x3);

\node[right = 3.1cm of c2](c2p){$c_2'$};

\node[left =.25cm of c2p, yshift=.8cm]  (t22) {$t_2(\bot_2)$};
\node[left =.25cm of c2p, yshift=-.8cm] (t23) {$t_2(\top_3)$};
\draw [-] (c2p) to (t23);
\alt<6>{\draw [-,line width = 2pt,red] (c2p) to (t22);}{\draw [-] (c2p) to (t22);}

\node[right = .3cm of c2p](c3){$c_3\phantom{'}$};
\alt<6>{\draw [-,line width = 2pt,red] (c2p) to (c3);}{\draw [-] (c2p) to (c3);}


\node[right =.25cm of c3, yshift=.8cm] (t3x1) {$t_3(\hat{x}_1)$};
\node[right =.25cm of c3, yshift=-.8cm] (t3x3) {$t_3(\hat{x}_3)$};
\draw [-] (c3) to (t3x1);
\alt<6>{\draw [-,line width = 2pt,red] (t3x3) to (c3);}{\draw [-] (t3x3) to (c3);}

\node[right = 3.1cm of c3](c3p){$c_3'$};

\node[left =.25cm of c3p, yshift=.8cm] (t31) {$t_3(\bot_1)$};
\node[left =.25cm of c3p, yshift=-.8cm] (t33) {$t_3(\bot_3)$};
\draw [-] (c3p) to (t31);
\alt<6>{\draw [-,line width = 2pt,red] (c3p) to (t33);}{\draw [-] (c3p) to (t33);}

\node [below =2cm of c2] (x1) {$\hat{x}_1$};
\node [below =.5cm of x1] (x2) {$\hat{x}_2$};
\node [below =.5cm of x2] (x3) {$\hat{x}_3$};

\node [right = of x1] (t1) {$\top_1$};
\node [left = of x1] (f1) {$\bot_1$};

\alt<4-6,8-9>{\draw [-,line width = 2pt,red] (x1) to (t1);}{\draw [-] (x1) to (t1);}
\draw [-] (x1) to (f1);

\node [right = of x2] (t2) {$\top_2$};
\node [left = of x2] (f2) {$\bot_2$};

\draw [-] (x2) to (t2);
\alt<4-6,8-9>{\draw [-,line width = 2pt,red] (x2) to (f2);}{\draw [-] (x2) to (f2);}

\node [right = of x3] (t3) {$\top_3$};
\node [left = of x3] (f3) {$\bot_3$};

\alt<8-9>{\draw [-,line width = 2pt,red] (x3) to (t3);}{\draw [-] (x3) to (t3);}
\alt<4-6>{\draw [-,line width = 2pt,red] (x3) to (f3);}{\draw [-] (x3) to (f3);}

\uncover<5-6,9>{\draw [dotted,red,line width = 2pt] ($(t1x1.east) -(3pt,0)$) to ($(t11.west)+(3pt,0)$);}
%\uncover<12-13>  {\draw [dotted,red,line width = 2pt] ($(t3x1.east) -(3pt,0)$) to ($(t31.west)+(3pt,0)$);}
\uncover<5-6,9>{\draw [dotted,red,line width = 2pt] ($(t2x2.east) -(3pt,0)$) to ($(t22.west)+(3pt,0)$);}
\uncover<5-6>{\draw [dotted,red,line width = 2pt] ($(t1x3.east) -(3pt,0)$) to ($(t13.west)+(3pt,0)$);}
\uncover<5-6>{\draw [dotted,red,line width = 2pt] ($(t3x3.east) -(3pt,0)$) to ($(t33.west)+(3pt,0)$);}

\uncover<2->{
\node [right =.3cm of t1, align=left,xshift=1.5cm] (assignments) {Assignment:};
}

\uncover<3->{
\node [below =.05cm of assignments] (i1) {
\alt<7->{
		\begin{tabular}{l}
	  $\mathcal{I}_2 = \{x_1,\neg x_2, x_3\}$\\
		$\mathcal{I}_2 \not\models \phi$\\
		\color{red}{No explanation}
		\end{tabular}
	}{
	\begin{tabular}{l}
		$\mathcal{I}_1 = \{x_1,\neg x_2, \neg x_3\}$\\
		$\mathcal{I}_1 \models \phi$\\
		\color{red}{Short explanation $E'$}
		\end{tabular}
	}
};
}
\end{tikzpicture}

}

%\centering \color{red}{Explanation $E'$ for $c_1 = c'_3$\only<15->{\\Size $11 = 3n + 4m - 1 (-3m)$\\$(-3m)$ for clauses $(x_1 \vee \neg x_1), (x_2 \vee \neg x_2), (x_3 \vee \neg x_3)$}}

\end{frame}

\begin{frame}

\frametitle{NP-completeness of short explanation problem}

\begin{subpart}{In NP}
	\item Guess explanation and check with congruence closure algorithm
\end{subpart}

\begin{subpart}{NP-hardness}
	\item Reduction of NP-hard problem SAT
\end{subpart}
$$\mathcal{\phi} \text{ with } n \text{ clauses and } m \text{ variables is satisfiable}$$
$$\textit{if and only if}$$
$$\text{There exists an explanation } E' \subseteq E \text{ of } s=t \text{ with } |E'| \leq 3n + 4m - 1$$


\end{frame}

%\begin{frame}
%
%\frametitle{Example of Reduction}
%
%\begin{equation*}
%\big\{C_1 = x_1 \vee x_2 \vee \neg x_3, C_2 = \neg x_2 \vee x_3, C_3 = \neg x_1 \vee \neg x_3\big\}.
%\end{equation*}
%%$$n = 3, m = 3$$
%\centering Input equations $E$\\
%\resizebox{\linewidth}{!}{
\centering
\begin{tikzpicture}[node distance=.8cm]
\node(c1){$c_1\phantom{'}$};


\node[right =.25cm of c1] (t1x2) {$t_1(\hat{x}_2)$};
\node[above =.8cm of t1x2] (t1x1) {$t_1(\hat{x}_1)$};
\node[below =.8cm of t1x2] (t1x3) {$t_1(\hat{x}_3)$};
\alt<2->{\draw [-,very thick,red] (c1) to (t1x1);}{\draw [-] (c1) to (t1x1);}
\draw [-] (c1) to (t1x2);
\draw [-] (c1) to (t1x3);

\node[right = 3.1cm of c1](c1p){$c_1'$};

\node[left =.25cm of c1p] (t12) {$t_1(\top_2)$};
\node[above=.8cm of t12] (t11) {$t_1(\top_1)$};
\node[below=.8cm of t12] (t13) {$t_1(\bot_3)$};

\alt<5->{\draw [-,very thick,red] (c1p) to (t11);}{\draw [-] (c1p) to (t11);}
\draw [-] (c1p) to (t12);
\draw [-] (c1p) to (t13);

\node[right = .3cm of c1p](c2){$c_2\phantom{'}$};
\alt<6->{\draw [-,very thick,red] (c1p) to (c2);}{\draw [-] (c1p) to (c2);}

\node[right=.25cm of c2, yshift=.8cm]  (t2x2) {$t_2(\hat{x}_2)$};
\node[right=.25cm of c2, yshift=-.8cm] (t2x3) {$t_2(\hat{x}_3)$};
\alt<7->{\draw [-,very thick,red] (c2) to (t2x2);}{\draw [-] (c2) to (t2x2);}
\draw [-] (c2) to (t2x3);

\node[right = 3.1cm of c2](c2p){$c_2'$};

\node[left =.25cm of c2p, yshift=.8cm]  (t22) {$t_2(\bot_2)$};
\node[left =.25cm of c2p, yshift=-.8cm] (t23) {$t_2(\top_3)$};
\draw [-] (c2p) to (t23);
\alt<10->{\draw [-,very thick,red] (c2p) to (t22);}{\draw [-] (c2p) to (t22);}

\node[right = .3cm of c2p](c3){$c_3\phantom{'}$};
\alt<11->{\draw [-,very thick,red] (c2p) to (c3);}{\draw [-] (c2p) to (c3);}


\node[right =.25cm of c3, yshift=.8cm] (t3x1) {$t_3(\hat{x}_1)$};
\node[right =.25cm of c3, yshift=-.8cm] (t3x3) {$t_3(\hat{x}_3)$};
\draw [-] (c3) to (t3x1);
\alt<12->{\draw [-,very thick,red] (t3x3) to (c3);}{\draw [-] (t3x3) to (c3);}

\node[right = 3.1cm of c3](c3p){$c_3'$};

\node[left =.25cm of c3p, yshift=.8cm] (t31) {$t_3(\bot_1)$};
\node[left =.25cm of c3p, yshift=-.8cm] (t33) {$t_3(\bot_3)$};
\draw [-] (c3p) to (t31);
\alt<14->{\draw [-,very thick,red] (c3p) to (t33);}{\draw [-] (c3p) to (t33);}

\node [below =1.3cm of c2, xshift=1.5cm] (x1) {$\hat{x}_1$};
\node [below =.5cm of x1] (x2) {$\hat{x}_2$};
\node [below =.5cm of x2] (x3) {$\hat{x}_3$};

\node [right = of x1] (t1) {$\top_1$};
\node [left = of x1] (f1) {$\bot_1$};

\alt<3->{\draw [-,very thick,red] (x1) to (t1);}{\draw [-] (x1) to (t1);}
\draw [-] (x1) to (f1);

\node [right = of x2] (t2) {$\top_2$};
\node [left = of x2] (f2) {$\bot_2$};

\draw [-] (x2) to (t2);
\alt<8->{\draw [-,very thick,red] (x2) to (f2);}{\draw [-] (x2) to (f2);}

\node [right = of x3] (t3) {$\top_3$};
\node [left = of x3] (f3) {$\bot_3$};

\draw [-] (x3) to (t3);
\alt<13->{\draw [-,very thick,red] (x3) to (f3);}{\draw [-] (x3) to (f3);}

\uncover<4->{\draw [dotted,red,line width = 2pt] ($(t1x1.east) -(3pt,0)$) to ($(t11.west)+(3pt,0)$);}
%\draw [dotted] (t1x3) to (t13);
\uncover<9->{\draw [dotted,red,line width = 2pt] ($(t2x2.east) -(3pt,0)$) to ($(t22.west)+(3pt,0)$);}
%\draw [dotted] (t2x3) to (t23);
\uncover<13->{\draw [dotted,red,line width = 2pt] ($(t3x3.east) -(3pt,0)$) to ($(t33.west)+(3pt,0)$);}

\end{tikzpicture}

}
%
%\centering \color{red}{Explanation $E'$ for $c_1 = c'_3$\only<15->{\\Size $11 = 3n + 4m - 1 (-3m)$\\$(-3m)$ for clauses $(x_1 \vee \neg x_1), (x_2 \vee \neg x_2), (x_3 \vee \neg x_3)$}}
%
%\end{frame}

\begin{frame}

\frametitle{Small explanations as shortest paths}


\centering
%\documentclass{standalone}
%\usepackage{tikz}
%\usetikzlibrary{positioning}
%\usetikzlibrary{arrows}
%\usetikzlibrary{calc}
%
%
 %\begin{document}

\centering
\begin{tikzpicture}[node distance=.8cm]

\node (fa) {$f(a)$};
\node[below =of fa] (a) {$a$};
\node[below =of a] (b) {$b$};
\node[below =of b] (fb) {$f(b)$};

\draw [-] (fa) to node[xshift=.2cm] {1} (a);

\draw [-] (a) to node[xshift=.2cm] {1} (b);

\draw [-] (b) to node[xshift=.2cm] {1} (fb);

\draw [-,in=120,out=240] (fa) to node[xshift=-.2cm] {1} (fb);

\draw [->,dotted,line width = 1.4pt]  ($(a) - (.6cm,.6cm)$) to ($(a) - (.1cm,.6cm)$);


\end{tikzpicture}

%\end{document}


\end{frame}

\begin{frame}

\frametitle{Small explanations as shortest paths}


\centering
%\documentclass{standalone}
%\usepackage{tikz}
%\usetikzlibrary{positioning}
%\usetikzlibrary{arrows}
%\usetikzlibrary{calc}
%
%
 %\begin{document}

\centering
\begin{tikzpicture}[node distance=.8cm]

\node (a) {$a$};
\node[below = of a] (b) {$b$};
\node[below = of b] (t1) {$t_1$};
\node[below = of t1] (t2) {$t_2$};
\node[below = of t2] (t3) {$t_3$};
\node[below = of t3] (c) {$c$};
\node[below = of c] (d) {$d$};

\node[right = of t1] (fac) {$f(a,c)$};
\node[right = of t3] (fbd) {$f(b,d)$};

\draw [-] (a) to node[xshift=-.2cm] {1} (b);

\draw [-] (b) to node[xshift=-.2cm] {1} (t1);

\draw [-] (t1) to node[xshift=-.2cm] {1} (t2);

\draw [-] (t2) to node[xshift=-.2cm] {1} (t3);

\draw [-] (t3) to node[xshift=-.2cm] {1} (c);

\draw [-] (c) to node[xshift=-.2cm] {1} (d);

\draw [-] (b) to node[xshift=.1cm,yshift=-.4cm] {1} (fac);

\draw [-] (c) to node[xshift=.1cm,yshift=.4cm] {1} (fbd);

\draw [-] (fac) to node[xshift=-.2cm] {2} (fbd);


\draw [->,dotted,out=30,in=350,bend=90,line width = 1.4pt]  ($(fac) - (-.2cm,1.1cm)$) to ($(a) - (-.1cm,.6cm)$);
\draw [->,dotted,out=330,in=10,bend=90,line width = 1.4pt]  ($(fac) - (-.2cm,1.3cm)$) to ($(c) - (-.1cm,.6cm)$);

\end{tikzpicture}

%\end{document}


\end{frame}

\begin{frame}

\frametitle{Conclusion}

\begin{itemize}
\setlength\itemsep{1cm}
\item Small conflict sets are desirable
%\begin{itemize}
%
	%\item Speed up decision procedure
	%\item Smaller proofs
%\end{itemize}
\item Obtaining small conflict sets is NP-complete
\item Find algorithms/heuristics to construct small conflict sets

\end{itemize}

\end{frame}

\begin{frame}

\center Thank you for your attention !
\center Questions ?

\end{frame}

%\begin{frame}
%
%\frametitle{Theory of equality and uninterpreted functions}
%
	%\begin{subpart}{Syntax}
		%\item Propositional logic $x_1,x_1 \wedge x_2,\neg x_3$
		%\item Terms $a, f(b), g(t_1,\ldots,t_n)$
		%\item Equation $a = f(b)$
		%\item Negated Equation $\neg (f(a) = f(b)) \leadsto f(a) \neq f(b)$
	%\end{subpart}
%
	%\begin{subpart}{Axioms of equality}
		%\item \makebox[2.2cm]{Reflexivity:\hfill} $t = t$
		%\item \makebox[2.2cm]{Symmetry:\hfill} $s = t$ implies $t = s$
		%\item \makebox[2.2cm]{Transitivity:\hfill} $t_1 = t_2$ and $t_2 = t_3$ implies $t_1 = t_3$
		%\item \makebox[2.2cm]{Congruence:\hfill} $t_1 = s_1$ and $\ldots$ $t_n = s_n$ implies \\\makebox[2.4cm]{}$f(t_1, \dots, t_n) = f(s_1, \dots, s_n) $
	%\end{subpart}
	%
%\end{frame}

\end{document}