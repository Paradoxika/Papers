\section{Resolution extended with equality}
\label{sec:calculus}

Equality is a well researched topic in computational logic.
Among the most prominent approaches to deal with this special predicate are first-order resolution with paramodulation \cite{Robinson1969}, the superposition calculus \cite{Nieuwenhuis2001} and term rewrite systems \cite{Baader1998}.
We present equality in a framework that is closer to the propositional resolution calculus.
In fact we extend the calculus defined in Section \ref{sec:resolution} to take into account the axioms of equality.
By doing this we create a calculus with proofs that can be compressed both with traditional propositional logic as well as our novel congruence closure compression algorithm.
Pure propositional logic compression algorithms can simply abstract away the semantics of equality and treat equations as normal literals.
We start by extending the notions of atoms, literals and clauses

\begin{definition}[Equality- Atom, Literal and Clause]

Let $\mathcal{T}$ be a set of terms and let $P$ be a finite set of propositional variables.
The set of \emph{equality atoms} is defined as $P \cup \mathcal{T} \times \mathcal{T}$.
An \emph{equality literal} is an equality atom $e$ or a negated equality atom $\neg e$.
An \emph{equality clause} is a set of equality literals.
For an equality clause $C$, we call the sets of equations $pos(C) := \{(u,v) \mid u = v \in C\}$ the \emph{positive part} and $neg(C) := \{(u,v) \mid u \neq v \in C\}$ the \emph{negative part} of $C$.
The empty clause is denoted by $\bot$.

\end{definition}

A set of equations can be interpreted as a set of clauses, if every equation is interpreted as the singleton clause containing just the equation itself.
In the context of equality atoms, we write equations $(s,t) \in \mathcal{T} \times \mathcal{T}$ as $s = t$ and $s \neq t$ for its negated version.
As usual, a clause represents the disjunction of its literals and a set of clauses represents the conjunction of its elements.\\

From hereon, we restrict our attention to sets of terms that, on top of constants, have at most one function symbol $f$, which is binary.
We justify this restriction in Section \ref{subsec:algorithms_preliminaries}.
The axioms defining congruence relations have to be reflected in our extended resolution calculus.
We achieve this by defining axiom schemas, that can be instantiated with concrete terms.

\begin{definition}[Axioms of Equality]

In the following axioms schemas, the variables $x_1,\ldots,x_n$ are placeholders for terms.
By simultaneously replacing all variables by terms of some set $\mathcal{T}$, one obtains an equality clause, which we call an \emph{instance w.r.t. $\mathcal{T}$} of the respective axiom of equality.

\begin{itemize}
	\item reflexivity: $\{x = x\}$
	\item symmetry: $\{x_1 \neq x_2, x_2 = x_1\}$
	\item transitivity: $\{x_1 \neq x_2, x_2 \neq x_3, \ldots, x_{n-1} \neq x_n, x_1 = x_n\}$
	\item compatability: $\{x_1 \neq x_3, x_2 \neq x_4, f(x_1,x_2) = f(x_3,x_4)\}$
\end{itemize}

\end{definition}

Next we will define the resolution calculus extended by congruence axioms.

\begin{definition}[Resolution with Equality]

Let $\ell$ be an equality literal and $C_1$, $C_2$ be equality clauses such that $\ell \in C_1$ and $\neg \ell \in C_2$.
The clause $C_1 \setminus \{\ell\} \cup C_2 \setminus \{\neg \ell\}$ is the \emph{resolvent} of $C_1$ and $C_2$ with \emph{pivot} $\ell$.

\noindent Let $F = \{C_1, \ldots, C_n\}$ be a set of equality clauses and let $E$ be the largest subset of $F$, such that every clause in $E$ is an equation.
The notion of a \emph{congruence derivation} for $F$ is defined inductively.
\begin{itemize}
	\item $\langle C_1, \ldots, C_n\rangle$ is a congruence derivation for $F$.
	\item If $\langle C_1, \ldots, C_m\rangle$ is a congruence derivation for $F$ then $\langle C_1, \ldots, C_{m+1} \rangle$ is a congruence derivation for $F$ if $C_{m+1}$ is an instance w.r.t. $\mathcal{T}_E$ of an axiom of equality or $C_{m+1}$ is a resolvent of $C_i$ and $C_j$ with $1 \leq i,j \leq m$.
\end{itemize}
A \emph{congruence refutation} is a congruence derivation containing the empty clause.

\noindent Let $D = \langle C_1, \ldots, C_m\rangle$ be a congruence derivation.
The longest subsequence $\langle C_{i_1}, \ldots, C_{i_k}\rangle$ of $D$, such that $C_{i_1}, \ldots, C_{i_k}$ all are instances of axioms of equality, is called the equality reasoning part of $D$.

\end{definition}

Proofs in this extended calculus are defined in the same manner as in Section \ref{sec:resolution}.
The following proposition proves the Sound- and Completeness of the extended calculus relative to the notion of congruence closure.
Its Sound- and Completeness w.r.t. propositional logic is not violated by adding new kinds of literals to the calculus.
As stated before, if the semantics of equality are ignored, the extended calculus is simply the propositional resolution calculus.

\begin{proposition}[Sound- \& Completeness]
\label{prop:extended_resolution}
Let $E$ be a set of equations and $s,t \in \mathcal{T}_E$, then $E \models s \thickapprox t$ if and only if there is a congruence refutation for $E \cup \{\{ s \neq t\}\}$

\end{proposition}

\begin{proof}

The existence of a congruence refutation in case $E \models s \thickapprox t$ is proven in terms of a proof producing algorithm, presented in Section \ref{sec:proofproduction}.
This algorithm produces a congruence derivation with last clause $\{u_1 \neq v_1,\ldots,u_n \neq v_n, s = t\}$ such that $\{(u_i,v_i) \mid i = 1,\ldots,n\} \subseteq E$.
Clearly this proof can be extended to a congruence refutation for $E \cup \{ s \neq t\}$.

Suppose there is a congruence refutation $\langle C_1, \ldots, C_n \rangle$ for $E \cup \{\{ s \neq t\}\}$.
%Clearly we can assume that the equality reasoning part of this derivation is the whole sequence.
Since every clause in $E \cup \{\{ s \neq t\}\}$ is singleton, none of its literals is in the resolvent of such a clause with any other clause.
Therefore we can assume that there is a $m < n$ such that $\{C_1, \ldots, C_m\}$ only contains of instances of equality axioms and recursive resolvents of such clauses.
Furthermore, since the whole sequence is a refutation, we can assume that $C_m$ is such that $neg(C_m) \subseteq E$ and $pos(C_m) = \{(s,t)\}$.

We show by induction on the clause structure, that for every clause $C \in \{C_1,\ldots,C_m\}$ that $pos(C) = \{(u,v)\}$ for some $(u,v) \in \mathcal{T}_E$ and that $neg(C) \models u \thickapprox v$.
Suppose that $C$ is an instance of an equality axiom. Clearly, the positive part contains of some equation $(u,v)$ and $neg(C) \models u \thickapprox v$ follows directly form the definition of congruence closure, and in case of the transitivity axiom also from the transitivity of equality.
Let $C$ be obtained by resolving the clauses $D_1$ and $D_2$, such that $pos(D_i) = \{(u_i,v_i)\}$ and $neg(D_i) \models u_i \thickapprox v_i$ for $i \in \{1,2\}$.
Suppose $D_1$ and $D_2$ were resolved using the equality literal $u_1 = v_1$ (the only other possibility is $u_2 = v_2$ and the cases are symmetric).
Then $pos(C) = \{(u_2,v_2)\}$ and $neg(C) = neg(D_1) \cup (neg(D_2) \setminus (u_1,v_1))$.
Using the monotonicity of the $\models$ relation (Proposition \ref{prop:models}) and the fact that $neg(D_1) \subseteq neg(C)$ and $neg(D_2) \subseteq neg(C) \cup \{(u_1,v_1)\}$, it follows that $neg(C) \models u_1 \thickapprox v_1$ and $neg(C) \cup \{u_1,v_1\} \models (u_2,v_2)$.
Using the consitency of the $\models$ relation (Proposition \ref{prop:models}), it follows that $neg(C) \models u_2 \thickapprox v_2$, which is the desired result.

\end{proof}