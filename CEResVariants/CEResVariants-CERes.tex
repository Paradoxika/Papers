\section{Cut-Elimination by Resolution}
\label{sec:CERes}


The $\CERes$ method consists of four steps. Firstly, the proof with cuts $\psi$ is skolemized into a proof $\psi'$. Secondly, a \emph{characteristic formula} $\struct{\psi}$ is extracted from $\psi'$. Thirdly, as this formula is always unsatisfiable, it can be converted to a \emph{characteristic clause set} and refuted by resolution. The resulting resolution refutation $\delta$ (whose existence is guaranteed by the completeness of the resolution calculus \cite{Leitsch1997The-resolution-calculus}) is made of resolution and factoring inferences. In the last step, a sequent calculus proof $\CEResNF{\psi,\delta}$ having the same end-sequent as $\psi'$ and whose cuts are atomic can be obtained by converting resolution and factoring inferences from $\delta$ to, respectively, atomic cuts and contractions and replacing the leaves from $\delta$ by cut-free parts of $\psi'$ known as \emph{projections}.


\begin{definition}[Characteristic Formula]
\label{definition:PertinentStruct}
The \emph{characteristic formula} $\struct{\varphi}$ of a proof $\varphi$ is the characteristic formula $\struct{\rho^*}$ where $\rho^*$ is its lowermost inference and $\struct{\rho}$ is defined for each inference $\rho$ as follows:

\begin{itemize}
	\item If $\rho$ is an axiom with sequent $A \seq A$, four cases are distinguished:
	\begin{itemize}	 	
	 	\item Only the succedent's $A$ is a cut-ancestor: $\struct{\rho} = A$
	 	\item Only the antecedent's $A$ is a cut-ancestor: $\struct{\rho} = \structdual A $
	 	\item None is a cut-ancestor: $\struct{\rho} = \structtimesEmpty $
	 	\item Both are cut-ancestors: $\struct{\rho} = \structplusEmpty$
	\end{itemize}

	\item If $\rho$ is an $n$-ary inference and $\rho_1,\ldots,\rho_n$ are the inferences deriving the premises of $\rho$, two cases are distinguished:
	\begin{itemize}
		\item $\rho$ operates on cut-ancestors:
$$
\struct{\rho} \defEq 
		\struct{\rho_1} \structplus \ldots \structplus \struct{\rho_n}
$$

	\item $\rho$ does not operate on cut-ancestors::
$$
\struct{\rho} \defEq 
		\struct{\rho_1} \structtimes \ldots \structtimes \struct{\rho_n}
$$
	\end{itemize}
\end{itemize}
\end{definition}


\begin{example}
\label{example:CutPertinentStruct}

Let $\varphi$ be the proof below:
\begin{prooftree}
\AXC{$A \seq \hA{A} $}
		\AXC{$B \seq \hB{B}$} \RightLabel{$\wedge^1_r$}
	\BIC{$A, B \seq A \wedge B$} \RightLabel{$\wedge^6_l$}
	\UIC{$A \wedge B \seq A \wedge B$}
				\AXC{$\hC{B} \seq B $}
						\AXC{$\hD{A} \seq A$} \RightLabel{$\wedge^2_r$}
					\BIC{$A, B \seq B \wedge A$} \RightLabel{$\wedge^7_l$}
					\UIC{$A \wedge B \seq B \wedge A$} \RightLabel{$cut^3$}
			\BIC{$A \wedge B \seq B \wedge A$}
							\AXC{$C \seq \hE{C}$} 
									\AXC{$\hF{C} \seq C$} \RightLabel{$cut^4$}
								\BIC{$C \seq C$} \RightLabel{$\vee^5_l$}
					\BIC{$(A \wedge B) \vee C \seq B \wedge A, C $} 
\end{prooftree}

\noindent
Its characteristic formula is:
$
\struct{\varphi} 
\equiv 
((\hA{A} \structplus^1 \hB{B}) \structplus^3 (\structdual{\hC{B}} \structtimes^2 \structdual{\hD{A}}))
\structtimes^5
(\hE{C} \structplus^4 \structdual{\hF{C}})
$
\hfill\QED
\end{example}

\begin{theorem}
\label{theorem:Unsatisfiability}
%For any proof $\varphi$, 
$\all \overline{\alpha_{\varphi}}. \struct{\varphi}$, where $\overline{\alpha_{\varphi}}$ are the eigenvariables of $\varphi$, is unsatisfiable.
\end{theorem}
\begin{proof}
Recursively transform each subproof $\psi$ of $\varphi$ having end-sequent $\Gamma, \Gamma^* \seq \Delta, \Delta^*$, where $\Gamma^* \seq \Delta^*$ are cut-ancestors, into a proof $\psi'$ of $\all \overline{\alpha_{\psi}}. \struct{\rho}, \Gamma^* \seq \Delta^*$. By doing so, $\varphi$ itself is transformed into a proof $\varphi'$ with end-sequent $\all \overline{\alpha_{\varphi}}. \struct{\varphi} \seq$.
\hfill\QED
\end{proof}






\begin{definition}[Simple Transformation to Conjunctive Normal Form]
\label{definition:NormalizationPlusTimes}
\index{Struct Normalization!Standard}
A formula in negative normal form can be transformed into conjunctive normal form by rewriting it according to the following rule:
$$
S \structtimes (S_1 \structplus \ldots \structplus S_n)   \normalizePlusTimes  (S \structtimes S_1) \structplus \ldots \structplus (S \structtimes S_n) 
$$
%$$
%(S_1 \structplus \ldots \structplus S_n) \structtimes S  \normalizePlusTimes  (S_1 \structtimes S) \structplus \ldots \structplus (S_n \structtimes S) 
%$$
\end{definition}


\begin{definition}[Sequent Notation]
\label{definition:Clausification}
A formula in conjunctive normal form 
$$
\structplusBig_{i\in I} (\structtimesBig_{1 \leq j' \leq j_i} \structdual{A_{ij'}} \structtimes \structtimesBig_{1 \leq h' \leq h_i} B_{ih'})
$$
can be written in sequent notation as the set
$
%\clausify{S} \defEq 
\{ A_{i1}, \ldots, A_{ij_i} \seq B_{i1},\ldots, B_{ih_i} | i \in I \}
$.
\end{definition}





\begin{definition}[Clause Set]
\label{definition:CutPertinentClauseSet}
\index{Clause Set!Standard}
The \emph{clause set} $\clauseset{\varphi}$ of a proof $\varphi$ is the conjunctive normal form of $\struct{\varphi}$ written in sequent notation.
\end{definition}


\begin{example}
\label{example:CutPertinentStandardClauseSet}
Let $\varphi$ be the proof shown in Example \ref{example:CutPertinentStruct}. Its characteristic formula $\struct{\varphi}$ normalizes as:
\begin{scriptsize}
$$
\begin{array}{rcl}
\struct{\varphi} 
& \normalizePlusTimes^* &
(A \structtimes C) \structplus (A \structtimes \structdual{C})
\structplus 
(B \structtimes C) \structplus (B \structtimes \structdual{C})  
\structplus 
(\structdual{B} \structtimes \structdual{A} \structtimes C) 
\structplus 
(\structdual{B} \structtimes \structdual{A}\structtimes \structdual{C})
\\
\end{array}
$$
\end{scriptsize}

\noindent
Hence, the clause set of $\varphi$ is:
%
$$
\clauseset{\varphi} \equiv \{ \seq A,C \ \ ; \ \ C \seq A \ \ ;\ \ \seq B,C \ \ ; \ \ C \seq B \ \ ; \ \ B, A \seq C \ \ ; \ \ B, A, C \seq  \}
$$
and it can be refuted by the following resolution refutation $\delta$:
\begin{small}
\begin{prooftree}
\AXC{$\seq A,C$}
		\AXC{$C \seq A$} \RightLabel{$r$}
	\BIC{$\seq A,A$} \RightLabel{$f_r$}
	\UIC{$\seq A$}
			\AXC{$\seq B,C$}
					\AXC{$C \seq B$} \RightLabel{$r$}
				\BIC{$\seq B,B$} \RightLabel{$f_r$}
				\UIC{$\seq B$}
							\AXC{$B, A \seq C$}
									\AXC{$C, B, A \seq $} \RightLabel{$r$}
								\BIC{$B, A, B, A \seq $} \RightLabel{$f_r$}
								\UIC{$B, A, B \seq $} \RightLabel{$f_r$}
								\UIC{$A, B \seq $} \RightLabel{$r$}
						\BIC{$A \seq $} \RightLabel{$r$}
		\BIC{$ \seq $}
\end{prooftree}
\end{small}
\hfill\QED
\end{example}


\noindent
Resolution and factoring inferences are essentially atomic cuts and contractions with unification. Therefore, to obtain the $\CERes$-normal-form (an $\LK$-proof, with only atomic cuts, of the skolemized end-sequent of the original proof with cuts), one can replace each leaf of the refutation by a \emph{projection} with an appropriate end-sequent, apply all the unifiers and convert resolution and factoring inferences to atomic cuts and contractions.
Since a projection's purpose is to replace a leaf in a refutation of a clause set, its end-sequent must contain the leaf's clause as a subsequent. Moreover, if its end-sequent contains any other formula, then this formula must appear in the end-sequent of the original proof with cuts, because this formula is propagated downward after the replacement and thus necessarily appears in the end-sequent of the $\CERes$-normal-form. Otherwise, if the formula were not in the end-sequent of the original proof, the $\CERes$-normal-form's end-sequent would be necessarily different from that of the skolemized proof with cuts. Finally, a projection must, of course, be cut-free, otherwise the $\CERes$-normal-form would contain more cuts in addition to the inessential atomic cuts originating from the refutation. These three conditions are formally expressed in Definition \ref{definition:Projection}.

\begin{definition}[Projection]
\label{definition:Projection}
\index{Projection}
Let $\varphi$ be a proof with end-sequent $\Gamma \seq \Delta$ and $\Gamma_c \seq \Delta_c \in \clauseset{\varphi}$. Any cut-free proof of $\Gamma', \Gamma_c \seq \Delta', \Delta_c$, where $\Gamma' \subseteq \Gamma$, $\Delta' \subseteq \Delta$, is a \emph{projection} of $\varphi$ with respect to $\Gamma_c \seq \Delta_c$.
\end{definition}

\noindent
Projections can be easily constructed by extracting cut-free parts of the original proof with cuts. 
The original method \cite{BaazLeitsch2009MethodsofCut-Elimination,BaazHetzlLeitschRichterSpohr2006ProofTransformationbyCERES,BaazLeitsch2000Cut-eliminationandRedundancy-eliminationbyResolution,BaazLeitsch1999MethodsofCut-Elimination,Richter2006ProofTransformationsbyResolution--ComputationalMethodsofCut-Elimination} generates projections where $\Gamma' = \Gamma$ and $\Delta' = \Delta$. Here a slightly more optimized method that constructs less redundant projections, where $\Gamma' \subseteq \Gamma$ and $\Delta' \subseteq \Delta$, is described. 


%The method follows roughly three steps: firstly, cut-pertinent inferences are ``deleted'' (so that cut-pertinent occurrences are propagated down to the end-sequent, and thus the end-sequent now contains every clause of the clause set); then, occurrences and inferences that are not in a certain sense relevant to the occurrences of the specific clause under consideration are also ``deleted'' (this guarantees that only that clause will occur in the end-sequent of the projection); and finally, the problems introduced by the previous two steps are fixed.


\begin{definition}[Algorithm for Constructing Projections] %\hspace*{\fill} \\
\label{definition:OProjection}
Let $\varphi$ be a proof and $c$ be a clause from $\clauseset{\varphi}$. Let $A$ be the set of axioms of $\varphi$ that contain formulas that contribute to $c$. Then, $\projection{\varphi}{c}$ is constructed by taking from $\varphi$ only the inferences that operate on formulas that are both descendents of axioms in $A$ and end-sequent ancestors, and adding weakening inferences when necessary.
\end{definition}

\begin{example}
\label{example:OProjections}
For $\varphi$ of Example \ref{example:CutPertinentStruct}, the projections $\projection{\varphi}{\seq \hA{A}, \hE{C}}$ and $\projection{\varphi}{\seq \hB{B}, \hE{C}}$ are:
\begin{small}
\begin{multicols}{2}{
\begin{prooftree}
	\AXC{$\hA{A} \seq \hA{A}$} \RightLabel{$w_l$}
	\UIC{$\hA{A}, \hG{B} \seq \hA{A}$} \RightLabel{$\wedge_l^6$}
	\UIC{$\hA{A} \wedge \hG{B} \seq \hA{A}$}
							\AXC{$\hE{C} \seq \hE{C}$} \RightLabel{$\vee_l^5$}
					\BIC{$(\hA{A} \wedge \hG{B}) \vee \hE{C} \seq \hA{A}, \hE{C}$} 
\end{prooftree}

\begin{prooftree}
	\AXC{$\hB{B} \seq \hB{B}$} \RightLabel{$w_l$}
	\UIC{$\hG{A}, \hB{B} \seq \hB{B}$} \RightLabel{$\wedge_l^6$}
	\UIC{$\hG{A} \wedge \hB{B} \seq \hB{B}$}
							\AXC{$\hE{C} \seq \hE{C}$} \RightLabel{$\vee_l^5$}
					\BIC{$(\hG{A} \wedge \hB{B}) \vee \hE{C} \seq \hB{B}, \hE{C}$} 
\end{prooftree}
}
\end{multicols}
\end{small}

\noindent
The projections $\projection{\varphi}{\hF{C} \seq \hA{A}}$ and $\projection{\varphi}{\hF{C} \seq \hB{B}}$ are:
\begin{small}
\begin{multicols}{2}{
\begin{prooftree}
\AXC{$\hA{A} \seq \hA{A} $} \RightLabel{$w_l$}
\UIC{$\hA{A}, \hG{B} \seq \hA{A}$} \RightLabel{$\wedge_l^6$}
\UIC{$\hA{A} \wedge \hG{B} \seq \hA{A}$}
								\AXC{$\hF{C} \seq \hF{C}$} \RightLabel{$w_l$}
								\UIC{$\hG{C},\hF{C} \seq \hF{C}$} \RightLabel{$\vee_l^5$}
					\BIC{$(\hA{A} \wedge \hG{B}) \vee \hG{C}, \hF{C} \seq \hA{A}, \hF{C} $} 
\end{prooftree}

\begin{prooftree}
\AXC{$\hB{B} \seq \hB{B} $} \RightLabel{$w_l$}
\UIC{$\hG{A}, \hB{B} \seq \hB{B}$} \RightLabel{$\wedge_l^6$}
\UIC{$\hG{A} \wedge \hB{B} \seq \hB{B}$}
								\AXC{$\hF{C} \seq \hF{C}$} \RightLabel{$w_l$}
								\UIC{$\hG{C},\hF{C} \seq \hF{C}$} \RightLabel{$\vee_l^5$}
					\BIC{$(\hG{A} \wedge \hB{B}) \vee \hG{C}, \hF{C} \seq \hB{B}, \hF{C} $} 
\end{prooftree}
}
\end{multicols}
\end{small}

\noindent
The projections $\projection{\varphi}{\hC{B}, \hD{A} \seq \hE{C}}$ and $\projection{\varphi}{\hC{B}, \hD{A}, \hF{C} \seq}$ are:
\begin{multicols}{2}{
\begin{scriptsize}
\begin{prooftree}
		\AXC{$\hC{B} \seq \hC{B} $}
				\AXC{$\hD{A} \seq \hD{A}$} \RightLabel{$\wedge_r^2$}
			\BIC{$\hD{A}, \hC{B} \seq \hC{B} \wedge \hD{A}$} \RightLabel{$w_l$}
			\UIC{$\hG{A} \wedge \hG{B}, \hD{A}, \hC{B} \seq \hC{B} \wedge \hD{A}$}
							\AXC{$\hE{C} \seq \hE{C}$} \RightLabel{$\vee_l^5$}
				\BIC{$(\hG{A} \wedge \hG{B}) \vee \hE{C}, \hD{A}, \hC{B} \seq \hC{B} \wedge \hD{A}, \hE{C} $} 
\end{prooftree}
\end{scriptsize}

\begin{scriptsize}
\begin{prooftree}
			\AXC{$\hC{B} \seq \hC{B} $}
					\AXC{$\hD{A} \seq \hD{A}$} \RightLabel{$\wedge_r^2$}
				\BIC{$\hD{A}, \hC{B} \seq \hC{B} \wedge \hD{A}$} \RightLabel{$w_l$}
				\UIC{$\hG{A} \wedge \hG{B}, \hD{A}, \hC{B} \seq \hC{B} \wedge \hD{A}$}
									\AXC{$\hF{C} \seq \hF{C}$} \RightLabel{$w_l$}
									\UIC{$\hG{C},\hE{C} \seq \hF{C}$} \RightLabel{$\vee_l^5$}
					\BIC{$(\hG{A} \wedge \hG{B}) \vee \hG{C}, \hD{A}, \hC{B} \seq \hC{B} \wedge \hD{A}, \hF{C} $} 
\end{prooftree}
\end{scriptsize}
}
\end{multicols}
\hfill\QED
\end{example}

\clearpage

\begin{theorem}
Let $\psi$ be a proof and $\Gamma_c \seq \Delta_c$ be a clause in its characteristic clause set. Then $\projection{\psi}{\Gamma_c \seq \Delta_c}$ is a projection of $\psi$ with respect to the clause $\Gamma_c \seq \Delta_c$.
\end{theorem}
\begin{proof}
A detailed proof is available in \cite{Woltzenlogel-Paleo2009A-General-Analysis-of-Cut-Elimination-by-CERes}. Here only a sketch is provided. Let $A$ be a set of axioms from $\psi$ that contain literals that contribute to $\Gamma_c \seq \Delta_c$. Note that, since the literals from $\Gamma_c \seq \Delta_c$ occur as cut-ancestors in the axioms in $A$, and inferences operating on cut-ancestors are not performed in the construction of $\projection{\psi}{\Gamma_c \seq \Delta_c}$, these literals are simply propagated down to the end-sequent of $\projection{\psi}{\Gamma_c \seq \Delta_c}$. Therefore, the end-sequent of $\projection{\psi}{\Gamma_c \seq \Delta_c}$ is a supersequent of $\Gamma_c \seq \Delta_c$. Among the formulas in the end-sequent of $\psi$, let $\Gamma' \seq \Delta'$ be the sequent containing all and only those formulas that contain descendents of axioms in $A$. Since $\projection{\psi}{\Gamma_c \seq \Delta_c}$ contains all the inferences that operate on formulas that are both end-sequent ancestors and descendents from axioms in $A$, the end-sequent of $\projection{\psi}{\Gamma_c \seq \Delta_c}$ is a supersequent of $\Gamma' \seq \Delta'$. Finally, $\projection{\psi}{\Gamma_c \seq \Delta_c}$ is cut-free, because it contains no inference operating on cut ancestors and thus contains no cut.
\hfill\QED
\end{proof}

\noindent
The construction of the characteristic clause set in original descriptions of the $\CERes$ method \cite{BaazLeitsch1999MethodsofCut-Elimination,BaazLeitsch2000Cut-eliminationandRedundancy-eliminationbyResolution,BaazLeitsch2006Towardsaclausalanalysisofcut-elimination} differs slightly from the construction presented in this paper. There, instead of a characteristic formula, one consructs a \emph{characteristic clause term}, that has $\oplus$ instead of $\wedge$ (for binary inferences operating on cut-ancestors), $\otimes$ instead of $\vee$ (for binary inferences operating on end-sequent-ancestors), and singleton clause sets instead of formulas (for axioms). The operators $\oplus$ and $\otimes$ are then interpreted respectively as a set union and as a clause set merge operation in order to generate the characteristic clause set. Both approaches are clearly equivalent, but the approach described here is simpler as it does not require the invention of new operators and relies on the standard technique of clause form transformation for the generation of the characteristic clause set. This is crucial for understanding why proofs generated by $\CERes$ may contain redundancies.


\begin{definition}[$\CERes$-normal-form]
The \emph{$\CERes$-normal-form} of the proof $\varphi$ w.r.t. the resolution refutation $\delta$ of its clause set $\clauseset{\varphi}$ is denoted $\CEResNF{\varphi}{\delta}$ and obtained by:
\begin{enumerate}
\item converting resolution and factoring inferences from $\delta$ to, respectively, atomic cuts and contractions, using a substitution $\sigma$ obtained by the composition of all unifiers.
\item replacing each axiom clause $c$ in $\delta$ by its corresponding projection $\projection{\psi}{c} \sigma$.
\item adding contractions in the bottom of the proof, if necessary.
\end{enumerate} 
\end{definition}



\begin{landscape}
\begin{example}
\label{example:CEResSONormalForm}
Let $\varphi$ be the proof shown in Example \ref{example:CutPertinentStruct} and $\delta$ be the refutation of its characteristic clause set, as shown in Example \ref{example:CutPertinentStandardClauseSet}. Then, $\CEResNF{\varphi}{\delta}$, obtained by replacing the leaves of $\delta$ by the respective projections shown in Example \ref{example:OProjections}, converting resolution and factoring inferences respectively to cuts and contractions and adding contractions at the bottom, is:

\begin{scriptsize}
\begin{prooftree}
\AXC{$\projection{\varphi}{\seq A, C}$} \noLine
\UIC{$(\hA{A} \wedge \hG{B}) \vee \hE{C} \seq \hA{A}, \hE{C}$}
		\AXC{$\projection{\varphi}{C \seq A}$} \noLine
		\UIC{$(\hA{A} \wedge \hG{B}) \vee \hG{C}, \hF{C} \seq \hA{A}, \hF{C} $} \RightLabel{$cut$}
	\BIC{$(\hA{A} \wedge \hG{B}) \vee \hE{C}, (\hA{A} \wedge \hG{B}) \vee \hG{C} \seq \hA{A},\hA{A}, \hF{C}$} \RightLabel{$c_r$}
	\UIC{$(\hA{A} \wedge \hG{B}) \vee \hE{C}, (\hA{A} \wedge \hG{B}) \vee \hG{C} \seq \hA{A}, \hF{C}$}
				\AXC{$ \psi $} \noLine 				
 				\UIC{$ \hD{A}, (\hG{A} \wedge \hB{B}) \vee \hE{C},(\hG{A} \wedge \hB{B}) \vee \hG{C}, (\hG{A} \wedge \hG{B}) \vee \hE{C}, (\hG{A} \wedge \hG{B}) \vee \hG{C} \seq \hC{B} \wedge \hD{A}, \hC{B} \wedge \hD{A}, \hF{C}$}\RightLabel{$cut$}
		\BIC{$(\hA{A} \wedge \hG{B}) \vee \hE{C}, (\hA{A} \wedge \hG{B}) \vee \hG{C}, (\hG{A} \wedge \hB{B}) \vee \hE{C},(\hG{A} \wedge \hB{B}) \vee \hG{C}, (\hG{A} \wedge \hG{B}) \vee \hE{C}, (\hG{A} \wedge \hG{B}) \vee \hG{C} \seq \hC{B} \wedge \hD{A}, \hC{B} \wedge \hD{A}, \hF{C}, \hF{C}$} \doubleLine \RightLabel{$c^*$}
		\UIC{$(A \wedge B) \vee C \seq B \wedge A, C$}
\end{prooftree}
\end{scriptsize}

\noindent
Where $\psi$ is:

\begin{scriptsize}
\begin{prooftree}
			\AXC{$\projection{\varphi}{\seq B, C}$} \noLine
			\UIC{$(\hG{A} \wedge \hB{B}) \vee \hE{C} \seq \hB{B}, \hE{C}$}
					\AXC{$\projection{\varphi}{C \seq B}$} \noLine
					\UIC{$(\hG{A} \wedge \hB{B}) \vee \hG{C}, \hF{C} \seq \hB{B}, \hF{C} $} \RightLabel{$cut$}
				\BIC{$ (\hG{A} \wedge \hB{B}) \vee \hE{C},(\hG{A} \wedge \hB{B}) \vee \hG{C} \seq \hB{B},\hB{B},\hF{C}$} \RightLabel{$c_r$}
				\UIC{$ (\hG{A} \wedge \hB{B}) \vee \hE{C},(\hG{A} \wedge \hB{B}) \vee \hG{C} \seq \hB{B},\hF{C}$}
							\AXC{$\projection{\varphi}{B, A \seq C}$} \noLine
							\UIC{$(\hG{A} \wedge \hG{B}) \vee \hE{C}, \hD{A}, \hC{B} \seq \hC{B} \wedge \hD{A}, \hE{C} $}
									\AXC{$\projection{\varphi}{B, A, C \seq }$} \noLine
									\UIC{$\hF{C}, (\hG{A} \wedge \hG{B}) \vee \hG{C}, \hD{A}, \hC{B} \seq \hC{B} \wedge \hD{A}, \hF{C} $} \RightLabel{$cut$}
								\BIC{$\hD{A}, \hC{B}, \hD{A}, \hC{B}, (\hG{A} \wedge \hG{B}) \vee \hE{C}, (\hG{A} \wedge \hG{B}) \vee \hG{C} \seq \hC{B} \wedge \hD{A}, \hC{B} \wedge \hD{A}, \hF{C}$} \RightLabel{$c_r$}
								\UIC{$\hD{A}, \hC{B}, \hC{B}, (\hG{A} \wedge \hG{B}) \vee \hE{C}, (\hG{A} \wedge \hG{B}) \vee \hG{C} \seq \hC{B} \wedge \hD{A}, \hC{B} \wedge \hD{A}, \hF{C} $} \RightLabel{$c_r$}
								\UIC{$\hD{A}, \hC{B}, (\hG{A} \wedge \hG{B}) \vee \hE{C}, (\hG{A} \wedge \hG{B}) \vee \hG{C} \seq \hC{B} \wedge \hD{A}, \hC{B} \wedge \hD{A}, \hF{C}$} \RightLabel{$cut$}
						\BIC{$\hD{A}, (\hG{A} \wedge \hB{B}) \vee \hE{C},(\hG{A} \wedge \hB{B}) \vee \hG{C}, (\hG{A} \wedge \hG{B}) \vee \hE{C}, (\hG{A} \wedge \hG{B}) \vee \hG{C} \seq \hC{B} \wedge \hD{A}, \hC{B} \wedge \hD{A}, \hF{C}$}
\end{prooftree}
\end{scriptsize}
\hfill\QED
\end{example}



\end{landscape}