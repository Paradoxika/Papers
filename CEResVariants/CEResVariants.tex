\documentclass{llncs}

\usepackage[utf8]{inputenc}
\usepackage{amssymb}
\usepackage{amsfonts}
\usepackage{pdflscape}
\usepackage{multicol}


\usepackage{bussproofs}
\usepackage{proof}
\usepackage{url}
\usepackage{color}

\usepackage{fancybox}
\usepackage{fancyvrb}

% Sequent Calculus Proof Settings
\EnableBpAbbreviations
\def\fCenter{\mbox{\ $\vdash$\ }}
\newcommand{\rl}[1]{\RightLabel{#1}}

\usepackage{commands}
\usepackage{comment}


\title{
  Reducing Redundancy in \\ 
  Cut-Elimination by Resolution
  \thanks{Supported by the Austrian Science Foundation (FWF) projects P19875 and P24300.}
}

\author{
  Bruno Woltzenlogel Paleo
}


\institute{
  Theory and Logic Group, Vienna University of Technology, Vienna, Austria \\
  \email{bruno@logic.at}
}


\begin{document}

\maketitle

\begin{abstract}
$\CERes$ is a method of cut-elimination that uses resolution proof search to avoid some kinds of redundancies that affect reductive cut-elimination methods. This paper shows that, unfortunately, there are also cases where $\CERes$ can produce proofs that are more redundant and even exponentially larger than the proofs produced by reductive cut-elimination methods. The paper then describes a few novel variants of $\CERes$ that are much less susceptible to these redundancies.
\end{abstract}


\section{Introduction}

The cut-elimination method $\CERes$ invented by A. Leitsch and M.
Baaz
\cite{BaazLeitsch1999MethodsofCut-Elimination,BaazLeitsch2000Cut-eliminationandRedundancy-eliminationbyResolution,BaazLeitsch2006Towardsaclausalanalysisofcut-elimination} 
uses resolution
proof search to avoid certain kinds of redundancies that affect reductive cut-
elimination methods: there are proofs for which reductive cut-elimination
methods may require non-elementary many reduction steps and produce non-
elementarily large intermediary proofs, while $\CERes$'s more global and
search-based approach produces a proof in atomic cut normal form without
performing such expensive intermediary steps 
\cite{BaazLeitsch2009MethodsofCut-Elimination}.

However, as explained in Section \ref{sec:Redundancy} of this paper, there are
also cases where reductive cut-elimination methods can produce short proofs
while the proofs generated by $\CERes$ are exponentially larger. Thanks to a
simplified description of $\CERes$ in Section \ref{sec:CERes}, it becomes
evident that the source of redundancy is the naive transformation to clause
form that is implicitly used by $\CERes$.

Sections \ref{sec:InferencePermutability} and \ref{sec:StructuralClauseForm}
develop two techniques to tame the redundancy. The first takes inference
permutability into account when performing the clause form transformation,
thus avoiding the duplication of literals when disjunctions are distributed
over conjunctions. The second proposes the use of structural clause form
transformation, which is known not to cause a worst-case exponential blow-up
in the formula size. These two techniques can be combined, resulting
in the least redundant $\CERes$ variant described in Section
\ref{sec:Combination}.

\paragraph{Acknowledgments:} the work presented in this paper is part of the
author's Ph.D. thesis
\cite{Woltzenlogel-Paleo2009A-General-Analysis-of-Cut-Elimination-by-CERes},
under the supervision of Prof. Alexander Leitsch.  Most
ideas discussed here originated from personal communication with him.

\section{Cut-Elimination by Resolution}
\label{sec:CERes}


The $\CERes$ method consists of four steps. Firstly, the proof with cuts $\psi$ is skolemized into a proof $\psi'$. Secondly, a \emph{characteristic formula} $\struct{\psi}$ is extracted from $\psi'$. Thirdly, as this formula is always unsatisfiable, it can be converted to a \emph{characteristic clause set} and refuted by resolution. The resulting resolution refutation $\delta$ (whose existence is guaranteed by the completeness of the resolution calculus \cite{Leitsch1997The-resolution-calculus}) is made of resolution and factoring inferences. In the last step, a sequent calculus proof $\CEResNF{\psi,\delta}$ having the same end-sequent as $\psi'$ and whose cuts are atomic can be obtained by converting resolution and factoring inferences from $\delta$ to, respectively, atomic cuts and contractions and replacing the leaves from $\delta$ by cut-free parts of $\psi'$ known as \emph{projections}.


\begin{definition}[Characteristic Formula]
\label{definition:PertinentStruct}
The \emph{characteristic formula} $\struct{\varphi}$ of a proof $\varphi$ is the characteristic formula $\struct{\rho^*}$ where $\rho^*$ is its lowermost inference and $\struct{\rho}$ is defined for each inference $\rho$ as follows:

\begin{itemize}
	\item If $\rho$ is an axiom with sequent $A \seq A$, four cases are distinguished:
	\begin{itemize}	 	
	 	\item Only the succedent's $A$ is a cut-ancestor: $\struct{\rho} = A$
	 	\item Only the antecedent's $A$ is a cut-ancestor: $\struct{\rho} = \structdual A $
	 	\item None is a cut-ancestor: $\struct{\rho} = \structtimesEmpty $
	 	\item Both are cut-ancestors: $\struct{\rho} = \structplusEmpty$
	\end{itemize}

	\item If $\rho$ is an $n$-ary inference and $\rho_1,\ldots,\rho_n$ are the inferences deriving the premises of $\rho$, two cases are distinguished:
	\begin{itemize}
		\item $\rho$ operates on cut-ancestors:
$$
\struct{\rho} \defEq 
		\struct{\rho_1} \structplus \ldots \structplus \struct{\rho_n}
$$

	\item $\rho$ does not operate on cut-ancestors::
$$
\struct{\rho} \defEq 
		\struct{\rho_1} \structtimes \ldots \structtimes \struct{\rho_n}
$$
	\end{itemize}
\end{itemize}
\end{definition}


\begin{example}
\label{example:CutPertinentStruct}

Let $\varphi$ be the proof below:
\begin{prooftree}
\AXC{$A \seq \hA{A} $}
		\AXC{$B \seq \hB{B}$} \RightLabel{$\wedge^1_r$}
	\BIC{$A, B \seq A \wedge B$} \RightLabel{$\wedge^6_l$}
	\UIC{$A \wedge B \seq A \wedge B$}
				\AXC{$\hC{B} \seq B $}
						\AXC{$\hD{A} \seq A$} \RightLabel{$\wedge^2_r$}
					\BIC{$A, B \seq B \wedge A$} \RightLabel{$\wedge^7_l$}
					\UIC{$A \wedge B \seq B \wedge A$} \RightLabel{$cut^3$}
			\BIC{$A \wedge B \seq B \wedge A$}
							\AXC{$C \seq \hE{C}$} 
									\AXC{$\hF{C} \seq C$} \RightLabel{$cut^4$}
								\BIC{$C \seq C$} \RightLabel{$\vee^5_l$}
					\BIC{$(A \wedge B) \vee C \seq B \wedge A, C $} 
\end{prooftree}

\noindent
Its characteristic formula is:
$
\struct{\varphi} 
\equiv 
((\hA{A} \structplus^1 \hB{B}) \structplus^3 (\structdual{\hC{B}} \structtimes^2 \structdual{\hD{A}}))
\structtimes^5
(\hE{C} \structplus^4 \structdual{\hF{C}})
$
\hfill\QED
\end{example}

\begin{theorem}
\label{theorem:Unsatisfiability}
%For any proof $\varphi$, 
$\all \overline{\alpha_{\varphi}}. \struct{\varphi}$, where $\overline{\alpha_{\varphi}}$ are the eigenvariables of $\varphi$, is unsatisfiable.
\end{theorem}
\begin{proof}
Recursively transform each subproof $\psi$ of $\varphi$ having end-sequent $\Gamma, \Gamma^* \seq \Delta, \Delta^*$, where $\Gamma^* \seq \Delta^*$ are cut-ancestors, into a proof $\psi'$ of $\all \overline{\alpha_{\psi}}. \struct{\rho}, \Gamma^* \seq \Delta^*$. By doing so, $\varphi$ itself is transformed into a proof $\varphi'$ with end-sequent $\all \overline{\alpha_{\varphi}}. \struct{\varphi} \seq$.
\hfill\QED
\end{proof}






\begin{definition}[Simple Transformation to Conjunctive Normal Form]
\label{definition:NormalizationPlusTimes}
\index{Struct Normalization!Standard}
A formula in negative normal form can be transformed into conjunctive normal form by rewriting it according to the following rule:
$$
S \structtimes (S_1 \structplus \ldots \structplus S_n)   \normalizePlusTimes  (S \structtimes S_1) \structplus \ldots \structplus (S \structtimes S_n) 
$$
%$$
%(S_1 \structplus \ldots \structplus S_n) \structtimes S  \normalizePlusTimes  (S_1 \structtimes S) \structplus \ldots \structplus (S_n \structtimes S) 
%$$
\end{definition}


\begin{definition}[Sequent Notation]
\label{definition:Clausification}
A formula in conjunctive normal form 
$$
\structplusBig_{i\in I} (\structtimesBig_{1 \leq j' \leq j_i} \structdual{A_{ij'}} \structtimes \structtimesBig_{1 \leq h' \leq h_i} B_{ih'})
$$
can be written in sequent notation as the set
$
%\clausify{S} \defEq 
\{ A_{i1}, \ldots, A_{ij_i} \seq B_{i1},\ldots, B_{ih_i} | i \in I \}
$.
\end{definition}





\begin{definition}[Clause Set]
\label{definition:CutPertinentClauseSet}
\index{Clause Set!Standard}
The \emph{clause set} $\clauseset{\varphi}$ of a proof $\varphi$ is the conjunctive normal form of $\struct{\varphi}$ written in sequent notation.
\end{definition}


\begin{example}
\label{example:CutPertinentStandardClauseSet}
Let $\varphi$ be the proof shown in Example \ref{example:CutPertinentStruct}. Its characteristic formula $\struct{\varphi}$ normalizes as:
\begin{scriptsize}
$$
\begin{array}{rcl}
\struct{\varphi} 
& \normalizePlusTimes^* &
(A \structtimes C) \structplus (A \structtimes \structdual{C})
\structplus 
(B \structtimes C) \structplus (B \structtimes \structdual{C})  
\structplus 
(\structdual{B} \structtimes \structdual{A} \structtimes C) 
\structplus 
(\structdual{B} \structtimes \structdual{A}\structtimes \structdual{C})
\\
\end{array}
$$
\end{scriptsize}

\noindent
Hence, the clause set of $\varphi$ is:
%
$$
\clauseset{\varphi} \equiv \{ \seq A,C \ \ ; \ \ C \seq A \ \ ;\ \ \seq B,C \ \ ; \ \ C \seq B \ \ ; \ \ B, A \seq C \ \ ; \ \ B, A, C \seq  \}
$$
and it can be refuted by the following resolution refutation $\delta$:
\begin{small}
\begin{prooftree}
\AXC{$\seq A,C$}
		\AXC{$C \seq A$} \RightLabel{$r$}
	\BIC{$\seq A,A$} \RightLabel{$f_r$}
	\UIC{$\seq A$}
			\AXC{$\seq B,C$}
					\AXC{$C \seq B$} \RightLabel{$r$}
				\BIC{$\seq B,B$} \RightLabel{$f_r$}
				\UIC{$\seq B$}
							\AXC{$B, A \seq C$}
									\AXC{$C, B, A \seq $} \RightLabel{$r$}
								\BIC{$B, A, B, A \seq $} \RightLabel{$f_r$}
								\UIC{$B, A, B \seq $} \RightLabel{$f_r$}
								\UIC{$A, B \seq $} \RightLabel{$r$}
						\BIC{$A \seq $} \RightLabel{$r$}
		\BIC{$ \seq $}
\end{prooftree}
\end{small}
\hfill\QED
\end{example}


\noindent
Resolution and factoring inferences are essentially atomic cuts and contractions with unification. Therefore, to obtain the $\CERes$-normal-form (an $\LK$-proof, with only atomic cuts, of the skolemized end-sequent of the original proof with cuts), one can replace each leaf of the refutation by a \emph{projection} with an appropriate end-sequent, apply all the unifiers and convert resolution and factoring inferences to atomic cuts and contractions.
Since a projection's purpose is to replace a leaf in a refutation of a clause set, its end-sequent must contain the leaf's clause as a subsequent. Moreover, if its end-sequent contains any other formula, then this formula must appear in the end-sequent of the original proof with cuts, because this formula is propagated downward after the replacement and thus necessarily appears in the end-sequent of the $\CERes$-normal-form. Otherwise, if the formula were not in the end-sequent of the original proof, the $\CERes$-normal-form's end-sequent would be necessarily different from that of the skolemized proof with cuts. Finally, a projection must, of course, be cut-free, otherwise the $\CERes$-normal-form would contain more cuts in addition to the inessential atomic cuts originating from the refutation. These three conditions are formally expressed in Definition \ref{definition:Projection}.

\begin{definition}[Projection]
\label{definition:Projection}
\index{Projection}
Let $\varphi$ be a proof with end-sequent $\Gamma \seq \Delta$ and $\Gamma_c \seq \Delta_c \in \clauseset{\varphi}$. Any cut-free proof of $\Gamma', \Gamma_c \seq \Delta', \Delta_c$, where $\Gamma' \subseteq \Gamma$, $\Delta' \subseteq \Delta$, is a \emph{projection} of $\varphi$ with respect to $\Gamma_c \seq \Delta_c$.
\end{definition}

\noindent
Projections can be easily constructed by extracting cut-free parts of the original proof with cuts. 
The original method \cite{BaazLeitsch2009MethodsofCut-Elimination,BaazHetzlLeitschRichterSpohr2006ProofTransformationbyCERES,BaazLeitsch2000Cut-eliminationandRedundancy-eliminationbyResolution,BaazLeitsch1999MethodsofCut-Elimination,Richter2006ProofTransformationsbyResolution--ComputationalMethodsofCut-Elimination} generates projections where $\Gamma' = \Gamma$ and $\Delta' = \Delta$. Here a slightly more optimized method that constructs less redundant projections, where $\Gamma' \subseteq \Gamma$ and $\Delta' \subseteq \Delta$, is described. 


%The method follows roughly three steps: firstly, cut-pertinent inferences are ``deleted'' (so that cut-pertinent occurrences are propagated down to the end-sequent, and thus the end-sequent now contains every clause of the clause set); then, occurrences and inferences that are not in a certain sense relevant to the occurrences of the specific clause under consideration are also ``deleted'' (this guarantees that only that clause will occur in the end-sequent of the projection); and finally, the problems introduced by the previous two steps are fixed.


\begin{definition}[Algorithm for Constructing Projections] %\hspace*{\fill} \\
\label{definition:OProjection}
Let $\varphi$ be a proof and $c$ be a clause from $\clauseset{\varphi}$. Let $A$ be the set of axioms of $\varphi$ that contain formulas that contribute to $c$. Then, $\projection{\varphi}{c}$ is constructed by taking from $\varphi$ only the inferences that operate on formulas that are both descendents of axioms in $A$ and end-sequent ancestors, and adding weakening inferences when necessary.
\end{definition}

\begin{example}
\label{example:OProjections}
For $\varphi$ of Example \ref{example:CutPertinentStruct}, the projections $\projection{\varphi}{\seq \hA{A}, \hE{C}}$ and $\projection{\varphi}{\seq \hB{B}, \hE{C}}$ are:
\begin{small}
\begin{multicols}{2}{
\begin{prooftree}
	\AXC{$\hA{A} \seq \hA{A}$} \RightLabel{$w_l$}
	\UIC{$\hA{A}, \hG{B} \seq \hA{A}$} \RightLabel{$\wedge_l^6$}
	\UIC{$\hA{A} \wedge \hG{B} \seq \hA{A}$}
							\AXC{$\hE{C} \seq \hE{C}$} \RightLabel{$\vee_l^5$}
					\BIC{$(\hA{A} \wedge \hG{B}) \vee \hE{C} \seq \hA{A}, \hE{C}$} 
\end{prooftree}

\begin{prooftree}
	\AXC{$\hB{B} \seq \hB{B}$} \RightLabel{$w_l$}
	\UIC{$\hG{A}, \hB{B} \seq \hB{B}$} \RightLabel{$\wedge_l^6$}
	\UIC{$\hG{A} \wedge \hB{B} \seq \hB{B}$}
							\AXC{$\hE{C} \seq \hE{C}$} \RightLabel{$\vee_l^5$}
					\BIC{$(\hG{A} \wedge \hB{B}) \vee \hE{C} \seq \hB{B}, \hE{C}$} 
\end{prooftree}
}
\end{multicols}
\end{small}

\noindent
The projections $\projection{\varphi}{\hF{C} \seq \hA{A}}$ and $\projection{\varphi}{\hF{C} \seq \hB{B}}$ are:
\begin{small}
\begin{multicols}{2}{
\begin{prooftree}
\AXC{$\hA{A} \seq \hA{A} $} \RightLabel{$w_l$}
\UIC{$\hA{A}, \hG{B} \seq \hA{A}$} \RightLabel{$\wedge_l^6$}
\UIC{$\hA{A} \wedge \hG{B} \seq \hA{A}$}
								\AXC{$\hF{C} \seq \hF{C}$} \RightLabel{$w_l$}
								\UIC{$\hG{C},\hF{C} \seq \hF{C}$} \RightLabel{$\vee_l^5$}
					\BIC{$(\hA{A} \wedge \hG{B}) \vee \hG{C}, \hF{C} \seq \hA{A}, \hF{C} $} 
\end{prooftree}

\begin{prooftree}
\AXC{$\hB{B} \seq \hB{B} $} \RightLabel{$w_l$}
\UIC{$\hG{A}, \hB{B} \seq \hB{B}$} \RightLabel{$\wedge_l^6$}
\UIC{$\hG{A} \wedge \hB{B} \seq \hB{B}$}
								\AXC{$\hF{C} \seq \hF{C}$} \RightLabel{$w_l$}
								\UIC{$\hG{C},\hF{C} \seq \hF{C}$} \RightLabel{$\vee_l^5$}
					\BIC{$(\hG{A} \wedge \hB{B}) \vee \hG{C}, \hF{C} \seq \hB{B}, \hF{C} $} 
\end{prooftree}
}
\end{multicols}
\end{small}

\noindent
The projections $\projection{\varphi}{\hC{B}, \hD{A} \seq \hE{C}}$ and $\projection{\varphi}{\hC{B}, \hD{A}, \hF{C} \seq}$ are:
\begin{multicols}{2}{
\begin{scriptsize}
\begin{prooftree}
		\AXC{$\hC{B} \seq \hC{B} $}
				\AXC{$\hD{A} \seq \hD{A}$} \RightLabel{$\wedge_r^2$}
			\BIC{$\hD{A}, \hC{B} \seq \hC{B} \wedge \hD{A}$} \RightLabel{$w_l$}
			\UIC{$\hG{A} \wedge \hG{B}, \hD{A}, \hC{B} \seq \hC{B} \wedge \hD{A}$}
							\AXC{$\hE{C} \seq \hE{C}$} \RightLabel{$\vee_l^5$}
				\BIC{$(\hG{A} \wedge \hG{B}) \vee \hE{C}, \hD{A}, \hC{B} \seq \hC{B} \wedge \hD{A}, \hE{C} $} 
\end{prooftree}
\end{scriptsize}

\begin{scriptsize}
\begin{prooftree}
			\AXC{$\hC{B} \seq \hC{B} $}
					\AXC{$\hD{A} \seq \hD{A}$} \RightLabel{$\wedge_r^2$}
				\BIC{$\hD{A}, \hC{B} \seq \hC{B} \wedge \hD{A}$} \RightLabel{$w_l$}
				\UIC{$\hG{A} \wedge \hG{B}, \hD{A}, \hC{B} \seq \hC{B} \wedge \hD{A}$}
									\AXC{$\hF{C} \seq \hF{C}$} \RightLabel{$w_l$}
									\UIC{$\hG{C},\hE{C} \seq \hF{C}$} \RightLabel{$\vee_l^5$}
					\BIC{$(\hG{A} \wedge \hG{B}) \vee \hG{C}, \hD{A}, \hC{B} \seq \hC{B} \wedge \hD{A}, \hF{C} $} 
\end{prooftree}
\end{scriptsize}
}
\end{multicols}
\hfill\QED
\end{example}

\clearpage

\begin{theorem}
Let $\psi$ be a proof and $\Gamma_c \seq \Delta_c$ be a clause in its characteristic clause set. Then $\projection{\psi}{\Gamma_c \seq \Delta_c}$ is a projection of $\psi$ with respect to the clause $\Gamma_c \seq \Delta_c$.
\end{theorem}
\begin{proof}
A detailed proof is available in \cite{Woltzenlogel-Paleo2009A-General-Analysis-of-Cut-Elimination-by-CERes}. Here only a sketch is provided. Let $A$ be a set of axioms from $\psi$ that contain literals that contribute to $\Gamma_c \seq \Delta_c$. Note that, since the literals from $\Gamma_c \seq \Delta_c$ occur as cut-ancestors in the axioms in $A$, and inferences operating on cut-ancestors are not performed in the construction of $\projection{\psi}{\Gamma_c \seq \Delta_c}$, these literals are simply propagated down to the end-sequent of $\projection{\psi}{\Gamma_c \seq \Delta_c}$. Therefore, the end-sequent of $\projection{\psi}{\Gamma_c \seq \Delta_c}$ is a supersequent of $\Gamma_c \seq \Delta_c$. Among the formulas in the end-sequent of $\psi$, let $\Gamma' \seq \Delta'$ be the sequent containing all and only those formulas that contain descendents of axioms in $A$. Since $\projection{\psi}{\Gamma_c \seq \Delta_c}$ contains all the inferences that operate on formulas that are both end-sequent ancestors and descendents from axioms in $A$, the end-sequent of $\projection{\psi}{\Gamma_c \seq \Delta_c}$ is a supersequent of $\Gamma' \seq \Delta'$. Finally, $\projection{\psi}{\Gamma_c \seq \Delta_c}$ is cut-free, because it contains no inference operating on cut ancestors and thus contains no cut.
\hfill\QED
\end{proof}

\noindent
The construction of the characteristic clause set in original descriptions of the $\CERes$ method \cite{BaazLeitsch1999MethodsofCut-Elimination,BaazLeitsch2000Cut-eliminationandRedundancy-eliminationbyResolution,BaazLeitsch2006Towardsaclausalanalysisofcut-elimination} differs slightly from the construction presented in this paper. There, instead of a characteristic formula, one consructs a \emph{characteristic clause term}, that has $\oplus$ instead of $\wedge$ (for binary inferences operating on cut-ancestors), $\otimes$ instead of $\vee$ (for binary inferences operating on end-sequent-ancestors), and singleton clause sets instead of formulas (for axioms). The operators $\oplus$ and $\otimes$ are then interpreted respectively as a set union and as a clause set merge operation in order to generate the characteristic clause set. Both approaches are clearly equivalent, but the approach described here is simpler as it does not require the invention of new operators and relies on the standard technique of clause form transformation for the generation of the characteristic clause set. This is crucial for understanding why proofs generated by $\CERes$ may contain redundancies.


\begin{definition}[$\CERes$-normal-form]
The \emph{$\CERes$-normal-form} of the proof $\varphi$ w.r.t. the resolution refutation $\delta$ of its clause set $\clauseset{\varphi}$ is denoted $\CEResNF{\varphi}{\delta}$ and obtained by:
\begin{enumerate}
\item converting resolution and factoring inferences from $\delta$ to, respectively, atomic cuts and contractions, using a substitution $\sigma$ obtained by the composition of all unifiers.
\item replacing each axiom clause $c$ in $\delta$ by its corresponding projection $\projection{\psi}{c} \sigma$.
\item adding contractions in the bottom of the proof, if necessary.
\end{enumerate} 
\end{definition}



\begin{landscape}
\begin{example}
\label{example:CEResSONormalForm}
Let $\varphi$ be the proof shown in Example \ref{example:CutPertinentStruct} and $\delta$ be the refutation of its characteristic clause set, as shown in Example \ref{example:CutPertinentStandardClauseSet}. Then, $\CEResNF{\varphi}{\delta}$, obtained by replacing the leaves of $\delta$ by the respective projections shown in Example \ref{example:OProjections}, converting resolution and factoring inferences respectively to cuts and contractions and adding contractions at the bottom, is:

\begin{scriptsize}
\begin{prooftree}
\AXC{$\projection{\varphi}{\seq A, C}$} \noLine
\UIC{$(\hA{A} \wedge \hG{B}) \vee \hE{C} \seq \hA{A}, \hE{C}$}
		\AXC{$\projection{\varphi}{C \seq A}$} \noLine
		\UIC{$(\hA{A} \wedge \hG{B}) \vee \hG{C}, \hF{C} \seq \hA{A}, \hF{C} $} \RightLabel{$cut$}
	\BIC{$(\hA{A} \wedge \hG{B}) \vee \hE{C}, (\hA{A} \wedge \hG{B}) \vee \hG{C} \seq \hA{A},\hA{A}, \hF{C}$} \RightLabel{$c_r$}
	\UIC{$(\hA{A} \wedge \hG{B}) \vee \hE{C}, (\hA{A} \wedge \hG{B}) \vee \hG{C} \seq \hA{A}, \hF{C}$}
				\AXC{$ \psi $} \noLine 				
 				\UIC{$ \hD{A}, (\hG{A} \wedge \hB{B}) \vee \hE{C},(\hG{A} \wedge \hB{B}) \vee \hG{C}, (\hG{A} \wedge \hG{B}) \vee \hE{C}, (\hG{A} \wedge \hG{B}) \vee \hG{C} \seq \hC{B} \wedge \hD{A}, \hC{B} \wedge \hD{A}, \hF{C}$}\RightLabel{$cut$}
		\BIC{$(\hA{A} \wedge \hG{B}) \vee \hE{C}, (\hA{A} \wedge \hG{B}) \vee \hG{C}, (\hG{A} \wedge \hB{B}) \vee \hE{C},(\hG{A} \wedge \hB{B}) \vee \hG{C}, (\hG{A} \wedge \hG{B}) \vee \hE{C}, (\hG{A} \wedge \hG{B}) \vee \hG{C} \seq \hC{B} \wedge \hD{A}, \hC{B} \wedge \hD{A}, \hF{C}, \hF{C}$} \doubleLine \RightLabel{$c^*$}
		\UIC{$(A \wedge B) \vee C \seq B \wedge A, C$}
\end{prooftree}
\end{scriptsize}

\noindent
Where $\psi$ is:

\begin{scriptsize}
\begin{prooftree}
			\AXC{$\projection{\varphi}{\seq B, C}$} \noLine
			\UIC{$(\hG{A} \wedge \hB{B}) \vee \hE{C} \seq \hB{B}, \hE{C}$}
					\AXC{$\projection{\varphi}{C \seq B}$} \noLine
					\UIC{$(\hG{A} \wedge \hB{B}) \vee \hG{C}, \hF{C} \seq \hB{B}, \hF{C} $} \RightLabel{$cut$}
				\BIC{$ (\hG{A} \wedge \hB{B}) \vee \hE{C},(\hG{A} \wedge \hB{B}) \vee \hG{C} \seq \hB{B},\hB{B},\hF{C}$} \RightLabel{$c_r$}
				\UIC{$ (\hG{A} \wedge \hB{B}) \vee \hE{C},(\hG{A} \wedge \hB{B}) \vee \hG{C} \seq \hB{B},\hF{C}$}
							\AXC{$\projection{\varphi}{B, A \seq C}$} \noLine
							\UIC{$(\hG{A} \wedge \hG{B}) \vee \hE{C}, \hD{A}, \hC{B} \seq \hC{B} \wedge \hD{A}, \hE{C} $}
									\AXC{$\projection{\varphi}{B, A, C \seq }$} \noLine
									\UIC{$\hF{C}, (\hG{A} \wedge \hG{B}) \vee \hG{C}, \hD{A}, \hC{B} \seq \hC{B} \wedge \hD{A}, \hF{C} $} \RightLabel{$cut$}
								\BIC{$\hD{A}, \hC{B}, \hD{A}, \hC{B}, (\hG{A} \wedge \hG{B}) \vee \hE{C}, (\hG{A} \wedge \hG{B}) \vee \hG{C} \seq \hC{B} \wedge \hD{A}, \hC{B} \wedge \hD{A}, \hF{C}$} \RightLabel{$c_r$}
								\UIC{$\hD{A}, \hC{B}, \hC{B}, (\hG{A} \wedge \hG{B}) \vee \hE{C}, (\hG{A} \wedge \hG{B}) \vee \hG{C} \seq \hC{B} \wedge \hD{A}, \hC{B} \wedge \hD{A}, \hF{C} $} \RightLabel{$c_r$}
								\UIC{$\hD{A}, \hC{B}, (\hG{A} \wedge \hG{B}) \vee \hE{C}, (\hG{A} \wedge \hG{B}) \vee \hG{C} \seq \hC{B} \wedge \hD{A}, \hC{B} \wedge \hD{A}, \hF{C}$} \RightLabel{$cut$}
						\BIC{$\hD{A}, (\hG{A} \wedge \hB{B}) \vee \hE{C},(\hG{A} \wedge \hB{B}) \vee \hG{C}, (\hG{A} \wedge \hG{B}) \vee \hE{C}, (\hG{A} \wedge \hG{B}) \vee \hG{C} \seq \hC{B} \wedge \hD{A}, \hC{B} \wedge \hD{A}, \hF{C}$}
\end{prooftree}
\end{scriptsize}
\hfill\QED
\end{example}



\end{landscape}
\section{Redundancy}
\label{sec:Redundancy}

A closer inspection of the projections in example \ref{example:OProjections} reveals that they are quite redundant. Inference $\wedge_l^6$, for example, repeatedly appears in four of the six projections; $\wedge_r^2$ appears in two projections; and $\vee_l^5$ appears in all six projections. Furthermore, all six projections derive the formula $(A \wedge B) \vee C$ in their end-sequents, and then all occurrences of this formula must be contracted in the bottom of $\CEResNF{\varphi}{\delta}$ shown in example \ref{example:CEResSONormalForm}. 

The redundancy is mainly a consequence of the simple transformation to conjunctive normal form used by $\CERes$. When disjunction is distributed over conjunction, it is often the case that literals must be duplicated. Hence, each projection w.r.t. a clause that contains a copy of a duplicated literal will have to contain a copy of every inference that operates on descendants of this literal.

In the worst case, by Theorem \ref{theorem:SizeOfCEResNormalForms}, this redundancy can make $\CERes$ normal forms exponentially larger than normal forms produced by reductive cut-elimination methods. Theorem \ref{theorem:SizeOfCEResNormalForms} is a corollary of Theorem \ref{theorem:SizeOfClauseSets}, which shows that exponential blow-up already occurs in the size of the characteristic clause set.


\begin{theorem}[Size of Characteristic Clause Set]
\label{theorem:SizeOfClauseSets}
There exist positive constants $k$ and $k'$ and a sequence of proofs $\varphi_1, \varphi_2, \ldots, \varphi_n, \ldots$ such that $\proofsizeSymbol{\varphi_n} \leq k 4^n$ and $|\clauseset{\varphi_n}| \geq k' 2^{2^n}$.
\end{theorem}
\begin{proof}
Let $\psi_1(s)$ be the following proof:
\begin{small}
\begin{prooftree}
\AXC{$A_1(s) \seq A_1(s)$}
		\AXC{$B_1(s) \seq B_1(s)$} \RightLabel{$\wedge_r$}
	\BIC{$A_1(s), B_1(s) \seq A_1(s) \wedge B_1(s)$} \RightLabel{$\wedge_l$}
	\UIC{$A_1(s)\wedge B_1(s) \seq A_1(s) \wedge B_1(s)$} 
				\AXC{$C_1(s) \seq C_1(s)$}
						\AXC{$D_1(s) \seq D_1(s)$} \RightLabel{$\wedge_r$}
					\BIC{$C_1(s), D_1(s) \seq C_1(s) \wedge D_1(s)$} \RightLabel{$\wedge_l$}
					\UIC{$C_1(s)\wedge D_1(s) \seq C_1(s) \wedge D_1(s)$} \RightLabel{$\vee_l$}
			\BIC{$(A_1(s)\wedge B_1(s)) \vee (C_1(s)\wedge D_1(s)) \seq A_1(s) \wedge B_1(s), C_1(s) \wedge D_1(s)$} \RightLabel{$\vee_r$}
			\UIC{$(A_1(s)\wedge B_1(s)) \vee (C_1(s)\wedge D_1(s)) \seq (A_1(s) \wedge B_1(s)) \vee (C_1(s) \wedge D_1(s))$}
\end{prooftree}
\end{small}

\noindent
And let $\psi_n(s)$ ($n>1$) be:
\begin{small}
\begin{prooftree}
\AXC{$\psi_{n-1}(a.s)$} \noLine
\UIC{$A_n(s) \seq A_n(s)$}
		\AXC{$\psi_{n-1}(b.s)$} \noLine
		\UIC{$B_n(s) \seq B_n(s)$} \RightLabel{$\wedge_r$}
	\BIC{$A_n(s), B_n(s) \seq A_n(s) \wedge B_n(s)$} \RightLabel{$\wedge_l$}
	\UIC{$A_n(s)\wedge B_n(s) \seq A_n(s) \wedge B_n(s)$} 
				\AXC{$\psi_{n-1}(c.s)$} \noLine
				\UIC{$C_n(s) \seq C_n(s)$}
						\AXC{$\psi_{n-1}(d.s)$} \noLine
						\UIC{$D_n(s) \seq D_n(s)$} \RightLabel{$\wedge_r$}
					\BIC{$C_n(s), D_n(s) \seq C_n(s) \wedge D_n(s)$} \RightLabel{$\wedge_l$}
					\UIC{$C_n(s)\wedge D_n(s) \seq C_n(s) \wedge D_n(s)$} \RightLabel{$\vee_l$}
			\BIC{$(A_n(s)\wedge B_n(s)) \vee (C_n(s)\wedge D_n(s)) \seq A_n(s) \wedge B_n(s), C_n(s) \wedge D_n(s)$} \RightLabel{$\vee_r$}
			\UIC{$(A_n(s)\wedge B_n(s)) \vee (C_n(s)\wedge D_n(s)) \seq (A_n(s) \wedge B_n(s)) \vee (C_n(s) \wedge D_n(s))$}
\end{prooftree}
\end{small}
where $x.s$ denotes the the result of prepending $x$ in the list $s$ and (for $n>1$):
$$
\begin{array}{rcl}
A_n(s) 	& \defEq & (A_{n-1}(a.s)\wedge B_{n-1}(a.s)) \vee (C_{n-1}(a.s)\wedge D_{n-1}(a.s)) \\
B_n(s) 	& \defEq & (A_{n-1}(b.s)\wedge B_{n-1}(b.s)) \vee (C_{n-1}(b.s)\wedge D_{n-1}(b.s)) \\
C_n(s) 	& \defEq & (A_{n-1}(c.s)\wedge B_{n-1}(c.s)) \vee (C_{n-1}(c.s)\wedge D_{n-1}(c.s)) \\
D_n(s) 	& \defEq & (A_{n-1}(d.s)\wedge B_{n-1}(d.s)) \vee (C_{n-1}(d.s)\wedge D_{n-1}(d.s)) \\
\end{array}
$$

\noindent
Let $\varphi_n$ be the proof below:
\begin{small}
\begin{prooftree}
\AXC{$\psi_n([])$}
		\AXC{$\psi_n([])$} \RightLabel{$cut$}
	\BIC{$(A_n([])\wedge B_n([])) \vee (C_n([])\wedge D_n([])) \seq (A_n([]) \wedge B_n([])) \vee (C_n([]) \wedge D_n([]))$}
\end{prooftree}
\end{small}

\noindent
Let $S_{\psi_k(s)}^l$ be the subformula of $\struct{\varphi_n}$ corresponding to the root inference of the subproof $\psi_k(s)$ in the left side of $\varphi_n$. Analogously, let $S_{\psi_k(s)}^r$ be the subformula of $\struct{\varphi_n}$ at the root inference of the subproof $\psi_k(s)$ in the right side of $\varphi_n$. Then:
$$
\struct{\varphi_n} = S_{\psi_n([])}^l \structplus S_{\psi_n([])}^r
$$
%
where:
$$
S_{\psi_j(s)}^l = \left\{ \begin{array}{ll}
(A_1(s) \structplus B_1(s)) \structtimes (C_1(s) \structplus D_1(s) & \textrm{, if } j=1 \\
(S_{\psi_{j-1}(a.s)}^l \structplus S_{\psi_{j-1}(b.s)}^l) \structtimes (S_{\psi_{j-1}(c.s)}^l \structplus S_{\psi_{j-1}(d.s)}^l) & \textrm{, otherwise }
\end{array}\right.
$$
$$
S_{\psi_j(s)}^r = \left\{ \begin{array}{ll}
  (\neg A_1(s) \structtimes \neg B_1(s)) 
  \structplus
  (\neg C_1(s) \structtimes \neg D_1(s))  & \textrm{, if } j=1 \\
  (S_{\psi_{j-1}(a.s)}^r \structtimes S_{\psi_{j-1}(b.s)}^r) 
  \structplus
  (S_{\psi_{j-1}(c.s)}^r \structtimes S_{\psi_{j-1}(d.s)}^r) & \textrm{, otherwise }
\end{array}\right.
$$

\noindent
Let $f_l(n)$ be the number of clauses in $\clauseset{\varphi_n}$ stemming from the left branch of the cut. Analogously, let $f_r(n)$ be the number of clauses stemming from the right branch of the cut. Clearly, $|\clauseset{\varphi_n}| = f_l(n) + f_r(n)$. By analyzing the structure of the subformulas of $\struct{\varphi_n}$, it is possible to see that:
$$
f_l(n) = \left\{ \begin{array}{ll}
4 & \textrm{, if } n=1 \\
4 (f_l(n-1))^2 & \textrm{, otherwise }
\end{array}\right.
$$
$$
f_r(n) = \left\{ \begin{array}{ll}
2 & \textrm{, if } n=1 \\
2 (f_r(n-1))^2 & \textrm{, otherwise }
\end{array}\right.
$$
%
It can be easily proved by induction that $f_l(n) = 4^{(2^n - 1)}$ and $f_r(n) = 2^{(2^n - 1)}$. Therefore:
$$
|\clauseset{\varphi_n}| = 4^{(2^n - 1)} + 2^{(2^n - 1)} \geq 2^{(2^n)} 
$$
\hfill\QED
\end{proof}



\begin{theorem}[Size of $\CERes$-Normal-Form]
\label{theorem:SizeOfCEResNormalForms}
There exist positive constants $k$ and $k''$ and a sequence of proofs $\varphi_1, \varphi_2, \ldots, \varphi_n, \ldots$ such that:
\begin{itemize}
\item $\proofsizeSymbol{\varphi_n} \leq k 4^n$.
\item $\proofsizeSymbol{\CEResNF{\varphi_n}{\delta_n}} \geq k'' 2^{2^n}$, for any refutation $\delta_n$.
\item $\proofsizeSymbol{\varphi_n^*} \leq k 4^n$, for any $\varphi_n^*$ obtained from $\varphi_n$ by reductive cut-elimination.
\end{itemize}
\end{theorem}
\begin{proof}
Consider the proofs $\varphi_n$ defined in the proof of the Theorem \ref{theorem:SizeOfClauseSets}. As proved there, 
$|\clauseset{\varphi_n}| \geq k' 2^{2^n}$, for some positive rational constant $k'$. Moreover, in any refutation $\delta_n$ of $\clauseset{\varphi_n}$, every clause of $\clauseset{\varphi_n}$ has to be used at least once. Therefore, $\proofsizeSymbol{\delta_n} \geq k'' 2^{2^n}$, for some $k'' > k'$. Since $\proofsizeSymbol{\CEResNF{\varphi_n}{\delta_n}} \geq \proofsizeSymbol{\delta_n}$, $\proofsizeSymbol{\CEResNF{\varphi_n}{\delta_n}} \geq k'' 2^{2^n}$ as well. As $\varphi_n$ contains neither implicit nor explicit contractions, the size strictly decreases with every reductive cut-elimination step. Hence, for any proof $\varphi_n^*$ obtained from $\varphi_n$ by reductive cut-elimination, $\proofsizeSymbol{\varphi_n^*} < \proofsizeSymbol{\varphi_n} < k 4^n$.
\hfill\QED
\end{proof}

\noindent
While the sequence of proofs used to prove Theorems \ref{theorem:SizeOfClauseSets} and \ref{theorem:SizeOfCEResNormalForms} is rather artificial, it is important to note that redundancy can be expected to occur often in practice as well. Whenever the input proof has a structure with alternations of inferences operating on cut-ancestors and end-sequent-ancestors, the characteristic formula has alternations of disjunctions and conjunctions, and its literals are duplicated during the clause form transformation of the characteristic formula. Therefore, this is an issue that must be addressed for $\CERes$ to be efficiently applicable to proofs with complex structure.

\section{Taking Inference Permutability into Account}
\label{sec:InferencePermutability}

Redundancy originates in the distribution of disjunction over conjunction during the clause form transformation of the characteristic formula. As conjunctions and disjunctions in the characteristic formula correspond to binary inferences operating, respectively, on cut-ancestors and end-sequent-ancestors in the proof, the amount of distributions can be reduced by first pre-processing the proof and permuting inferences operating on cut-ancestors downward. The rules for inference permutation are shown in Appendix \ref{appendix:InferencePermutation} and in \cite{Woltzenlogel-Paleo2009A-General-Analysis-of-Cut-Elimination-by-CERes}.

\begin{example}
\label{example:InferencePermutation}
Downward permutation of all inferences operating on cut ancestors transforms proof $\varphi$ from Example \ref{example:CutPertinentStruct} into the proof $\psi$ shown below:
\begin{small}
\begin{prooftree}
			\AXC{$\psi'$} \noLine
			\UIC{$(\hF{A} \wedge \hF{B}) \vee \hF{C} \seq \hA{A} \wedge \hB{B}, \hF{C}$}
						\AXC{$C \seq C$} \RightLabel{$cut_2$}
			\BIC{$(\hF{A} \wedge \hF{B}) \vee \hF{C} \seq \hA{A} \wedge \hB{B}, C$}
					\AXC{$\hC{B} \seq \hC{B} $}
							\AXC{$\hC{A} \seq \hC{A}$} \RightLabel{$\wedge_r$}
						\BIC{$\hC{A}, \hC{B} \seq \hC{B} \wedge \hC{A}$} \RightLabel{$\wedge_l$}
						\UIC{$\hC{A} \wedge \hC{B} \seq \hC{B} \wedge \hC{A}$}  \RightLabel{$cut_1$}
				\BIC{$(\hF{A} \wedge \hF{B}) \vee \hF{C} \seq B \wedge A, C $}
\end{prooftree}
\end{small}
where $\psi'$ is:
\begin{small}
\begin{prooftree}
\AXC{$\hA{A} \seq \hA{A} $} \RightLabel{$w_l$}
\UIC{$\hA{A}, \hA{B} \seq \hA{A}$} \RightLabel{$\wedge_l$}
\UIC{$\hA{A} \wedge \hA{B} \seq \hA{A}$} 
		\AXC{$\hA{C} \seq \hA{C}$} \RightLabel{$\hA{\vee_l}$}  		
	\BIC{$(\hA{A} \wedge \hA{B}) \vee \hA{C}, \seq \hA{A}, \hA{C}$}
				\AXC{$\hB{B} \seq \hB{B}$} \RightLabel{$w_l$}
				\UIC{$\hB{A}, \hB{B} \seq \hB{B}$} \RightLabel{$\wedge_l$}
				\UIC{$\hB{A} \wedge \hB{B} \seq \hB{B}$}
						\AXC{$\hB{C} \seq \hB{C}$} \RightLabel{$\hB{\vee_l}$}  		
					\BIC{$(\hB{A} \wedge \hB{B}) \vee \hB{C} \seq \hB{B}, \hB{C}$}\RightLabel{$\wedge_r$}
			\BIC{$(\hA{A} \wedge \hA{B}) \vee \hA{C}, (\hB{A} \wedge \hB{B}) \vee \hB{C} \seq \hA{A} \wedge \hB{B},\hA{C},\hB{C}$} \RightLabel{$c_l$} 		
			\UIC{$(\hF{A} \wedge \hF{B}) \vee \hF{C} \seq \hA{A} \wedge \hB{B}, \hA{C}, \hB{C}$}
\RightLabel{$c_r$} 		
			\UIC{$(\hF{A} \wedge \hF{B}) \vee \hF{C} \seq \hA{A} \wedge \hB{B}, \hF{C}$}
\end{prooftree}
\end{small}

\noindent
Its characteristic formula is:
$$
\struct{\psi} 
\equiv 
(((A\structtimes C) \structplus (B\structtimes C)) \structplus \structdual{C}) \structplus (\structdual{B} \structtimes \structdual{A})
$$

\noindent
And its characteristic clause set is: 
$$
\clauseset{\psi} \equiv \{ \seq A,C \ \ ; \ \ \seq B,C \ \ ; \ \ C \seq \ \ ; \ \ B, A \seq \ \  \}
$$

\noindent 
Thanks to inference permutations, fewer duplications occur in $\clauseset{\psi}$ than in $\clauseset{\varphi}$.
\hfill\QED
\end{example}


\noindent
However, local proof rewriting such as inference permutation are inefficient. It is typical of reductive cut-elimination methods and it is something that cut-elimination by resolution strives to avoid and improve. Therefore, a method for transforming the characteristic formula into clause form that takes the possibility of inference permutation into account without actually having to perform inference permutations is desirable. Such a method ($\normalizePlusTimesS$) is defined in this section, and it is shown in Lemma \ref{lemma:SwapNormalizeCorrespondence} that every rewriting of the characteristic formula according to $\normalizePlusTimesS$ corresponds to a sequence of inference permutation steps according to the rules in Appendix \ref{appendix:InferencePermutation}. Thus, while $\normalizePlusTimes$ does full distribution of disjunction over conjunction, as if inferences operating on cut-ancestors were always indirectly dependent on the inferences operating on end-sequent ancestors above them, $\normalizePlusTimesS$ does partial distribution when the corresponding inferences are independent and permutable without the need for duplications. For the partial distribution to be possible, the characteristic formula must contain extra information to allow the retrieval of the dependencies between the branching inferences.



\begin{definition}
\label{definition:InferenceOccurrences}
Let $\rho$ be an inference in a proof $\varphi$. Then
$
\occInference{\varphi}{\rho}
$
denotes the set of descendents of formulas that occur in axiom sequents containing ancestors of auxiliary formulas of $\rho$.
\end{definition}


\newcommand{\marked}[1]{#1^*}

\begin{example}
In the proof $\varphi$ below, the formulas belonging to $\occInference{\varphi}{\vee_l}$ have been highlighted in blue:
\begin{prooftree}
\AXC{$\hB{A} \seq \hB{A} $}
		\AXC{$\hB{B} \seq \hB{B}$} \RightLabel{$\wedge_r$}
	\BIC{$\hB{A}, \hB{B} \seq \hB{A} \wedge \hB{B}$} \RightLabel{$\wedge_l$}
	\UIC{$\hB{A} \wedge \hB{B} \seq \hB{A} \wedge \hB{B}$}
				\AXC{$B \seq B $}
						\AXC{$A \seq A$} \RightLabel{$\wedge_r$}
					\BIC{$A, B \seq B \wedge A$} \RightLabel{$\wedge_l$}
					\UIC{$A \wedge B \seq B \wedge A$} \RightLabel{$cut_1$}
			\BIC{$\hB{A} \wedge \hB{B} \seq B \wedge A$}
							\AXC{$\hB{C} \seq \hB{C}$} 
									\AXC{$C \seq C$} \RightLabel{$cut_2$}
								\BIC{$\hB{C} \seq C$} \RightLabel{$\vee_l$}
					\BIC{$(\hB{A} \wedge \hB{B}) \vee \hB{C} \seq B \wedge A, C $} 
\end{prooftree}

\noindent
And below, the formulas belonging to $\occInference{\varphi}{cut_1}$ have been highlighted in blue:
\begin{prooftree}
\AXC{$\hB{A} \seq \hB{A} $}
		\AXC{$\hB{B} \seq \hB{B}$} \RightLabel{$\wedge_r$}
	\BIC{$\hB{A}, \hB{B} \seq \hB{A} \wedge \hB{B}$} \RightLabel{$\wedge_l$}
	\UIC{$\hB{A} \wedge \hB{B} \seq \hB{A} \wedge \hB{B}$}
				\AXC{$\hB{B} \seq \hB{B} $}
						\AXC{$\hB{A} \seq \hB{A}$} \RightLabel{$\wedge_r$}
					\BIC{$\hB{A}, \hB{B} \seq \hB{B} \wedge \hB{A}$} \RightLabel{$\wedge_l$}
					\UIC{$\hB{A} \wedge \hB{B} \seq \hB{B} \wedge \hB{A}$} \RightLabel{$cut_1$}
			\BIC{$\hB{A} \wedge \hB{B} \seq \hB{B} \wedge \hB{A}$}
							\AXC{$C \seq C$} 
									\AXC{$C \seq C$} \RightLabel{$cut_2$}
								\BIC{$C \seq C$} \RightLabel{$\vee_l$}
					\BIC{$(\hB{A} \wedge \hB{B}) \vee C \seq \hB{B} \wedge \hB{A}, C $} 
\end{prooftree}
%\hfill\QED
\end{example}






\begin{definition}[$\normalizePlusTimesS$]
\label{definition:NormalizationPlusTimesSwap}
\index{Struct Normalization!Swapped}
%\newcommand{\hF}[1]{{\color{brickred} #1}}
In the rewriting rules below, let $\rho$ be the inference in $\varphi$ corresponding to $\hF{\structtimes_{\rho}}$. For the rewriting rules to be applicable, 
$\hF{S_{n+1}}, \ldots, \hF{S_{n+m}}$ and $\hF{S}$ must contain at least one formula from $\occInference{\varphi}{\rho}$ each (i.e. there must be an atomic subformula $\hF{A}$ of $\hF{S_{n+k}}$ such that $\hF{A} \in \occInference{\varphi}{\rho}$), and $S_1, \ldots, S_n$ and $S'$ and $S''$ should not contain any formula from $\occInference{\varphi}{\rho}$. Moreover, an innermost rewriting strategy is enforced.
$$
\hF{S} \hF{\structtimes_{\rho}} (S_1 \structplus \ldots \structplus S_n \structplus \hF{S_{n+1}} \structplus \ldots \structplus \hF{S_{n+m}}) \normalizePlusTimesS  S_1 \structplus \ldots \structplus S_n \structplus (\hF{S} \hF{\structtimes_{\rho}} \hF{S_{n+1}}) \structplus \ldots \structplus (\hF{S} \hF{\structtimes_{\rho}} \hF{S_{n+m}})
$$
\begin{multicols}{3}{
$$
\hF{S} \hF{\structplus_{\rho}} S' \normalizePlusTimesS  S'
$$

$$
S' \hF{\structtimes_{\rho}} S'' \normalizePlusTimesS  S'
$$

$$
S' \hF{\structplus_{\rho}} S'' \normalizePlusTimesS  S'
$$
}\end{multicols}

\end{definition}

\begin{definition}[Degenerate Inferences]
An inference $\rho$ in a proof $\varphi$ is \emph{degenerate} when all its 
auxiliary formulas are descendants of main formulas of 
weakening inferences. When only some auxiliary (sub)formulas of $\rho$ are descendants of main formulas of weakening inferences, $\rho$ is \emph{partially degenerate}.
\end{definition}


\begin{remark}
The last three rules in Definition \ref{definition:NormalizationPlusTimesSwap} handle connectives that correspond to (partially) \emph{degenerate inferences}, whose auxiliary formulas are introduced by weakening. These rules are related to downward permutation of weakening inferences, as shown in Lemma \ref{lemma:SwapNormalizeCorrespondence}. Because of the last two rules, $\normalizePlusTimesS$ is not confluent.
\end{remark}


\begin{definition}[Swapped Clause Set]
\label{definition:CutPertinentClauseSetSwap}
A \emph{swapped clause set} $\clausesetSwap{\varphi}{S}$ of a proof $\varphi$ is the $\normalizePlusTimesS$-normal-form $S$ of $\struct{\varphi}$ written in sequent notation.
\end{definition}

\begin{remark}
In cases where $\clausesetSwap{\varphi}{S_1} = \clausesetSwap{\varphi}{S_1}$ for any $\normalizePlusTimesS$-normal-forms $S_1$ and $S_2$ of $\struct{\varphi}$, the unique swapped clause set is denoted simply as $\clausesetSwapUnique{\varphi}$.
\end{remark}


\begin{remark}
Swapped clause sets are very similar to \emph{profiles}, which have been defined in \cite{Hetzl2007CharacteristicClauseSetsandProofTransformations}. In fact, the concept of swapped clause set evolved from attempts to find a simpler explanation for profiles. They differ only on proofs that contain  degenerate inferences. In such cases, swapped clause sets are always smaller \cite{Woltzenlogel-Paleo2009A-General-Analysis-of-Cut-Elimination-by-CERes}.
\end{remark}


\begin{definition}[$\CEResSwap$-normal-form]
$\CEResNFSwap{\varphi}{\delta}$ denotes \emph{$\CEResSwap$ normal form} of the proof $\varphi$ w.r.t. the resolution refutation $\delta$ of any swapped clause set $\clausesetSwap{\varphi}{S}$. It is obtained in the same way as a $\CERes$-normal-form, but using a swapped clause set $\clausesetSwap{\varphi}{S}$ instead of the clause set $\clauseset{\varphi}$.
\end{definition}


\begin{example}
\label{example:PlusTimesSwapNormalization}
Let $\varphi$ be the proof below:
\begin{prooftree}
\AXC{$\hA{A} \seq \hA{A} $}
		\AXC{$\hB{B} \seq \hB{B}$} \RightLabel{$\wedge_r^1$}
	\BIC{$\hA{A}, \hB{B} \seq \hA{A} \wedge \hB{B}$} \RightLabel{$\wedge_l$}
	\UIC{$\hA{A} \wedge \hB{B} \seq \hA{A} \wedge \hB{B}$}
				\AXC{$\hC{B} \seq \hC{B} $}
						\AXC{$\hD{A} \seq \hD{A}$} \RightLabel{$\wedge_r^2$}
					\BIC{$\hD{A}, \hC{B} \seq \hC{B} \wedge \hD{A}$} \RightLabel{$\wedge_l$}
					\UIC{$\hD{A} \wedge \hC{B} \seq \hC{B} \wedge \hD{A}$} \RightLabel{$cut^3$}
			\BIC{$\hA{A} \wedge \hB{B} \seq \hC{B} \wedge \hD{A}$}
							\AXC{$\hE{C} \seq \hE{C}$} 
									\AXC{$\hF{C} \seq \hF{C}$} \RightLabel{$cut^4$}
								\BIC{$\hE{C} \seq \hF{C}$} \RightLabel{$\vee_l^5$}
					\BIC{$(\hA{A} \wedge \hB{B}) \vee \hE{C} \seq \hC{B} \wedge \hD{A}, \hF{C} $} 
\end{prooftree}

\noindent
Its characteristic formula is:
$$
\struct{\varphi} 
\equiv 
((\hA{A} \structplus^1 \hB{B}) \structplus^3 (\hC{\structdual{B}} \structtimes^2 \hD{\structdual{A}}))
\structtimes^5
(\hE{C} \structplus^4 \hF{\structdual{C}})
$$

\noindent
Considering that $\{ \hA{A}, \hB{B}, \hE{C}\} \subset \occInference{\varphi}{\vee_l^5}$ and $\{ \hD{A}, \hC{B}, \hF{C} \} \cap \occInference{\varphi}{\vee_l^5} = \emptyset$, the characteristic formula $\struct{\varphi}$ can be normalized in the two ways shown below:
$$
\begin{array}{rcl}
\struct{\varphi} 
& \equiv &
((\hA{A} \structplus^1 \hB{B}) \structplus^3 (\hC{\structdual{B}} \structtimes^2 \hD{\structdual{A}}))
\structtimes^5
(\hE{C} \structplus^4 \hF{\structdual{C}}) \\
%
& \normalizePlusTimesS &
	((\hA{A} \structplus^1 \hB{B})\structtimes^5 (\hE{C} \structplus^4 \hF{\structdual{C}}))
\structplus^3 
	(\hC{\structdual{B}} \structtimes^2 \hD{\structdual{A}}) \\
%
& \normalizePlusTimesS &
((\hA{A} \structtimes^5
(\hE{C} \structplus^4 \hF{\structdual{C}})) \structplus^1 (\hB{B}\structtimes^5
(\hE{C} \structplus^4 \hF{\structdual{C}}))) \structplus^3 (\hC{\structdual{B}} \structtimes^2 \hD{\structdual{A}}) \\
%
& \normalizePlusTimesS &
(
		(
			(
				\hA{A} 
			\structtimes^5 
				\hE{C}
			) 
		\structplus^4 
			\hF{\structdual{C}}
		) 
	\structplus^1 
		(
			(
				\hB{B}
			\structtimes^5 
				\hE{C}
			) 
		\structplus^4 
			\hF{\structdual{C}}
		)
) 
\structplus^3 
(
	\hC{\structdual{B}} \structtimes^2 \hD{\structdual{A}}
) \\
& \equiv &	
	(
		\hA{A} 
	\structtimes^5 
		\hE{C}
	) 
\structplus 
	\hF{\structdual{C}}	 
\structplus 	
	(
		\hB{B}
	\structtimes^5 
		\hE{C}
	) 
\structplus 
	\hF{\structdual{C}}
\structplus 
(
	\hC{\structdual{B}} \structtimes^2 \hD{\structdual{A}}
) \\
& \equiv & S_1
\end{array}
$$

$$
\begin{array}{rcl}
\struct{\varphi} 
& \equiv &
((\hA{A} \structplus^1 \hB{B}) \structplus^3 (\hC{\structdual{B}} \structtimes^2 \hD{\structdual{A}}))
\structtimes^5
(\hE{C} \structplus^4 \hF{\structdual{C}}) \\
%
& \normalizePlusTimesS &
	((\hA{A} \structplus^1 \hB{B})\structtimes^5 (\hE{C} \structplus^4 \hF{\structdual{C}}))
\structplus^3 
	(\hC{\structdual{B}} \structtimes^2 \hD{\structdual{A}}) \\
%
& \normalizePlusTimesS &
	((((\hA{A} \structplus^1 \hB{B})\structtimes^5 \hE{C}) \structplus^4 \hF{\structdual{C}}))
\structplus^3 
	(\hC{\structdual{B}} \structtimes^2 \hD{\structdual{A}}) \\
%
%
& \normalizePlusTimesS &
	((((\hA{A}\structtimes^5 \hE{C}) \structplus^1 (\hB{B}\structtimes^5 \hE{C})) \structplus^4 \hF{\structdual{C}}))
\structplus^3 
	(\hC{\structdual{B}} \structtimes^2 \hD{\structdual{A}}) \\
%
& \equiv &
	(\hA{A}\structtimes^5 \hE{C}) \structplus (\hB{B}\structtimes^5 \hE{C}) \structplus \hF{\structdual{C}}
\structplus 
	(\hC{\structdual{B}} \structtimes^2 \hD{\structdual{A}}) \\
%
& \equiv &
	S_2 \\
\end{array}
$$

\noindent
The swapped clause sets are:
$$
\clausesetSwap{\varphi}{S_1} = \{\ \ \seq \hA{A}, \hE{C} \ \ ; \ \ \seq \hB{B}, \hE{C} \ \ ;
								\ \ \hF{C} \seq \ \ ; \ \ \hF{C} \seq \ \ ; \ \ \hC{B}, \hD{A} \seq \ \ \}
$$
$$
\clausesetSwap{\varphi}{S_2} = \{\ \ \seq \hA{A}, \hE{C} \ \ ; \ \ \seq \hB{B}, \hE{C} \ \ ;
							 \ \ \hF{C} \seq \ \ ; \ \ \hC{B}, \hD{A} \seq \ \ \}
$$

\noindent
It is interesting to note that $\clausesetSwap{\varphi}{S_1} = \clausesetSwap{\varphi}{S_2}$ (because they are sets). This is not a coincidence. It always occurs when the non-confluence is due to non-degenerated applications of the first rewriting rule. A refutation $\delta$ of $\clausesetSwapUnique{\varphi}$ is shown below:
\begin{small}
\begin{prooftree}
\AXC{$\seq \hA{A}, \hE{C}$}
			\AXC{$\seq \hB{B}, \hE{C}$}
					\AXC{$\hC{B}, \hD{A} \seq$} \RightLabel{$r$}
				\BIC{$\hD{A} \seq \hE{C}$} \RightLabel{$r$}
	\BIC{$\seq \hE{C}, \hE{C}$} \RightLabel{$f_r$}
	\UIC{$\seq \hE{C}$} 
			\AXC{$\hF{C} \seq$} \RightLabel{$r$}
		\BIC{$\seq$}
\end{prooftree}
\end{small}

\noindent
$\CEResNFSwap{\varphi}{\delta}$ is the $\CERes$-normal-form obtained with the swapped clause set:
\begin{scriptsize}
\begin{prooftree}
		\AXC{$ \varphi_0 $} \noLine
		\UIC{$(\hA{A} \wedge \hG{B}) \vee \hE{C}, (\hG{A} \wedge \hB{B}) \vee \hE{C} \seq \hC{B} \wedge \hD{A}, \hE{C}$} 
				\AXC{$\hF{C} \seq \hF{C}$} \RightLabel{$cut$}
			\BIC{$(\hA{A} \wedge \hG{B}) \vee \hE{C}, (\hG{A} \wedge \hB{B}) \vee \hE{C} \seq \hC{B} \wedge \hD{A}, \hF{C}$} \RightLabel{$c_l$}
			\UIC{$(A \wedge B) \vee C \seq \hC{B} \wedge \hD{A}, \hF{C}$}
\end{prooftree}
\end{scriptsize}
where $\varphi_0$ is:
\begin{scriptsize}
\begin{prooftree}
\AXC{$\hA{A} \seq \hA{A}$} \RightLabel{$w_l$}
\UIC{$\hA{A}, \hG{B} \seq \hA{A}$} \RightLabel{$\wedge_l$}
\UIC{$\hA{A} \wedge \hG{B} \seq \hA{A}$}
		\AXC{$\hE{C} \seq \hE{C}$} \RightLabel{$\vee_l$}
	\BIC{$(\hA{A} \wedge \hG{B}) \vee \hE{C} \seq \hA{A}, \hE{C}$} 
				\AXC{$\hB{B} \seq \hB{B}$} \RightLabel{$w_l$}
				\UIC{$\hG{A}, \hB{B} \seq \hB{B}$} \RightLabel{$\wedge_l$}
				\UIC{$\hG{A} \wedge \hB{B} \seq \hB{B}$}
						\AXC{$\hE{C} \seq \hE{C}$} \RightLabel{$\vee_l$}
					\BIC{$(\hG{A} \wedge \hB{B}) \vee \hE{C} \seq \hB{B}, \hE{C}$} 
							\AXC{$\hC{B} \seq \hC{B} $}
									\AXC{$\hD{A} \seq \hD{A}$} \RightLabel{$\wedge_r$}
								\BIC{$\hD{A}, \hC{B} \seq \hC{B} \wedge \hD{A}$} \RightLabel{$cut$}
						\BIC{$(\hG{A} \wedge \hB{B}) \vee \hE{C}, \hD{A} \seq \hC{B} \wedge \hD{A}, \hE{C}$} \RightLabel{$cut$}
		\BIC{$(\hA{A} \wedge \hG{B}) \vee \hE{C}, (\hG{A} \wedge \hB{B}) \vee \hE{C} \seq \hC{B} \wedge \hD{A}, \hE{C}, \hE{C}$} \RightLabel{$c_r$}
		\UIC{$(\hA{A} \wedge \hG{B}) \vee \hE{C}, (\hG{A} \wedge \hB{B}) \vee \hE{C} \seq \hC{B} \wedge \hD{A}, \hE{C}$}
\end{prooftree}
\end{scriptsize}

\noindent
As expected, less redundant clause sets and projections result in a $\CEResNFSwap{\varphi}{\delta}$ significantly smaller than $\CEResNF{\varphi}{\delta}$ shown in Example \ref{example:CEResSONormalForm}. 
\hfill\QED
\end{example}


\begin{lemma}[Correspondence between $\normalizePlusTimesS$ and $\swap$]
\label{lemma:SwapNormalizeCorrespondence}
If $\varphi$ is skolemized and $\struct{\varphi} \normalizePlusTimesS S$, then there exists a proof $\psi$ such that $\varphi \swap^* \psi$ and $\struct{\psi} = S$. 
\end{lemma}
\begin{proof}
In Appendix \ref{sec:Correspondence}.
\end{proof}


\begin{theorem}[Unsatisfiability of the Swapped Clause Set]
\label{theorem:UnsatisfiabilityOfClauseSetSwap}\\
For any skolemized proof $\varphi$ and $\normalizePlusTimesS$-normal-form $S$ of $\struct{\varphi}$, $\clausesetSwap{\varphi}{S}$ is unsatisfiable.
\end{theorem}
\begin{proof}
By Lemma \ref{lemma:SwapNormalizeCorrespondence},
there is $\psi$ such that $\struct{\psi} = S$. Clearly, 
$\clausesetSwap{\varphi}{S} = \clausesetSwapUnique{\psi}$. 
But $\clausesetSwapUnique{\psi} = \clauseset{\psi}$, 
since $S$ is also a $\normalizePlusTimes$-normal-form. 
Therefore, $\clausesetSwap{\varphi}{S} = \clauseset{\psi}$ and, 
by Theorem \ref{theorem:Unsatisfiability} and the fact that 
$\clauseset{\psi}$ is equisatisfiable with $\struct{\psi}$, 
$\clausesetSwap{\varphi}{S}$ is unsatisfiable.
\hfill\QED
\end{proof}


\begin{remark}
It is not possible to prove Theorem \ref{theorem:UnsatisfiabilityOfClauseSetSwap} analogously to the proof of the unsatisfiability of the profile shown in \cite{Hetzl2007CharacteristicClauseSetsandProofTransformations,Hetzl2008ProofProfiles.CharacteristicClauseSetsandProofTransformations}, which is essentially based on the fact that the profile of $\varphi$ subsumes $\clauseset{\varphi}$. Unfortunately, $\clausesetSwap{\varphi}{S}$ does not subsume $\clauseset{\varphi}$ in general. In particular, the subsumption fails when $\varphi$ contains degenerate inferences, in which case $\normalizePlusTimesS$ prunes too much of the characteristic formula and then some clauses of $\clauseset{\varphi}$ are not subsumed by any clause of $\clausesetSwap{\varphi}{S}$.
\end{remark}


\noindent
Although $\CEResSwap$-normal-forms are usually much less redundant than $\CERes$-normal-forms. In the worst case, there is still an exponential blow-up when the characteristic formula is converted to a swapped clause set.

\begin{theorem}[Size of Swapped Clause Set]
\label{theorem:SizeOfSwappedClauseSets}
There exist positive constants $k$ and $k'$ and a sequence of proofs $\varphi_1, \varphi_2, \ldots, \varphi_n, \ldots$ such that $\proofsizeSymbol{\varphi_n} \leq k 4^n$ and $|\clausesetSwapUnique{\varphi_n}| \geq k' 2^{2^n}$.
\end{theorem}

\begin{theorem}[Size of $\CEResSwap$-Normal-Form]
\label{theorem:SizeOfCEResSwapNormalForms}
There exist positive constants $k$ and $k''$ and a sequence of proofs $\varphi_1, \varphi_2, \ldots, \varphi_n, \ldots$ such that:
\begin{itemize}
\item $\proofsizeSymbol{\varphi_n} \leq k 4^n$.
\item $\proofsizeSymbol{\CEResNFSwap{\varphi_n}{\delta_n}} \geq k'' 2^{2^n}$, for any refutation $\delta_n$.
\end{itemize}
\end{theorem}
\begin{proof}
For both theorems above, the same sequence of proofs used in 
Theorem \ref{theorem:SizeOfClauseSets} can be used, 
because $\clauseset{\varphi_k} = \clausesetSwapUnique{\varphi_k}$ for any proof $\varphi_k$ in that sequence.
\hfill\QED
\end{proof}

\section{Using Structural Clause Form Transformation}
\label{sec:StructuralClauseForm}


The construction of standard clause sets from structs is analogous to the standard transformation of formulas to conjunctive normal forms. Consequently, it has the same well-known disadvantage of increasing the size significantly in the worst case. Indeed, the size of a standard clause set can be exponential with respect to the size of the struct from which it is constructed, in the same way that the size of a clause normal form of a formula can be exponential with respect to the size of the formula itself. There exists, however, an improved technique known as \emph{structural clause form transformation} \cite{BaazEglyLeitsch2001NormalFormTransformations}, based on the extension principle. By using this technique, it can be shown that the atomic size of the clause normal form of a formula is in the worst case only linearly bigger\footnote{However, the symbolic size can be quadratic due to Skolem terms produced by skolemization.} than the size of the formula itself. The price paid is that the structural conjunctive normal form of a formula is not logically equivalent to the formula anymore, because new defined predicate symbols are added, thus extending the signature. Nevertheless, satisfiability-equivalence is preserved: the formula is unsatisfiable if and only if its structural clause form is unsatisfiable.

Although profile clause sets and swapped clause sets are great improvements of the standard clause set, it is not so hard to see that, in the worst case, the size of these clause sets is still exponential with respect to the size of the struct, because distributive duplications still occur (in cases corresponding to swapping of indirectly dependent inferences). It is therefore only natural to investigate the possibility of adapting the idea of structural clause form transformation to the construction of clause sets from structs, in order to avoid the exponential blow-up in size in the worst case\footnote{Since structs do not contain quantifiers, no skolemization is necessary. Therefore, by adapting the structural clause form transformation technique to structs, not only the atomic size of the clause set remains linearly bounded with respect to the atomic size of the struct but also its symbolic size remains linearly bounded with respect to the symbolic size of the struct.}. The purpose of this section is to show how this can be done.


\subsection{Cut-Pertinent Definitional Clause Set}

Definition \ref{definition:NormalizationPlusTimesDefinitional} adapts to structs the idea of structural conjunctive normal form transformation. Every substruct is given a new name, a new predicate symbol defined to be equivalent to the substruct. The defining formulas are very shallow formulas and can be easily transformed to $\structplus$-junctions of $\structtimes$-junctions. 


\newcommand{\normalizePlusTimesD}{\leadsto_{\structplus\structtimes_{D}}}
\begin{definition}[$\normalizePlusTimesD$]
\label{definition:NormalizationPlusTimesDefinitional}
\index{Struct Normalization!Definitional}
%\newcommand{\hF}[1]{{\color{brickred} #1}}

Let $S$ be a struct. For every non-literal substruct $S' \equiv S'_1 \structtimes \ldots \structtimes S'_n$ or $S' \equiv S'_1 \structplus \ldots \structplus S'_n$ of $S$, a new predicate symbol can be created together with a corresponding defining formula:
%
$$
\mathit{Def}_{S'} \ \ \defEq \ \  N_{S'}(x_1,\ldots, x_m) \biimp N_{n(S'_1) \structtimes \ldots \structtimes n(S'_n)}(x_1,\ldots, x_m) \biimp n(S'_1) \structtimes \ldots \structtimes n(S'_n)
$$
%
or 
%
$$
\mathit{Def}_{S'} \ \ \defEq \ \ N_{S'}(x_1,\ldots, x_m) \biimp N_{n(S'_1) \structplus \ldots \structplus n(S'_n)}(x_1,\ldots, x_m) \biimp n(S'_1) \structplus \ldots \structplus n(S'_n)
$$
%
where $x_1,\ldots, x_m$ are the free variables of $S'$, $n(S'_k)$ is $S'_k$ if $S'_k$ is a literal struct and $N_{S'_k}(y_1,\ldots, y_j)$ if $S'_k$ is a non-literal struct with free variables $y_1,\ldots, y_j$. The connective $\biimp$ is considered to be just an abbreviation\footnote{The abbreviation can be intuitively understood due to the analogy of $\structtimes$ with $\vee$ and $\structplus$ with $\wedge$.}:
$$
A \biimp B_1 \structtimes \ldots \structtimes B_n \defEq 
(\overline{A} \structtimes B_1 \structtimes \ldots \structtimes B_n) \structplus 
(\overline{B_1} \structtimes A) \structplus \ldots \structplus 
(\overline{B_n} \structtimes A)
$$
$$
A \biimp B_1 \structplus \ldots \structplus B_n \defEq 
(\overline{B_1} \structtimes \ldots \structtimes \overline{B_n} \structtimes A ) \structplus 
(\overline{A} \structtimes B_1) \structplus \ldots \structplus 
(\overline{A} \structtimes B_n)
$$
where $\overline{C}$ is $\structdual D$, if $C = D$, and $D$, if $C = \structdual D$.

Then:

$$
S \normalizePlusTimesD S^*
$$

where:

$$
S^* \defEq n(S) \structplus \structplusBig_{\scriptscriptstyle \textrm{non-literal substructs } S'\textrm{ of }S } \mathit{Def}_{S'}
$$

Each defining formula $\mathit{Def}_{S'}$ originates so-called \emph{definitional $\structtimes$-junctions}. All other $\structtimes$-junctions (e.g. $n(S)$) are called \emph{proper $\structtimes$-junctions}.
\end{definition}


\begin{example}[$\structplus\structtimes_{D}$-Normalization]
\label{example:PlusTimesDefinitionalNormalization}


Let $\varphi$ be the proof below:

\begin{prooftree}
\AXC{$\hA{A} \seq \hA{A} $}
		\AXC{$\hB{B} \seq \hB{B}$} \RightLabel{$\wedge_r^1$}
	\BIC{$\hA{A}, \hB{B} \seq \hA{A} \wedge \hB{B}$} \RightLabel{$\wedge_l$}
	\UIC{$\hA{A} \wedge \hB{B} \seq \hA{A} \wedge \hB{B}$}
				\AXC{$\hC{B} \seq \hC{B} $}
						\AXC{$\hD{A} \seq \hD{A}$} \RightLabel{$\wedge_r^2$}
					\BIC{$\hD{A}, \hC{B} \seq \hC{B} \wedge \hD{A}$} \RightLabel{$\wedge_l$}
					\UIC{$\hD{A} \wedge \hC{B} \seq \hC{B} \wedge \hD{A}$} \RightLabel{$cut^3$}
			\BIC{$\hA{A} \wedge \hB{B} \seq \hC{B} \wedge \hD{A}$}
							\AXC{$\hE{C} \seq \hE{C}$} 
									\AXC{$\hF{C} \seq \hF{C}$} \RightLabel{$cut^4$}
								\BIC{$\hE{C} \seq \hF{C}$} \RightLabel{$\vee_l^5$}
					\BIC{$(\hA{A} \wedge \hB{B}) \vee \hE{C} \seq \hC{B} \wedge \hD{A}, \hF{C} $} 
\end{prooftree}

Its cut-pertinent struct is:

$$
\struct{\varphi} 
\equiv 
((\hA{A} \structplus \hB{B}) \structplus (\hC{\structdual{B}} \structtimes \hD{\structdual{A}}))
\structtimes
(\hE{C} \structplus \hF{\structdual{C}})
$$

New predicate symbols can be created and defined by the following formulas: 

\begin{itemize}
\item $D \biimp \hE{C} \structplus \hF{\structdual{C}}$

\item $E \biimp \hC{\structdual{B}} \structtimes \hD{\structdual{A}}$

\item $F \biimp \hA{A} \structplus \hB{B}$

\item $G \biimp F \structplus E$

\item $H \biimp G \structtimes D$
\end{itemize}

Finally, the $\normalizePlusTimesD$-normal-form of $\struct{\varphi}$ is:

$$
\begin{array}{rcl}
S^*	
& \defEq & H \structplus \\
&			& 
(\structdual D \structtimes \hE{C})
\structplus
(\structdual D \structtimes \structdual \hF{C})
\structplus
( \structdual \hE{C} \structtimes D \structtimes \hF{C}) \structplus \\
&			&
(\structdual E \structtimes \structdual \hC{B} \structtimes \structdual \hD{A} )
\structplus
(E \structtimes \hC{B})
\structplus
(E \structtimes \hD{A}) \structplus \\
&			&
(\structdual F \structtimes \hA{A})
\structplus
(\structdual F \structtimes \hB{B})
\structplus
(\structdual \hA{A} \structtimes \structdual \hB{B} \structtimes F) \structplus \\
&			&
(\structdual G \structtimes F)
\structplus
(\structdual G \structtimes E)
\structplus
(\structdual E \structtimes \structdual F \structtimes G) \structplus \\
&			&
(\structdual H \structtimes G \structtimes D)
\structplus
(\structdual G \structtimes H)
\structplus
(\structdual D \structtimes H) \\
\end{array}
$$

Then, its definitional clause set $\clausesetDefinitional{\varphi}$ consists of the following clauses. The proper clause is $\seq H$. All other clauses are definitional clauses.

\begin{multicols}{3}{
{
$D \seq \hE{C}$

$E, \hC{B}, \hD{A} \seq $

$F \seq \hA{A}$

$G \seq F$

$H \seq G, D$

$\seq H$
}

{
$D, \hF{C} \seq $

$\seq E, \hC{B}$

$F \seq \hB{B}$

$G \seq E$

$G \seq H$

$\phantom{\seq H}$
}

{
$ \hE{C} \seq D, \hF{C}$

$\seq E, \hD{A}$

$\hA{A}, \hB{B} \seq F$

$E, F \seq G$

$D \seq H$

$\phantom{\seq H}$
}
}\end{multicols}

\end{example}

\subsection{Projections}

The construction of projections requires special care when definitional clause sets are used. The reason is that the clauses now contain many new predicate symbols which do not occur in the proof. Since S-projections and O-projections contain only symbols that occur in the proof, it is clear that they cannot be used with definitional clause sets. New kinds of projections, called \emph{D-projections} have to be developed.


\subsubsection{D-Projections}

For all definitional clauses of a definitional clause set, projections can be constructed very easily by using definition rules, even without any dependence on the proof. These projections are the \emph{definitional D-projections} explained in Definition \ref{definition:DProjectionDefinitional}. However, in every definitional clause set there is exactly one clause, namely the proper clause, for which definitional D-projections do not work. Then a \emph{proper D-projection} (Definition \ref{definition:DProjectionProper}) is necessary. It is called proper, because it actually depends on the proof.



\begin{definition}[Definitional D-Projection]
\label{definition:DProjectionDefinitional}
\index{Projection!Definitional D-Projection}
Let $\varphi$ be a proof and $c$ a definitional clause in $\clausesetDefinitional{\varphi}$. The \emph{Definitional D-projection} $\projectionDDefinitional{\varphi}{c}$ with respect to the clause $c$ can be easily constructed by using definition rules, as exemplified below:

Assume $c$ is one of the definitional clauses originating from the following defining formula:

$$
\mathit{Def}_{S'} \ \ \defEq \ \  N_{n(S'_1) \structtimes \ldots \structtimes n(S'_n)}(x_1,\ldots, x_m) \biimp n(S'_1) \structtimes \ldots \structtimes n(S'_n)
$$

Then $c$ is one of the following clauses:
\begin{itemize}
\item $N_{n(S'_1) \structtimes \ldots \structtimes n(S'_n)}(x_1,\ldots, x_m) \seq  n(S'_1), \ldots , n(S'_n)$
\item $n(S'_1) \seq N_{n(S'_1) \structtimes \ldots \structtimes n(S'_n)}(x_1,\ldots, x_m)$
\item \ldots
\item $n(S'_n) \seq N_{n(S'_1) \structtimes \ldots \structtimes n(S'_n)}(x_1,\ldots, x_m)$
\end{itemize}

And the definitional D-projections are:

\begin{small}
\begin{multicols}{2}{
$\projectionDDefinitional{\varphi}{N_{S'_1 \structtimes \ldots \structtimes S'_n}(x_1,\ldots, x_m) \seq  n(S'_1), \ldots , n(S'_n)}$:
\begin{prooftree}
\AXC{$n(S'_1) \seq n(S'_1)$}
	\AXC{$\ldots$}
		\AXC{$n(S'_n) \seq n(S'_n)$} \doubleLine \RightLabel{$\vee^*_l$}
	\TIC{$n(S'_1) \vee \ldots \vee n(S'_n) \seq n(S'_1), \ldots , n(S'_n)$} \RightLabel{$d_l$}
	\UIC{$N_{n(S'_1) \structtimes \ldots \structtimes n(S'_n)}(x_1,\ldots, x_m) \seq  n(S'_1), \ldots , n(S'_n)$} 
\end{prooftree}


$\projectionDDefinitional{\varphi}{n(S'_k) \seq N_{S'_1 \structtimes \ldots \structtimes S'_n}(x_1,\ldots, x_m)}$:
\begin{prooftree}
\AXC{$n(S'_k) \seq n(S'_k)$} \doubleLine \RightLabel{$w^*_r$}
\UIC{$n(S'_k) \seq n(S'_1), \ldots, n(S'_n)$} \doubleLine \RightLabel{$\vee^*_r$}
\UIC{$n(S'_k) \seq n(S'_1) \vee \ldots \vee n(S'_n)$} \RightLabel{$d_r$}
\UIC{$n(S'_k) \seq N_{n(S'_1) \structtimes \ldots \structtimes n(S'_n)}(x_1,\ldots, x_m)$} 
\end{prooftree}
}\end{multicols}
\end{small}


If $c$, on the other hand is one of the definitional clauses originating from the following defining formula:

$$
\mathit{Def}_{S'} \ \ \defEq \ \ N_{n(S'_1) \structplus \ldots \structplus n(S'_n)}(x_1,\ldots, x_m) \biimp n(S'_1) \structplus \ldots \structplus n(S'_n)
$$

Then $c$ is one of the following clauses:
\begin{itemize}
\item $n(S'_1), \ldots , n(S'_n) \seq N_{n(S'_1) \structplus \ldots \structplus n(S'_n)}(x_1,\ldots, x_m)$
\item $N_{n(S'_1) \structplus \ldots \structplus n(S'_n)}(x_1,\ldots, x_m) \seq n(S'_1)$
\item \ldots
\item $N_{n(S'_1) \structplus \ldots \structplus n(S'_n)}(x_1,\ldots, x_m) \seq n(S'_n)$
\end{itemize}

And the definitional D-projections are:

\begin{small}
\begin{multicols}{2}{
$\projectionDDefinitional{\varphi}{N_{S'_1 \structtimes \ldots \structtimes S'_n}(x_1,\ldots, x_m) \seq  n(S'_1), \ldots , n(S'_n)}$:
\begin{prooftree}
\AXC{$n(S'_1) \seq n(S'_1)$}
	\AXC{$\ldots$}
		\AXC{$n(S'_n) \seq n(S'_n)$} \doubleLine \RightLabel{$\wedge^*_r$}
	\TIC{$n(S'_1), \ldots , n(S'_n) \seq n(S'_1) \wedge \ldots \wedge n(S'_n)$} \RightLabel{$d_r$}
	\UIC{$n(S'_1), \ldots , n(S'_n) \seq  N_{n(S'_1) \structplus \ldots \structplus n(S'_n)}(x_1,\ldots, x_m)$} 
\end{prooftree}


$\projectionDDefinitional{\varphi}{n(S'_k) \seq N_{S'_1 \structtimes \ldots \structtimes S'_n}(x_1,\ldots, x_m)}$:
\begin{prooftree}
\AXC{$n(S'_k) \seq n(S'_k)$} \doubleLine \RightLabel{$w^*_l$}
\UIC{$n(S'_1), \ldots, n(S'_n)\seq n(S'_k)$} \doubleLine \RightLabel{$\wedge^*_l$}
\UIC{$n(S'_1) \wedge \ldots \wedge n(S'_n) \seq n(S'_k)$} \RightLabel{$d_l$}
\UIC{$N_{n(S'_1) \structplus \ldots \structplus n(S'_n)}(x_1,\ldots, x_m) \seq n(S'_k)$} 
\end{prooftree}
}\end{multicols}
\end{small}

If $S'_k$ is a negative literal, it is necessary to add negation inferences to the definitional D-projections above.
\end{definition}

\begin{example}[Definitional D-Projection]
\label{example:DProjectionDefinitional}


The simple D-projections are:


\begin{multicols}{3}{
$\projectionDDefinitional{\varphi}{D \seq \hE{C}}$:
\begin{prooftree}
\AXC{$\hE{C} \seq \hE{C}$} \RightLabel{$w_l$}
\UIC{$\hE{C}, \neg \hF{C} \seq \hE{C}$} \RightLabel{$\wedge_l$}
\UIC{$\hE{C} \wedge \neg \hF{C} \seq \hE{C}$} \RightLabel{$d_l$}
\UIC{$D \seq \hE{C}$} 
\end{prooftree}

$\projectionDDefinitional{\varphi}{D, \hF{C} \seq }$:
\begin{prooftree}
\AXC{$\hF{C} \seq \hF{C}$} \RightLabel{$w_l$}
\UIC{$\hE{C}, \hF{C} \seq \hF{C}$} \RightLabel{$\neg_l$}
\UIC{$\hE{C}, \neg \hF{C}, \hF{C} \seq $} \RightLabel{$\wedge_l$}
\UIC{$\hE{C} \wedge \neg \hF{C}, \hF{C} \seq $} \RightLabel{$d_l$}
\UIC{$D, \hF{C} \seq $} 
\end{prooftree}

$\projectionDDefinitional{\varphi}{\hE{C} \seq D, \hF{C}}$:
\begin{prooftree}
\AXC{$\hE{C} \seq \hE{C}$}
		\AXC{$\hF{C} \seq \hF{C}$} \RightLabel{$\neg_r$}
		\UIC{$ \seq  \hF{C}, \neg \hF{C}$} \RightLabel{$\wedge_r$}
	\BIC{$\hE{C} \seq \hF{C}, \hE{C} \wedge \neg \hF{C}$} \RightLabel{$d_r$}
	\UIC{$\hE{C} \seq \hF{C}, D$} 
\end{prooftree}
}
\end{multicols}


\begin{multicols}{3}{
$\projectionDDefinitional{\varphi}{\seq E, \hD{A}}$:
\begin{prooftree}
\AXC{$\hD{A} \seq \hD{A}$} \RightLabel{$\neg_r$}
\UIC{$\seq \neg \hD{A}, \hD{A}$} \RightLabel{$w_r$}
\UIC{$\seq \neg \hC{B}, \neg \hD{A}, \hD{A}$} \RightLabel{$\vee_r$}
\UIC{$\seq \neg \hC{B} \vee \neg \hD{A}, \hD{A}$} \RightLabel{$d_r$}
\UIC{$\seq E, \hD{A}$} 
\end{prooftree}

$\projectionDDefinitional{\varphi}{\seq E, \hC{B}}$:
\begin{prooftree}
\AXC{$\hC{B} \seq \hC{B}$} \RightLabel{$\neg_r$}
\UIC{$\seq \neg \hC{B}, \hC{B}$} \RightLabel{$w_r$}
\UIC{$\seq \neg \hC{B}, \neg \hD{A}, \hC{B}$} \RightLabel{$\vee_r$}
\UIC{$\seq \neg \hC{B} \vee \neg \hD{A}, \hC{B}$} \RightLabel{$d_r$}
\UIC{$\seq E, \hC{B}$} 
\end{prooftree}

$\projectionDDefinitional{\varphi}{E, \hD{A}, \hC{B} \seq }$:
\begin{small}
\begin{prooftree}
\AXC{$\hC{B} \seq \hC{B}$} \RightLabel{$\neg_l$}
\UIC{$\neg \hC{B}, \hC{B} \seq $}
		\AXC{$\hD{A} \seq \hD{A}$} \RightLabel{$\neg_l$}
		\UIC{$\neg \hD{A}, \hD{A} \seq $} \RightLabel{$\vee_l$}
	\BIC{$\neg \hC{B} \vee \neg \hD{A}, \hC{B}, \hD{A} \seq $} \RightLabel{$d_l$}
	\UIC{$E, \hC{B}, \hD{A} \seq $} 
\end{prooftree}
\end{small}
}
\end{multicols}




\begin{multicols}{3}{
$\projectionDDefinitional{\varphi}{F \seq \hA{A}}$:
\begin{prooftree}
\AXC{$\hA{A} \seq \hA{A}$} \RightLabel{$w_l$}
\UIC{$\hA{A}, \hB{B} \seq \hA{A}$} \RightLabel{$\wedge_l$}
\UIC{$\hA{A} \wedge \hB{B} \seq \hA{A}$} \RightLabel{$d_l$}
\UIC{$F \seq \hA{A}$} 
\end{prooftree}

$\projectionDDefinitional{\varphi}{F \seq \hB{B}}$:
\begin{prooftree}
\AXC{$\hB{B} \seq \hB{B}$} \RightLabel{$w_l$}
\UIC{$\hA{A}, \hB{B} \seq \hB{B}$} \RightLabel{$\wedge_l$}
\UIC{$\hA{A} \wedge \hB{B} \seq \hB{B}$} \RightLabel{$d_l$}
\UIC{$F \seq \hB{B}$} 
\end{prooftree}

$\projectionDDefinitional{\varphi}{\hA{A}, \hB{B} \seq F}$:
\begin{prooftree}
\AXC{$\hA{A} \seq \hA{A}$}
		\AXC{$\hB{B} \seq \hB{B}$} \RightLabel{$\wedge_r$}
	\BIC{$\hA{A}, \hB{B} \seq \hA{A} \wedge \hB{B}$} \RightLabel{$d_r$}
	\UIC{$\hA{A}, \hB{B} \seq F$} 
\end{prooftree}
}
\end{multicols}





\begin{multicols}{3}{
$\projectionDDefinitional{\varphi}{G \seq F}$:
\begin{prooftree}
\AXC{$F \seq F$} \RightLabel{$w_l$}
\UIC{$F, E \seq F$} \RightLabel{$\wedge_l$}
\UIC{$F \wedge E \seq F$} \RightLabel{$d_l$}
\UIC{$G \seq F$} 
\end{prooftree}

$\projectionDDefinitional{\varphi}{G \seq E}$:
\begin{prooftree}
\AXC{$E \seq E$} \RightLabel{$w_l$}
\UIC{$F, E \seq E$} \RightLabel{$\wedge_l$}
\UIC{$F \wedge E \seq E$} \RightLabel{$d_l$}
\UIC{$G \seq E$} 
\end{prooftree}

$\projectionDDefinitional{\varphi}{F, E \seq G}$:
\begin{prooftree}
\AXC{$F \seq F$}
		\AXC{$E \seq E$} \RightLabel{$\wedge_r$}
	\BIC{$F, E \seq F \wedge E$} \RightLabel{$d_r$}
	\UIC{$F, E \seq G$} 
\end{prooftree}
}
\end{multicols}


\begin{multicols}{3}{
$\projectionDDefinitional{\varphi}{D \seq H}$:
\begin{prooftree}
\AXC{$D \seq D$} \RightLabel{$w_r$}
\UIC{$D \seq G, D$} \RightLabel{$\vee_r$}
\UIC{$D \seq G\vee D$} \RightLabel{$d_r$}
\UIC{$D \seq H$} 
\end{prooftree}

$\projectionDDefinitional{\varphi}{G \seq H}$:
\begin{prooftree}
\AXC{$G \seq G$} \RightLabel{$w_r$}
\UIC{$G \seq G, D$} \RightLabel{$\vee_r$}
\UIC{$G \seq G\vee D$} \RightLabel{$d_r$}
\UIC{$G \seq H$} 
\end{prooftree}

$\projectionDDefinitional{\varphi}{H \seq G, D}$:
\begin{prooftree}
\AXC{$G \seq G$}
		\AXC{$D \seq D$} \RightLabel{$\vee_l$}
	\BIC{$G \vee D \seq G, D$} \RightLabel{$d_l$}
	\UIC{$H \seq G, D$} 
\end{prooftree}
}
\end{multicols}

\end{example}



\begin{definition}[Proper D-Projection]
\label{definition:DProjectionProper}
\index{Projection!Proper D-Projection}

Let $\varphi$ be a proof and $\struct{\varphi}$ its cut-pertinent struct. Then, the \emph{proper D-projection} $\projectionDProper{\varphi}{\seq n(\struct{\varphi})}$ can be constructed inductively. Let $\varphi'$ be a subproof of $\varphi$ having $\rho$ as its last inference and let $S'$ be the corresponding substruct of $\struct{\varphi}$. The following cases can be distinguished:

\begin{itemize}
\item $\rho$ is an axiom inference: Then $\varphi'$ is of the form:

\begin{prooftree}
\AXC{$ $} \RightLabel{$\rho$}
\UIC{$A \seq A$}
\end{prooftree}

	\begin{itemize}
	\item If both occurrences of $A$ are in $\occCutPert{\varphi}$ (i.e. they are ancestors of  cut-formulas), then $\varphi''$ is defined as:

	\begin{prooftree}
	\AXC{$ $} \RightLabel{$\rho$}
	\UIC{$A \seq A$} \RightLabel{$\neg_r$}
	\UIC{$\seq \neg A, A$} \RightLabel{$\vee_r$}
	\UIC{$\seq \neg A \vee A$} \RightLabel{$d_r$}
	\UIC{$\seq n(S')$}
	\end{prooftree}

	\item If only the occurrence of $A$ in the antecedent is in $\occCutPert{\varphi}$ (i.e. an ancestor of a cut-formula), then $\varphi''$ is defined as:

	\begin{prooftree}
	\AXC{$ $} \RightLabel{$\rho$}
	\UIC{$A \seq A$} \RightLabel{$\neg_r$}
	\UIC{$\seq \neg A, A$}
	\end{prooftree}

 
	\item Otherwise, $\varphi'' \defEq \varphi'$
	\end{itemize}

\item $\rho$ is a n-ary inference (with $n \geq 2$): Then $\varphi'$ is of the form:

\begin{prooftree}
\AXC{$\psi'_1 $} \noLine
\UIC{$\Gamma'_1 \seq \Delta'_1$}
	\AXC{$\ldots$}
		\AXC{$\psi'_n $} \noLine
		\UIC{$\Gamma'_n \seq \Delta'_n$}		\RightLabel{$\rho$}
	\TIC{$\Gamma' \seq \Delta'$}
\end{prooftree}

By induction, $\psi''_k$ is of the form:

\begin{prooftree}
\AXC{$\psi''_k $} \noLine
\UIC{$\Gamma''_1 \seq \Delta''_1, n(S'_{\psi'_k})$}
\end{prooftree}

where $S'_{\psi'_k}$ is the substruct of $S'$ corresponding to $\psi'_k$.

	\begin{itemize}
	\item $\rho$ is cut-impertinent: Then $\varphi''$ is defined as:

\begin{prooftree}
\AXC{$\psi''_1 $} \noLine
\UIC{$\Gamma''_1 \seq \Delta''_1, n(S'_{\psi'_1})$}
	\AXC{$\ldots$}
		\AXC{$\psi''_n $} \noLine
		\UIC{$\Gamma''_n \seq \Delta''_n, n(S'_{\psi'_n})$}		\RightLabel{$\rho$}
	\TIC{$\Gamma'' \seq \Delta'', n(S'_{\psi'_1}), \ldots, n(S'_{\psi'_n})$} \RightLabel{$\vee_r$}
	\UIC{$\Gamma'' \seq \Delta'', n(S'_{\psi'_1}) \vee \ldots \vee n(S'_{\psi'_n})$}\RightLabel{$d_r$}
	\UIC{$\Gamma'' \seq \Delta'', n(S')$}
\end{prooftree}

	More informally, after the cut-impertinent inference $\rho$, the defining components of $n(S')$ are available to be combined disjunctively. By the defining formula of $n(S')$, a $d_r$ inference can be used to encapsulate the disjunction in the single defined predicate symbol $n(S')$. 

	\item $\rho$ is cut-pertinent: Then $\varphi''$ is defined as:

\begin{prooftree}
\AXC{$\psi''_1 $} \noLine
\UIC{$\Gamma''_1 \seq \Delta''_1, n(S'_{\psi'_1})$}
	\AXC{$\ldots$}
		\AXC{$\psi''_n $} \noLine
		\UIC{$\Gamma''_n \seq \Delta''_n, n(S'_{\psi'_n})$}		\RightLabel{$\wedge_r$}
	\TIC{$\Gamma'' \seq \Delta'', n(S'_{\psi'_1}) \wedge \ldots \wedge n(S'_{\psi'_n})$} \RightLabel{$d_r$}
	\UIC{$\Gamma'' \seq \Delta'', n(S')$}
\end{prooftree}

	More informally, the cut-pertinent inference $\rho$ can be replaced by a $\wedge_r$ inference, which combines the defining components of $n(S')$ conjunctively. By the defining formula of $n(S')$, a $d_r$ inference can be used to encapsulate the conjunction in the single defined predicate symbol $n(S')$. 


	\end{itemize}


\item $\rho$ is a unary inference: Then $\varphi'$ is of the form:

\begin{prooftree}
\AXC{$\psi' $} \RightLabel{$\rho$}
\UIC{$\Gamma' \seq \Delta'$}
\end{prooftree}

	\begin{itemize}

	\item $\rho$ is cut-pertinent: then $\varphi''$ is defined as:

\begin{prooftree}
\AXC{$\psi'' $} 
\end{prooftree}

	More informally, $\rho$ is simply skipped.

	\item $\rho$ is cut-impertinent: then $\varphi''$ is defined as:

\begin{prooftree}
\AXC{$\psi'' $} \RightLabel{$\rho$}
\UIC{$\Gamma'' \seq \Delta''$}
\end{prooftree}

	More informally, $\rho$ is simply kept and nothing changes, except for the downward propagation of changes that occurred in transforming the proof $\psi'$ above to $\psi''$.

	\end{itemize}
	



\end{itemize}

The \emph{proper D-projection} $\projectionDProper{\varphi}{\seq n(\struct{\varphi})}$ is the final result of this inductive construction, i.e. it is $\varphi''$ when the subproof $\varphi'$ coincides with the whole proof $\varphi$.
\end{definition}



\begin{example}[Proper D-Projection]
\label{example:DProjectionProper}


Consider again the proof $\varphi$ from previous examples:

\begin{prooftree}
\AXC{$\hA{A} \seq \hA{A} $}
		\AXC{$\hB{B} \seq \hB{B}$} \RightLabel{$\wedge_r^1$}
	\BIC{$\hA{A}, \hB{B} \seq \hA{A} \wedge \hB{B}$} \RightLabel{$\wedge_l$}
	\UIC{$\hA{A} \wedge \hB{B} \seq \hA{A} \wedge \hB{B}$}
				\AXC{$\hC{B} \seq \hC{B} $}
						\AXC{$\hD{A} \seq \hD{A}$} \RightLabel{$\wedge_r^2$}
					\BIC{$\hD{A}, \hC{B} \seq \hC{B} \wedge \hD{A}$} \RightLabel{$\wedge_l$}
					\UIC{$\hD{A} \wedge \hC{B} \seq \hC{B} \wedge \hD{A}$} \RightLabel{$cut^3$}
			\BIC{$\hA{A} \wedge \hB{B} \seq \hC{B} \wedge \hD{A}$}
							\AXC{$\hE{C} \seq \hE{C}$} 
									\AXC{$\hF{C} \seq \hF{C}$} \RightLabel{$cut^4$}
								\BIC{$\hE{C} \seq \hF{C}$} \RightLabel{$\vee_l^5$}
					\BIC{$(\hA{A} \wedge \hB{B}) \vee \hE{C} \seq \hC{B} \wedge \hD{A}, \hF{C} $} 
\end{prooftree}

Below the inductive construction of the proper D-projection $\projectionDDefinitional{\varphi}{ \seq H}$ is shown step-by-step. An informal skeleton of the original proof is shown in every step, just to emphasize that the construction follows the structure of the original proof.

\begin{prooftree}
\AXC{$\hA{A} \seq \hA{A} $}
		\AXC{$\hB{B} \seq \hB{B}$} \RightLabel{$\wedge_r^1$}
	\BIC{$\phantom{\hA{A}, \hB{B} \seq \hA{A} \wedge \hB{B}}$} \RightLabel{$\wedge_l$}
	\UIC{$\phantom{\hA{A} \wedge \hB{B} \seq \hA{A} \wedge \hB{B}}$}
				\AXC{$\hC{B} \seq \hC{B} $}
						\AXC{$\hD{A} \seq \hD{A}$} \RightLabel{$\wedge_r^2$}
					\BIC{$\phantom{\hD{A}, \hC{B} \seq \hC{B} \wedge \hD{A}}$} \RightLabel{$\wedge_l$}
					\UIC{$\phantom{\hD{A} \wedge \hC{B} \seq \hC{B} \wedge \hD{A}}$} \RightLabel{$cut^3$}
			\BIC{$\phantom{\hA{A} \wedge \hB{B} \seq \hC{B} \wedge \hD{A}}$}
							\AXC{$\hE{C} \seq \hE{C}$} 
									\AXC{$\hF{C} \seq \hF{C}$} \RightLabel{$cut^4$}
								\BIC{$\phantom{\hE{C} \seq \hF{C}}$} \RightLabel{$\vee_l^5$}
					\BIC{$\phantom{(\hA{A} \wedge \hB{B}) \vee \hE{C} \seq \hC{B} \wedge \hD{A}, \hF{C} }$} 
\end{prooftree}

Some of the axiom sequents contain cut-pertinent formula occurrences in the antecedents. It is necessary, therefore, to add $\neg_r$ inferences to move these formula occurrences to the consequents:

\begin{small}
\begin{prooftree}
\AXC{$\hA{A} \seq \hA{A} $}
		\AXC{$\hB{B} \seq \hB{B}$} \RightLabel{$\wedge_r^1$}
	\BIC{$\phantom{\hA{A}, \hB{B} \seq \hA{A} \wedge \hB{B}}$} \RightLabel{$\wedge_l$}
	\UIC{$\phantom{\hA{A} \wedge \hB{B} \seq \hA{A} \wedge \hB{B}}$}
				\AXC{$\hC{B} \seq \hC{B} $}\RightLabel{$\hG{\neg_r}$}
				\UIC{$ \seq \neg \hG{\hC{B}}, \hC{B} $}
						\AXC{$\hD{A} \seq \hD{A}$}\RightLabel{$\hG{\neg_r}$}
						\UIC{$ \seq \neg \hG{\hD{A}}, \hD{A}$}  \RightLabel{$\wedge_r^2$}
					\BIC{$\phantom{\hD{A}, \hC{B} \seq \hC{B} \wedge \hD{A}}$} \RightLabel{$\wedge_l$}
					\UIC{$\phantom{\hD{A} \wedge \hC{B} \seq \hC{B} \wedge \hD{A}}$} \RightLabel{$cut^3$}
			\BIC{$\phantom{\hA{A} \wedge \hB{B} \seq \hC{B} \wedge \hD{A}}$}
							\AXC{$\hE{C} \seq \hE{C}$} 
									\AXC{$\hF{C} \seq \hF{C}$} \RightLabel{$\hG{\neg_r}$}
									\UIC{$\seq \neg \hG{\hF{C}}, \hF{C}$}\RightLabel{$cut^4$}
								\BIC{$\phantom{\hE{C} \seq \hF{C}}$} \RightLabel{$\vee_l^5$}
					\BIC{$\phantom{(\hA{A} \wedge \hB{B}) \vee \hE{C} \seq \hC{B} \wedge \hD{A}, \hF{C} }$} 
\end{prooftree}
\end{small}

$\wedge_r^1$ and $cut^4$ are cut-pertinent inferences, and hence they must be replaced by $\wedge_r$ inferences followed by appropriate $d_r$ inferences. $\wedge_r^2$, on the other hand, is a cut-impertinent inference. Therefore, a $\vee_r$ inference and a $d_r$ inference must be added after $\wedge_r^2$:

\begin{prooftree}
\AXC{$\hA{A} \seq \hA{A} $}
		\AXC{$\hB{B} \seq \hB{B}$} \RightLabel{$\wedge$}
	\BIC{$\hA{A}, \hB{B} \seq \hA{A} \wedge \hB{B}$} \RightLabel{$\hG{d_r}$}
	\UIC{$\hA{A}, \hB{B} \seq \hG{F}$} \RightLabel{$\wedge_l$}
	\UIC{$\phantom{\hA{A} \wedge \hB{B} \seq \hA{A} \wedge \hB{B}}$}
				\AXC{$\hC{B} \seq \hC{B} $}\RightLabel{$\hG{\neg_r}$}
				\UIC{$ \seq \neg \hG{\hC{B}}, \hC{B} $}
						\AXC{$\hD{A} \seq \hD{A}$}\RightLabel{$\hG{\neg_r}$}
						\UIC{$ \seq \neg \hG{\hD{A}}, \hD{A}$}  \RightLabel{$\wedge_r^2$}
					\BIC{$ \seq \neg \hG{\hC{B}}, \neg \hG{\hD{A}}, \hC{B} \wedge \hD{A}$}\RightLabel{$\hG{\vee_r}$}
					\UIC{$ \seq \neg \hG{\hC{B}} \vee \neg \hG{\hD{A}}, \hC{B} \wedge \hD{A}$} \RightLabel{$\hG{d_r}$}
					\UIC{$ \seq \hG{E}, \hC{B} \wedge \hD{A}$} \RightLabel{$\wedge_l$}
					\UIC{$\phantom{\hD{A} \wedge \hC{B} \seq \hC{B} \wedge \hD{A}}$} \RightLabel{$cut^3$}
			\BIC{$\phantom{\hA{A} \wedge \hB{B} \seq \hC{B} \wedge \hD{A}}$}
							\AXC{$\hE{C} \seq \hE{C}$} 
									\AXC{$\hF{C} \seq \hF{C}$} \RightLabel{$\hG{\neg_r}$}
									\UIC{$\seq \neg \hG{\hF{C}}, \hF{C}$}\RightLabel{$\hG{\wedge_r}$}
								\BIC{$\hE{C} \seq \hE{C} \wedge \neg \hG{\hF{C}}, \hF{C}$} \RightLabel{$\hG{d_r}$} 
								\UIC{$\hE{C} \seq \hG{D}, \hF{C}$} \RightLabel{$\vee_l^5$}
					\BIC{$\phantom{(\hA{A} \wedge \hB{B}) \vee \hE{C} \seq \hC{B} \wedge \hD{A}, \hF{C} }$} 
\end{prooftree}

The leftmost $\wedge_l$ unary inference is cut-impertinent, and hence must be kept. The rightmost $\wedge_l$ unary inference, on the other hand, is cut-impertinent, and hence must be skipped.

\begin{prooftree}
\AXC{$\hA{A} \seq \hA{A} $}
		\AXC{$\hB{B} \seq \hB{B}$} \RightLabel{$\wedge$}
	\BIC{$\hA{A}, \hB{B} \seq \hA{A} \wedge \hB{B}$} \RightLabel{$\hG{d_r}$}
	\UIC{$\hA{A}, \hB{B} \seq \hG{F}$} \RightLabel{$\wedge_l$}
	\UIC{$\hA{A} \wedge \hB{B} \seq \hG{F}$}
				\AXC{$\hC{B} \seq \hC{B} $}\RightLabel{$\hG{\neg_r}$}
				\UIC{$ \seq \neg \hG{\hC{B}}, \hC{B} $}
						\AXC{$\hD{A} \seq \hD{A}$}\RightLabel{$\hG{\neg_r}$}
						\UIC{$ \seq \neg \hG{\hD{A}}, \hD{A}$}  \RightLabel{$\wedge_r^2$}
					\BIC{$ \seq \neg \hG{\hC{B}}, \neg \hG{\hD{A}}, \hC{B} \wedge \hD{A}$}\RightLabel{$\hG{\vee_r}$}
					\UIC{$ \seq \neg \hG{\hC{B}} \vee \neg \hG{\hD{A}}, \hC{B} \wedge \hD{A}$} \RightLabel{$\hG{d_r}$}
					\UIC{$ \seq \hG{E}, \hC{B} \wedge \hD{A}$} \RightLabel{$cut^3$}
			\BIC{$\phantom{\hA{A} \wedge \hB{B} \seq \hC{B} \wedge \hD{A}}$}
							\AXC{$\hE{C} \seq \hE{C}$} 
									\AXC{$\hF{C} \seq \hF{C}$} \RightLabel{$\hG{\neg_r}$}
									\UIC{$\seq \neg \hG{\hF{C}}, \hF{C}$}\RightLabel{$\hG{\wedge_r}$}
								\BIC{$\hE{C} \seq \hE{C} \wedge \neg \hG{\hF{C}}, \hF{C}$} \RightLabel{$\hG{d_r}$} 
								\UIC{$\hE{C} \seq \hG{D}, \hF{C}$} \RightLabel{$\vee_l^5$}
					\BIC{$\phantom{(\hA{A} \wedge \hB{B}) \vee \hE{C} \seq \hC{B} \wedge \hD{A}, \hF{C} }$} 
\end{prooftree}

The procedure for $cut^3$ is analogous. It must be replaced by $\wedge^r$ and $d^r$:

\begin{prooftree}
\AXC{$\hA{A} \seq \hA{A} $}
		\AXC{$\hB{B} \seq \hB{B}$} \RightLabel{$\wedge$}
	\BIC{$\hA{A}, \hB{B} \seq \hA{A} \wedge \hB{B}$} \RightLabel{$\hG{d_r}$}
	\UIC{$\hA{A}, \hB{B} \seq \hG{F}$} \RightLabel{$\wedge_l$}
	\UIC{$\hA{A} \wedge \hB{B} \seq \hG{F}$}
				\AXC{$\hC{B} \seq \hC{B} $}\RightLabel{$\hG{\neg_r}$}
				\UIC{$ \seq \neg \hG{\hC{B}}, \hC{B} $}
						\AXC{$\hD{A} \seq \hD{A}$}\RightLabel{$\hG{\neg_r}$}
						\UIC{$ \seq \neg \hG{\hD{A}}, \hD{A}$}  \RightLabel{$\wedge_r^2$}
					\BIC{$ \seq \neg \hG{\hC{B}}, \neg \hG{\hD{A}}, \hC{B} \wedge \hD{A}$}\RightLabel{$\hG{\vee_r}$}
					\UIC{$ \seq \neg \hG{\hC{B}} \vee \neg \hG{\hD{A}}, \hC{B} \wedge \hD{A}$} \RightLabel{$\hG{d_r}$}
					\UIC{$ \seq \hG{E}, \hC{B} \wedge \hD{A}$} \RightLabel{$\hG{\wedge_r}$}
			\BIC{$\hA{A} \wedge \hB{B} \seq \hG{F} \wedge \hG{E}, \hC{B} \wedge \hD{A}$} \RightLabel{$\hG{d_r}$}
			\UIC{$\hA{A} \wedge \hB{B} \seq \hG{G}, \hC{B} \wedge \hD{A}$}
							\AXC{$\hE{C} \seq \hE{C}$} 
									\AXC{$\hF{C} \seq \hF{C}$} \RightLabel{$\hG{\neg_r}$}
									\UIC{$\seq \neg \hG{\hF{C}}, \hF{C}$}\RightLabel{$\hG{\wedge_r}$}
								\BIC{$\hE{C} \seq \hE{C} \wedge \neg \hG{\hF{C}}, \hF{C}$} \RightLabel{$\hG{d_r}$} 
								\UIC{$\hE{C} \seq \hG{D}, \hF{C}$} \RightLabel{$\vee_l^5$}
					\BIC{$\phantom{(\hA{A} \wedge \hB{B}) \vee \hE{C} \seq \hC{B} \wedge \hD{A}, \hF{C} }$} 
\end{prooftree}

Finally, $\vee_r$ and $d_r$ are added after the cut-impertinent $\vee_l^5$ inference, thus resulting in the following proper D-projection:

\begin{prooftree}
\AXC{$\hA{A} \seq \hG{\hA{A}} $}
		\AXC{$\hB{B} \seq \hG{\hB{B}}$} \RightLabel{$\hG{\wedge_r}$}
	\BIC{$\hA{A}, \hB{B} \seq \hG{\hA{A}} \wedge \hG{\hB{B}}$} \RightLabel{$\hG{d_r}$}
	\UIC{$\hA{A}, \hB{B} \seq \hG{F}$} \RightLabel{$\wedge_l$}
	\UIC{$\hA{A} \wedge \hB{B} \seq \hG{F}$}
				\AXC{$\hG{\hC{B}} \seq \hC{B} $} \RightLabel{$\hG{\neg_r}$}
				\UIC{$ \seq \neg \hG{\hC{B}}, \hC{B} $}
						\AXC{$\hG{\hD{A}} \seq \hD{A}$} \RightLabel{$\hG{\neg_r}$}
						\UIC{$ \seq \neg \hG{\hD{A}}, \hD{A}$} \RightLabel{$\wedge_r$}
					\BIC{$ \seq \neg \hG{\hC{B}}, \neg \hG{\hD{A}}, \hC{B} \wedge \hD{A}$} \RightLabel{$\hG{\vee_r}$}
					\UIC{$ \seq \neg \hG{\hC{B}} \vee \neg \hG{\hD{A}}, \hC{B} \wedge \hD{A}$} \RightLabel{$\hG{d_r}$}
					\UIC{$ \seq \hG{E}, \hC{B} \wedge \hD{A}$} \RightLabel{$\hG{\wedge_r}$}
			\BIC{$\hA{A} \wedge \hB{B} \seq \hG{F} \wedge \hG{E}, \hC{B} \wedge \hD{A}$} \RightLabel{$\hG{d_r}$}
			\UIC{$\hA{A} \wedge \hB{B} \seq \hG{G}, \hC{B} \wedge \hD{A}$}
							\AXC{$\hE{C} \seq \hG{\hE{C}}$} 
									\AXC{$\hG{\hF{C}} \seq \hF{C}$} \RightLabel{$\hG{\neg_r}$}
									\UIC{$\seq \neg \hG{\hF{C}}, \hF{C}$}	 \RightLabel{$\hG{\wedge_r}$}
								\BIC{$\hE{C} \seq \hE{C} \wedge \neg \hG{\hF{C}}, \hF{C}$} \RightLabel{$\hG{d_r}$} 
								\UIC{$\hE{C} \seq \hG{D}, \hF{C}$} \RightLabel{$\vee_l$}
					\BIC{$(\hA{A} \wedge \hB{B}) \vee \hE{C} \seq \hG{G}, \hG{D}, \hC{B} \wedge \hD{A}, \hF{C}$} \RightLabel{$\hG{\vee_r}$} 
					\UIC{$(\hA{A} \wedge \hB{B}) \vee \hE{C} \seq \hG{G} \vee \hG{D}, \hC{B} \wedge \hD{A}, \hF{C}$} \RightLabel{$\hG{d_r}$}
					\UIC{$(\hA{A} \wedge \hB{B}) \vee \hE{C} \seq \hG{H}, \hC{B} \wedge \hD{A}, \hF{C}$}
\end{prooftree}

\end{example}


\begin{example}[$\CEResDD$-Normal-Form]


Consider the definitional clause set $\clausesetDefinitional{\varphi}$ of the proof $\varphi$ shown in Example \ref{example:DefinitionalClauseSet}:

\begin{multicols}{3}{
{
$D \seq \hE{C}$

$E, \hC{B}, \hD{A} \seq $

$F \seq \hA{A}$

$G \seq F$

$H \seq G, D$

$\seq H$
}

{
$D, \hF{C} \seq $

$\seq E, \hC{B}$

$F \seq \hB{B}$

$G \seq E$

$G \seq H$

$\phantom{\seq H}$
}

{
$ \hE{C} \seq D, \hF{C}$

$\seq E, \hD{A}$

$\hA{A}, \hB{B} \seq F$

$E, F \seq G$

$D \seq H$

$\phantom{\seq H}$
}
}\end{multicols}

The shortest refutation $\delta$ of $\clausesetDefinitional{\varphi}$ is shown below:

\begin{footnotesize}
\begin{prooftree}
\AXC{$\seq H$}
		\AXC{$H \seq G, D$}
				\AXC{$G\seq F$}
						\AXC{$G\seq E$}
								\AXC{$F \seq \hA{A}$}
										\AXC{$F \seq \hB{B}$}
												\AXC{$E, \hC{B}, \hD{A} \seq $} \RightLabel{$r$}
											\BIC{$F, E, \hD{A} \seq $} \RightLabel{$r$}
									\BIC{$F, F, E \seq $} \RightLabel{$f_l$}
									\UIC{$F, E \seq$} \RightLabel{$r$}
							\BIC{$G, F \seq$} \RightLabel{$r$}
					\BIC{$G, G \seq$} \RightLabel{$f_l$}
					\UIC{$G \seq$} \RightLabel{$r$}
			\BIC{$H \seq D$}
												\AXC{$D \seq \hE{C}$}
														\AXC{$D, \hF{C} \seq$} \RightLabel{$r$}
													\BIC{$D, D \seq$} \RightLabel{$f_l$}
													\UIC{$D \seq$} \RightLabel{$r$}
								\BIC{$H \seq$} \RightLabel{$r$}
	\BIC{$\seq$}
\end{prooftree}
\end{footnotesize}

By using the proper D-projection shown in Example \ref{example:DProjectionProper} and definitional D-projections shown in Example \ref{example:DProjectionDefinitional}, $\CEResNFDD{\varphi}{\delta}$ is:

\begin{tiny}
\begin{prooftree}
\AXC{$\hA{A} \seq \hG{\hA{A}} $}
		\AXC{$\hB{B} \seq \hG{\hB{B}}$} \RightLabel{$\hG{\wedge_r}$}
	\BIC{$\hA{A}, \hB{B} \seq \hG{\hA{A}} \wedge \hG{\hB{B}}$} \RightLabel{$\hG{d_r}$}
	\UIC{$\hA{A}, \hB{B} \seq \hG{F}$} \RightLabel{$\wedge_l$}
	\UIC{$\hA{A} \wedge \hB{B} \seq \hG{F}$}
				\AXC{$\hG{\hC{B}} \seq \hC{B} $} \RightLabel{$\hG{\neg_r}$}
				\UIC{$ \seq \neg \hG{\hC{B}}, \hC{B} $}
						\AXC{$\hG{\hD{A}} \seq \hD{A}$} \RightLabel{$\hG{\neg_r}$}
						\UIC{$ \seq \neg \hG{\hD{A}}, \hD{A}$} \RightLabel{$\wedge_r$}
					\BIC{$ \seq \neg \hG{\hC{B}}, \neg \hG{\hD{A}}, \hC{B} \wedge \hD{A}$} \RightLabel{$\hG{\vee_r}$}
					\UIC{$ \seq \neg \hG{\hC{B}} \vee \neg \hG{\hD{A}}, \hC{B} \wedge \hD{A}$} \RightLabel{$\hG{d_r}$}
					\UIC{$ \seq \hG{E}, \hC{B} \wedge \hD{A}$} \RightLabel{$\hG{\wedge_r}$}
			\BIC{$\hA{A} \wedge \hB{B} \seq \hG{F} \wedge \hG{E}, \hC{B} \wedge \hD{A}$} \RightLabel{$\hG{d_r}$}
			\UIC{$\hA{A} \wedge \hB{B} \seq \hG{G}, \hC{B} \wedge \hD{A}$}
							\AXC{$\hE{C} \seq \hG{\hE{C}}$} 
									\AXC{$\hG{\hF{C}} \seq \hF{C}$} \RightLabel{$\hG{\neg_r}$}
									\UIC{$\seq \neg \hG{\hF{C}}, \hF{C}$}	 \RightLabel{$\hG{\wedge_r}$}
								\BIC{$\hE{C} \seq \hE{C} \wedge \neg \hG{\hF{C}}, \hF{C}$} \RightLabel{$\hG{d_r}$} 
								\UIC{$\hE{C} \seq \hG{D}, \hF{C}$} \RightLabel{$\vee_l$}
					\BIC{$(\hA{A} \wedge \hB{B}) \vee \hE{C} \seq \hG{G}, \hG{D}, \hC{B} \wedge \hD{A}, \hF{C}$} \RightLabel{$\hG{\vee_r}$} 
					\UIC{$(\hA{A} \wedge \hB{B}) \vee \hE{C} \seq \hG{G} \vee \hG{D}, \hC{B} \wedge \hD{A}, \hF{C}$} \RightLabel{$\hG{d_r}$}
					\UIC{$(\hA{A} \wedge \hB{B}) \vee \hE{C} \seq \hG{H}, \hC{B} \wedge \hD{A}, \hF{C}$}
		\AXC{$\projectionDDefinitional{\varphi}{H \seq G, D}$}
				\AXC{$\projectionDDefinitional{\varphi}{G\seq F}$}
						\AXC{$\projectionDDefinitional{\varphi}{G\seq E}$}
								\AXC{$\projectionDDefinitional{\varphi}{F \seq \hA{A}}$}
										\AXC{$\projectionDDefinitional{\varphi}{F \seq \hB{B}}$}
												\AXC{$\projectionDDefinitional{\varphi}{E, \hC{B}, \hD{A} \seq }$} \RightLabel{$cut$}
											\BIC{$F, E, \hD{A} \seq $} \RightLabel{$cut$}
									\BIC{$F, F, E \seq $} \RightLabel{$c_l$}
									\UIC{$F, E \seq$} \RightLabel{$cut$}
							\BIC{$G, F \seq$} \RightLabel{$cut$}
					\BIC{$G, G \seq$} \RightLabel{$c_l$}
					\UIC{$G \seq$} \RightLabel{$cut$}
			\BIC{$H \seq D$}
												\AXC{$\projectionDDefinitional{\varphi}{D \seq \hE{C}}$}
														\AXC{$\projectionDDefinitional{\varphi}{D, \hF{C} \seq}$} \RightLabel{$cut$}
													\BIC{$D, D \seq$} \RightLabel{$c_l$}
													\UIC{$D \seq$} \RightLabel{$cut$}
								\BIC{$H \seq$} \RightLabel{$cut$}
	\BIC{$(\hA{A} \wedge \hB{B}) \vee \hE{C} \seq \hC{B} \wedge \hD{A}, \hF{C}$}
\end{prooftree}
\end{tiny}

\end{example}


\section{A Combined Approach}
\label{sec:Combination}

Although the number of defined symbols introduced by the construction of definitional clause sets is only linearly bounded with respect to the size of the structs, it is still far from optimal. A technique that combines ideas from swapped clause sets and from definitional clause sets can be used to significantly reduce this number. Once again, the difficulty lies in the projections. As in the case of definitional clause sets, a new notion of projection has to be developed.




\subsection{Cut-Pertinent Swapped Definitional Clause Set}

\newcommand{\normalizePlusTimesDW}{\leadsto_{\structplus\structtimes_{DW}}}
\newcommand{\normalizePlusTimesDWw}{\leadsto_{\structplus\structtimes_{DW_W}}}
\newcommand{\normalizePlusTimesDWd}{\leadsto_{\structplus\structtimes_{DW_D}}}

Swapped definitional clause sets are obtained by a straightforward combination of the normalizations used for swapped clause sets and for definitional clause sets. Initially, a restricted form of $\normalizePlusTimesS$-normalization (namely $\normalizePlusTimesDWw$) can be applied as long as no duplications of substructs occur. Subsequently, a limited form of $\normalizePlusTimesD$ (namely $\normalizePlusTimesDWd$) can be applied with the restriction that only substructs that are $\structplus$-junctions nested within $\structtimes$-junctions are replaced by new defined predicates.


\begin{definition}[$\normalizePlusTimesDW$]
\label{definition:NormalizationPlusTimesDefinitionalSwap}

In the struct rewriting rules below, let $\rho$ be the inference in $\varphi$ corresponding to $\hF{\structtimes_{\rho}}$. For the rewriting rules to be applicable, $\hF{S_k}$ and $\hF{S}$ must contain at least one occurrence from $\occInference{\varphi}{\rho}$ each (i.e. there is an atomic substruct $\hF{S'_{k}}$ of $\hF{S_{k}}$ such that $\hF{S'_{k}} \in \occInference{\varphi}{\rho}$)\footnote{
An atomic substruct is a formula occurrence. Therefore it makes sense to talk about pertinence of atomic substructs in $\occInference{\varphi}{\rho}$, even though it might look strange at first.}, and $S_1, \ldots, S_n$ and $S_l$ and $S_r$ should not contain any occurrence from $\occInference{\varphi}{\rho}$. Moreover, an innermost rewriting strategy is enforced: only minimal reducible substructs (i.e. structs having no reducible proper substruct) can be rewritten.

$$
\hF{S} \hF{\structtimes} (S_1 \structplus \ldots \structplus \hF{S_k} \structplus \ldots \structplus S_n) \normalizePlusTimesDWw  S_1 \structplus \ldots \structplus (\hF{S} \hF{\structtimes} \hF{S_{k}}) \structplus \ldots \structplus S_n
$$

$$
(S_1 \structplus \ldots \structplus \hF{S_k} \structplus \ldots \structplus S_n) \hF{\structtimes} \hF{S} \normalizePlusTimesDWw  S_1 \structplus \ldots \structplus S_n \structplus ( \hF{S_k} \hF{\structtimes} \hF{S}) \structplus \ldots \structplus S_n  
$$

\begin{multicols}{4}{

$$
\hF{S} \hF{\structtimes} S_r \normalizePlusTimesDWw  S_r
$$

$$
S_l \hF{\structtimes} \hF{S} \normalizePlusTimesDWw  S_l
$$

$$
\hF{S} \hF{\structplus} S_r \normalizePlusTimesDWw  S_r
$$

$$
S_l \hF{\structplus} \hF{S} \normalizePlusTimesDWw  S_l
$$

}\end{multicols}


\begin{multicols}{4}{

$$
S_l \hF{\structtimes} S_r \normalizePlusTimesDWw  S_l
$$

$$
S_l \hF{\structtimes} S_r \normalizePlusTimesDWw  S_r
$$

$$
S_l \hF{\structplus} S_r \normalizePlusTimesDWw  S_l
$$

$$
S_l \hF{\structplus} S_r \normalizePlusTimesDWw  S_r
$$

}\end{multicols}

In the struct rewriting rule below, $C[\phantom{S}]$ is a struct context (i.e. $C[S]$ indicates that the struct $S$ is a substruct of a struct $C[S]$). For the rewriting rule to be applicable, $S \equiv S_1 \structplus \ldots \structplus S_n$ must be a $\structtimes$-junct in $C[S]$. Moreover, an innermost rewriting strategy is enforced: $S \equiv S_1 \structplus \ldots \structplus S_n$ can be replaced by $N_{S}(x_1,\ldots, x_m)$ only if $S$ has no substruct $S'$ that is a $\structtimes$-junction of $\structplus$-junctions (if this were the case, then $S'$ must be replaced before).


$$
\begin{array}{rcl}
C[S]  & \equiv & C[S_1 \structplus \ldots \structplus S_n] \\ 
		& \normalizePlusTimesDWd & 
C[N_{S}(x_1,\ldots, x_m)] \structplus (N_{S}(x_1,\ldots, x_m) \biimp S'_1 \structplus \ldots \structplus S'_n)
\end{array}
$$

\medskip

The relation $\normalizePlusTimesDW$ is the composition of $\normalizePlusTimesDWw^*$ and $\normalizePlusTimesDWd^*$ (i.e. $S \normalizePlusTimesDW S^*$ if and only if there exists $S'$ such that $S \normalizePlusTimesDWw^* S'$ and $S' \normalizePlusTimesDWd^* S^*$).

The $\structtimes$-junctions of a struct in $\normalizePlusTimesDW$ can be classified in the following way:
\begin{itemize}
	\item If the $\structtimes$-junction originates from a defining equation, it is called a \emph{definitional $\structtimes$-junction}.

	\item Otherwise:
		\begin{itemize}
			\item If the $\structtimes$-junction does not contain new defined predicate symbols, it is called \emph{pure}.

			\item Otherwise, it is called \emph{mixed}.
		\end{itemize}

\end{itemize}

\end{definition}


\begin{example}[$\structplus\structtimes_{DW}$-Normalization]
\label{example:PlusTimesSwappedDefinitionalNormalization}
Let $\varphi$ be the proof below:

\begin{prooftree}
\AXC{$\hA{A} \seq \hA{A} $}
		\AXC{$\hB{B} \seq \hB{B}$} \RightLabel{$\wedge_r^1$}
	\BIC{$\hA{A}, \hB{B} \seq \hA{A} \wedge \hB{B}$} \RightLabel{$\wedge_l$}
	\UIC{$\hA{A} \wedge \hB{B} \seq \hA{A} \wedge \hB{B}$}
				\AXC{$\hC{B} \seq \hC{B} $}
						\AXC{$\hD{A} \seq \hD{A}$} \RightLabel{$\wedge_r^2$}
					\BIC{$\hD{A}, \hC{B} \seq \hC{B} \wedge \hD{A}$} \RightLabel{$\wedge_l$}
					\UIC{$\hD{A} \wedge \hC{B} \seq \hC{B} \wedge \hD{A}$} \RightLabel{$cut^3$}
			\BIC{$\hA{A} \wedge \hB{B} \seq \hC{B} \wedge \hD{A}$}
							\AXC{$\hE{C} \seq \hE{C}$} 
									\AXC{$\hF{C} \seq \hF{C}$} \RightLabel{$cut^4$}
								\BIC{$\hE{C} \seq \hF{C}$} \RightLabel{$\vee_l^5$}
					\BIC{$(\hA{A} \wedge \hB{B}) \vee \hE{C} \seq \hC{B} \wedge \hD{A}, \hF{C} $} 
\end{prooftree}

Its cut-pertinent struct is:

$$
\struct{\varphi} 
\equiv 
((\hA{A} \structplus^1 \hB{B}) \structplus^3 (\hC{\structdual{B}} \structtimes^2 \hD{\structdual{A}}))
\structtimes^5
(\hE{C} \structplus^4 \hF{\structdual{C}})
$$

Considering that $\{ \hA{A}, \hB{B}, \hE{C}\} \subset \occInference{\varphi}{\vee_l^5}$ and $\{ \hD{A}, \hC{B}, \hF{C} \} \cap \occInference{\varphi}{\vee_l^5} = \emptyset$, the struct can be normalized in the way shown below:

$$
\begin{array}{rcl}
\struct{\varphi} 
& \equiv &
((\hA{A} \structplus^1 \hB{B}) \structplus^3 (\hC{\structdual{B}} \structtimes^2 \hD{\structdual{A}}))
\structtimes^5
(\hE{C} \structplus^4 \hF{\structdual{C}}) \\
%
& \normalizePlusTimesDWw &
	((\hA{A} \structplus^1 \hB{B})\structtimes^5 (\hE{C} \structplus^4 \hF{\structdual{C}}))
\structplus^3 
	(\hC{\structdual{B}} \structtimes^2 \hD{\structdual{A}}) \\
%
& \normalizePlusTimesDWw &
	((((\hA{A} \structplus^1 \hB{B})\structtimes^5 \hE{C}) \structplus^4 \hF{\structdual{C}}))
\structplus^3 
	(\hC{\structdual{B}} \structtimes^2 \hD{\structdual{A}}) \\
%
& \normalizePlusTimesDWd &
	(((D_{\hA{A} \structplus \hB{B}} \structtimes^5 \hE{C}) \structplus^4 \hF{\structdual{C}}))
\structplus^3 
	(\hC{\structdual{B}} \structtimes^2 \hD{\structdual{A}}) 
\structplus
(D_{\hA{A} \structplus \hB{B}} \biimp (\hA{A} \structplus^1 \hB{B})) \\
%
& \equiv &
	(D_{\hA{A} \structplus \hB{B}} \structtimes^5 \hE{C}) 
\structplus^4 
	\hF{\structdual{C}}
\structplus^3 
	(\hC{\structdual{B}} \structtimes^2 \hD{\structdual{A}}) 
\structplus \\
&		   &	
	(\neg D_{\hA{A} \structplus \hB{B}} \structtimes \hA{A} ) 
\structplus
	(\neg D_{\hA{A} \structplus \hB{B}} \structtimes \hB{B} )
\structplus
	(D_{\hA{A} \structplus \hB{B}} \structtimes (\neg \hA{A} \structtimes \neg \hB{B})) \\
\end{array}
$$

$(\neg D_{\hA{A} \structplus \hB{B}} \structtimes \hA{A} )$, $(\neg D_{\hA{A} \structplus \hB{B}} \structtimes \hB{B} )$ and $(D_{\hA{A} \structplus \hB{B}} \structtimes (\neg \hA{A} \structtimes \neg \hB{B}))$ are definitional $\structtimes$-junctions. $\hF{\structdual{C}}$ and $(\hC{\structdual{B}} \structtimes^2 \hD{\structdual{A}})$ are pure $\structtimes$-junctions. And, finally, $(D_{\hA{A} \structplus \hB{B}} \structtimes^5 \hE{C})$ is a mixed $\structtimes$-junction.
\end{example}



\begin{definition}[Cut-pertinent Swapped Definitional Clause Set]
\label{definition:CutPertinentClauseSetSwappedDefinitional}
A \emph{cut-pertinent definitional clause set} of a proof $\varphi$ is:

$$
\clausesetSwappedDefinitional{\varphi}{S'} \defEq \clausify{S}
$$
where $S'$ is a $\normalizePlusTimesDWw$-normal-form of $\struct{\varphi}$ and $S' \normalizePlusTimesDWd^* S$.


The clauses corresponding to definitional $\structtimes$-junctions are called \emph{definitional clauses}. The clauses corresponding to pure $\structtimes$-junctions are called \emph{pure clauses}. The clauses corresponding to mixed $\structtimes$-junctions are called \emph{mixed clauses}.

If the $\normalizePlusTimesDWw$-normal-form $S'$ is unique or clear from the context, then it can be omitted. The swapped definitional clause set is then denoted simply as $\clausesetSwappedDefinitionalUnique{\varphi}$.
\end{definition}

\begin{example}[Swapped Definitional Clause Set]
\label{example:SwappedDefinitionalClauseSet}


Let $\varphi$ be the proof in Example \ref{example:PlusTimesSwapNormalization}. Then one of its swapped definitional clause sets is:

$$
\clausesetSwappedDefinitionalUnique{\varphi} \equiv  \left\{ \begin{array}{l}
	\seq D_{\hA{A} \structplus \hB{B}} , \hE{C} 
\ \ \ ; \\
	\hF{C} \seq 
\ \ \ ; \\
	\hC{B} , \hD{A} \seq
\ \ \ ; \\
	D_{\hA{A} \structplus \hB{B}} \seq \hA{A}  
\ \ \ ; \\
	D_{\hA{A} \structplus \hB{B}} \seq \hB{B} 
\ \ \ ; \\
	\hA{A}, \hB{B} \seq D_{\hA{A} \structplus \hB{B}} 
\end{array} \right\}
$$

The clauses $D_{\hA{A} \structplus \hB{B}} \seq \hA{A}$, $D_{\hA{A} \structplus \hB{B}} \seq \hB{B}$ and 
$\hA{A}, \hB{B} \seq D_{\hA{A} \structplus \hB{B}}$ are definitional clauses. $\hF{C} \seq $ and 
$\hC{B} , \hD{\structdual{A}} \seq$ are pure clauses. And $\seq D_{\hA{A} \structplus \hB{B}} , \hE{C}$ is a mixed clause.
\end{example}


\subsection{Projections}

While construction of swapped clause sets is reasonably straightforward, the construction of projections for some of the clauses presents some difficulties. As in the case of definitional clause sets, some clauses of swapped definitional clause sets are merely definitional, and hence corresponding definitional D-projections can be easily constructed. Other clauses are pure in the sense that they do not contain any defined predicate symbol, and hence O-Projections can be constructed for such clauses. However, there are mixed clauses for which none of the previously defined notions of projection work, because these clauses contain a mix of defined and undefined predicate symbols. 

\subsubsection{DW-Projections}

The new notion of projection required by mixed clauses is called mixed DW-Projection (Definition \ref{definition:DWProjectionMixed}) and it is essentially a combination of O-projection and proper D-projection. It requires the auxiliary concepts of encapsulated formula occurrences (Definition \ref{definition:EncapsulatedFormulaOccurrences}) and encapsulated inferences (Definition \ref{definition:EncapsulatedInferences}). Roughly, constructing a mixed DW-projection is initially similar to constructing an O-projection, taking care to include encapsulated formula occurrences in the slice. Later cut-pertinent inferences are replaced by $\wedge_r$ and $d_r$ inferences, similarly to what is done during the construction of proper D-projections, in order to re-encapsulate the encapsulated formula occurrences into the defined predicate symbol.

\begin{definition}[Encapsulated Formula Occurrences]
\label{definition:EncapsulatedFormulaOccurrences}
Let $S$ be a struct and $S'$ be a substruct of $S$. Let $N_{S'}$ be the defined predicate for $S'$. Then, the \emph{encapsulated occurrences} of $N_{S'}$ are all the atomic occurrences of $S'$.
\end{definition}

\begin{example}[Encapsulated Formula Occurrences]
\label{example:EncapsulatedFormulaOccurrences}


The encapsulated formula occurrences of the defined predicate $D_{\hA{A} \structplus \hB{B}}$ of the $\normalizePlusTimesDW$-normal-form of the struct $\struct{\varphi}$ shown in Example \ref{example:PlusTimesSwappedDefinitionalNormalization} are: $\hA{A}$ and $\hB{B}$.
\end{example}


\begin{definition}[Encapsulated Inferences]
\label{definition:EncapsulatedInferences}
Let $S$ be a cut-pertinent struct of a proof $\varphi$ and $S'$ be a substruct of $S$. Let $N_{S'}$ be the defined predicate for $S'$. Then, every inference $\rho$ of $\varphi$ which corresponds to a connective $\structplus_{\rho}$ or $\structtimes_{\rho}$ in $S'$ or that is an axiom inference having a formula occurrence of $S'$ in its conclusion sequent is an \emph{encapsulated inference} of $N_{S'}$.
\end{definition}

\begin{example}[Encapsulated Inferences]
\label{example:EncapsulatedInferences}


The encapsulated inferences of the defined predicate $D_{\hA{A} \structplus \hB{B}}$ of the $\normalizePlusTimesDW$-normal-form of the struct $\struct{\varphi}$ shown in Example \ref{example:PlusTimesSwappedDefinitionalNormalization} are: $\wedge^1_r$ and the axiom inferences having $\hA{A} \seq \hA{A}$ and $\hB{B} \seq \hB{B}$ as conclusion sequents.
\end{example}


\begin{definition}[Definitional DW-Projection]
\label{definition:DWProjectionDefinitional}
\index{Projection!Definitional DW-Projection}
Let $\varphi$ be a proof and $c$ a definitional clause in $\clausesetSwappedDefinitional{\varphi}{S}$. Then the \emph{definitional DW-projection} of $\varphi$ with respect to $c$ is constructed in the same way as a definitional D-projection and thus simply defined as:

$$
\projectionDWDefinitional{\varphi}{c} \defEq \projectionDDefinitional{\varphi}{c}
$$
\end{definition}


\begin{definition}[Pure DW-Projection]
\label{definition:DWProjectionPure}
\index{Projection!Pure DW-Projection}
Let $\varphi$ be a proof and $c$ a pure clause in $\clausesetSwappedDefinitional{\varphi}{S}$. Then the \emph{pure DW-projection} of $\varphi$ with respect to $c$ is constructed in the same way as a O-projection and thus simply defined as:

$$
\projectionDWDefinitional{\varphi}{c} \defEq \projectionO{\varphi}{c}
$$
\end{definition}


\begin{definition}[Mixed DW-Projection]
\label{definition:DWProjectionMixed}
\index{Projection!Mixed DW-Projection}
Let $\varphi$ be a proof and $c$ a mixed clause in $\clausesetSwappedDefinitional{\varphi}{S}$. Let $\Omega_E$ and $\Upsilon_E$ be the sets of, respectively, encapsulated formula occurrences and encapsulated inferences of defined predicates occurring in $c$. Let $\Omega_c$ be the set of undefined formula occurrences in $c$. Then the \emph{mixed DW-projection} of $\varphi$ with respect to $c$ can be computed according to the following steps:

\begin{enumerate}
%\item Construct $\varphi^1 \defEq \slice{\varphi}{\Omega_E \cup \Omega_c}$.
\item Replace the inferences of $\Upsilon_E$ in $\varphi^1$ by $\neg_r$, $\wedge_r$, $\vee_r$ and $d_r$ (analogously to what is done in the construction of proper D-projections). Let $\varphi^2$ be the resulting proofoid.
\item Construct $\varphi^3 \defEq \replacePert{\varphi^2}{\occCutPert{\varphi^2}}$ by replacing the cut-pertinent inferences of $\varphi^2$ by $Y$-inferences.
\item Construct $\varphi^4 \defEq \WFix{\varphi^3}$ by fixing broken inferences with weakening.
%\item Finally, construct the mixed DW-projection $\projectionDWMixed{\varphi}{c} \defEq \EliminateY{\varphi^4}$ by eliminating the $Y$-inferences from $\varphi^4$.
\end{enumerate}
\end{definition}


\begin{example}[Mixed DW-Projection]
\label{example:DWProjectionMixed}



Let ${\varphi}$ be the proof shown in Example \ref{example:PlusTimesSwappedDefinitionalNormalization}, which is displayed again for convenience below:

\begin{prooftree}
\AXC{$\hA{A} \seq \hA{A} $}
		\AXC{$\hB{B} \seq \hB{B}$} \RightLabel{$\wedge_r^1$}
	\BIC{$\hA{A}, \hB{B} \seq \hA{A} \wedge \hB{B}$} \RightLabel{$\wedge_l$}
	\UIC{$\hA{A} \wedge \hB{B} \seq \hA{A} \wedge \hB{B}$}
				\AXC{$\hC{B} \seq \hC{B} $}
						\AXC{$\hD{A} \seq \hD{A}$} \RightLabel{$\wedge_r^2$}
					\BIC{$\hD{A}, \hC{B} \seq \hC{B} \wedge \hD{A}$} \RightLabel{$\wedge_l$}
					\UIC{$\hD{A} \wedge \hC{B} \seq \hC{B} \wedge \hD{A}$} \RightLabel{$cut^3$}
			\BIC{$\hA{A} \wedge \hB{B} \seq \hC{B} \wedge \hD{A}$}
							\AXC{$\hE{C} \seq \hE{C}$} 
									\AXC{$\hF{C} \seq \hF{C}$} \RightLabel{$cut^4$}
								\BIC{$\hE{C} \seq \hF{C}$} \RightLabel{$\vee_l^5$}
					\BIC{$(\hA{A} \wedge \hB{B}) \vee \hE{C} \seq \hC{B} \wedge \hD{A}, \hF{C} $} 
\end{prooftree}

The first step in the construction of the mixed DW-projection $\projectionDWMixed{\varphi}{\seq D_{\hA{A} \structplus \hB{B}} , \hE{C}}$ is the slicing with respect to $\Omega_E \cup \Omega_c$ where $\Omega_E = \{\hA{A}, \hB{B}\}$ and $\Omega_c = \{\hE{C}\}$:

\renewcommand{\hC}[1]{\phantom{#1}}
\renewcommand{\hD}[1]{\phantom{#1}}
\renewcommand{\hF}[1]{\phantom{#1}}

\begin{prooftree}
\AXC{$\hA{A} \seq \hA{A} $}
		\AXC{$\hB{B} \seq \hB{B}$} \RightLabel{$\wedge_r^1$}
	\BIC{$\hA{A}, \hB{B} \seq \hA{A} \wedge \hB{B}$} \RightLabel{$\wedge_l$}
	\UIC{$\hA{A} \wedge \hB{B} \seq \hA{A} \wedge \hB{B}$}
				\AXC{$\hC{B} \seq \hC{B} $}
						\AXC{$\hD{A} \seq \hD{A}$} \RightLabel{$Y$}
					\BIC{$\hD{A}\phantom{,} \hC{B} \seq \hC{B} \phantom{\wedge} \hD{A}$} \RightLabel{$Y$}
					\UIC{$\hD{A} \phantom{\wedge} \hC{B} \seq \hC{B} \phantom{\wedge} \hD{A}$} \RightLabel{$Y$}
			\BIC{$\hA{A} \wedge \hB{B} \seq \hA{A} \wedge \hB{B}$}
							\AXC{$\hE{C} \seq \hE{C}$} 
									\AXC{$\hF{C} \seq \hF{C}$} \RightLabel{$Y$}
								\BIC{$\hE{C} \seq \hE{C}$} \RightLabel{$\vee_l^5$}
					\BIC{$(\hA{A} \wedge \hB{B}) \vee \hE{C} \seq \hA{A} \wedge \hB{B}, \hC{B} \phantom{\wedge} \hD{A}\phantom{,} \hE{C} $} 
\end{prooftree}


The second step is the introduction of definition inferences, resulting in the proofoid $\varphi^2$ below:

\begin{prooftree}
\AXC{$\hA{A} \seq \hA{A} $}
		\AXC{$\hB{B} \seq \hB{B}$} \RightLabel{$\wedge_r$}
	\BIC{$\hA{A}, \hB{B} \seq \hA{A} \wedge \hB{B}$} \RightLabel{$d_r$}
	\UIC{$\hA{A}, \hB{B} \seq D_{\hA{A} \structplus \hB{B}}$} \RightLabel{$\wedge_l$}
	\UIC{$\hA{A} \wedge \hB{B} \seq D_{\hA{A} \structplus \hB{B}}$}
				\AXC{$\hC{B} \seq \hC{B} $}
						\AXC{$\hD{A} \seq \hD{A}$} \RightLabel{$Y$}
					\BIC{$\hD{A}\phantom{,} \hC{B} \seq \hC{B} \phantom{\wedge} \hD{A}$} \RightLabel{$Y$}
					\UIC{$\hD{A} \phantom{\wedge} \hC{B} \seq \hC{B} \phantom{\wedge} \hD{A}$} \RightLabel{$Y$}
			\BIC{$\hA{A} \wedge \hB{B} \seq D_{\hA{A} \structplus \hB{B}}$}
							\AXC{$\hE{C} \seq \hE{C}$} 
									\AXC{$\hF{C} \seq \hF{C}$} \RightLabel{$Y$}
								\BIC{$\hE{C} \seq \hE{C}$} \RightLabel{$\vee_l^5$}
					\BIC{$(\hA{A} \wedge \hB{B}) \vee \hE{C} \seq D_{\hA{A} \structplus \hB{B}},  \hC{B} \phantom{\wedge} \hD{A}\phantom{,} \hE{C} $} 
\end{prooftree}

Subsequently, cut-pertinent inferences of $\varphi^2$ should be replaced by $Y$-inferences. However, since $\varphi^2$ has no cuts, there is nothing to be replaced, and hence $\varphi^3 = \varphi^2$. Subsequently, broken inferences of $\varphi^3$ should be W-fixed. However, there are no broken inferences in $\varphi^3$. Therefore, only the last step of eliminating $Y$-inferences remains and its result is the mixed DW-projection $\projectionDWMixed{\varphi}{\seq D_{\hA{A} \structplus \hB{B}} , \hE{C}}$ shown below:

\begin{prooftree}
\AXC{$\hA{A} \seq \hA{A} $}
		\AXC{$\hB{B} \seq \hB{B}$} \RightLabel{$\wedge_r$}
	\BIC{$\hA{A}, \hB{B} \seq \hA{A} \wedge \hB{B}$} \RightLabel{$d_r$}
	\UIC{$\hA{A}, \hB{B} \seq D_{\hA{A} \structplus \hB{B}}$} \RightLabel{$\wedge_l$}
	\UIC{$\hA{A} \wedge \hB{B} \seq D_{\hA{A} \structplus \hB{B}}$}
							\AXC{$\hE{C} \seq \hE{C}$} \RightLabel{$\vee_l^5$}
					\BIC{$(\hA{A} \wedge \hB{B}) \vee \hE{C} \seq D_{\hA{A} \structplus \hB{B}}, \hE{C} $} 
\end{prooftree}

\end{example}


\begin{example}[$\CEResDSwapDSwap$-Normal-Form]

Consider again the swapped definitional clause set of the proof $\varphi$ shown in Example \ref{example:SwappedDefinitionalClauseSet}:

$$
\clausesetSwappedDefinitionalUnique{\varphi} \equiv  \left\{
	\seq D_{\hA{A} \structplus \hB{B}} , \hE{C} 
\ \ \ ; \ \
	\hF{C} \seq 
\ \ \ ; \ \
	\hC{B} , \hD{A} \seq
\ \ \ ; \ \
	D_{\hA{A} \structplus \hB{B}} \seq \hA{A}  
\ \ \ ; \ \
	D_{\hA{A} \structplus \hB{B}} \seq \hB{B} 
\ \ \ ; \ \
	\hA{A}, \hB{B} \seq D_{\hA{A} \structplus \hB{B}} 
\right\}
$$

The shortest refutation $\delta$ of $\clausesetSwappedDefinitionalUnique{\varphi}$ is shown below:

\begin{prooftree}
\AXC{$\seq D_{\hA{A} \structplus \hB{B}} , \hE{C}$}
		\AXC{$\hF{C} \seq$} \RightLabel{$r$}
	\BIC{$\seq D_{\hA{A} \structplus \hB{B}}$}
				\AXC{$D_{\hA{A} \structplus \hB{B}} \seq \hA{A}$}
						\AXC{$D_{\hA{A} \structplus \hB{B}} \seq \hB{B}$}
								\AXC{$\hC{B} , \hD{A} \seq$} \RightLabel{$r$}
							\BIC{$D_{\hA{A} \structplus \hB{B}}, \hD{A} \seq $} \RightLabel{$r$}
					\BIC{$D_{\hA{A} \structplus \hB{B}}, D_{\hA{A} \structplus \hB{B}} \seq $} \RightLabel{$f_l$}
					\UIC{$D_{\hA{A} \structplus \hB{B}} \seq $} \RightLabel{$r$}
			\BIC{$ \seq $}
\end{prooftree}

By using the mixed DW-projection shown in Example \ref{example:DWProjectionMixed}, pure DW-projections shown in Example \ref{example:OProjections} and definitional DW-projections shown in Example \ref{example:DProjectionDefinitional}, $\CEResNFDSwapDSwap{\varphi}{\delta}$ is:


%\begin{small}
\begin{prooftree}
\AXC{$\hA{A} \seq \hA{A} $}
		\AXC{$\hB{B} \seq \hB{B}$} \RightLabel{$\wedge_r$}
	\BIC{$\hA{A}, \hB{B} \seq \hA{A} \wedge \hB{B}$} \RightLabel{$d_r$}
	\UIC{$\hA{A}, \hB{B} \seq D_{\hA{A} \structplus \hB{B}}$} \RightLabel{$\wedge_l$}
	\UIC{$\hA{A} \wedge \hB{B} \seq D_{\hA{A} \structplus \hB{B}}$}
							\AXC{$\hE{C} \seq \hE{C}$} \RightLabel{$\vee_l^5$}
					\BIC{$(\hA{A} \wedge \hB{B}) \vee \hE{C} \seq D_{\hA{A} \structplus \hB{B}}, \hE{C} $} 
               		\AXC{$\hF{C} \seq \hF{C}$} \RightLabel{$cut$}
               	\BIC{$(\hA{A} \wedge \hB{B}) \vee \hE{C} \seq D_{\hA{A} \structplus \hB{B}}, \hF{C}$}
	            				\AXC{$\hA{A} \seq \hA{A}$} \RightLabel{$w_l$}
               				\UIC{$\hA{A}, \hB{B} \seq \hA{A}$} \RightLabel{$\wedge_l$}
               				\UIC{$\hA{A} \wedge \hB{B} \seq \hA{A}$} \RightLabel{$d_l$}
               				\UIC{$D_{\hA{A} \structplus \hB{B}} \seq \hA{A}$}
               						\AXC{$\hB{B} \seq \hB{B}$} \RightLabel{$w_l$}
               						\UIC{$\hA{A}, \hB{B} \seq \hB{B}$} \RightLabel{$\wedge_l$}
               						\UIC{$\hA{A} \wedge \hB{B} \seq \hB{B}$} \RightLabel{$d_l$}
               						\UIC{$D_{\hA{A} \structplus \hB{B}} \seq \hB{B}$}
               								\AXC{$\hC{B} \seq \hC{B} $}
															\AXC{$\hD{A} \seq \hD{A}$} \RightLabel{$\wedge_r$}
														\BIC{$ \hC{B}, \hD{A} \seq \hC{B} \wedge \hD{A}$} \RightLabel{$cut$}
               							\BIC{$D_{\hA{A} \structplus \hB{B}}, \hD{A} \seq \hC{B} \wedge \hD{A}$} \RightLabel{$cut$}
               					\BIC{$D_{\hA{A} \structplus \hB{B}}, D_{\hA{A} \structplus \hB{B}} \seq \hC{B} \wedge \hD{A}$} \RightLabel{$c_l$}
               					\UIC{$D_{\hA{A} \structplus \hB{B}} \seq \hC{B} \wedge \hD{A}$} \RightLabel{$cut$}
               			\BIC{$(\hA{A} \wedge \hB{B}) \vee \hE{C} \seq \hC{B} \wedge \hD{A}, \hF{C}$}
\end{prooftree}
%\end{small}


\end{example}


\section{Ignoring Atomic and Quantifier-Free Cuts}
\label{sec:CutEliminationByResolution:CEResIgnoringAtomicCuts}

If $\CERes$ is applied to a proof containing only atomic cuts, $\CERes$ still transforms the proof into a new proof containing only atomic cuts, but with additional structural inferences and with the atomic cuts located in the bottom of the proof. This is clearly non-ideal, because the proof could be simply left unchanged. More generally, if $\CERes$ is applied to a proof containing complex cuts and atomic cuts, $\CERes$ unnecessarily includes the atomic cuts in the process of reduction, even though atomic cuts cannot be reduced further. The inclusion of atomic cuts results in larger clause sets that are more costly to refute, and in normal forms with possibly additional structural inferences. This indicates that there is a very simple and evident improvement of the $\CERes$ method that has been thoroughly overlooked so far: instead of distinguishing between cut-pertinent and cut-impertinent formula occurrences (i.e. between ancestors and non-ancestors of \emph{all} cut formula occurrences) and cut-pertinent and cut-impertinent inferences (i.e inferences that operate on the ancestors and on the non-ancestors of cut formula occurrences), it suffices to distinguish between ancestors of \emph{complex} cut formula occurrences and ancestors of either occurrences in the end-sequent or of atomic cut-formula occurrences.


\newcommand{\occCutPertComplex}[1]{\Omega_{CCP}(#1)}   % Cut-pertinent occurrences of a proof. Argument: proof.
\newcommand{\occCutImpertComplex}[1]{\Omega_{CCI}(#1)}   % Cut-impertinent occurrences of a proof. Argument: proof.
\begin{definition}[Complex-Cut-Pertinent and Complex-Cut-Impertinent Occurrences]
\label{definition:ComplexCutPertinenceOccurrences}
A formula occurrence is \emph{complex-cut-pertinent} if and only if it is an ancestor of a non-atomic cut formula occurrence. The \emph{set of complex-cut-pertinent formula occurrences} of a proof $\varphi$ is denoted $\occCutPertComplex{\varphi}$.

A formula occurrence is \emph{complex-cut-impertinent} if and only if it is not complex-cut-pertinent. The \emph{set of complex-cut-impertinent formula occurrences} of a proof $\varphi$ is denoted $\occCutImpertComplex{\varphi}$.
\end{definition}

\begin{definition}[Complex-Cut-Pertinence]
\label{definition:CutPertinenceInferences}
An inference $\rho$ is \emph{complex-cut-pertinent} if and only if $\rho$ is $\occCutPertComplex{\varphi}$-pertinent.

An inference $\rho$ is \emph{complex-cut-impertinent} if and only if $\rho$ is $\occCutImpertComplex{\varphi}$-pertinent.
\end{definition}

\begin{definition}[$\CERes$-Normal-Form Ignoring Atomic Cuts]
\label{definition:CEResNormalFormIgnoringAtomicCuts} \hspace*{\fill} \\
The \emph{$\CERes$-normal-form ignoring atomic cuts} $\CEResNFComplex{\varphi}{\delta}$ of a proof $\varphi$ is obtained in the same way as $\CEResNF{\varphi}{\delta}$ except that, in all manipulations and constructions of structs, clause sets and projections, $\occCutPertComplex{\varphi}$ is used instead of $\occCutPert{\varphi}$, $\occCutImpertComplex{\varphi}$ is used instead of $\occCutImpert{\varphi}$ and complex-cut-pertinence of inferences is used instead of cut-pertinence of inferences.
\end{definition}

\begin{example}[$\CERes$-Normal-Form Ignoring Atomic Cuts]
\label{example:CEResNormalFormIgnoringAtomicCuts}

Let $\varphi$ be the proof below:

\begin{prooftree}
\AXC{$\hA{A} \seq \hA{A} $}
		\AXC{$\hB{B} \seq \hB{B}$} \RightLabel{$\wedge_r^1$}
	\BIC{$\hA{A}, \hB{B} \seq \hA{A} \wedge \hB{B}$} \RightLabel{$\wedge_l$}
	\UIC{$\hA{A} \wedge \hB{B} \seq \hA{A} \wedge \hB{B}$}
				\AXC{$\hC{B} \seq \hC{B} $}
						\AXC{$\hD{A} \seq \hD{A}$} \RightLabel{$\wedge_r^2$}
					\BIC{$\hD{A}, \hC{B} \seq \hC{B} \wedge \hD{A}$} \RightLabel{$\wedge_l$}
					\UIC{$\hD{A} \wedge \hC{B} \seq \hC{B} \wedge \hD{A}$} \RightLabel{$cut^3$}
			\BIC{$\hA{A} \wedge \hB{B} \seq \hC{B} \wedge \hD{A}$}
							\AXC{$\hE{C} \seq \hE{C}$} 
									\AXC{$\hF{C} \seq \hF{C}$} \RightLabel{$cut^4$}
								\BIC{$\hE{C} \seq \hF{C}$} \RightLabel{$\vee_l^5$}
					\BIC{$(\hA{A} \wedge \hB{B}) \vee \hE{C} \seq \hC{B} \wedge \hD{A}, \hF{C} $} 
\end{prooftree}

Its complex-cut-pertinent struct is shown below. It is interesting to note that $cut^4$ now corresponds to a $\structtimes$ connective, because $cut^4$ is complex-cut-impertinent.
$$
\struct{\varphi}^C 
\equiv 
((\hA{A} \structplus^1 \hB{B}) \structplus^3 (\hC{\structdual{B}} \structtimes^2 \hD{\structdual{A}}))
\structtimes^5
(\hE{\structtimesEmpty} \structtimes^4 \hF{\structtimesEmpty})
$$

The struct can be $\normalizePlusTimesS$-normalized to:
$$
S
\equiv
	(\hA{A}\structtimes^5 \hE{\structtimesEmpty} \structtimes^4 \hF{\structtimesEmpty}) 
\structplus^1 
	(\hB{B}\structtimes^5 \hE{\structtimesEmpty} \structtimes^4 \hF{\structtimesEmpty}) 
\structplus^3 
	(\hC{\structdual{B}} \structtimes^2 \hD{\structdual{A}})
$$

And the corresponding clause set is:
$$
\clauseset{\varphi}
\equiv
	\{ \ \ \seq \hA{A} \ \ ; \ \ \seq \hB{B} \ \ ; \ \ \hC{B}, \hD{A} \seq \ \ \}
$$

It can be refuted by the refutation $\delta$ shown below:

\begin{prooftree}
\AXC{$\seq \hA{A}$} 
		\AXC{$\seq \hB{B}$}
				\AXC{$\hC{B}, \hD{A} \seq$} \RightLabel{$r$}
			\BIC{$\hD{A} \seq$} \RightLabel{$r$}
	\BIC{$\seq$}
\end{prooftree}


The O-projection $\projectionO{\varphi}{\seq \hA{A}}$ is shown below. Interestingly, projections can now contain atomic cuts because they are complex-cut-impertinent inferences.

\begin{prooftree}
	\AXC{$\hA{A} \seq \hA{A} $} \RightLabel{$w_l$}
	\UIC{$\hA{A}, \hG{B} \seq \hA{A}$} \RightLabel{$\wedge_l$}
	\UIC{$\hA{A} \wedge \hG{B} \seq \hA{A}$}
							\AXC{$\hE{C} \seq \hE{C}$} 
									\AXC{$\hF{C} \seq \hF{C}$} \RightLabel{$cut^4$}
								\BIC{$\hE{C} \seq \hF{C}$} \RightLabel{$\vee_l^5$}
					\BIC{$(\hA{A} \wedge \hG{B}) \vee \hE{C} \seq \hA{A}, \hF{C} $} 
\end{prooftree}


Analogously, the O-projection $\projectionO{\varphi}{\seq \hB{B}}$ is:

\begin{prooftree}
	\AXC{$\hB{B} \seq \hB{B}$} \RightLabel{$w_l$}
	\UIC{$\hG{A}, \hB{B} \seq \hB{B}$} \RightLabel{$\wedge_l$}
	\UIC{$\hG{A} \wedge \hB{B} \seq \hB{B}$}
							\AXC{$\hE{C} \seq \hE{C}$} 
									\AXC{$\hF{C} \seq \hF{C}$} \RightLabel{$cut^4$}
								\BIC{$\hE{C} \seq \hF{C}$} \RightLabel{$\vee_l^5$}
					\BIC{$(\hG{A} \wedge \hB{B}) \vee \hE{C} \seq \hB{B}, \hF{C} $} 
\end{prooftree}

And the O-projection $\projectionO{\varphi}{\hC{B}, \hD{A} \seq}$ is:

\begin{prooftree}
				\AXC{$\hC{B} \seq \hC{B} $}
						\AXC{$\hD{A} \seq \hD{A}$} \RightLabel{$\wedge_r^2$}
					\BIC{$\hD{A}, \hC{B} \seq \hC{B} \wedge \hD{A}$} 
\end{prooftree}

\begin{landscape}
Combining the refutation and the projections as usual, $\CEResNFComplexSwapO{\varphi}{\delta}$ is obtained:

%\begin{footnotesize}
\begin{prooftree}
\AXC{$\hA{A} \seq \hA{A} $} \RightLabel{$w_l$}
\UIC{$\hA{A}, \hG{B} \seq \hA{A}$} \RightLabel{$\wedge_l$}
\UIC{$\hA{A} \wedge \hG{B} \seq \hA{A}$}
		\AXC{$\hE{C} \seq \hE{C}$} 
				\AXC{$\hF{C} \seq \hF{C}$} \RightLabel{$cut^4$}
			\BIC{$\hE{C} \seq \hF{C}$} \RightLabel{$\vee_l^5$}
	\BIC{$(\hA{A} \wedge \hG{B}) \vee \hE{C} \seq \hA{A}, \hF{C} $} 
						\AXC{$\hB{B} \seq \hB{B}$} \RightLabel{$w_l$}
						\UIC{$\hG{A}, \hB{B} \seq \hB{B}$} \RightLabel{$\wedge_l$}
						\UIC{$\hG{A} \wedge \hB{B} \seq \hB{B}$}
								\AXC{$\hE{C} \seq \hE{C}$} 
										\AXC{$\hF{C} \seq \hF{C}$} \RightLabel{$cut^4$}
									\BIC{$\hE{C} \seq \hF{C}$} \RightLabel{$\vee_l^5$}
							\BIC{$(\hG{A} \wedge \hB{B}) \vee \hE{C} \seq \hB{B}, \hF{C} $}
				\AXC{$\hC{B} \seq \hC{B} $}
						\AXC{$\hD{A} \seq \hD{A}$} \RightLabel{$\wedge_r^2$}
					\BIC{$\hD{A}, \hC{B} \seq \hC{B} \wedge \hD{A}$} \RightLabel{$cut$}
			\BIC{$(\hG{A} \wedge \hB{B}) \vee \hE{C}, \hD{A} \seq \hC{B} \wedge \hD{A}, \hF{C}$} \RightLabel{$cut$}
	\BIC{$(\hA{A} \wedge \hG{B}) \vee \hE{C}, (\hG{A} \wedge \hB{B}) \vee \hE{C} \seq \hC{B} \wedge \hD{A}, \hF{C}, \hF{C}$} \doubleLine \RightLabel{$c^*$}
	\UIC{$(A \wedge B) \vee C \seq \hC{B} \wedge \hD{A}, C$}
\end{prooftree}
%\end{footnotesize}
\end{landscape}

\end{example}


\label{sec:CutEliminationByResolution:CEResIgnoringQuantifierFreeCuts}


In fact, for some applications, such as Herbrand sequent extraction \cite{BrunoWoltzenlogelPaleoMestradoEMCL2007,BrunoWoltzenlogelPaleoHerbrandSequentBook2008}, it suffices to eliminate only cuts that have quantifiers\footnote{In fact, even if the cut formula occurrences of a cut $\rho$ in a proof $\varphi$ do contain quantifiers, if these quantifiers are dummy in the sense that they were introduced by weakening inferences instead of being properly introduced by quantifier inferences, then $\rho$ could also be considered ``quantifier-free'' and therefore be ignored. Nevertheless, for simplicity, this additional improvement is not considered in detail here.} in their cut formulas.


\newcommand{\occCutPertQuant}[1]{\Omega_{QCP}(#1)}   % Cut-pertinent occurrences of a proof. Argument: proof.
\newcommand{\occCutImpertQuant}[1]{\Omega_{QCI}(#1)}   % Cut-impertinent occurrences of a proof. Argument: proof.
\begin{definition}[Quantified-Cut-Pertinent and Quantified-Cut-Impertinent Occurrences]
\label{definition:ComplexCutPertinenceOccurrences}
A formula occurrence is \emph{quantified-cut-pertinent} if and only if it is an ancestor of a cut formula occurrence that contains quantifiers. The \emph{set of quantified-cut-pertinent formula occurrences} of a proof $\varphi$ is denoted $\occCutPertQuant{\varphi}$.

A formula occurrence is \emph{quantified-cut-impertinent} if and only if it is not quantified-cut-pertinent. The \emph{set of quantified-cut-impertinent formula occurrences} of a proof $\varphi$ is denoted $\occCutImpertQuant{\varphi}$.
\end{definition}

\begin{definition}[Quantified-Cut-Pertinence]
\label{definition:CutPertinenceInferences}
An inference $\rho$ is \emph{quantified-cut-pertinent} if and only if $\rho$ is $\occCutPertQuant{\varphi}$-pertinent.

An inference $\rho$ is \emph{quantified-cut-impertinent} if and only if $\rho$ is $\occCutImpertQuant{\varphi}$-pertinent.
\end{definition}

\begin{definition}[$\CERes$-Normal-Form Ignoring Quantifier-Free Cuts]
\label{definition:CEResNormalFormIgnoringQuantifierFreeCuts} \hspace*{\fill} \\
The \emph{$\CERes$-normal-form ignoring quantifier-free cuts} $\CEResNFQuant{\varphi}{\delta}$ of a proof $\varphi$ is obtained in the same way as $\CEResNF{\varphi}{\delta}$ except that, in all manipulations and constructions of structs, clause sets and projections, $\occCutPertQuant{\varphi}$ is used instead of $\occCutPert{\varphi}$, $\occCutImpertQuant{\varphi}$ is used instead of $\occCutImpert{\varphi}$ and quantified-cut-pertinence of inferences is used instead of cut-pertinence of inferences.
\end{definition}


\section{A Combined Approach}
\label{sec:Combination}

Although the number of defined symbols introduced by the construction of
definitional clause sets is bounded linearly with respect to the size of the
characteristic formula, it still creates a new symbol for every subformula of
the characteristic formula. The number of new symbols can be reduced with a
technique that combines ideas from swapped and definitional clause sets. The
idea is to use $\normalizePlusTimesS$ as long as no duplications occur and
then use $\normalizePlusTimesD$ only for the subformulas that cannot be
normalized with $\normalizePlusTimesS$ without duplications.


\begin{definition}[$\normalizePlusTimesSD$]
\label{definition:NormalizationPlusTimesDefinitionalSwap}
$\normalizePlusTimesSDs$ denotes a restricted form of $\normalizePlusTimesS$ where the distribution of disjunction over conjunction cannot lead to duplications. The first rewriting rule from Definition \ref{definition:NormalizationPlusTimesSwap} is replaced by the rewriting rule below, where $\hF{S}$ is distributed to at most one conjunct $\hF{S_k}$:
$$
\hF{S} \hF{\structtimes} (S_1 \structplus \ldots \structplus \hF{S_k} \structplus \ldots \structplus S_n) \normalizePlusTimesSDs  S_1 \structplus \ldots \structplus (\hF{S} \hF{\structtimes} \hF{S_{k}}) \structplus \ldots \structplus S_n
$$

\noindent
$\normalizePlusTimesSDd$ denotes a restricted form of $\normalizePlusTimesD$, defined by the following rewriting rule, which can be applied only if $S \vee (S_1 \structplus \ldots \structplus S_n)$ is already in $\normalizePlusTimesSDs$-normal-form:
\begin{small}
$$
C[S \vee (S_1 \structplus \ldots \structplus S_n)]   
\normalizePlusTimesSDd 
C[S \vee N(x_1,\ldots, x_m)] \structplus (N(x_1,\ldots, x_m) \biimp S_1 \structplus \ldots \structplus S_n)
$$
\end{small}
where $N$ is a new symbol and $x_1, \ldots, x_m$ are free-variables of $(S_1 \structplus \ldots \structplus S_n)$.

\medskip
\noindent
The relation $\normalizePlusTimesSD$ is the union of $\normalizePlusTimesSDs$ and $\normalizePlusTimesSDd$.
\end{definition}

\begin{definition}[Definitional Swapped Clause Set]
\label{definition:CutPertinentClauseSetSwappeDefinitional}
A \emph{definitional swapped clause set} $\clausesetSwapDef{\varphi}{S}$ of a proof $\varphi$ w.r.t. to a $\normalizePlusTimesSD$-normal-form $S$ of $\struct{\varphi}$ is $S$ written in sequent notation.
Clauses originating from defining equations introduced by $\normalizePlusTimesSDd$ are \emph{definitional clauses}. Non-definitional clauses not containing new symbols are \emph{pure clauses}. All other clauses are \emph{mixed clauses}.
\end{definition}

\begin{remark}
In cases where $\clausesetSwapDef{\varphi}{S_1} = \clausesetSwapDef{\varphi}{S_2}$ for any $S_1$ and $S_2$, the unique definitional swapped clause set is denoted simply as $\clausesetSwapDefUnique{\varphi}$.
\end{remark}



\begin{example}
\label{example:PlusTimesSwappeDefinitionalNormalization}
Let $\varphi$ be the proof shown in Example \ref{example:PlusTimesSwapNormalization}.
Its characteristic formula can be normalized as follows:
$$
\begin{array}{rcl}
\struct{\varphi} 
& \equiv &
((\hA{A} \structplus^1 \hB{B}) \structplus^3 (\hC{\structdual{B}} \structtimes^2 \hD{\structdual{A}}))
\structtimes^5
(\hE{C} \structplus^4 \hF{\structdual{C}}) \\
%
& \normalizePlusTimesSDs &
	((\hA{A} \structplus^1 \hB{B})\structtimes^5 (\hE{C} \structplus^4 \hF{\structdual{C}}))
\structplus^3 
	(\hC{\structdual{B}} \structtimes^2 \hD{\structdual{A}}) \\
%
& \normalizePlusTimesSDs &
	((((\hA{A} \structplus^1 \hB{B})\structtimes^5 \hE{C}) \structplus^4 \hF{\structdual{C}}))
\structplus^3 
	(\hC{\structdual{B}} \structtimes^2 \hD{\structdual{A}}) \\
%
& \normalizePlusTimesSDd &
	(((D_{\hA{A} \structplus \hB{B}} \structtimes^5 \hE{C}) \structplus^4 \hF{\structdual{C}}))
\structplus^3 
	(\hC{\structdual{B}} \structtimes^2 \hD{\structdual{A}}) 
\structplus
(D_{\hA{A} \structplus \hB{B}} \biimp (\hA{A} \structplus^1 \hB{B})) \\
%
& \equiv &
	(D_{\hA{A} \structplus \hB{B}} \structtimes^5 \hE{C}) 
\structplus^4 
	\hF{\structdual{C}}
\structplus^3 
	(\hC{\structdual{B}} \structtimes^2 \hD{\structdual{A}}) 
\structplus \\
&		   &	
	(\neg D_{\hA{A} \structplus \hB{B}} \structtimes \hA{A} ) 
\structplus
	(\neg D_{\hA{A} \structplus \hB{B}} \structtimes \hB{B} )
\structplus
	(D_{\hA{A} \structplus \hB{B}} \structtimes (\neg \hA{A} \structtimes \neg \hB{B})) \\
\end{array}
$$

\noindent
And the corresponding definitional swapped clause set is:
$$
\clausesetSwapDefUnique{\varphi} \equiv  \left\{ \begin{array}{l}
	\seq D_{\hA{A} \structplus \hB{B}} , \hE{C} 
\ \ \ ; \\
	\hF{C} \seq 
\ \ \ ; \\
	\hC{B} , \hD{A} \seq
\ \ \ ; \\
	D_{\hA{A} \structplus \hB{B}} \seq \hA{A}  
\ \ \ ; \\
	D_{\hA{A} \structplus \hB{B}} \seq \hB{B} 
\ \ \ ; \\
	\hA{A}, \hB{B} \seq D_{\hA{A} \structplus \hB{B}} 
\end{array} \right\}
$$

\noindent
$D_{\hA{A} \structplus \hB{B}} \seq \hA{A}$, $D_{\hA{A} \structplus \hB{B}} \seq \hB{B}$ 
and $\hA{A}, \hB{B} \seq D_{\hA{A} \structplus \hB{B}}$ are definitional clauses. 
$\hF{C} \seq $ and 
$\hC{B} , \hD{\structdual{A}} \seq$ are pure clauses. 
And $\seq D_{\hA{A} \structplus \hB{B}} , \hE{C}$ is a mixed clause.
\hfill\QED
\end{example}


\noindent
While construction of definitional swapped clause sets is reasonably
straightforward, the construction of projections
presents some difficulties. As in the case of definitional clause sets, some
clauses in definitional swapped clause sets are definitional, and their projections can be easily constructed according to Definition \ref{ToDo}. Other
clauses are pure in the sense that they do not contain any defined predicate
symbol, and hence their projections can be constructed in the standard way explained in Definition \ref{ToDo}. However, for mixed clauses, which contain a mix of defined and undefined predicate symbols, it is necessary to construct a \emph{mixed projection}, which combines the construction methods of standard and proper projections (from Definition \ref{ToDo}). 



\begin{definition}[Encapsulated Formulas]
\label{definition:EncapsulatedFormulaOccurrences}
Let $S$ be a characteristic formula and $S'$ be a subformula of $S$ having $N$ as the new predicate symbol for $S'$ created during the $\normalizePlusTimesSD$-normalization of $S$. Then, the \emph{encapsulated formulas} of $N$ are all the atomic formulas of $S'$.
\end{definition}

\begin{example}
\label{example:EncapsulatedFormulaOccurrences}
The formulas encapsulated by the new predicate symbol $D_{\hA{A} \structplus \hB{B}}$ of the $\normalizePlusTimesSD$-normal-form of the struct $\struct{\varphi}$ shown in Example \ref{example:PlusTimesSwappeDefinitionalNormalization} are: $\hA{A}$ and $\hB{B}$.
\hfill\QED
\end{example}


% \begin{definition}[Encapsulated Inferences]
% \label{definition:EncapsulatedInferences}
% Let $S$ be a cut-pertinent struct of a proof $\varphi$ and $S'$ be a substruct of $S$. Let $N_{S'}$ be the defined predicate for $S'$. Then, every inference $\rho$ of $\varphi$ which corresponds to a connective $\structplus_{\rho}$ or $\structtimes_{\rho}$ in $S'$ or that is an axiom inference having a formula occurrence of $S'$ in its conclusion sequent is an \emph{encapsulated inference} of $N_{S'}$.
% \end{definition}

% \begin{example}[Encapsulated Inferences]
% \label{example:EncapsulatedInferences}


% The encapsulated inferences of the defined predicate $D_{\hA{A} \structplus \hB{B}}$ of the $\normalizePlusTimesSD$-normal-form of the struct $\struct{\varphi}$ shown in Example \ref{example:PlusTimesSwappeDefinitionalNormalization} are: $\wedge^1_r$ and the axiom inferences having $\hA{A} \seq \hA{A}$ and $\hB{B} \seq \hB{B}$ as conclusion sequents.
% \end{example}


Roughly, constructing a mixed DW-projection is initially similar to constructing an O-projection, taking care to include encapsulated formula occurrences in the slice. Later cut-pertinent inferences are replaced by $\wedge_r$ and $d_r$ inferences, similarly to what is done during the construction of proper D-projections, in order to re-encapsulate the encapsulated formula occurrences into the defined predicate symbol.


\begin{definition}[Mixed DW-Projection]
\label{definition:DWProjectionMixed}
\index{Projection!Mixed DW-Projection}
Let $\varphi$ be a proof and $c$ a mixed clause in $\clausesetSwapDef{\varphi}{S}$. Let $\Omega_E$ and $\Upsilon_E$ be the sets of, respectively, encapsulated formula occurrences and encapsulated inferences of defined predicates occurring in $c$. Let $\Omega_c$ be the set of undefined formula occurrences in $c$. Then the \emph{mixed DW-projection} of $\varphi$ with respect to $c$ can be computed according to the following steps:

\begin{enumerate}
%\item Construct $\varphi^1 \defEq \slice{\varphi}{\Omega_E \cup \Omega_c}$.
\item Replace the inferences of $\Upsilon_E$ in $\varphi^1$ by $\neg_r$, $\wedge_r$, $\vee_r$ and $d_r$ (analogously to what is done in the construction of proper D-projections). Let $\varphi^2$ be the resulting proofoid.
\item Construct $\varphi^3 \defEq \replacePert{\varphi^2}{\occCutPert{\varphi^2}}$ by replacing the cut-pertinent inferences of $\varphi^2$ by $Y$-inferences.
\item Construct $\varphi^4 \defEq \WFix{\varphi^3}$ by fixing broken inferences with weakening.
%\item Finally, construct the mixed DW-projection $\projectionDWMixed{\varphi}{c} \defEq \EliminateY{\varphi^4}$ by eliminating the $Y$-inferences from $\varphi^4$.
\end{enumerate}
\end{definition}


\begin{example}[Mixed DW-Projection]
\label{example:DWProjectionMixed}



Let ${\varphi}$ be the proof shown in Example \ref{example:PlusTimesSwappeDefinitionalNormalization}, which is displayed again for convenience below:

\begin{prooftree}
\AXC{$\hA{A} \seq \hA{A} $}
		\AXC{$\hB{B} \seq \hB{B}$} \RightLabel{$\wedge_r^1$}
	\BIC{$\hA{A}, \hB{B} \seq \hA{A} \wedge \hB{B}$} \RightLabel{$\wedge_l$}
	\UIC{$\hA{A} \wedge \hB{B} \seq \hA{A} \wedge \hB{B}$}
				\AXC{$\hC{B} \seq \hC{B} $}
						\AXC{$\hD{A} \seq \hD{A}$} \RightLabel{$\wedge_r^2$}
					\BIC{$\hD{A}, \hC{B} \seq \hC{B} \wedge \hD{A}$} \RightLabel{$\wedge_l$}
					\UIC{$\hD{A} \wedge \hC{B} \seq \hC{B} \wedge \hD{A}$} \RightLabel{$cut^3$}
			\BIC{$\hA{A} \wedge \hB{B} \seq \hC{B} \wedge \hD{A}$}
							\AXC{$\hE{C} \seq \hE{C}$} 
									\AXC{$\hF{C} \seq \hF{C}$} \RightLabel{$cut^4$}
								\BIC{$\hE{C} \seq \hF{C}$} \RightLabel{$\vee_l^5$}
					\BIC{$(\hA{A} \wedge \hB{B}) \vee \hE{C} \seq \hC{B} \wedge \hD{A}, \hF{C} $} 
\end{prooftree}

The first step in the construction of the mixed DW-projection $\projectionDWMixed{\varphi}{\seq D_{\hA{A} \structplus \hB{B}} , \hE{C}}$ is the slicing with respect to $\Omega_E \cup \Omega_c$ where $\Omega_E = \{\hA{A}, \hB{B}\}$ and $\Omega_c = \{\hE{C}\}$:

\renewcommand{\hC}[1]{\phantom{#1}}
\renewcommand{\hD}[1]{\phantom{#1}}
\renewcommand{\hF}[1]{\phantom{#1}}

\begin{prooftree}
\AXC{$\hA{A} \seq \hA{A} $}
		\AXC{$\hB{B} \seq \hB{B}$} \RightLabel{$\wedge_r^1$}
	\BIC{$\hA{A}, \hB{B} \seq \hA{A} \wedge \hB{B}$} \RightLabel{$\wedge_l$}
	\UIC{$\hA{A} \wedge \hB{B} \seq \hA{A} \wedge \hB{B}$}
				\AXC{$\hC{B} \seq \hC{B} $}
						\AXC{$\hD{A} \seq \hD{A}$} \RightLabel{$Y$}
					\BIC{$\hD{A}\phantom{,} \hC{B} \seq \hC{B} \phantom{\wedge} \hD{A}$} \RightLabel{$Y$}
					\UIC{$\hD{A} \phantom{\wedge} \hC{B} \seq \hC{B} \phantom{\wedge} \hD{A}$} \RightLabel{$Y$}
			\BIC{$\hA{A} \wedge \hB{B} \seq \hA{A} \wedge \hB{B}$}
							\AXC{$\hE{C} \seq \hE{C}$} 
									\AXC{$\hF{C} \seq \hF{C}$} \RightLabel{$Y$}
								\BIC{$\hE{C} \seq \hE{C}$} \RightLabel{$\vee_l^5$}
					\BIC{$(\hA{A} \wedge \hB{B}) \vee \hE{C} \seq \hA{A} \wedge \hB{B}, \hC{B} \phantom{\wedge} \hD{A}\phantom{,} \hE{C} $} 
\end{prooftree}


The second step is the introduction of definition inferences, resulting in the proofoid $\varphi^2$ below:

\begin{prooftree}
\AXC{$\hA{A} \seq \hA{A} $}
		\AXC{$\hB{B} \seq \hB{B}$} \RightLabel{$\wedge_r$}
	\BIC{$\hA{A}, \hB{B} \seq \hA{A} \wedge \hB{B}$} \RightLabel{$d_r$}
	\UIC{$\hA{A}, \hB{B} \seq D_{\hA{A} \structplus \hB{B}}$} \RightLabel{$\wedge_l$}
	\UIC{$\hA{A} \wedge \hB{B} \seq D_{\hA{A} \structplus \hB{B}}$}
				\AXC{$\hC{B} \seq \hC{B} $}
						\AXC{$\hD{A} \seq \hD{A}$} \RightLabel{$Y$}
					\BIC{$\hD{A}\phantom{,} \hC{B} \seq \hC{B} \phantom{\wedge} \hD{A}$} \RightLabel{$Y$}
					\UIC{$\hD{A} \phantom{\wedge} \hC{B} \seq \hC{B} \phantom{\wedge} \hD{A}$} \RightLabel{$Y$}
			\BIC{$\hA{A} \wedge \hB{B} \seq D_{\hA{A} \structplus \hB{B}}$}
							\AXC{$\hE{C} \seq \hE{C}$} 
									\AXC{$\hF{C} \seq \hF{C}$} \RightLabel{$Y$}
								\BIC{$\hE{C} \seq \hE{C}$} \RightLabel{$\vee_l^5$}
					\BIC{$(\hA{A} \wedge \hB{B}) \vee \hE{C} \seq D_{\hA{A} \structplus \hB{B}},  \hC{B} \phantom{\wedge} \hD{A}\phantom{,} \hE{C} $} 
\end{prooftree}

Subsequently, cut-pertinent inferences of $\varphi^2$ should be replaced by $Y$-inferences. However, since $\varphi^2$ has no cuts, there is nothing to be replaced, and hence $\varphi^3 = \varphi^2$. Subsequently, broken inferences of $\varphi^3$ should be W-fixed. However, there are no broken inferences in $\varphi^3$. Therefore, only the last step of eliminating $Y$-inferences remains and its result is the mixed DW-projection $\projectionDWMixed{\varphi}{\seq D_{\hA{A} \structplus \hB{B}} , \hE{C}}$ shown below:

\begin{prooftree}
\AXC{$\hA{A} \seq \hA{A} $}
		\AXC{$\hB{B} \seq \hB{B}$} \RightLabel{$\wedge_r$}
	\BIC{$\hA{A}, \hB{B} \seq \hA{A} \wedge \hB{B}$} \RightLabel{$d_r$}
	\UIC{$\hA{A}, \hB{B} \seq D_{\hA{A} \structplus \hB{B}}$} \RightLabel{$\wedge_l$}
	\UIC{$\hA{A} \wedge \hB{B} \seq D_{\hA{A} \structplus \hB{B}}$}
							\AXC{$\hE{C} \seq \hE{C}$} \RightLabel{$\vee_l^5$}
					\BIC{$(\hA{A} \wedge \hB{B}) \vee \hE{C} \seq D_{\hA{A} \structplus \hB{B}}, \hE{C} $} 
\end{prooftree}

\end{example}


\begin{example}[$\CEResDSwapDSwap$-Normal-Form]

Consider again the swapped definitional clause set of the proof $\varphi$ shown in Example \ref{example:SwappeDefinitionalClauseSet}:

$$
\clausesetSwapDefUnique{\varphi} \equiv  \left\{
	\seq D_{\hA{A} \structplus \hB{B}} , \hE{C} 
\ \ \ ; \ \
	\hF{C} \seq 
\ \ \ ; \ \
	\hC{B} , \hD{A} \seq
\ \ \ ; \ \
	D_{\hA{A} \structplus \hB{B}} \seq \hA{A}  
\ \ \ ; \ \
	D_{\hA{A} \structplus \hB{B}} \seq \hB{B} 
\ \ \ ; \ \
	\hA{A}, \hB{B} \seq D_{\hA{A} \structplus \hB{B}} 
\right\}
$$

The shortest refutation $\delta$ of $\clausesetSwapDefUnique{\varphi}$ is shown below:

\begin{prooftree}
\AXC{$\seq D_{\hA{A} \structplus \hB{B}} , \hE{C}$}
		\AXC{$\hF{C} \seq$} \RightLabel{$r$}
	\BIC{$\seq D_{\hA{A} \structplus \hB{B}}$}
				\AXC{$D_{\hA{A} \structplus \hB{B}} \seq \hA{A}$}
						\AXC{$D_{\hA{A} \structplus \hB{B}} \seq \hB{B}$}
								\AXC{$\hC{B} , \hD{A} \seq$} \RightLabel{$r$}
							\BIC{$D_{\hA{A} \structplus \hB{B}}, \hD{A} \seq $} \RightLabel{$r$}
					\BIC{$D_{\hA{A} \structplus \hB{B}}, D_{\hA{A} \structplus \hB{B}} \seq $} \RightLabel{$f_l$}
					\UIC{$D_{\hA{A} \structplus \hB{B}} \seq $} \RightLabel{$r$}
			\BIC{$ \seq $}
\end{prooftree}

By using the mixed DW-projection shown in Example \ref{example:DWProjectionMixed}, pure DW-projections shown in Example \ref{example:OProjections} and definitional DW-projections shown in Example \ref{example:DProjectionDefinitional}, $\CEResNFDSwapDSwap{\varphi}{\delta}$ is:


%\begin{small}
\begin{prooftree}
\AXC{$\hA{A} \seq \hA{A} $}
		\AXC{$\hB{B} \seq \hB{B}$} \RightLabel{$\wedge_r$}
	\BIC{$\hA{A}, \hB{B} \seq \hA{A} \wedge \hB{B}$} \RightLabel{$d_r$}
	\UIC{$\hA{A}, \hB{B} \seq D_{\hA{A} \structplus \hB{B}}$} \RightLabel{$\wedge_l$}
	\UIC{$\hA{A} \wedge \hB{B} \seq D_{\hA{A} \structplus \hB{B}}$}
							\AXC{$\hE{C} \seq \hE{C}$} \RightLabel{$\vee_l^5$}
					\BIC{$(\hA{A} \wedge \hB{B}) \vee \hE{C} \seq D_{\hA{A} \structplus \hB{B}}, \hE{C} $} 
               		\AXC{$\hF{C} \seq \hF{C}$} \RightLabel{$cut$}
               	\BIC{$(\hA{A} \wedge \hB{B}) \vee \hE{C} \seq D_{\hA{A} \structplus \hB{B}}, \hF{C}$}
	            				\AXC{$\hA{A} \seq \hA{A}$} \RightLabel{$w_l$}
               				\UIC{$\hA{A}, \hB{B} \seq \hA{A}$} \RightLabel{$\wedge_l$}
               				\UIC{$\hA{A} \wedge \hB{B} \seq \hA{A}$} \RightLabel{$d_l$}
               				\UIC{$D_{\hA{A} \structplus \hB{B}} \seq \hA{A}$}
               						\AXC{$\hB{B} \seq \hB{B}$} \RightLabel{$w_l$}
               						\UIC{$\hA{A}, \hB{B} \seq \hB{B}$} \RightLabel{$\wedge_l$}
               						\UIC{$\hA{A} \wedge \hB{B} \seq \hB{B}$} \RightLabel{$d_l$}
               						\UIC{$D_{\hA{A} \structplus \hB{B}} \seq \hB{B}$}
               								\AXC{$\hC{B} \seq \hC{B} $}
															\AXC{$\hD{A} \seq \hD{A}$} \RightLabel{$\wedge_r$}
														\BIC{$ \hC{B}, \hD{A} \seq \hC{B} \wedge \hD{A}$} \RightLabel{$cut$}
               							\BIC{$D_{\hA{A} \structplus \hB{B}}, \hD{A} \seq \hC{B} \wedge \hD{A}$} \RightLabel{$cut$}
               					\BIC{$D_{\hA{A} \structplus \hB{B}}, D_{\hA{A} \structplus \hB{B}} \seq \hC{B} \wedge \hD{A}$} \RightLabel{$c_l$}
               					\UIC{$D_{\hA{A} \structplus \hB{B}} \seq \hC{B} \wedge \hD{A}$} \RightLabel{$cut$}
               			\BIC{$(\hA{A} \wedge \hB{B}) \vee \hE{C} \seq \hC{B} \wedge \hD{A}, \hF{C}$}
\end{prooftree}
%\end{small}


\end{example}


\section{Ignoring Atomic and Quantifier-Free Cuts}
\label{sec:CutEliminationByResolution:CEResIgnoringAtomicCuts}

If $\CERes$ is applied to a proof containing only atomic cuts, $\CERes$ still transforms the proof into a new proof containing only atomic cuts, but with additional structural inferences and with the atomic cuts located in the bottom of the proof. This is clearly non-ideal, because the proof could be simply left unchanged. More generally, if $\CERes$ is applied to a proof containing complex cuts and atomic cuts, $\CERes$ unnecessarily includes the atomic cuts in the process of reduction, even though atomic cuts cannot be reduced further. The inclusion of atomic cuts results in larger clause sets that are more costly to refute, and in normal forms with possibly additional structural inferences. This indicates that there is a very simple and evident improvement of the $\CERes$ method that has been thoroughly overlooked so far: instead of distinguishing between cut-pertinent and cut-impertinent formula occurrences (i.e. between ancestors and non-ancestors of \emph{all} cut formula occurrences) and cut-pertinent and cut-impertinent inferences (i.e inferences that operate on the ancestors and on the non-ancestors of cut formula occurrences), it suffices to distinguish between ancestors of \emph{complex} cut formula occurrences and ancestors of either occurrences in the end-sequent or of atomic cut-formula occurrences.



\section{Ignoring Quantifier-Free Cuts}
\label{sec:CutEliminationByResolution:CEResIgnoringAtomicCuts}

When all that is desired is the possibility to 
summarize the proof by means of Herbrand sequents 
\cite{Paleo2007Herbrand-Sequent-Extraction,Paleo2008Herbrand-Sequent-Extraction,HetzlLeitschWellerPaleo2008Herbrand-Sequent-Extraction}, 
it is only necessary to remove cuts that contain quantifiers. 
This can be achieved with an easy modification of $\CERes$: 
in the construction of the characteristic formula and of the projections, 
ancestors of quantifier-free cuts should be treated in the same way 
as end-sequent ancestors.

Any of the previously defined variants of $\CERes$ can be modified in this
manner. When this is done, more axioms are mapped to $\bot$ in the
characteristic formula, and some inferences that were previously mapped to
conjunctions are now mapped to disjunctions. Consequently, fewer
distributions, redundant duplications and new predicate symbols are necessary.
This leads to smaller and more easily refutable clause sets.

\section{Conclusion}
\label{sec:Conclusion}


Remark about difficulties in intuitionistic ceres

\appendix

\section{Sequent Calculus}
\label{sec:Calculus}

\newcommand{\m}[1]{#1}

The sequent calculus used in this paper is shown below. 
$\m{\Gamma}$,$\m{\Delta}$,$\m{\Gamma_1}$,$\m{\Delta_1}$,$\m{\Gamma_2}$,$\m{\Delta_2}$
are multisets of formulas called \emph{contexts}. 
For each rule, the active formula below the line, colored in red, 
is its \emph{main} formula, while the active formulas in the premises, 
colored in blue, are its \emph{auxiliary} formulas.


\begin{scriptsize}
$$
\infer[axiom]{\hA{\m{A}} \seq \hA{\m{A}}}{ }
\qquad
\infer[cut]{\m{\Gamma_1},\m{\Gamma_2}\seq\m{\Delta_1},\m{\Delta_2}}{
    \m{\Gamma_1}\seq\m{\Delta_1},\hB{\m{F}}
    &
    \hB{\m{F}},\m{\Gamma_2}\seq\m{\Gamma_2}
}
$$

    $$
    \infer[w_l]{\hA{\m{F}},\m{\Gamma}\seq\m{\Delta}}{\m{\Gamma}\seq\m{\Delta}}
    \qquad
    \infer[w_r]{\m{\Gamma}\seq\m{\Delta},\hA{\m{F}}}{\m{\Gamma}\seq\m{\Delta}}
    \qquad
    \infer[c_l]{ \hA{\m{F}}, \m{\Gamma} \seq \m{\Delta}} { \hB{\m{F}}, \hB{\m{F}}, \m{\Gamma}
    \seq \m{\Delta}}
    \qquad
    \infer[c_r]{\m{\Gamma}\seq \m{\Delta}, \hA{\m{F}}} {\m{\Gamma}\seq
    \m{\Delta}, \hB{\m{F}}, \hB{\m{F}}}
    $$

$$
\infer[\wedge_l]{\hA{\m{F_1}\land \m{F_2}},\m{\Gamma}\seq\m{\Delta}}{
    \hB{\m{F_1}},\hB{\m{F_2}},\m{\Gamma}\seq\m{\Delta}
}
\qquad
\infer[\wedge_r]{\m{\Gamma_1},\m{\Gamma_2}\seq\m{\Delta_1},\m{\Delta_2},\hA{\m{F_1}\land \m{F_2}}}{
    \m{\Gamma_1}\seq\m{\Delta_1},\hB{\m{F_1}}
    &
    \m{\Gamma_2}\seq\m{\Delta_2},\hB{\m{F_2}}
}
\qquad
\infer[\neg_l]{\hA{\neg \m{F}},\m{\Gamma}\seq\m{\Delta}}{\m{\Gamma}\seq\m{\Delta}, \hB{\m{F}}}
$$

$$
\infer[\vee_l]{\hA{\m{F_1}\lor \m{F_2}},\m{\Gamma_1},\m{\Gamma_2}\seq\m{\Delta_1},\m{\Delta_2}}{
    \hB{\m{F_1}},\m{\Gamma_1}\seq\m{\Delta_1}
    &
    \hB{\m{F_2}},\m{\Gamma_2}\seq \m{\Delta_2}
}
\qquad
\infer[\vee_r]{\m{\Gamma}\seq\m{\Delta},\hA{\m{F_1}\lor \m{F_2}}}{
    \m{\Gamma}\seq\m{\Delta},\hB{\m{F_1}},\hB{\m{F_2}}
}
\qquad
\infer[\neg_r]{\m{\Gamma}\seq\m{\Delta},\hA{\neg \m{F}}}{\hB{\m{F}},\m{\Gamma}\seq\m{\Delta}}
$$

$$
\infer[\imp_l]{\hA{\m{F_1}\imp \m{F_2}},\m{\Gamma_1},\m{\Gamma_2}\seq\m{\Delta_1},\m{\Delta_2}}{
    \m{\Gamma_1}\seq\m{\Delta_1},\hB{\m{F_1}}
    &
    \hB{\m{F_2}},\m{\Gamma_2}\seq \m{\Delta_2}
}
\qquad
\infer[\imp_r]{\m{\Gamma}\seq\m{\Delta},\hA{\m{F_1}\imp \m{F_2}}}{
    \hB{\m{F_1}},\m{\Gamma}\seq\m{\Delta},\hB{\m{F_2}}
}
$$

$$
\infer[\forall_l]{\hA{(\forall x)\m{F}},\m{\Gamma}\seq\m{\Delta}}{\hB{\m{F}\subst{x}{t}},\m{\Gamma}\seq\m{\Delta}}
\qquad
\infer[\forall_r]{\m{\Gamma}\seq\m{\Delta},\hA{(\forall x)\m{F}}}{\m{\Gamma}\seq\m{\Delta},\hB{\m{F}\subst{x}{\alpha}}}
\qquad
\infer[\exists_l]{\hA{(\exists x)\m{F}},\m{\Gamma}\seq\m{\Delta}}{\hB{\m{F}\subst{x}{\alpha}},\m{\Gamma}\seq\m{\Delta}}
\qquad
\infer[\exists_r]{\m{\Gamma}\seq\m{\Delta},\hA{(\exists x)\m{F}}}{\m{\Gamma}\seq\m{\Delta},\hB{\m{F}\subst{x}{t}}}
$$
%
\begin{center}
$\alpha$ should occur neither in $\m{\Gamma}$ nor in $\m{\Delta}$ nor in $\m{F}$.
$t$ must not contain a variable that is bound in $\m{F}$.
\end{center}


$$
\infer[d_l]{\hA{\m{P(x_1,\ldots, x_n)}},\m{\Gamma}\seq\m{\Delta}}
{\hB{\m{F[x_1,\ldots, x_n]}},\m{\Gamma}\seq\m{\Delta}}
\qquad
\infer[d_r]{\m{\Gamma}\seq\m{\Delta},\hA{\m{P(x_1,\ldots, x_n)}}}
{\m{\Gamma}\seq\m{\Delta},\hB{\m{F[x_1,\ldots, x_n]}}}
$$
%
\begin{center}
where $x_1, \ldots, x_n$ are the free variables of $F$ and $P$ is defined by:
$P(x_1,\ldots, x_n) \biimp F[x_1,\ldots, x_n]$.
\end{center}
\end{scriptsize}



\section{Inference Permutation}
\label{appendix:InferencePermutation}

In this section, a proof rewriting system (Definition \ref{definition:Swapping}) for inference swapping
is described. It is subdivided according to the kind of dependence (Definition \ref{definition:InferenceDependence}) between the inferences that are being swapped. If the lower inference is independent of the upper inference, then they can easily be swapped (Definition \ref{definition:SwappingOfIndependentInferences}), with no increase of proof size. However, if the lower inference is indirectly dependent on the upper inference, then swapping requires a duplication of the lower inference, as well as the introduction of weakening and contraction inferences (Definition \ref{definition:SwappingOfIndirectlyDependentInferences}). The case of eigen-variable dependence can be avoided by considering skolemized proofs only. Even though two inferences cannot generally be swapped if there is direct dependence between them, swapping is possible in the particular case when the upper inference is a contraction (Definition \ref{definition:SwappingContraction}) or a weakening (downward swapping of weakening inferences, Definition \ref{definition:SwappingContraction}). 



\begin{definition}[Inference Dependence]
\label{definition:InferenceDependence}
\index{Inference Dependence}
An inference $\rho_1$ is \emph{directly dependent} on another inference $\rho_2$, denoted $\rho_1 \dependentD \rho_2$, if and only if a main occurrence of $\rho_2$ is an ancestor of an auxiliary occurrence of $\rho_1$. 

A strong quantifier inference $\rho_1$ is \emph{eigenvariable-dependent} on another inference $\rho_2$ occurring above $\rho_1$, denoted $\rho_1 \dependentQ \rho_2$, if and only if the substitution term of $\rho_2$ contains an occurrence of the eigenvariable of $\rho_1$.

An inference $\rho_1$ is \emph{indirectly dependent} on another inference $\rho_2$ occurring above $\rho_1$, denoted $\rho_1 \dependentI \rho_2$, if and only if it is not directly dependent on $\rho_2$ and the auxiliary occurrences of $\rho_1$ have ancestors in more than one premise sequent of $\rho_2$.

An inference $\rho_1$ is \emph{independent} of another inference $\rho_2$ if and only if $\rho_1$ is neither directly dependent nor eigenvariable-dependent nor indirectly dependent on $\rho_2$. 
\end{definition}





\begin{definition}[$\swapI$]
\label{definition:SwappingOfIndependentInferences}
Swapping of Independent Inferences:

\begin{prooftree}
\AXC{$\varphi_1$}\noLine
\UIC{$\hA{\Gamma_1^{\rho_1}}, \hB{\Gamma_1^{\rho_2}}, \Gamma_1  \seq \hA{\Delta_1^{\rho_1}}, \hB{\Delta_1^{\rho_2}}, \Delta_1$}\RightLabel{$\rho_1$}
\UIC{$\hA{\Gamma^{\rho_1}}, \hB{\Gamma_1^{\rho_2}}, \Gamma_1  \seq \hA{\Delta^{\rho_1}}, \hB{\Delta_1^{\rho_2}}, \Delta_1$}\RightLabel{$\rho_2$}
\UIC{$\hA{\Gamma^{\rho_1}}, \hB{\Gamma^{\rho_2}}, \Gamma_1  \seq \hA{\Delta^{\rho_1}}, \hB{\Delta^{\rho_2}}, \Delta_1$} 
\end{prooftree}
$$
\Downarrow
$$
\begin{prooftree}
\AXC{$\varphi_1$}\noLine
\UIC{$\hA{\Gamma_1^{\rho_1}}, \hB{\Gamma_1^{\rho_2}}, \Gamma_1  \seq \hA{\Delta_1^{\rho_1}}, \hB{\Delta_1^{\rho_2}}, \Delta_1$}\RightLabel{$\rho_2$}
\UIC{$\hA{\Gamma_1^{\rho_1}}, \hB{\Gamma^{\rho_2}}, \Gamma_1  \seq \hA{\Delta_1^{\rho_1}}, \hB{\Delta^{\rho_2}}, \Delta_1$}  \RightLabel{$\rho_1$}
\UIC{$\hA{\Gamma^{\rho_1}}, \hB{\Gamma^{\rho_2}}, \Gamma_1 \seq \hA{\Delta^{\rho_1}}, \hB{\Delta^{\rho_2}}, \Delta_1$} 
\end{prooftree}


\begin{prooftree}
\AXC{$\varphi_1$}\noLine
\UIC{$\hA{\Gamma_1^{\rho_1}}, \hB{\Gamma_1^{\rho_2}}, \Gamma_1  \seq \hA{\Delta_1^{\rho_1}}, \hB{\Delta_1^{\rho_2}}, \Delta_1$}
		\AXC{$\varphi_2$}\noLine
		\UIC{$\hA{\Gamma_2^{\rho_1}}, \Gamma_2  \seq \hA{\Delta_2^{\rho_1}}, \Delta_2$}  \RightLabel{$\rho_1$}
	\BIC{$\hA{\Gamma^{\rho_1}}, \hB{\Gamma_1^{\rho_2}}, \Gamma_1, \Gamma_2  \seq \hA{\Delta^{\rho_1}}, \hB{\Delta_1^{\rho_2}}, \Delta_1, \Delta_2$}\RightLabel{$\rho_2$}
	\UIC{$\hA{\Gamma^{\rho_1}}, \hB{\Gamma^{\rho_2}}, \Gamma_1, \Gamma_2  \seq \hA{\Delta^{\rho_1}}, \hB{\Delta^{\rho_2}}, \Delta_1, \Delta_2$} 
\end{prooftree}
$$
\Downarrow
$$
\begin{prooftree}
\AXC{$\varphi_1$}\noLine
\UIC{$\hA{\Gamma_1^{\rho_1}}, \hB{\Gamma_1^{\rho_2}}, \Gamma_1  \seq \hA{\Delta_1^{\rho_1}}, \hB{\Delta_1^{\rho_2}}, \Delta_1$}\RightLabel{$\rho_2$}
\UIC{$\hA{\Gamma_1^{\rho_1}}, \hB{\Gamma^{\rho_2}}, \Gamma_1  \seq \hA{\Delta_1^{\rho_1}}, \hB{\Delta^{\rho_2}}, \Delta_1$}
				\AXC{$\varphi_2$}\noLine
				\UIC{$\hA{\Gamma_2^{\rho_1}}, \Gamma_2  \seq \hA{\Delta_2^{\rho_1}}, \Delta_2$}  \RightLabel{$\rho_1$}
		\BIC{$\hA{\Gamma^{\rho_1}}, \hB{\Gamma^{\rho_2}}, \Gamma_1, \Gamma_2 \seq \hA{\Delta^{\rho_1}}, \hB{\Delta^{\rho_2}}, \Delta_1, \Delta_2$} 
\end{prooftree}



%\begin{prooftree}
%\AXC{$\varphi_2$}\noLine
%\UIC{$\hA{\Gamma_2^{\rho_1}}, \Gamma_2  \seq \hA{\Delta_2^{\rho_1}}, \Delta_2$} 
%		\AXC{$\varphi_1$}\noLine
%		\UIC{$\hA{\Gamma_1^{\rho_1}}, \hB{\Gamma_1^{\rho_2}}, \Gamma_1  \seq \hA{\Delta_1^{\rho_1}}, \hB{\Delta_1^{\rho_2}}, \Delta_1$} \RightLabel{$\rho_1$}
%	\BIC{$\hA{\Gamma^{\rho_1}}, \hB{\Gamma_1^{\rho_2}}, \Gamma_1, \Gamma_2  \seq \hA{\Delta^{\rho_1}}, \hB{\Delta_1^{\rho_2}}, \Delta_1, \Delta_2$}\RightLabel{$\rho_2$}
%	\UIC{$\hA{\Gamma^{\rho_1}}, \hB{\Gamma^{\rho_2}}, \Gamma_1, \Gamma_2  \seq \hA{\Delta^{\rho_1}}, \hB{\Delta^{\rho_2}}, \Delta_1, \Delta_2$} 
%\end{prooftree}
%$$
%\Downarrow
%$$
%\begin{prooftree}
%\AXC{$\varphi_2$}\noLine
%\UIC{$\hA{\Gamma_2^{\rho_1}}, \Gamma_2  \seq \hA{\Delta_2^{\rho_1}}, \Delta_2$}  
%			\AXC{$\varphi_1$}\noLine
%			\UIC{$\hA{\Gamma_1^{\rho_1}}, \hB{\Gamma_1^{\rho_2}}, \Gamma_1  \seq \hA{\Delta_1^{\rho_1}}, \hB{\Delta_1^{\rho_2}}, \Delta_1$}\RightLabel{$\rho_2$}
%			\UIC{$\hA{\Gamma_1^{\rho_1}}, \hB{\Gamma^{\rho_2}}, \Gamma_1  \seq \hA{\Delta_1^{\rho_1}}, \hB{\Delta^{\rho_2}}, \Delta_1$} \RightLabel{$\rho_1$}
%		\BIC{$\hA{\Gamma^{\rho_1}}, \hB{\Gamma^{\rho_2}}, \Gamma_1, \Gamma_2 \seq \hA{\Delta^{\rho_1}}, \hB{\Delta^{\rho_2}}, \Delta_1, \Delta_2$} 
%\end{prooftree}




\begin{prooftree}
\AXC{$\varphi_1$}\noLine
\UIC{$\hA{\Gamma_1^{\rho_1}}, \hB{\Gamma_1^{\rho_2}}, \Gamma_1  \seq \hA{\Delta_1^{\rho_1}}, \hB{\Delta_1^{\rho_2}}, \Delta_1$}  \RightLabel{$\rho_1$}
\UIC{$\hA{\Gamma^{\rho_1}}, \hB{\Gamma_1^{\rho_2}}, \Gamma_1  \seq \hA{\Delta^{\rho_1}}, \hB{\Delta_1^{\rho_2}}, \Delta_1$}
				\AXC{$\varphi_2$}\noLine
				\UIC{$\hB{\Gamma_2^{\rho_2}}, \Gamma_2  \seq \hB{\Delta_2^{\rho_2}}, \Delta_2$} \RightLabel{$\rho_2$}
			\BIC{$\hA{\Gamma^{\rho_1}}, \hB{\Gamma^{\rho_2}}, \Gamma_1, \Gamma_2  \seq \hA{\Delta^{\rho_1}}, \hB{\Delta^{\rho_2}}, \Delta_1, \Delta_2$} 
\end{prooftree}
$$
\Downarrow
$$
\begin{prooftree}
\AXC{$\varphi_1$}\noLine
\UIC{$\hA{\Gamma_1^{\rho_1}}, \hB{\Gamma_1^{\rho_2}}, \Gamma_1  \seq \hA{\Delta_1^{\rho_1}}, \hB{\Delta_1^{\rho_2}}, \Delta_1$}
		\AXC{$\varphi_2$}\noLine
		\UIC{$\hB{\Gamma_2^{\rho_2}}, \Gamma_2  \seq \hB{\Delta_2^{\rho_2}}, \Delta_2$} \RightLabel{$\rho_2$}
	\BIC{$\hA{\Gamma_1^{\rho_1}}, \hB{\Gamma^{\rho_2}}, \Gamma_1, \Gamma_2  \seq \hA{\Delta_1^{\rho_1}}, \hB{\Delta^{\rho_2}}, \Delta_1, \Delta_2$} \RightLabel{$\rho_1$}
	\UIC{$\hA{\Gamma^{\rho_1}}, \hB{\Gamma^{\rho_2}}, \Gamma_1, \Gamma_2  \seq \hA{\Delta^{\rho_1}}, \hB{\Delta^{\rho_2}}, \Delta_1, \Delta_2$} 
\end{prooftree}

%\begin{prooftree}
%\AXC{$\varphi_2$}\noLine
%\UIC{$\hB{\Gamma_2^{\rho_2}}, \Gamma_2  \seq \hB{\Delta_2^{\rho_2}}, \Delta_2$}
%		\AXC{$\varphi_1$}\noLine
%		\UIC{$\hA{\Gamma_1^{\rho_1}}, \hB{\Gamma_1^{\rho_2}}, \Gamma_1  \seq \hA{\Delta_1^{\rho_1}}, \hB{\Delta_1^{\rho_2}}, \Delta_1$}  \RightLabel{$\rho_1$}
%		\UIC{$\hA{\Gamma^{\rho_1}}, \hB{\Gamma_1^{\rho_2}}, \Gamma_1  \seq \hA{\Delta^{\rho_1}}, \hB{\Delta_1^{\rho_2}}, \Delta_1$} \RightLabel{$\rho_2$}
%	\BIC{$\hA{\Gamma^{\rho_1}}, \hB{\Gamma^{\rho_2}}, \Gamma_1, \Gamma_2  \seq \hA{\Delta^{\rho_1}}, \hB{\Delta^{\rho_2}}, \Delta_1, \Delta_2$} 
%\end{prooftree}
%$$
%\Downarrow
%$$
%\begin{prooftree}
%\AXC{$\varphi_2$}\noLine
%\UIC{$\hB{\Gamma_2^{\rho_2}}, \Gamma_2  \seq \hB{\Delta_2^{\rho_2}}, \Delta_2$} 
%		\AXC{$\varphi_1$}\noLine
%		\UIC{$\hA{\Gamma_1^{\rho_1}}, \hB{\Gamma_1^{\rho_2}}, \Gamma_1  \seq \hA{\Delta_1^{\rho_1}}, \hB{\Delta_1^{\rho_2}}, \Delta_1$} \RightLabel{$\rho_2$}
%	\BIC{$\hA{\Gamma_1^{\rho_1}}, \hB{\Gamma^{\rho_2}}, \Gamma_1, \Gamma_2  \seq \hA{\Delta_1^{\rho_1}}, \hB{\Delta^{\rho_2}}, \Delta_1, \Delta_2$} \RightLabel{$\rho_1$}
%	\UIC{$\hA{\Gamma^{\rho_1}}, \hB{\Gamma^{\rho_2}}, \Gamma_1, \Gamma_2  \seq \hA{\Delta^{\rho_1}}, \hB{\Delta^{\rho_2}}, \Delta_1, \Delta_2$} 
%\end{prooftree}



\begin{prooftree}
\AXC{$\varphi_1$}\noLine
\UIC{$\hA{\Gamma_1^{\rho_1}}, \hB{\Gamma_1^{\rho_2}}, \Gamma_1  \seq \hA{\Delta_1^{\rho_1}}, \hB{\Delta_1^{\rho_2}}, \Delta_1$}
		\AXC{$\varphi_2$}\noLine
		\UIC{$\hA{\Gamma_2^{\rho_1}}, \Gamma_2  \seq \hA{\Delta_2^{\rho_1}}, \Delta_2$}  \RightLabel{$\rho_1$}
	\BIC{$\hA{\Gamma^{\rho_1}}, \hB{\Gamma_1^{\rho_2}}, \Gamma_1, \Gamma_2  \seq \hA{\Delta^{\rho_1}}, \hB{\Delta_1^{\rho_2}}, \Delta_1, \Delta_2$}
				\AXC{$\varphi_3$}\noLine
				\UIC{$\hB{\Gamma_3^{\rho_2}}, \Gamma_3  \seq \hB{\Delta_3^{\rho_2}}, \Delta_3$} \RightLabel{$\rho_2$}
			\BIC{$\hA{\Gamma^{\rho_1}}, \hB{\Gamma^{\rho_2}}, \Gamma_1, \Gamma_2, \Gamma_3  \seq \hA{\Delta^{\rho_1}}, \hB{\Delta^{\rho_2}}, \Delta_1, \Delta_2, \Delta_3$} 
\end{prooftree}
$$
\Downarrow
$$
\begin{prooftree}
\AXC{$\varphi_1$}\noLine
\UIC{$\hA{\Gamma_1^{\rho_1}}, \hB{\Gamma_1^{\rho_2}}, \Gamma_1  \seq \hA{\Delta_1^{\rho_1}}, \hB{\Delta_1^{\rho_2}}, \Delta_1$}
		\AXC{$\varphi_3$}\noLine
		\UIC{$\hB{\Gamma_3^{\rho_2}}, \Gamma_3  \seq \hB{\Delta_3^{\rho_2}}, \Delta_3$} \RightLabel{$\rho_2$}
	\BIC{$\hA{\Gamma_1^{\rho_1}}, \hB{\Gamma^{\rho_2}}, \Gamma_1, \Gamma_3  \seq \hA{\Delta_1^{\rho_1}}, \hB{\Delta^{\rho_2}}, \Delta_1, \Delta_3$}
				\AXC{$\varphi_2$}\noLine
				\UIC{$\hA{\Gamma_2^{\rho_1}}, \Gamma_2  \seq \hA{\Delta_2^{\rho_1}}, \Delta_2$}  \RightLabel{$\rho_1$}
		\BIC{$\hA{\Gamma^{\rho_1}}, \hB{\Gamma^{\rho_2}}, \Gamma_1, \Gamma_2, \Gamma_3  \seq \hA{\Delta^{\rho_1}}, \hB{\Delta^{\rho_2}}, \Delta_1, \Delta_2, \Delta_3$} 
\end{prooftree}

%\begin{prooftree}
%\AXC{$\varphi_1$}\noLine
%\UIC{$\hA{\Gamma_1^{\rho_1}}, \Gamma_1  \seq \hA{\Delta_1^{\rho_1}}, \Delta_1$}
%		\AXC{$\varphi_2$}\noLine
%		\UIC{$\hA{\Gamma_2^{\rho_1}}, \hB{\Gamma_2^{\rho_2}}, \Gamma_2  \seq \hA{\Delta_2^{\rho_1}}, \hB{\Delta_2^{\rho_2}}, \Delta_2$}  \RightLabel{$\rho_1$}
%	\BIC{$\hA{\Gamma^{\rho_1}}, \hB{\Gamma_2^{\rho_2}}, \Gamma_1, \Gamma_2  \seq \hA{\Delta^{\rho_1}}, \hB{\Delta_2^{\rho_2}}, \Delta_1, \Delta_2$}
%				\AXC{$\varphi_3$}\noLine
%				\UIC{$\hB{\Gamma_3^{\rho_2}}, \Gamma_3  \seq \hB{\Delta_3^{\rho_2}}, \Delta_3$} \RightLabel{$\rho_2$}
%			\BIC{$\hA{\Gamma^{\rho_1}}, \hB{\Gamma^{\rho_2}}, \Gamma_1, \Gamma_2, \Gamma_3  \seq \hA{\Delta^{\rho_1}}, \hB{\Delta^{\rho_2}}, \Delta_1, \Delta_2, \Delta_3$} 
%\end{prooftree}
%$$
%\Downarrow
%$$
%\begin{prooftree}
%\AXC{$\varphi_1$}\noLine
%\UIC{$\hA{\Gamma_1^{\rho_1}}, \Gamma_1  \seq \hA{\Delta_1^{\rho_1}}, \Delta_1$}
%		\AXC{$\varphi_2$}\noLine
%		\UIC{$\hA{\Gamma_2^{\rho_1}}, \hB{\Gamma_2^{\rho_2}}, \Gamma_2  \seq \hA{\Delta_2^{\rho_1}}, \hB{\Delta_2^{\rho_2}}, \Delta_2$}  
%				\AXC{$\varphi_3$}\noLine
%				\UIC{$\hB{\Gamma_3^{\rho_2}}, \Gamma_3  \seq \hB{\Delta_3^{\rho_2}}, \Delta_3$} \RightLabel{$\rho_2$}
%			\BIC{$\hA{\Gamma_2^{\rho_1}}, \hB{\Gamma^{\rho_2}}, \Gamma_2, \Gamma_3  \seq \hA{\Delta_2^{\rho_1}}, \hB{\Delta^{\rho_2}}, \Delta_2, \Delta_3$} 
%\RightLabel{$\rho_1$}
%	\BIC{$\hA{\Gamma^{\rho_1}}, \hB{\Gamma^{\rho_2}}, \Gamma_1, \Gamma_2, \Gamma_3  \seq \hA{\Delta^{\rho_1}}, \hB{\Delta^{\rho_2}}, \Delta_1, \Delta_2, \Delta_3$}
%\end{prooftree}
%
%\begin{prooftree}
%\AXC{$\varphi_3$}\noLine
%\UIC{$\hB{\Gamma_3^{\rho_2}}, \Gamma_3  \seq \hB{\Delta_3^{\rho_2}}, \Delta_3$}
%		\AXC{$\varphi_1$}\noLine
%		\UIC{$\hA{\Gamma_1^{\rho_1}}, \hB{\Gamma_1^{\rho_2}}, \Gamma_1  \seq \hA{\Delta_1^{\rho_1}}, \hB{\Delta_1^{\rho_2}}, \Delta_1$}
%				\AXC{$\varphi_2$}\noLine
%				\UIC{$\hA{\Gamma_2^{\rho_1}}, \Gamma_2  \seq \hA{\Delta_2^{\rho_1}}, \Delta_2$}  \RightLabel{$\rho_1$}
%			\BIC{$\hA{\Gamma^{\rho_1}}, \hB{\Gamma_1^{\rho_2}}, \Gamma_1, \Gamma_2  \seq \hA{\Delta^{\rho_1}}, \hB{\Delta_1^{\rho_2}}, \Delta_1, \Delta_2$}
%				 \RightLabel{$\rho_2$}
%	\BIC{$\hA{\Gamma^{\rho_1}}, \hB{\Gamma^{\rho_2}}, \Gamma_1, \Gamma_2, \Gamma_3  \seq \hA{\Delta^{\rho_1}}, \hB{\Delta^{\rho_2}}, \Delta_1, \Delta_2, \Delta_3$} 
%\end{prooftree}
%$$
%\Downarrow
%$$
%\begin{prooftree}
%\AXC{$\varphi_3$}\noLine
%\UIC{$\hB{\Gamma_3^{\rho_2}}, \Gamma_3  \seq \hB{\Delta_3^{\rho_2}}, \Delta_3$}
%		\AXC{$\varphi_1$}\noLine
%		\UIC{$\hA{\Gamma_1^{\rho_1}}, \hB{\Gamma_1^{\rho_2}}, \Gamma_1  \seq \hA{\Delta_1^{\rho_1}}, \hB{\Delta_1^{\rho_2}}, \Delta_1$} \RightLabel{$\rho_2$}
%	\BIC{$\hA{\Gamma_1^{\rho_1}}, \hB{\Gamma^{\rho_2}}, \Gamma_1, \Gamma_3  \seq \hA{\Delta_1^{\rho_1}}, \hB{\Delta^{\rho_2}}, \Delta_1, \Delta_3$}
%				\AXC{$\varphi_2$}\noLine
%				\UIC{$\hA{\Gamma_2^{\rho_1}}, \Gamma_2  \seq \hA{\Delta_2^{\rho_1}}, \Delta_2$}  \RightLabel{$\rho_1$}
%		\BIC{$\hA{\Gamma^{\rho_1}}, \hB{\Gamma^{\rho_2}}, \Gamma_1, \Gamma_2, \Gamma_3  \seq \hA{\Delta^{\rho_1}}, \hB{\Delta^{\rho_2}}, \Delta_1, \Delta_2, \Delta_3$} 
%\end{prooftree}
%
%\begin{prooftree}
%\AXC{$\varphi_3$}\noLine
%\UIC{$\hB{\Gamma_3^{\rho_2}}, \Gamma_3  \seq \hB{\Delta_3^{\rho_2}}, \Delta_3$}
%		\AXC{$\varphi_1$}\noLine
%		\UIC{$\hA{\Gamma_1^{\rho_1}}, \Gamma_1  \seq \hA{\Delta_1^{\rho_1}}, \Delta_1$}
%				\AXC{$\varphi_2$}\noLine
%				\UIC{$\hA{\Gamma_2^{\rho_1}}, \hB{\Gamma_2^{\rho_2}}, \Gamma_2  \seq \hA{\Delta_2^{\rho_1}}, \hB{\Delta_2^{\rho_2}}, \Delta_2$}  \RightLabel{$\rho_1$}
%			\BIC{$\hA{\Gamma^{\rho_1}}, \hB{\Gamma_2^{\rho_2}}, \Gamma_1, \Gamma_2  \seq \hA{\Delta^{\rho_1}}, \hB{\Delta_2^{\rho_2}}, \Delta_1, \Delta_2$} \RightLabel{$\rho_2$}
%	\BIC{$\hA{\Gamma^{\rho_1}}, \hB{\Gamma^{\rho_2}}, \Gamma_1, \Gamma_2, \Gamma_3  \seq \hA{\Delta^{\rho_1}}, \hB{\Delta^{\rho_2}}, \Delta_1, \Delta_2, \Delta_3$} 
%\end{prooftree}
%$$
%\Downarrow
%$$
%\begin{prooftree}
%\AXC{$\varphi_1$}\noLine
%\UIC{$\hA{\Gamma_1^{\rho_1}}, \Gamma_1  \seq \hA{\Delta_1^{\rho_1}}, \Delta_1$}
%		\AXC{$\varphi_3$}\noLine
%		\UIC{$\hB{\Gamma_3^{\rho_2}}, \Gamma_3  \seq \hB{\Delta_3^{\rho_2}}, \Delta_3$}
%				\AXC{$\varphi_2$}\noLine
%				\UIC{$\hA{\Gamma_2^{\rho_1}}, \hB{\Gamma_2^{\rho_2}}, \Gamma_2  \seq \hA{\Delta_2^{\rho_1}}, \hB{\Delta_2^{\rho_2}}, \Delta_2$}  \RightLabel{$\rho_2$}
%			\BIC{$\hA{\Gamma_2^{\rho_1}}, \hB{\Gamma^{\rho_2}}, \Gamma_2, \Gamma_3  \seq \hA{\Delta_2^{\rho_1}}, \hB{\Delta^{\rho_2}}, \Delta_2, \Delta_3$} 
%\RightLabel{$\rho_1$}
%	\BIC{$\hA{\Gamma^{\rho_1}}, \hB{\Gamma^{\rho_2}}, \Gamma_1, \Gamma_2, \Gamma_3  \seq \hA{\Delta^{\rho_1}}, \hB{\Delta^{\rho_2}}, \Delta_1, \Delta_2, \Delta_3$}
%\end{prooftree}


\end{definition}



\begin{definition}[$\swapID$]
\label{definition:SwappingOfIndirectlyDependentInferences}
Distributional Swapping of Indirectly Dependent Inferences:

\begin{prooftree}
\AXC{$\varphi_1$}\noLine
\UIC{$\hA{\Gamma_1^{\rho_1}}, \hB{\Gamma_1^{\rho_2}}, \Gamma_1  \seq \hA{\Delta_1^{\rho_1}}, \hB{\Delta_1^{\rho_2}}, \Delta_1$}
		\AXC{$\varphi_2$}\noLine
		\UIC{$\hA{\Gamma_2^{\rho_1}}, \hB{\Gamma_2^{\rho_2}}, \Gamma_2  \seq \hA{\Delta_2^{\rho_1}}, \hB{\Delta_2^{\rho_2}}, \Delta_2$}  \RightLabel{$\rho_1$}
	\BIC{$\hA{\Gamma^{\rho_1}}, \hB{\Gamma_1^{\rho_2}}, \hB{\Gamma_2^{\rho_2}},\Gamma_1, \Gamma_2  \seq \hA{\Delta^{\rho_1}}, \hB{\Delta_1^{\rho_2}}, \hB{\Gamma_2^{\rho_2}}, \Delta_1, \Delta_2$}\RightLabel{$\rho_2$}
	\UIC{$\hA{\Gamma^{\rho_1}}, \hB{\Gamma^{\rho_2}}, \Gamma_1, \Gamma_2  \seq \hA{\Delta^{\rho_1}}, \hB{\Delta^{\rho_2}}, \Delta_1, \Delta_2$} 
\end{prooftree}
$$
\Downarrow
$$
\begin{prooftree}
\AXC{$\varphi_1$}\noLine
\UIC{$\hA{\Gamma_1^{\rho_1}}, \hB{\Gamma_1^{\rho_2}}, \Gamma_1  \seq \hA{\Delta_1^{\rho_1}}, \hB{\Delta_1^{\rho_2}}, \Delta_1$} \doubleLine \RightLabel{$w^*$}
\UIC{$\hA{\Gamma_1^{\rho_1}}, \hB{\Gamma_1^{\rho_2}}, \hB{\Gamma_2^{\rho_2}}, \Gamma_1  \seq \hA{\Delta_1^{\rho_1}}, \hB{\Delta_1^{\rho_2}}, \hB{\Delta_2^{\rho_2}} \Delta_1$} \RightLabel{$\rho_2$}
\UIC{$\hA{\Gamma_1^{\rho_1}}, \hB{\Gamma^{\rho_2}}, \Gamma_1  \seq \hA{\Delta_1^{\rho_1}}, \hB{\Delta^{\rho_2}}, \Delta_1$} 
		\AXC{$\varphi_2$}\noLine
		\UIC{$\hA{\Gamma_2^{\rho_1}}, \hB{\Gamma_2^{\rho_2}}, \Gamma_2  \seq \hA{\Delta_2^{\rho_1}}, \hB{\Delta_2^{\rho_2}}, \Delta_2$} \doubleLine \RightLabel{$w^*$}
		\UIC{$\hA{\Gamma_2^{\rho_1}}, \hB{\Gamma_1^{\rho_2}}, \hB{\Gamma_2^{\rho_2}}, \Gamma_2  \seq \hA{\Delta_2^{\rho_1}}, \hB{\Delta_1^{\rho_2}}, \hB{\Delta_2^{\rho_2}}, \Delta_2$} \RightLabel{$\rho_2$}
		\UIC{$\hA{\Gamma_2^{\rho_1}}, \hB{\Gamma^{\rho_2}}, \Gamma_2  \seq \hA{\Delta_2^{\rho_1}}, \hB{\Delta^{\rho_2}}, \Delta_2$} 
 \RightLabel{$\rho_1$}
	\BIC{$\hA{\Gamma^{\rho_1}}, \hB{\Gamma^{\rho_2}},\hB{\Gamma^{\rho_2}}, \Gamma_1, \Gamma_2  \seq \hA{\Delta^{\rho_1}}, \hB{\Delta^{\rho_2}},\hB{\Delta^{\rho_2}}, \Delta_1, \Delta_2$} \doubleLine \RightLabel{$c^*$}
	\UIC{$\hA{\Gamma^{\rho_1}}, \hB{\Gamma^{\rho_2}}, \Gamma_1, \Gamma_2  \seq \hA{\Delta^{\rho_1}}, \hB{\Delta^{\rho_2}}, \Delta_1, \Delta_2$}
\end{prooftree}
\end{definition}


\begin{remark}
While the inference $\rho_2$ in the proof rewriting rules of Definition \ref{definition:SwappingOfIndirectlyDependentInferences} can be a contraction, there are cases in which contractions can be swapped upward in a smarter way, as shown in Definition \ref{definition:SwappingIndirectlyDependentContraction}
\end{remark}




\begin{definition}[$\swapIDC$]
\label{definition:SwappingIndirectlyDependentContraction}
Swapping of indirectly dependent contractions:

\begin{prooftree}
\AXC{$\varphi_1$} \noLine
\UIC{$\Gamma_1, \hA{\Gamma_{\rho}},\hB{\Gamma_{\rho}} \seq \Delta_1, \hA{\Delta_{\rho}},\hB{\Delta_{\rho}}$} \RightLabel{$\hB{\rho}$}
\UIC{$\Gamma_1, \hA{\Gamma_{\rho}},\hB{\Pi_{\rho}} \seq \Delta_1, \hA{\Delta_{\rho}}, \hB{\Lambda_{\rho}}$}\RightLabel{$\hA{\rho}$}
\UIC{$\Gamma_1, \hA{\Pi_{\rho}}, \hB{\Pi_{\rho}} \seq \Delta_1, \hA{\Lambda_{\rho}}, \hB{\Lambda_{\rho}}$} \doubleLine \RightLabel{$c^*$}
\UIC{$\Gamma_1, \hC{\Pi_{\rho}} \seq \Delta_1, \hC{\Lambda_{\rho}}$}
\end{prooftree}
$$
\Downarrow
$$
\begin{prooftree}
\AXC{$\varphi_1$} \noLine
\UIC{$\Gamma_1, \hA{\Gamma_{\rho}}, \hB{\Gamma_{\rho}} \seq \Delta_1, \hA{\Delta_{\rho}}, \hB{\Delta_{\rho}}$} \doubleLine \RightLabel{$c^*$}
\UIC{$\Gamma_1, \hC{\Gamma_{\rho}} \seq \Delta_1, \hC{\Delta_{\rho}}$} \RightLabel{$\hC{\rho}$}
\UIC{$\Gamma_1, \hC{\Pi_{\rho}} \seq \Delta_1, \hC{\Lambda_{\rho}}$}
\end{prooftree}

\begin{prooftree}
\AXC{$\varphi_1$} \noLine
\UIC{$\Gamma_1, \hA{\Gamma_1^{\rho}},\hB{\Gamma_1^{\rho}} \seq \Delta_1, \hA{\Delta_{\rho}},\hB{\Delta_{\rho}}$}
		\AXC{$\varphi_2$} \noLine
		\UIC{$\Gamma_2, \hB{\Gamma_2^{\rho}} \seq \Delta_2, \hB{\Delta_2^{\rho}}$} \RightLabel{$\hB{\rho}$}
	\BIC{$\Gamma_1, \hA{\Gamma_{\rho}},\hB{\Pi_{\rho}} \seq \Delta_1, \hA{\Delta_{\rho}}, \hA{\Lambda_{\rho}}$}
			\AXC{$\varphi_2$} \noLine
			\UIC{$\Gamma_2, \hA{\Gamma_2^{\rho}} \seq \Delta_2, \hA{\Delta_2^{\rho}}$}\RightLabel{$\hB{\rho}$}
		\BIC{$\Gamma_1, \hA{\Pi_{\rho}}, \hB{\Pi_{\rho}} \seq \Delta_1, \hA{\Lambda_{\rho}}, \hB{\Lambda_{\rho}}$} \doubleLine \RightLabel{$c^*$}
		\UIC{$\Gamma_1, \hC{\Pi_{\rho}} \seq \Delta_1, \hC{\Lambda_{\rho}}$}
\end{prooftree}
$$
\Downarrow
$$
\begin{prooftree}
\AXC{$\varphi_1$} \noLine
\UIC{$\Gamma_1, \hA{\Gamma_1^{\rho}}, \hB{\Gamma_1^{\rho}} \seq \Delta_1, \hA{\Delta_1^{\rho}}, \hB{\Delta_1^{\rho}}$} \doubleLine \RightLabel{$c^*$}
\UIC{$\Gamma_1, \hC{\Gamma_{\rho}} \seq \Delta_1, \hC{\Delta_{\rho}}$}
		\AXC{$\varphi_2$} \noLine
		\UIC{$\Gamma_2, \hC{\Gamma_2^{\rho}} \seq \Delta_2, \hC{\Delta_2^{\rho}}$}\RightLabel{$\hC{\rho}$}
	\BIC{$\Gamma_1, \hC{\Pi_{\rho}} \seq \Delta_1, \hC{\Lambda_{\rho}}$}
\end{prooftree}

%\begin{prooftree}
%\AXC{$\varphi_2$} \noLine
%\UIC{$\Gamma_2, \hA{\Gamma_2^{\rho}} \seq \Delta_2, \hA{\Delta_2^{\rho}}$}
%		\AXC{$\varphi_2$} \noLine
%		\UIC{$\Gamma_2, \hB{\Gamma_2^{\rho}} \seq \Delta_2, \hB{\Delta_2^{\rho}}$}
%				\AXC{$\varphi_1$} \noLine
%				\UIC{$\Gamma_1, \hA{\Gamma_1^{\rho}},\hB{\Gamma_1^{\rho}} \seq \Delta_1, \hA{\Delta_{\rho}},\hB{\Delta_{\rho}}$} \RightLabel{$\hB{\rho}$}
%			\BIC{$\Gamma_1, \hA{\Gamma_{\rho}},\hB{\Pi_{\rho}} \seq \Delta_1, \hA{\Delta_{\rho}}, \hB{\Lambda_{\rho}}$}\RightLabel{$\hA{\rho}$}
%	\BIC{$\Gamma_1, \hA{\Pi_{\rho}}, \hB{\Pi_{\rho}} \seq \Delta_1, \hA{\Lambda_{\rho}}, \hB{\Lambda_{\rho}}$} \doubleLine \RightLabel{$c^*$}
%	\UIC{$\Gamma_1, \hC{\Pi_{\rho}} \seq \Delta_1, \hC{\Lambda_{\rho}}$}
%\end{prooftree}
%$$
%\Downarrow
%$$
%\begin{prooftree}
%\AXC{$\varphi_2$} \noLine
%\UIC{$\Gamma_2, \hA{\Gamma_2^{\rho}} \seq \Delta_2, \hA{\Delta_2^{\rho}}$}
%		\AXC{$\varphi_1$} \noLine
%		\UIC{$\Gamma_1, \hA{\Gamma_1^{\rho}}, \hB{\Gamma_1^{\rho}} \seq \Delta_1, \hA{\Delta_1^{\rho}}, \hB{\Delta_1^{\rho}}$} \doubleLine \RightLabel{$c^*$}
%		\UIC{$\Gamma_1, \hC{\Gamma_{\rho}} \seq \Delta_1, \hC{\Delta_{\rho}}$}\RightLabel{$\hC{\rho}$}
%	\BIC{$\Gamma_1, \hC{\Pi_{\rho}} \seq \Delta_1, \hC{\Lambda_{\rho}}$}
%\end{prooftree}


\end{definition}



\begin{definition}[$\swapC$]
\label{definition:SwappingContraction}
Distributional Swapping over contractions:

\begin{prooftree}
\AXC{$\varphi_1$} \noLine
\UIC{$\Gamma_1, \hA{\Gamma_{\rho}}, \hA{\Gamma'_{\rho}} \seq \Delta_1, \hA{\Delta_{\rho}}, \hA{\Delta'_{\rho}}$} \doubleLine \RightLabel{$c^*$}
\UIC{$\Gamma_1, \hA{\Gamma_{\rho}} \seq \Delta_1, \hA{\Delta_{\rho}}$} \RightLabel{$\rho$}
\UIC{$\Gamma_1, \hA{\Pi_{\rho}} \seq \Delta_1, \hA{\Lambda_{\rho}}$}
\end{prooftree}
$$
\Downarrow
$$
\begin{prooftree}
\AXC{$\varphi_1$} \noLine
\UIC{$\Gamma_1, \hA{\Gamma_{\rho}}, \hB{\Gamma'_{\rho}} \seq \Delta_1, \hA{\Delta_{\rho}}, \hB{\Delta'_{\rho}}$} \doubleLine \RightLabel{$w^*$}
\UIC{$\Gamma_1, \hA{\Gamma_{\rho}},\hB{\Gamma_{\rho}} \seq \Delta_1, \hA{\Delta_{\rho}},\hB{\Delta_{\rho}}$} \RightLabel{$\hB{\rho}$}
\UIC{$\Gamma_1, \hA{\Gamma_{\rho}},\hB{\Pi_{\rho}} \seq \Delta_1, \hA{\Delta_{\rho}}, \hB{\Lambda_{\rho}}$}\RightLabel{$\hA{\rho}$}
\UIC{$\Gamma_1, \hA{\Pi_{\rho}}, \hB{\Pi_{\rho}} \seq \Delta_1, \hA{\Lambda_{\rho}}, \hB{\Lambda_{\rho}}$} \doubleLine \RightLabel{$c^*$}
\UIC{$\Gamma_1, \hC{\Pi_{\rho}} \seq \Delta_1, \hC{\Lambda_{\rho}}$}
\end{prooftree}

\begin{prooftree}
\AXC{$\varphi_1$} \noLine
\UIC{$\Gamma_1, \hA{\Gamma_1^{\rho}}, \hA{\Gamma_1{\rho'}} \seq \Delta_1, \hA{\Delta_1^{\rho}}, \hA{\Delta_1^{\rho'}}$} \doubleLine \RightLabel{$c^*$}
\UIC{$\Gamma_1, \hA{\Gamma_{\rho}} \seq \Delta_1, \hA{\Delta_{\rho}}$}
		\AXC{$\varphi_2$} \noLine
		\UIC{$\Gamma_2, \hA{\Gamma_2^{\rho}} \seq \Delta_2, \hA{\Delta_2^{\rho}}$}\RightLabel{$\rho$}
	\BIC{$\Gamma_1, \hA{\Pi_{\rho}} \seq \Delta_1, \hA{\Lambda_{\rho}}$}
\end{prooftree}
$$
\Downarrow
$$
\begin{prooftree}
\AXC{$\varphi_1$} \noLine
\UIC{$\Gamma_1, \hA{\Gamma_1^{\rho}}, \hB{\Gamma_1^{\rho'}} \seq \Delta_1, \hA{\Delta_1^{\rho}}, \hB{\Delta_1^{\rho'}}$} \doubleLine \RightLabel{$w^*$}
\UIC{$\Gamma_1, \hA{\Gamma_1^{\rho}},\hB{\Gamma_1^{\rho}} \seq \Delta_1, \hA{\Delta_{\rho}},\hB{\Delta_{\rho}}$}
		\AXC{$\varphi_2$} \noLine
		\UIC{$\Gamma_2, \hB{\Gamma_2^{\rho}} \seq \Delta_2, \hB{\Delta_2^{\rho}}$} \RightLabel{$\hB{\rho}$}
	\BIC{$\Gamma_1, \hA{\Gamma_{\rho}},\hB{\Pi_{\rho}} \seq \Delta_1, \hA{\Delta_{\rho}}, \hB{\Lambda_{\rho}}$}
			\AXC{$\varphi_2$} \noLine
			\UIC{$\Gamma_2, \hA{\Gamma_2^{\rho}} \seq \Delta_2, \hA{\Delta_2^{\rho}}$}\RightLabel{$\hA{\rho}$}
		\BIC{$\Gamma_1, \hA{\Pi_{\rho}}, \hB{\Pi_{\rho}} \seq \Delta_1, \hA{\Lambda_{\rho}}, \hB{\Lambda_{\rho}}$} \doubleLine \RightLabel{$c^*$}
		\UIC{$\Gamma_1, \hC{\Pi_{\rho}} \seq \Delta_1, \hC{\Lambda_{\rho}}$}
\end{prooftree}

%\begin{prooftree}
%\AXC{$\varphi_2$} \noLine
%\UIC{$\Gamma_2, \hA{\Gamma_2^{\rho}} \seq \Delta_2, \hA{\Delta_2^{\rho}}$}
%		\AXC{$\varphi_1$} \noLine
%		\UIC{$\Gamma_1, \hA{\Gamma_1^{\rho}}, \hA{\Gamma_1^{\rho'}} \seq \Delta_1, \hA{\Delta_1^{\rho}}, \hA{\Delta_1^{\rho'}}$} \doubleLine \RightLabel{$c^*$}
%		\UIC{$\Gamma_1, \hA{\Gamma_{\rho}} \seq \Delta_1, \hA{\Delta_{\rho}}$}\RightLabel{$\rho$}
%	\BIC{$\Gamma_1, \hA{\Pi_{\rho}} \seq \Delta_1, \hA{\Lambda_{\rho}}$}
%\end{prooftree}
%$$
%\Downarrow
%$$
%\begin{prooftree}
%\AXC{$\varphi_2$} \noLine
%\UIC{$\Gamma_2, \hA{\Gamma_2^{\rho}} \seq \Delta_2, \hA{\Delta_2^{\rho}}$}
%		\AXC{$\varphi_2$} \noLine
%		\UIC{$\Gamma_2, \hB{\Gamma_2^{\rho}} \seq \Delta_2, \hB{\Delta_2^{\rho}}$}
%				\AXC{$\varphi_1$} \noLine
%				\UIC{$\Gamma_1, \hA{\Gamma_1^{\rho}}, \hB{\Gamma_1^{\rho'}} \seq \Delta_1, \hA{\Delta_1^{\rho}}, \hB{\Delta_1^{\rho'}}$} \doubleLine \RightLabel{$w^*$}
%				\UIC{$\Gamma_1, \hA{\Gamma_1^{\rho}},\hB{\Gamma_1^{\rho}} \seq \Delta_1, \hA{\Delta_{\rho}},\hB{\Delta_{\rho}}$} \RightLabel{$\hB{\rho}$}
%			\BIC{$\Gamma_1, \hA{\Gamma_{\rho}},\hB{\Pi_{\rho}} \seq \Delta_1, \hA{\Delta_{\rho}}, \hB{\Lambda_{\rho}}$}\RightLabel{$\hA{\rho}$}
%	\BIC{$\Gamma_1, \hA{\Pi_{\rho}}, \hB{\Pi_{\rho}} \seq \Delta_1, \hA{\Lambda_{\rho}}, \hB{\Lambda_{\rho}}$} \doubleLine \RightLabel{$c^*$}
%	\UIC{$\Gamma_1, \hC{\Pi_{\rho}} \seq \Delta_1, \hC{\Lambda_{\rho}}$}
%\end{prooftree}


\end{definition}





\begin{definition}[$\swapWI$]
Downward swapping of weakening inferences over independent inferences.

\begin{multicols}{3}{
\begin{prooftree}
\AXC{$\varphi_1$}\noLine
\UIC{$\hB{\Gamma_1^{\rho}}, \Gamma  \seq \hB{\Delta_1^{\rho}}, \Delta$}\RightLabel{$w_l$}
\UIC{$\hA{F}, \hB{\Gamma_1^{\rho}}, \Gamma  \seq \hB{\Delta_1^{\rho}}, \Delta$}\RightLabel{$\rho$}
\UIC{$\hA{F}, \hB{\Gamma^{\rho}}, \Gamma  \seq \hB{\Delta^{\rho}}, \Delta$} 
\end{prooftree}
$$
\Rightarrow
$$
\begin{prooftree}
\AXC{$\varphi_1$}\noLine
\UIC{$\hB{\Gamma_1^{\rho}}, \Gamma  \seq \hB{\Delta_1^{\rho}}, \Delta$} \RightLabel{$\rho$}
\UIC{$\hB{\Gamma^{\rho}}, \Gamma  \seq \hB{\Delta^{\rho}}, \Delta$} \RightLabel{$w_l$}
\UIC{$\hA{F}, \hB{\Gamma^{\rho}}, \Gamma  \seq \hB{\Delta^{\rho}}, \Delta$} 
\end{prooftree}

%\begin{prooftree}
%\AXC{$\varphi_1$}\noLine
%\UIC{$\hB{\Gamma_1^{\rho}}, \Gamma  \seq \hB{\Delta_1^{\rho}}, \Delta$}\RightLabel{$w_r$}
%\UIC{$\hB{\Gamma_1^{\rho}}, \Gamma  \seq \hB{\Delta_1^{\rho}}, \Delta, \hA{F}$}\RightLabel{$\rho$}
%\UIC{$\hB{\Gamma^{\rho}}, \Gamma  \seq \hB{\Delta^{\rho}}, \Delta, \hA{F}$} 
%\end{prooftree}
%$$
%\Downarrow
%$$
%\begin{prooftree}
%\AXC{$\varphi_1$}\noLine
%\UIC{$\hB{\Gamma_1^{\rho}}, \Gamma  \seq \hB{\Delta_1^{\rho}}, \Delta$} \RightLabel{$\rho$}
%\UIC{$\hB{\Gamma^{\rho}}, \Gamma  \seq \hB{\Delta^{\rho}}, \Delta$} \RightLabel{$w_l$}
%\UIC{$\hB{\Gamma^{\rho}}, \Gamma  \seq \hB{\Delta^{\rho}}, \Delta, \hA{F}$} 
%\end{prooftree}
}\end{multicols}

\begin{small}
%\begin{multicols}{2}{
\begin{prooftree}
\AXC{$\varphi_1$}\noLine
\UIC{$\hB{\Gamma_1^{\rho}}, \Gamma_1  \seq \hB{\Delta_1^{\rho}}, \Delta_1$}  \RightLabel{$w_l$}
\UIC{$\hA{F}, \hB{\Gamma_1^{\rho}}, \Gamma_1  \seq \hB{\Delta_1^{\rho}}, \Delta_1$}
				\AXC{$\varphi_2$}\noLine
				\UIC{$\hB{\Gamma_2^{\rho}}, \Gamma_2  \seq \hB{\Delta_2^{\rho}}, \Delta_2$} \RightLabel{$\rho$}
			\BIC{$\hA{F}, \hB{\Gamma^{\rho}}, \Gamma_1, \Gamma_2  \seq \hB{\Delta^{\rho}}, \Delta_1, \Delta_2$} 
\end{prooftree}
$$
\Downarrow
$$
\begin{prooftree}
\AXC{$\varphi_1$}\noLine
\UIC{$\hB{\Gamma_1^{\rho}}, \Gamma_1  \seq \hB{\Delta_1^{\rho}}, \Delta_1$}
				\AXC{$\varphi_2$}\noLine
				\UIC{$\hB{\Gamma_2^{\rho}}, \Gamma_2  \seq \hB{\Delta_2^{\rho}}, \Delta_2$} \RightLabel{$\rho$}
			\BIC{$\hB{\Gamma^{\rho}}, \Gamma_1, \Gamma_2  \seq \hB{\Delta^{\rho}}, \Delta_1, \Delta_2$} \RightLabel{$w_l$}
			\UIC{$\hA{F}, \hB{\Gamma^{\rho}}, \Gamma_1, \Gamma_2  \seq \hB{\Delta^{\rho}}, \Delta_1, \Delta_2$}
\end{prooftree}

%\begin{prooftree}
%\AXC{$\varphi_1$}\noLine
%\UIC{$\hB{\Gamma_1^{\rho}}, \Gamma_1  \seq \hB{\Delta_1^{\rho}}, \Delta_1$}  \RightLabel{$w_r$}
%\UIC{$\hB{\Gamma_1^{\rho}}, \Gamma_1  \seq \hB{\Delta_1^{\rho}}, \Delta_1, \hA{F}$}
%				\AXC{$\varphi_2$}\noLine
%				\UIC{$\hB{\Gamma_2^{\rho}}, \Gamma_2  \seq \hB{\Delta_2^{\rho}}, \Delta_2$} \RightLabel{$\rho$}
%			\BIC{$\hB{\Gamma^{\rho}}, \Gamma_1, \Gamma_2  \seq \hB{\Delta^{\rho}}, \Delta_1, \hA{F}, \Delta_2$} 
%\end{prooftree}
%$$
%\Downarrow
%$$
%\begin{prooftree}
%\AXC{$\varphi_1$}\noLine
%\UIC{$\hB{\Gamma_1^{\rho}}, \Gamma_1  \seq \hB{\Delta_1^{\rho}}, \Delta_1$}
%				\AXC{$\varphi_2$}\noLine
%				\UIC{$\hB{\Gamma_2^{\rho}}, \Gamma_2  \seq \hB{\Delta_2^{\rho}}, \Delta_2$} \RightLabel{$\rho$}
%			\BIC{$\hB{\Gamma^{\rho}}, \Gamma_1, \Gamma_2  \seq \hB{\Delta^{\rho}}, \Delta_1, \Delta_2$} \RightLabel{$w_r$}
%			\UIC{$\hB{\Gamma^{\rho}}, \Gamma_1, \Gamma_2  \seq \hB{\Delta^{\rho}}, \Delta_1, \hA{F}, \Delta_2$}
%\end{prooftree}
%}\end{multicols}

%\begin{multicols}{2}{
%\begin{prooftree}
%\AXC{$\varphi_1$}\noLine
%\UIC{$\hB{\Gamma_1^{\rho}}, \Gamma_1  \seq \hB{\Delta_1^{\rho}}, \Delta_1$}  
%			\AXC{$\varphi_2$}\noLine
%			\UIC{$\hB{\Gamma_2^{\rho}}, \Gamma_2  \seq \hB{\Delta_2^{\rho}}, \Delta_2$} \RightLabel{$w_l$}
%			\UIC{$\hA{F}, \hB{\Gamma_2^{\rho}}, \Gamma_2  \seq \hB{\Delta_2^{\rho}}, \Delta_2$}\RightLabel{$\rho$}
%		\BIC{$\hA{F}, \hB{\Gamma^{\rho}}, \Gamma_1, \Gamma_2  \seq \hB{\Delta^{\rho}}, \Delta_1, \Delta_2$} 
%\end{prooftree}
%$$
%\Downarrow
%$$
%\begin{prooftree}
%\AXC{$\varphi_1$}\noLine
%\UIC{$\hB{\Gamma_1^{\rho}}, \Gamma_1  \seq \hB{\Delta_1^{\rho}}, \Delta_1$}
%				\AXC{$\varphi_2$}\noLine
%				\UIC{$\hB{\Gamma_2^{\rho}}, \Gamma_2  \seq \hB{\Delta_2^{\rho}}, \Delta_2$} \RightLabel{$\rho$}
%			\BIC{$\hB{\Gamma^{\rho}}, \Gamma_1, \Gamma_2  \seq \hB{\Delta^{\rho}}, \Delta_1, \Delta_2$} \RightLabel{$w_l$}
%			\UIC{$\hA{F}, \hB{\Gamma^{\rho}}, \Gamma_1, \Gamma_2  \seq \hB{\Delta^{\rho}}, \Delta_1, \Delta_2$}
%\end{prooftree}
%
%\begin{prooftree}
%\AXC{$\varphi_1$}\noLine
%\UIC{$\hB{\Gamma_1^{\rho}}, \Gamma_1  \seq \hB{\Delta_1^{\rho}}, \Delta_1$}
%			\AXC{$\varphi_2$}\noLine
%			\UIC{$\hB{\Gamma_2^{\rho}}, \Gamma_2  \seq \hB{\Delta_2^{\rho}}, \Delta_2$} \RightLabel{$w_r$}
%			\UIC{$\hB{\Gamma_2^{\rho}}, \Gamma_2  \seq \hB{\Delta_2^{\rho}}, \Delta_2, \hA{F}$}\RightLabel{$\rho$}
%		\BIC{$\hB{\Gamma^{\rho}}, \Gamma_1, \Gamma_2  \seq \hB{\Delta^{\rho}}, \Delta_1, \hA{F}, \Delta_2$} 
%\end{prooftree}
%$$
%\Downarrow
%$$
%\begin{prooftree}
%\AXC{$\varphi_1$}\noLine
%\UIC{$\hB{\Gamma_1^{\rho}}, \Gamma_1  \seq \hB{\Delta_1^{\rho}}, \Delta_1$}
%				\AXC{$\varphi_2$}\noLine
%				\UIC{$\hB{\Gamma_2^{\rho}}, \Gamma_2  \seq \hB{\Delta_2^{\rho}}, \Delta_2$} \RightLabel{$\rho$}
%			\BIC{$\hB{\Gamma^{\rho}}, \Gamma_1, \Gamma_2  \seq \hB{\Delta^{\rho}}, \Delta_1, \Delta_2$} \RightLabel{$w_r$}
%			\UIC{$\hB{\Gamma^{\rho}}, \Gamma_1, \Gamma_2  \seq \hB{\Delta^{\rho}}, \Delta_1, \hA{F}, \Delta_2$}
%\end{prooftree}
%}\end{multicols}
\end{small}


\end{definition}


\begin{definition}[Degenerate Inferences]
An inference $\rho$ in a proof $\varphi$ is \emph{degenerate} when all its auxiliary formula occurrences are descendants of main formula occurrences of weakening inferences. When only some auxiliary (sub)-formula occurrences of $\rho$ are descendants of main formula occurrences of weakening inferences, $\rho$ is \emph{partially degenerate}.
\end{definition}


\begin{definition}[$\swapWD$]
Downward swapping of weakening inferences over directly dependent inferences.

\begin{prooftree}
\AXC{$\varphi_1$}\noLine
\UIC{$\Gamma  \seq \Delta$} \doubleLine \RightLabel{$w^*$}
\UIC{$\hB{\Gamma_1^{\rho}}, \Gamma  \seq \hB{\Delta_1^{\rho}}, \Delta$}\RightLabel{$\rho$}
\UIC{$\hB{\Gamma^{\rho}}, \Gamma  \seq \hB{\Delta^{\rho}}, \Delta$} 
\end{prooftree}
$$
\Downarrow
$$
\begin{prooftree}
\AXC{$\varphi_1$}\noLine
\UIC{$\Gamma  \seq \Delta$} \doubleLine \RightLabel{$w^*$}
\UIC{$\hB{\Gamma^{\rho}}, \Gamma  \seq \hB{\Delta^{\rho}}, \Delta$}  
\end{prooftree}

\begin{small}
%\begin{multicols}{2}{
\begin{prooftree}
\AXC{$\varphi_1$}\noLine
\UIC{$\Gamma_1  \seq \Delta_1$} \doubleLine \RightLabel{$w^*$}
\UIC{$\hB{\Gamma_1^{\rho}}, \Gamma_1  \seq \hB{\Delta_1^{\rho}}, \Delta_1$}
		\AXC{$\varphi_2$}\noLine
		\UIC{$\hB{\Gamma_2^{\rho}}, \Gamma_2  \seq \hB{\Delta_2^{\rho}}, \Delta_2$} \RightLabel{$\rho$}
	\BIC{$\hB{\Gamma^{\rho}}, \Gamma_1, \Gamma_2  \seq \hB{\Delta^{\rho}}, \Delta_1, \Delta_2$} 
\end{prooftree}
$$
\Downarrow
$$
\begin{prooftree}
\AXC{$\varphi_1$}\noLine
\UIC{$\Gamma_1  \seq \Delta_1$} \doubleLine \RightLabel{$w^*$}
\UIC{$\hB{\Gamma^{\rho}}, \Gamma_1, \Gamma_2  \seq \hB{\Delta^{\rho}}, \Delta_1, \Delta_2$}  
\end{prooftree}

%\begin{prooftree}
%\AXC{$\varphi_2$}\noLine
%\UIC{$\hB{\Gamma_2^{\rho}}, \Gamma_2  \seq \hB{\Delta_2^{\rho}}, \Delta_2$} 
%		\AXC{$\varphi_1$}\noLine
%		\UIC{$\Gamma_1  \seq \Delta_1$} \doubleLine \RightLabel{$w^*$}
%		\UIC{$\hB{\Gamma_1^{\rho}}, \Gamma_1  \seq \hB{\Delta_1^{\rho}}, \Delta_1$}\RightLabel{$\rho$}
%	\BIC{$\hB{\Gamma^{\rho}}, \Gamma_2, \Gamma_1  \seq \hB{\Delta^{\rho}}, \Delta_2, \Delta_1$} 
%\end{prooftree}
%$$
%\Downarrow
%$$
%\begin{prooftree}
%\AXC{$\varphi_1$}\noLine
%\UIC{$\Gamma_1  \seq \Delta_1$} \doubleLine \RightLabel{$w^*$}
%\UIC{$\hB{\Gamma^{\rho}}, \Gamma_2, \Gamma_1  \seq \hB{\Delta^{\rho}}, \Delta_2, \Delta_1$}  
%\end{prooftree}
%}\end{multicols}
\end{small}


\end{definition}




\begin{definition}[$\swapW$]
\label{definition:SwappingWeakening}
\index{Inference Swapping!Swapping of Weakening}
The proof rewriting relation for \emph{downward swapping of weakening} is:
$$
\swapW \ \defEq \ (\swapWI \cup \swapWD)
$$
\end{definition}


\begin{definition}[$\swap$] 
\label{definition:Swapping}
The proof rewriting relation for \emph{inference swapping} is:
$$
\swap \ \defEq \ (\swapI \cup \swapID \cup \swapIDC \cup \swapC \cup \swapW)
$$
\end{definition}

\section{Correspondence between $\normalizePlusTimesS$ and $\swap$}
\label{sec:Correspondence}

\begin{lemma}%[Correspondence between $\normalizePlusTimesS$ and $\swap$]
%\label{lemma:SwapNormalizeCorrespondence}
If $\varphi$ is skolemized and $\struct{\varphi} \normalizePlusTimesS S$, then there exists a proof $\psi$ such that $\varphi \swap^* \psi$ and $\struct{\psi} = S$. 
\end{lemma}
\begin{proof}

The proof can be subdivided according to all possible cases of rewriting according to $\normalizePlusTimesS$. Here only three cases are shown, but every other case is either symmetric or analogous to one of these three cases.


\paragraph{\textbf{Case of rewriting with no duplication,}} 
when the redex has the form $\hF{S} \hF{\structtimes_{\rho_2}} (S' \structplus_{\rho_1} \hF{S''})$ and is rewritten to $S' \structplus_{\rho_1} (\hF{S} \hF{\structtimes_{\rho_2}} \hF{S''})$):
Since $\rho_1$ operates on cut ancestors and $\rho_2$ operates on end-sequent ancestors, $\rho_2$ is not directly dependent on $\rho_1$. Moreover, $\varphi$ is skolemized and thus $\rho_2$ is also not eigen-variable dependent on $\rho_1$. 
Since $\hF{S}$ is distributed only to $\hF{S''}$, only $\hF{S''}$ contains formulas from $\occInference{\varphi}{\rho_2}$. Therefore, $\occInference{\varphi}{\rho_2}$ has formulas in at most one premise of $\rho_1$, and hence ancestors of auxiliary formulas of $\rho_2$ occur in at most one premise of $\rho_1$. Therefore $\rho_2$ is independent of $\rho_1$. Moreover, any inference $\rho_i$ on the path between $\rho_2$ and $\rho_1$ and on which $\rho_2$ directly depends is also independent of $\rho_1$. Consequently, there exists a proof $\psi$ with $\varphi \swapI^* \psi$ where $\rho_2$ and all inferences $\rho_i$ on which it depends have been swapped above $\rho_1$, so that $\struct{\psi}$ is equal to $\struct{\varphi}$ with $\hF{S} \hF{\structtimes} (S' \structplus \hF{S''})$ rewritten to $S' \structplus (\hF{S} \hF{\structtimes} \hF{S''})$.


\paragraph{\textbf{Case of rewriting with duplication,}} when the redex has the form $(\hF{S'} \structplus_{\rho_1} \hF{S''}) \hF{\structtimes_{\rho_2}} \hF{S}$ and is rewritten to $(\hF{S'} \hF{\structtimes_{\rho_2}} \hF{S}) \structplus_{\rho_1} (\hF{S''} \hF{\structtimes_{\rho_2}} \hF{S})$:
Since $\rho_1$ operates on cut ancestors and $\rho_2$ operates on end-sequent ancestors, $\rho_2$ is not directly dependent on $\rho_1$. Moreover, $\varphi$ is skolemized and thus $\rho_2$ is also not eigen-variable dependent on $\rho_1$. However, as both $\hF{S'}$ and $\hF{S''}$ contain formulas from $\occInference{\varphi}{\rho_2}$, it must be the case that $\rho_2 \dependentI \rho_1$. In the sequent calculus $\LK$, this can only happen if there exists a sequence of unary\footnote{If the inferences were not unary, the redex would not be of the form $(\hF{S'} \structplus_{\rho_1} \hF{S''}) \hF{\structtimes_{\rho_2}} \hF{S}$.} inferences $\rho^*_D \equiv (\rho_{D_1},\ldots,\rho_{D_n})$ on the path between $\rho_2$ and $\rho_1$ such that $\rho_{D_i}$ is indirectly dependent on $\rho_1$ and $\rho_2$ depends on $\rho_{D_i}$, for any $i$ such that $1 \leq i \leq n$. Moreover, any inference $\rho_{D_i}$ (on the path between $\rho_2$ and $\rho_1$) on which $\rho_2$ directly depends is also independent of $\rho_1$.

Let $\varphi'$ be the subproof of $\varphi$ having the conclusion sequent of $\rho_2$ as its end-sequent. Then, there exists a proof $\varphi''$ with $\varphi' \swap^* \varphi''$ (permuting $\rho^*_D$ and $\rho_2$ above all inferences on which they do not depend) such that $\varphi''$ has the following form (or a form symmetric to it):
\begin{scriptsize}
\begin{prooftree}
\AXC{$\varphi_1$} \noLine
\UIC{$\Gamma_1, \hF{\Gamma_1^{\rho_2}}, \Gamma_1^{\rho_1} \seq \Delta_1, \hF{\Delta_1^{\rho_2}}, \Delta_1^{\rho_1}$}	
		\AXC{$\varphi_2$} \noLine
		\UIC{$\Gamma_2, \hF{\Gamma_2^{\rho_2}}, \Gamma_2^{\rho_1} \seq \Delta_2, \hF{\Delta_2^{\rho_2}}, \Delta_2^{\rho_1}$}	\RightLabel{$\rho_1$}
	\BIC{$\Gamma_1, \Gamma_2, \hF{\Gamma_1^{\rho_2}}, \hF{\Gamma_2^{\rho_2}}, \Gamma^{\rho_1} \seq \Delta_1, \Delta_2, \hF{\Delta_1^{\rho_2}}, \hF{\Delta_2^{\rho_2}}, \Delta^{\rho_1}$} \doubleLine \RightLabel{$\rho^*_D$}
	\UIC{$\Gamma_1, \Gamma_2, \Gamma^{\rho_1} \seq \Delta_1, \Delta_2, \hF{F_{12}^{\rho_2}}, \Delta^{\rho_1}$}
					\AXC{$\varphi_3$} \noLine
					\UIC{$\Gamma_3 \seq \Delta_3, \hF{F_3^{\rho_2}}$}	\RightLabel{$\rho_2$}
			\BIC{$\Gamma_1,\Gamma_2,\Gamma_3, \Gamma^{\rho_1} \seq \Delta_1,\Delta_2,\Delta_3,\hF{F^{\rho_2}},\Delta^{\rho_1} $}
\end{prooftree}
\end{scriptsize}

\begin{landscape}
\noindent
Then there exists a proof $\varphi'''$ with $\varphi'' \swap^* \varphi'''$ (permuting $\rho^*_D$ above  $\rho_1$) such that $\varphi'''$ has the following form:
\begin{scriptsize}
\begin{prooftree}
\AXC{$\varphi_1$} \noLine
\UIC{$\Gamma_1, \hF{\Gamma_1^{\rho_2}}, \Gamma_1^{\rho_1} \seq \Delta_1, \hF{\Delta_1^{\rho_2}}, \Delta_1^{\rho_1}$}\doubleLine \RightLabel{$w^*$}
\UIC{$\Gamma_1, \hF{\Gamma_1^{\rho_2}}, \hF{\Gamma_2^{\rho_2}}, \Gamma_1^{\rho_1} \seq \Delta_1, \hF{\Delta_1^{\rho_2}}, \hF{\Delta_2^{\rho_2}}, \Delta_1^{\rho_1}$} \doubleLine \RightLabel{$\rho^*_D$}
\UIC{$\Gamma_1, \Gamma_1^{\rho_1} \seq \Delta_1, \hF{F_{12}^{\rho_2}}, \Delta_1^{\rho_1}$}
		\AXC{$\varphi_2$} \noLine
		\UIC{$\Gamma_2, \hF{\Gamma_2^{\rho_2}}, \Gamma_2^{\rho_1} \seq \Delta_2, \hF{\Delta_2^{\rho_2}}, \Delta_2^{\rho_1}$} \doubleLine \RightLabel{$w^*$}
		\UIC{$\Gamma_2, \hF{\Gamma_1^{\rho_2}}, \hF{\Gamma_2^{\rho_2}}, \Gamma_2^{\rho_1} \seq \Delta_2, \hF{\Delta_1^{\rho_2}}, \hF{\Delta_2^{\rho_2}}, \Delta_2^{\rho_1}$} \doubleLine \RightLabel{$\rho^*_D$}
		\UIC{$\Gamma_2, \Gamma_2^{\rho_1} \seq \Delta_2, \hF{F_{12}^{\rho_2}}, \Delta_2^{\rho_1}$}	\RightLabel{$\rho_1$}
	\BIC{$\Gamma_1, \Gamma_2, \Gamma^{\rho_1} \seq \Delta_1, \Delta_2, \hF{F_{12}^{\rho_2}}, \hF{F_{12}^{\rho_2}}, \Delta^{\rho_1}$} \RightLabel{$c_r$}
	\UIC{$\Gamma_1, \Gamma_2, \Gamma^{\rho_1} \seq \Delta_1, \Delta_2, \hF{F_{12}^{\rho_2}}, \Delta^{\rho_1}$}
					\AXC{$\varphi_3$} \noLine
					\UIC{$\Gamma_3 \seq \Delta_3, \hF{F_3^{\rho_2}}$}	\RightLabel{$\rho_2$}
			\BIC{$\Gamma_1,\Gamma_2,\Gamma_3, \Gamma^{\rho_1} \seq \Delta_1,\Delta_2,\Delta_3,\hF{F^{\rho_2}},\Delta^{\rho_1} $}
\end{prooftree}
\end{scriptsize}

\noindent
Then there exists a proof $\varphi^{(4)}$ with $\varphi''' \swapC \varphi^{(4)}$ (permuting $\rho_2$ above the contraction) such that $\varphi^{(4)}$ has the following form:
\begin{tiny}
\begin{prooftree}
\AXC{$\varphi_1$} \noLine
\UIC{$\Gamma_1, \hF{\Gamma_1^{\rho_2}}, \Gamma_1^{\rho_1} \seq \Delta_1, \hF{\Delta_1^{\rho_2}}, \Delta_1^{\rho_1}$}\doubleLine \RightLabel{$w^*$}
\UIC{$\Gamma_1, \hF{\Gamma_1^{\rho_2}}, \hF{\Gamma_2^{\rho_2}}, \Gamma_1^{\rho_1} \seq \Delta_1, \hF{\Delta_1^{\rho_2}}, \hF{\Delta_2^{\rho_2}}, \Delta_1^{\rho_1}$} \doubleLine \RightLabel{$\rho^*_D$}
\UIC{$\Gamma_1, \Gamma_1^{\rho_1} \seq \Delta_1, \hF{F_{12}^{\rho_2}}, \Delta_1^{\rho_1}$}
		\AXC{$\varphi_2$} \noLine
		\UIC{$\Gamma_2, \hB{\Gamma_2^{\rho_2}}, \Gamma_2^{\rho_1} \seq \Delta_2, \hB{\Delta_2^{\rho_2}}, \Delta_2^{\rho_1}$} \doubleLine \RightLabel{$w^*$}
		\UIC{$\Gamma_2, \hB{\Gamma_1^{\rho_2}}, \hB{\Gamma_2^{\rho_2}}, \Gamma_2^{\rho_1} \seq \Delta_2, \hB{\Delta_1^{\rho_2}}, \hB{\Delta_2^{\rho_2}}, \Delta_2^{\rho_1}$} \doubleLine \RightLabel{$\rho^*_D$}
		\UIC{$\Gamma_2, \Gamma_2^{\rho_1} \seq \Delta_2, \hB{F_{12}^{\rho_2}}, \Delta_2^{\rho_1}$}	\RightLabel{$\rho_1$}
	\BIC{$\Gamma_1, \Gamma_2, \Gamma^{\rho_1} \seq \Delta_1, \Delta_2, \hF{F_{12}^{\rho_2}}, \hB{F_{12}^{\rho_2}}, \Delta^{\rho_1}$} 
			\AXC{$\varphi_3$} \noLine
			\UIC{$\Gamma_3 \seq \Delta_3, \hF{F_3^{\rho_2}}$}	\RightLabel{$\rho_2$}
		\BIC{$\Gamma_1,\Gamma_2,\Gamma_3, \Gamma^{\rho_1} \seq \Delta_1,\Delta_2,\Delta_3,\hF{F_{12}^{\rho_2}},\hF{F^{\rho_2}},\Delta^{\rho_1} $}
					\AXC{$\varphi_3$} \noLine
					\UIC{$\Gamma_3 \seq \Delta_3, \hB{F_3^{\rho_2}}$}	\RightLabel{$\rho_2$}
			  \BIC{$\Gamma_1,\Gamma_2,\Gamma_3,\Gamma_3, \Gamma^{\rho_1} \seq \Delta_1,\Delta_2,\Delta_3,\Delta_3,\hF{F^{\rho_2}},\hB{F^{\rho_2}},\Delta^{\rho_1} $}
 \RightLabel{$c_r$}
	        \UIC{$\Gamma_1,\Gamma_2,\Gamma_3,\Gamma_3, \Gamma^{\rho_1} \seq \Delta_1,\Delta_2,\Delta_3,\Delta_3,\hE{F^{\rho_2}},\Delta^{\rho_1} $} \doubleLine \RightLabel{$c^*$}
	        \UIC{$\Gamma_1,\Gamma_2,\Gamma_3, \Gamma^{\rho_1} \seq \Delta_1,\Delta_2,\Delta_3,\hE{F^{\rho_2}},\Delta^{\rho_1} $}
\end{prooftree}
\end{tiny}

\noindent
Finally, there exists a proof $\psi'$ with $\varphi^{(4)} \swapI^* \psi'$ (permuting each copy of $\rho_2$ above $\rho_1$) such that $\psi'$ has the following form:
\begin{tiny}
\begin{prooftree}
\AXC{$\varphi_1$} \noLine
\UIC{$\Gamma_1, \hF{\Gamma_1^{\rho_2}}, \Gamma_1^{\rho_1} \seq \Delta_1, \hF{\Delta_1^{\rho_2}}, \Delta_1^{\rho_1}$}\doubleLine \RightLabel{$w^*$}
\UIC{$\Gamma_1, \hF{\Gamma_1^{\rho_2}}, \hF{\Gamma_2^{\rho_2}}, \Gamma_1^{\rho_1} \seq \Delta_1, \hF{\Delta_1^{\rho_2}}, \hF{\Delta_2^{\rho_2}}, \Delta_1^{\rho_1}$} \doubleLine \RightLabel{$\rho^*_D$}
\UIC{$\Gamma_1, \Gamma_1^{\rho_1} \seq \Delta_1, \hF{F_{12}^{\rho_2}}, \Delta_1^{\rho_1}$}
		\AXC{$\varphi_3$} \noLine
		\UIC{$\Gamma_3 \seq \Delta_3, \hF{F_3^{\rho_2}}$}	\RightLabel{$\rho_2$}
	\BIC{$\Gamma_1,\Gamma_3, \Gamma_1^{\rho_1} \seq \Delta_1,\Delta_3,\hF{F^{\rho_2}},\Delta_1^{\rho_1} $}
			\AXC{$\varphi_2$} \noLine
			\UIC{$\Gamma_2, \hB{\Gamma_2^{\rho_2}}, \Gamma_2^{\rho_1} \seq \Delta_2, \hB{\Delta_2^{\rho_2}}, \Delta_2^{\rho_1}$} \doubleLine \RightLabel{$w^*$}
			\UIC{$\Gamma_2, \hB{\Gamma_1^{\rho_2}}, \hB{\Gamma_2^{\rho_2}}, \Gamma_2^{\rho_1} \seq \Delta_2, \hB{\Delta_1^{\rho_2}}, \hB{\Delta_2^{\rho_2}}, \Delta_2^{\rho_1}$} \doubleLine \RightLabel{$\rho^*_D$}
			\UIC{$\Gamma_2, \Gamma_2^{\rho_1} \seq \Delta_2, \hB{F_{12}^{\rho_2}}, \Delta_2^{\rho_1}$}	
					\AXC{$\varphi_3$} \noLine
					\UIC{$\Gamma_3 \seq \Delta_3, \hB{F_3^{\rho_2}}$}	\RightLabel{$\rho_2$}
			  \BIC{$\Gamma_2,\Gamma_3, \Gamma_2^{\rho_1} \seq \Delta_2, \Delta_3, \hB{F^{\rho_2}}, \Delta_2^{\rho_1} $} \RightLabel{$\rho_1$}
		\BIC{$\Gamma_1, \Gamma_2, \Gamma^{\rho_1} \seq \Delta_1, \Delta_2, \hF{F_{12}^{\rho_2}}, \hB{F_{12}^{\rho_2}}, \Delta^{\rho_1}$} \RightLabel{$c_r$}
	   \UIC{$\Gamma_1,\Gamma_2,\Gamma_3,\Gamma_3, \Gamma^{\rho_1} \seq \Delta_1,\Delta_2,\Delta_3,\Delta_3,\hE{F^{\rho_2}},\Delta^{\rho_1} $} \doubleLine \RightLabel{$c^*$}
	   \UIC{$\Gamma_1,\Gamma_2,\Gamma_3, \Gamma^{\rho_1} \seq \Delta_1,\Delta_2,\Delta_3,\hE{F^{\rho_2}},\Delta^{\rho_1} $}
\end{prooftree}
\end{tiny}
\end{landscape}

\noindent
Consequently, there exists a proof $\psi$ with $\varphi \swap^* \psi$ (namely, the proof obtained from $\varphi$ by rewriting its subproof $\varphi'$ to $\psi'$ as shown above) where $\rho_2$ and all unary inferences $\rho_{D_i}$ on which it depends have been permuted above $\rho_1$, so that $\struct{\psi}$ is equal to $\struct{\varphi}$ with $(\hF{S'} \structplus \hF{S''}) \hF{\structtimes} \hF{S}$ rewritten to $(\hF{S'} \hF{\structtimes} \hF{S}) \structplus (\hF{S''} \hF{\structtimes} \hF{S})$.



\paragraph{\textbf{Degenerate case,}} when the redex has the form $\hF{S} \hF{\structtimes_{\rho}} (S_1 \structplus \ldots \structplus S_n)$ and is rewritten to $S_1 \structplus \ldots \structplus S_n$:
Let $\varphi'$ be the subproof ending with $\rho$, $\varphi'_1$ be its left subproof (corresponding to $\hF{S}$) and $\varphi'_2$ be its right subproof (corresponding to $(S_1 \structplus \ldots \structplus S_n)$).
Since $\varphi'_2$ contains no formula from $\occInference{\varphi}{\rho}$, it must be the case that all auxiliary formulas of $\rho$ occurring in its right premise are descendants of main formulas of weakening inferences. $\rho$ is a (partially) degenerate inference.
Moreover, since the innermost rewriting strategy guarantees that $\hF{S} \hF{\structtimes} (S_1 \structplus \ldots \structplus S_n)$ is a minimal redex, the auxiliary formulas in the right premise of $\rho$ are not ancestors of any binary inference operating (for if they were, there would be a redex in $(S_1 \structplus \ldots \structplus S_n)$). Therefore, in the sequence rewriting $\varphi'_2$ into its $\swapW$-normal-form $\varphi''_2$, none of the rewriting rules of $\swapWD$ that delete binary inferences is used. Consequently, the characteristic formula remains unchanged when $\varphi'_2$ is rewritten into $\varphi''_2$. 
Let $\varphi''$ be the result of replacing $\varphi'_2$ by $\varphi''_2$ in $\varphi'$.$\varphi''$ is of the following form:
\begin{prooftree}
\AXC{$\varphi'_1$}\noLine
\UIC{$\hB{\Gamma_1^{\rho}}, \Gamma_1  \seq \hB{\Delta_1^{\rho}}, \Delta_1$} 
		\AXC{$\varphi'''_2$}\noLine
		\UIC{$\Gamma_2  \seq \Delta_2$} \doubleLine \RightLabel{$w^*$}
		\UIC{$\hB{\Gamma_2^{\rho}}, \Gamma_2  \seq \hB{\Delta_2^{\rho}}, \Delta_2$}\RightLabel{$\rho$}
	\BIC{$\hB{\Gamma^{\rho}}, \Gamma_2, \Gamma_1  \seq \hB{\Delta^{\rho}}, \Delta_2, \Delta_1$} 
\end{prooftree}
%
with $\varphi''_2$ being:
%
\begin{prooftree}
		\AXC{$\varphi'''_2$}\noLine
		\UIC{$\Gamma_2  \seq \Delta_2$} \doubleLine \RightLabel{$w^*$}
		\UIC{$\hB{\Gamma_2^{\rho}}, \Gamma_2  \seq \hB{\Delta_2^{\rho}}, \Delta_2$}
\end{prooftree}

\noindent
By using one of the rewriting rules of $\swapWD$, $\varphi''$ can be rewritten to $\psi'$ below:
\begin{prooftree}
\AXC{$\varphi'''_2$}\noLine
\UIC{$\Gamma_2  \seq \Delta_2$} \doubleLine \RightLabel{$w^*$}
\UIC{$\hB{\Gamma^{\rho}}, \Gamma_2, \Gamma_1  \seq \hB{\Delta^{\rho}}, \Delta_2, \Delta_1$}  
\end{prooftree}

\noindent
Therefore, there exists a proof $\psi$ (namely, the proof obtainable from $\varphi$ by replacing its subproof $\varphi'$ by $\psi'$) such that $\varphi \swapW \psi$ and $\struct{\psi}$ is $\struct{\varphi}$ with $\hF{S} \hF{\structtimes} (S_1 \structplus \ldots \structplus S_n)$ rewritten to $S_1 \structplus \ldots \structplus S_n$.
\hfill\QED
\end{proof}



\bibliographystyle{plain}
\bibliography{Bibliography}



\end{document}
