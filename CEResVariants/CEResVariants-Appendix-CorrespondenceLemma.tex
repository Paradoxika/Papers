\section{Correspondence between $\normalizePlusTimesS$ and $\swap$}
\label{sec:Correspondence}

\begin{lemma}%[Correspondence between $\normalizePlusTimesS$ and $\swap$]
%\label{lemma:SwapNormalizeCorrespondence}
If $\varphi$ is skolemized and $\struct{\varphi} \normalizePlusTimesS S$, then there exists a proof $\psi$ such that $\varphi \swap^* \psi$ and $\struct{\psi} = S$. 
\end{lemma}
\begin{proof}

The proof can be subdivided according to all possible cases of rewriting according to $\normalizePlusTimesS$. Here only three cases are shown, but every other case is either symmetric or analogous to one of these three cases.


\paragraph{\textbf{Case of rewriting with no duplication,}} 
when the redex has the form $\hF{S} \hF{\structtimes_{\rho_2}} (S' \structplus_{\rho_1} \hF{S''})$ and is rewritten to $S' \structplus_{\rho_1} (\hF{S} \hF{\structtimes_{\rho_2}} \hF{S''})$):
Since $\rho_1$ operates on cut ancestors and $\rho_2$ operates on end-sequent ancestors, $\rho_2$ is not directly dependent on $\rho_1$. Moreover, $\varphi$ is skolemized and thus $\rho_2$ is also not eigen-variable dependent on $\rho_1$. 
Since $\hF{S}$ is distributed only to $\hF{S''}$, only $\hF{S''}$ contains formulas from $\occInference{\varphi}{\rho_2}$. Therefore, $\occInference{\varphi}{\rho_2}$ has formulas in at most one premise of $\rho_1$, and hence ancestors of auxiliary formulas of $\rho_2$ occur in at most one premise of $\rho_1$. Therefore $\rho_2$ is independent of $\rho_1$. Moreover, any inference $\rho_i$ on the path between $\rho_2$ and $\rho_1$ and on which $\rho_2$ directly depends is also independent of $\rho_1$. Consequently, there exists a proof $\psi$ with $\varphi \swapI^* \psi$ where $\rho_2$ and all inferences $\rho_i$ on which it depends have been swapped above $\rho_1$, so that $\struct{\psi}$ is equal to $\struct{\varphi}$ with $\hF{S} \hF{\structtimes} (S' \structplus \hF{S''})$ rewritten to $S' \structplus (\hF{S} \hF{\structtimes} \hF{S''})$.


\paragraph{\textbf{Case of rewriting with duplication,}} when the redex has the form $(\hF{S'} \structplus_{\rho_1} \hF{S''}) \hF{\structtimes_{\rho_2}} \hF{S}$ and is rewritten to $(\hF{S'} \hF{\structtimes_{\rho_2}} \hF{S}) \structplus_{\rho_1} (\hF{S''} \hF{\structtimes_{\rho_2}} \hF{S})$:
Since $\rho_1$ operates on cut ancestors and $\rho_2$ operates on end-sequent ancestors, $\rho_2$ is not directly dependent on $\rho_1$. Moreover, $\varphi$ is skolemized and thus $\rho_2$ is also not eigen-variable dependent on $\rho_1$. However, as both $\hF{S'}$ and $\hF{S''}$ contain formulas from $\occInference{\varphi}{\rho_2}$, it must be the case that $\rho_2 \dependentI \rho_1$. In the sequent calculus $\LK$, this can only happen if there exists a sequence of unary\footnote{If the inferences were not unary, the redex would not be of the form $(\hF{S'} \structplus_{\rho_1} \hF{S''}) \hF{\structtimes_{\rho_2}} \hF{S}$.} inferences $\rho^*_D \equiv (\rho_{D_1},\ldots,\rho_{D_n})$ on the path between $\rho_2$ and $\rho_1$ such that $\rho_{D_i}$ is indirectly dependent on $\rho_1$ and $\rho_2$ depends on $\rho_{D_i}$, for any $i$ such that $1 \leq i \leq n$. Moreover, any inference $\rho_{D_i}$ (on the path between $\rho_2$ and $\rho_1$) on which $\rho_2$ directly depends is also independent of $\rho_1$.

Let $\varphi'$ be the subproof of $\varphi$ having the conclusion sequent of $\rho_2$ as its end-sequent. Then, there exists a proof $\varphi''$ with $\varphi' \swap^* \varphi''$ (permuting $\rho^*_D$ and $\rho_2$ above all inferences on which they do not depend) such that $\varphi''$ has the following form (or a form symmetric to it):
\begin{scriptsize}
\begin{prooftree}
\AXC{$\varphi_1$} \noLine
\UIC{$\Gamma_1, \hF{\Gamma_1^{\rho_2}}, \Gamma_1^{\rho_1} \seq \Delta_1, \hF{\Delta_1^{\rho_2}}, \Delta_1^{\rho_1}$}	
		\AXC{$\varphi_2$} \noLine
		\UIC{$\Gamma_2, \hF{\Gamma_2^{\rho_2}}, \Gamma_2^{\rho_1} \seq \Delta_2, \hF{\Delta_2^{\rho_2}}, \Delta_2^{\rho_1}$}	\RightLabel{$\rho_1$}
	\BIC{$\Gamma_1, \Gamma_2, \hF{\Gamma_1^{\rho_2}}, \hF{\Gamma_2^{\rho_2}}, \Gamma^{\rho_1} \seq \Delta_1, \Delta_2, \hF{\Delta_1^{\rho_2}}, \hF{\Delta_2^{\rho_2}}, \Delta^{\rho_1}$} \doubleLine \RightLabel{$\rho^*_D$}
	\UIC{$\Gamma_1, \Gamma_2, \Gamma^{\rho_1} \seq \Delta_1, \Delta_2, \hF{F_{12}^{\rho_2}}, \Delta^{\rho_1}$}
					\AXC{$\varphi_3$} \noLine
					\UIC{$\Gamma_3 \seq \Delta_3, \hF{F_3^{\rho_2}}$}	\RightLabel{$\rho_2$}
			\BIC{$\Gamma_1,\Gamma_2,\Gamma_3, \Gamma^{\rho_1} \seq \Delta_1,\Delta_2,\Delta_3,\hF{F^{\rho_2}},\Delta^{\rho_1} $}
\end{prooftree}
\end{scriptsize}

\begin{landscape}
\noindent
Then there exists a proof $\varphi'''$ with $\varphi'' \swap^* \varphi'''$ (permuting $\rho^*_D$ above  $\rho_1$) such that $\varphi'''$ has the following form:
\begin{scriptsize}
\begin{prooftree}
\AXC{$\varphi_1$} \noLine
\UIC{$\Gamma_1, \hF{\Gamma_1^{\rho_2}}, \Gamma_1^{\rho_1} \seq \Delta_1, \hF{\Delta_1^{\rho_2}}, \Delta_1^{\rho_1}$}\doubleLine \RightLabel{$w^*$}
\UIC{$\Gamma_1, \hF{\Gamma_1^{\rho_2}}, \hF{\Gamma_2^{\rho_2}}, \Gamma_1^{\rho_1} \seq \Delta_1, \hF{\Delta_1^{\rho_2}}, \hF{\Delta_2^{\rho_2}}, \Delta_1^{\rho_1}$} \doubleLine \RightLabel{$\rho^*_D$}
\UIC{$\Gamma_1, \Gamma_1^{\rho_1} \seq \Delta_1, \hF{F_{12}^{\rho_2}}, \Delta_1^{\rho_1}$}
		\AXC{$\varphi_2$} \noLine
		\UIC{$\Gamma_2, \hF{\Gamma_2^{\rho_2}}, \Gamma_2^{\rho_1} \seq \Delta_2, \hF{\Delta_2^{\rho_2}}, \Delta_2^{\rho_1}$} \doubleLine \RightLabel{$w^*$}
		\UIC{$\Gamma_2, \hF{\Gamma_1^{\rho_2}}, \hF{\Gamma_2^{\rho_2}}, \Gamma_2^{\rho_1} \seq \Delta_2, \hF{\Delta_1^{\rho_2}}, \hF{\Delta_2^{\rho_2}}, \Delta_2^{\rho_1}$} \doubleLine \RightLabel{$\rho^*_D$}
		\UIC{$\Gamma_2, \Gamma_2^{\rho_1} \seq \Delta_2, \hF{F_{12}^{\rho_2}}, \Delta_2^{\rho_1}$}	\RightLabel{$\rho_1$}
	\BIC{$\Gamma_1, \Gamma_2, \Gamma^{\rho_1} \seq \Delta_1, \Delta_2, \hF{F_{12}^{\rho_2}}, \hF{F_{12}^{\rho_2}}, \Delta^{\rho_1}$} \RightLabel{$c_r$}
	\UIC{$\Gamma_1, \Gamma_2, \Gamma^{\rho_1} \seq \Delta_1, \Delta_2, \hF{F_{12}^{\rho_2}}, \Delta^{\rho_1}$}
					\AXC{$\varphi_3$} \noLine
					\UIC{$\Gamma_3 \seq \Delta_3, \hF{F_3^{\rho_2}}$}	\RightLabel{$\rho_2$}
			\BIC{$\Gamma_1,\Gamma_2,\Gamma_3, \Gamma^{\rho_1} \seq \Delta_1,\Delta_2,\Delta_3,\hF{F^{\rho_2}},\Delta^{\rho_1} $}
\end{prooftree}
\end{scriptsize}

\noindent
Then there exists a proof $\varphi^{(4)}$ with $\varphi''' \swapC \varphi^{(4)}$ (permuting $\rho_2$ above the contraction) such that $\varphi^{(4)}$ has the following form:
\begin{tiny}
\begin{prooftree}
\AXC{$\varphi_1$} \noLine
\UIC{$\Gamma_1, \hF{\Gamma_1^{\rho_2}}, \Gamma_1^{\rho_1} \seq \Delta_1, \hF{\Delta_1^{\rho_2}}, \Delta_1^{\rho_1}$}\doubleLine \RightLabel{$w^*$}
\UIC{$\Gamma_1, \hF{\Gamma_1^{\rho_2}}, \hF{\Gamma_2^{\rho_2}}, \Gamma_1^{\rho_1} \seq \Delta_1, \hF{\Delta_1^{\rho_2}}, \hF{\Delta_2^{\rho_2}}, \Delta_1^{\rho_1}$} \doubleLine \RightLabel{$\rho^*_D$}
\UIC{$\Gamma_1, \Gamma_1^{\rho_1} \seq \Delta_1, \hF{F_{12}^{\rho_2}}, \Delta_1^{\rho_1}$}
		\AXC{$\varphi_2$} \noLine
		\UIC{$\Gamma_2, \hB{\Gamma_2^{\rho_2}}, \Gamma_2^{\rho_1} \seq \Delta_2, \hB{\Delta_2^{\rho_2}}, \Delta_2^{\rho_1}$} \doubleLine \RightLabel{$w^*$}
		\UIC{$\Gamma_2, \hB{\Gamma_1^{\rho_2}}, \hB{\Gamma_2^{\rho_2}}, \Gamma_2^{\rho_1} \seq \Delta_2, \hB{\Delta_1^{\rho_2}}, \hB{\Delta_2^{\rho_2}}, \Delta_2^{\rho_1}$} \doubleLine \RightLabel{$\rho^*_D$}
		\UIC{$\Gamma_2, \Gamma_2^{\rho_1} \seq \Delta_2, \hB{F_{12}^{\rho_2}}, \Delta_2^{\rho_1}$}	\RightLabel{$\rho_1$}
	\BIC{$\Gamma_1, \Gamma_2, \Gamma^{\rho_1} \seq \Delta_1, \Delta_2, \hF{F_{12}^{\rho_2}}, \hB{F_{12}^{\rho_2}}, \Delta^{\rho_1}$} 
			\AXC{$\varphi_3$} \noLine
			\UIC{$\Gamma_3 \seq \Delta_3, \hF{F_3^{\rho_2}}$}	\RightLabel{$\rho_2$}
		\BIC{$\Gamma_1,\Gamma_2,\Gamma_3, \Gamma^{\rho_1} \seq \Delta_1,\Delta_2,\Delta_3,\hF{F_{12}^{\rho_2}},\hF{F^{\rho_2}},\Delta^{\rho_1} $}
					\AXC{$\varphi_3$} \noLine
					\UIC{$\Gamma_3 \seq \Delta_3, \hB{F_3^{\rho_2}}$}	\RightLabel{$\rho_2$}
			  \BIC{$\Gamma_1,\Gamma_2,\Gamma_3,\Gamma_3, \Gamma^{\rho_1} \seq \Delta_1,\Delta_2,\Delta_3,\Delta_3,\hF{F^{\rho_2}},\hB{F^{\rho_2}},\Delta^{\rho_1} $}
 \RightLabel{$c_r$}
	        \UIC{$\Gamma_1,\Gamma_2,\Gamma_3,\Gamma_3, \Gamma^{\rho_1} \seq \Delta_1,\Delta_2,\Delta_3,\Delta_3,\hE{F^{\rho_2}},\Delta^{\rho_1} $} \doubleLine \RightLabel{$c^*$}
	        \UIC{$\Gamma_1,\Gamma_2,\Gamma_3, \Gamma^{\rho_1} \seq \Delta_1,\Delta_2,\Delta_3,\hE{F^{\rho_2}},\Delta^{\rho_1} $}
\end{prooftree}
\end{tiny}

\noindent
Finally, there exists a proof $\psi'$ with $\varphi^{(4)} \swapI^* \psi'$ (permuting each copy of $\rho_2$ above $\rho_1$) such that $\psi'$ has the following form:
\begin{tiny}
\begin{prooftree}
\AXC{$\varphi_1$} \noLine
\UIC{$\Gamma_1, \hF{\Gamma_1^{\rho_2}}, \Gamma_1^{\rho_1} \seq \Delta_1, \hF{\Delta_1^{\rho_2}}, \Delta_1^{\rho_1}$}\doubleLine \RightLabel{$w^*$}
\UIC{$\Gamma_1, \hF{\Gamma_1^{\rho_2}}, \hF{\Gamma_2^{\rho_2}}, \Gamma_1^{\rho_1} \seq \Delta_1, \hF{\Delta_1^{\rho_2}}, \hF{\Delta_2^{\rho_2}}, \Delta_1^{\rho_1}$} \doubleLine \RightLabel{$\rho^*_D$}
\UIC{$\Gamma_1, \Gamma_1^{\rho_1} \seq \Delta_1, \hF{F_{12}^{\rho_2}}, \Delta_1^{\rho_1}$}
		\AXC{$\varphi_3$} \noLine
		\UIC{$\Gamma_3 \seq \Delta_3, \hF{F_3^{\rho_2}}$}	\RightLabel{$\rho_2$}
	\BIC{$\Gamma_1,\Gamma_3, \Gamma_1^{\rho_1} \seq \Delta_1,\Delta_3,\hF{F^{\rho_2}},\Delta_1^{\rho_1} $}
			\AXC{$\varphi_2$} \noLine
			\UIC{$\Gamma_2, \hB{\Gamma_2^{\rho_2}}, \Gamma_2^{\rho_1} \seq \Delta_2, \hB{\Delta_2^{\rho_2}}, \Delta_2^{\rho_1}$} \doubleLine \RightLabel{$w^*$}
			\UIC{$\Gamma_2, \hB{\Gamma_1^{\rho_2}}, \hB{\Gamma_2^{\rho_2}}, \Gamma_2^{\rho_1} \seq \Delta_2, \hB{\Delta_1^{\rho_2}}, \hB{\Delta_2^{\rho_2}}, \Delta_2^{\rho_1}$} \doubleLine \RightLabel{$\rho^*_D$}
			\UIC{$\Gamma_2, \Gamma_2^{\rho_1} \seq \Delta_2, \hB{F_{12}^{\rho_2}}, \Delta_2^{\rho_1}$}	
					\AXC{$\varphi_3$} \noLine
					\UIC{$\Gamma_3 \seq \Delta_3, \hB{F_3^{\rho_2}}$}	\RightLabel{$\rho_2$}
			  \BIC{$\Gamma_2,\Gamma_3, \Gamma_2^{\rho_1} \seq \Delta_2, \Delta_3, \hB{F^{\rho_2}}, \Delta_2^{\rho_1} $} \RightLabel{$\rho_1$}
		\BIC{$\Gamma_1, \Gamma_2, \Gamma^{\rho_1} \seq \Delta_1, \Delta_2, \hF{F_{12}^{\rho_2}}, \hB{F_{12}^{\rho_2}}, \Delta^{\rho_1}$} \RightLabel{$c_r$}
	   \UIC{$\Gamma_1,\Gamma_2,\Gamma_3,\Gamma_3, \Gamma^{\rho_1} \seq \Delta_1,\Delta_2,\Delta_3,\Delta_3,\hE{F^{\rho_2}},\Delta^{\rho_1} $} \doubleLine \RightLabel{$c^*$}
	   \UIC{$\Gamma_1,\Gamma_2,\Gamma_3, \Gamma^{\rho_1} \seq \Delta_1,\Delta_2,\Delta_3,\hE{F^{\rho_2}},\Delta^{\rho_1} $}
\end{prooftree}
\end{tiny}
\end{landscape}

\noindent
Consequently, there exists a proof $\psi$ with $\varphi \swap^* \psi$ (namely, the proof obtained from $\varphi$ by rewriting its subproof $\varphi'$ to $\psi'$ as shown above) where $\rho_2$ and all unary inferences $\rho_{D_i}$ on which it depends have been permuted above $\rho_1$, so that $\struct{\psi}$ is equal to $\struct{\varphi}$ with $(\hF{S'} \structplus \hF{S''}) \hF{\structtimes} \hF{S}$ rewritten to $(\hF{S'} \hF{\structtimes} \hF{S}) \structplus (\hF{S''} \hF{\structtimes} \hF{S})$.



\paragraph{\textbf{Degenerate case,}} when the redex has the form $\hF{S} \hF{\structtimes_{\rho}} (S_1 \structplus \ldots \structplus S_n)$ and is rewritten to $S_1 \structplus \ldots \structplus S_n$:
Let $\varphi'$ be the subproof ending with $\rho$, $\varphi'_1$ be its left subproof (corresponding to $\hF{S}$) and $\varphi'_2$ be its right subproof (corresponding to $(S_1 \structplus \ldots \structplus S_n)$).
Since $\varphi'_2$ contains no formula from $\occInference{\varphi}{\rho}$, it must be the case that all auxiliary formulas of $\rho$ occurring in its right premise are descendants of main formulas of weakening inferences. $\rho$ is a (partially) degenerate inference.
Moreover, since the innermost rewriting strategy guarantees that $\hF{S} \hF{\structtimes} (S_1 \structplus \ldots \structplus S_n)$ is a minimal redex, the auxiliary formulas in the right premise of $\rho$ are not ancestors of any binary inference operating (for if they were, there would be a redex in $(S_1 \structplus \ldots \structplus S_n)$). Therefore, in the sequence rewriting $\varphi'_2$ into its $\swapW$-normal-form $\varphi''_2$, none of the rewriting rules of $\swapWD$ that delete binary inferences is used. Consequently, the characteristic formula remains unchanged when $\varphi'_2$ is rewritten into $\varphi''_2$. 
Let $\varphi''$ be the result of replacing $\varphi'_2$ by $\varphi''_2$ in $\varphi'$.$\varphi''$ is of the following form:
\begin{prooftree}
\AXC{$\varphi'_1$}\noLine
\UIC{$\hB{\Gamma_1^{\rho}}, \Gamma_1  \seq \hB{\Delta_1^{\rho}}, \Delta_1$} 
		\AXC{$\varphi'''_2$}\noLine
		\UIC{$\Gamma_2  \seq \Delta_2$} \doubleLine \RightLabel{$w^*$}
		\UIC{$\hB{\Gamma_2^{\rho}}, \Gamma_2  \seq \hB{\Delta_2^{\rho}}, \Delta_2$}\RightLabel{$\rho$}
	\BIC{$\hB{\Gamma^{\rho}}, \Gamma_2, \Gamma_1  \seq \hB{\Delta^{\rho}}, \Delta_2, \Delta_1$} 
\end{prooftree}
%
with $\varphi''_2$ being:
%
\begin{prooftree}
		\AXC{$\varphi'''_2$}\noLine
		\UIC{$\Gamma_2  \seq \Delta_2$} \doubleLine \RightLabel{$w^*$}
		\UIC{$\hB{\Gamma_2^{\rho}}, \Gamma_2  \seq \hB{\Delta_2^{\rho}}, \Delta_2$}
\end{prooftree}

\noindent
By using one of the rewriting rules of $\swapWD$, $\varphi''$ can be rewritten to $\psi'$ below:
\begin{prooftree}
\AXC{$\varphi'''_2$}\noLine
\UIC{$\Gamma_2  \seq \Delta_2$} \doubleLine \RightLabel{$w^*$}
\UIC{$\hB{\Gamma^{\rho}}, \Gamma_2, \Gamma_1  \seq \hB{\Delta^{\rho}}, \Delta_2, \Delta_1$}  
\end{prooftree}

\noindent
Therefore, there exists a proof $\psi$ (namely, the proof obtainable from $\varphi$ by replacing its subproof $\varphi'$ by $\psi'$) such that $\varphi \swapW \psi$ and $\struct{\psi}$ is $\struct{\varphi}$ with $\hF{S} \hF{\structtimes} (S_1 \structplus \ldots \structplus S_n)$ rewritten to $S_1 \structplus \ldots \structplus S_n$.
\hfill\QED
\end{proof}

