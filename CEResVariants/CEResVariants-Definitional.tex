\section{Using Structural Clause Form Transformation}
\label{sec:StructuralClauseForm}


In order to avoid the exponential blow-up in the size of the clause set altogether, this section investigates the use of \emph{structural clause form transformation} \cite{BaazEglyLeitsch2001NormalFormTransformations}, which avoids the distribution of disjunction over conjunction by introducing new defined predicate symbols for each subformula of the formula to be normalized. The number of new predicate symbols is, therefore, linear w.r.t. to the size of the formula. Moreover, each literal is duplicated only a constant number of times. 


\begin{definition}[$\normalizePlusTimesD$]
\label{definition:NormalizationPlusTimesDefinitional}
For every conjunctive or disjunctive formula $S$ with free variables $x_1,\ldots, x_m$, a new predicate symbol $N_{S}(x_1,\ldots, x_m)$ can be created together with its corresponding defining formula:
%
\begin{small}
$$
D_{S} \ \ \defEq \ \
N_{S_1 \structplus \ldots \structplus S_n}(x_1,\ldots, x_m) 
%\biimp N_{n(S'_1) \structplus \ldots \structplus n(S'_n)}(x_1,\ldots, x_m) 
\biimp n(S_1) \structplus \ldots \structplus n(S_n)
$$
$$
D_{S} \ \ \defEq \ \  
N_{S_1 \structtimes \ldots \structtimes S_n}(x_1,\ldots, x_m) 
%\biimp N_{n(S'_1) \structtimes \ldots \structtimes n(S'_n)}(x_1,\ldots, x_m) 
\biimp n(S_1) \structtimes \ldots \structtimes n(S_n)
$$
\end{small}
where:
\begin{small}
$$
n(S_k) = \left\{ \begin{array}{ll}
S_k & \textrm{, if } S_k \textrm{ is a literal} \\
N_{S_k}(y_1,\ldots, y_j) & \textrm{, if } S_k \textrm{ is a non-literal with free variables } y_1,\ldots, y_j
\end{array}\right.
$$
\end{small}

\noindent
Then:
\begin{small}
$$
S \normalizePlusTimesD n(S) \structplus \structplusBig_{\scriptscriptstyle S' \in sub(S) } D_{S'}
$$
\end{small}
where $sub(S)$ is the set of non-literal subformulas of $S$.
\end{definition}

\begin{remark}
The connective $\biimp$ is considered to be just an abbreviation:
\begin{small}
$$
A \biimp B_1 \structtimes \ldots \structtimes B_n  \ \ \defEq  \ \
(\overline{A} \structtimes B_1 \structtimes \ldots \structtimes B_n) \structplus 
(\overline{B_1} \structtimes A) \structplus \ldots \structplus 
(\overline{B_n} \structtimes A)
$$
$$
A \biimp B_1 \structplus \ldots \structplus B_n \ \ \defEq \ \  
(\overline{B_1} \structtimes \ldots \structtimes \overline{B_n} \structtimes A ) \structplus 
(\overline{A} \structtimes B_1) \structplus \ldots \structplus 
(\overline{A} \structtimes B_n)
$$
\end{small}
where $\overline{C}$ is $\structdual D$, if $C = D$, and $D$, if $C = \structdual D$.
\end{remark}


\begin{definition}[Definitional Clause Set]
\label{definition:ClauseSetDefinitional}
The \emph{definitional clause set} $\clausesetDefinitional{\varphi}$ of a 
proof $\varphi$ is the $\normalizePlusTimesD$-normal-form of $\struct{\varphi}$ 
written in sequent notation. Clauses in $\clausesetDefinitional{\varphi}$ are either the \emph{proper clause} $\seq n(\struct{\varphi})$ or \emph{definitional clauses} originating from the defining formulas $D_{S'}$ for each subformula $S'$ of $\struct{\varphi}$.
\end{definition}

\begin{example}
\label{example:PlusTimesDefinitionalNormalization}
Let $\varphi$ be the proof shown in Example \ref{example:CutPertinentStruct}, whose characteristic formulas is:
% \begin{prooftree}
% \AXC{$\hA{A} \seq \hA{A} $}
% 		\AXC{$\hB{B} \seq \hB{B}$} \RightLabel{$\wedge_r^1$}
% 	\BIC{$\hA{A}, \hB{B} \seq \hA{A} \wedge \hB{B}$} \RightLabel{$\wedge_l$}
% 	\UIC{$\hA{A} \wedge \hB{B} \seq \hA{A} \wedge \hB{B}$}
% 				\AXC{$\hC{B} \seq \hC{B} $}
% 						\AXC{$\hD{A} \seq \hD{A}$} \RightLabel{$\wedge_r^2$}
% 					\BIC{$\hD{A}, \hC{B} \seq \hC{B} \wedge \hD{A}$} \RightLabel{$\wedge_l$}
% 					\UIC{$\hD{A} \wedge \hC{B} \seq \hC{B} \wedge \hD{A}$} \RightLabel{$cut^3$}
% 			\BIC{$\hA{A} \wedge \hB{B} \seq \hC{B} \wedge \hD{A}$}
% 							\AXC{$\hE{C} \seq \hE{C}$} 
% 									\AXC{$\hF{C} \seq \hF{C}$} \RightLabel{$cut^4$}
% 								\BIC{$\hE{C} \seq \hF{C}$} \RightLabel{$\vee_l^5$}
% 					\BIC{$(\hA{A} \wedge \hB{B}) \vee \hE{C} \seq \hC{B} \wedge \hD{A}, \hF{C} $} 
% \end{prooftree}
$$
\struct{\varphi} 
\equiv 
((\hA{A} \structplus \hB{B}) \structplus (\hC{\structdual{B}} \structtimes \hD{\structdual{A}}))
\structtimes
(\hE{C} \structplus \hF{\structdual{C}})
$$

\noindent
New predicate symbols can be created and defined by the following formulas: 
$$
D \biimp \hE{C} \structplus \hF{\structdual{C}}
\qquad
E \biimp \hC{\structdual{B}} \structtimes \hD{\structdual{A}}
\qquad
F \biimp \hA{A} \structplus \hB{B}
\qquad
G \biimp F \structplus E
\qquad
H \biimp G \structtimes D
$$

\noindent
The $\normalizePlusTimesD$-normal-form of $\struct{\varphi}$ is:
$$
\begin{array}{rcl}
%S^*	
&           & H \ \structplus \\
&			& 
(\structdual D \structtimes \hE{C})
\structplus
(\structdual D \structtimes \structdual \hF{C})
\structplus
( \structdual \hE{C} \structtimes D \structtimes \hF{C}) \ \structplus \\
&			&
(\structdual E \structtimes \structdual \hC{B} \structtimes \structdual \hD{A} )
\structplus
(E \structtimes \hC{B})
\structplus
(E \structtimes \hD{A}) \ \structplus \\
&			&
(\structdual F \structtimes \hA{A})
\structplus
(\structdual F \structtimes \hB{B})
\structplus
(\structdual \hA{A} \structtimes \structdual \hB{B} \structtimes F) \ \structplus \\
&			&
(\structdual G \structtimes F)
\structplus
(\structdual G \structtimes E)
\structplus
(\structdual E \structtimes \structdual F \structtimes G) \ \structplus \\
&			&
(\structdual H \structtimes G \structtimes D)
\structplus
(\structdual G \structtimes H)
\structplus
(\structdual D \structtimes H) \\
\end{array}
$$

\noindent
The definitional clause set $\clausesetDefinitional{\varphi}$ consists of the following clauses. The proper clause is $\seq H$. All other clauses are definitional clauses.

\begin{multicols}{3}{
{
$\seq H$

$D \seq \hE{C}$

$E, \hC{B}, \hD{A} \seq $

$F \seq \hA{A}$

$G \seq F$

$H \seq G, D$
}

{
$\phantom{\seq H}$

$D, \hF{C} \seq $

$\seq E, \hC{B}$

$F \seq \hB{B}$

$G \seq E$

$G \seq H$
}

{
$\phantom{\seq H}$

$ \hE{C} \seq D, \hF{C}$

$\seq E, \hD{A}$

$\hA{A}, \hB{B} \seq F$

$E, F \seq G$

$D \seq H$
}
}\end{multicols}
\hfill\QED
\end{example}

\noindent
The construction of projections requires special care when definitional clause sets are used. The reason is that the clauses now contain many new predicate symbols which do not occur in the input proof. Fortunately, for all definitional clauses of a definitional clause set, projections can be constructed easily and independently of the input proof, by using definition inference rules in the sequent calculus.

\begin{definition}[Definitional Projection]
\label{definition:DProjectionDefinitional}
The \emph{definitional projection} $\projectionDefinitional{\varphi}{c}$ of 
a proof $\varphi$ w.r.t. a definitional clause $c$ from 
$\clausesetDefinitional{\varphi}$ is constructed according 
to one of the following templates:

\begin{tiny}
\begin{multicols}{2}{
\begin{prooftree}
\AXC{$n(S_1) \seq n(S_1)$}
	\AXC{$\ldots$}
		\AXC{$n(S_n) \seq n(S_n)$} \RightLabel{$\vee^*_l$}
	\TIC{$n(S_1) \vee \ldots \vee n(S_n) \seq n(S_1), \ldots , n(S_n)$} \RightLabel{$d_l$}
	\UIC{$N_{S_1 \structtimes \ldots \structtimes S_n}(x_1,\ldots, x_m) \seq  n(S_1), \ldots , n(S_n)$} 
\end{prooftree}


\begin{prooftree}
\AXC{$n(S_k) \seq n(S_k)$} \doubleLine \RightLabel{$w^*_r$}
\UIC{$n(S_k) \seq n(S_1), \ldots, n(S_n)$} \RightLabel{$\vee^*_r$}
\UIC{$n(S_k) \seq n(S_1) \vee \ldots \vee n(S_n)$} \RightLabel{$d_r$}
\UIC{$n(S_k) \seq N_{S_1 \structtimes \ldots \structtimes S_n}(x_1,\ldots, x_m)$}
\end{prooftree}
}\end{multicols}
\end{tiny}


\begin{tiny}
\begin{multicols}{2}{
\begin{prooftree}
\AXC{$n(S_1) \seq n(S_1)$}
	\AXC{$\ldots$}
		\AXC{$n(S_n) \seq n(S_n)$} \RightLabel{$\wedge^*_r$}
	\TIC{$n(S_1), \ldots , n(S_n) \seq n(S_1) \wedge \ldots \wedge n(S_n)$} \RightLabel{$d_r$}
	\UIC{$n(S_1), \ldots , n(S_n) \seq  N_{S_1 \structplus \ldots \structplus S_n}(x_1,\ldots, x_m)$} 
\end{prooftree}


\begin{prooftree}
\AXC{$n(S_k) \seq n(S_k)$} \doubleLine \RightLabel{$w^*_l$}
\UIC{$n(S_1), \ldots, n(S_n)\seq n(S_k)$} \RightLabel{$\wedge^*_l$}
\UIC{$n(S_1) \wedge \ldots \wedge n(S_n) \seq n(S_k)$} \RightLabel{$d_l$}
\UIC{$N_{S_1 \structplus \ldots \structplus S_n}(x_1,\ldots, x_m) \seq n(S_k)$} 
\end{prooftree}
}\end{multicols}
\end{tiny}

\noindent
If $S_k$ is a negative literal, negation inferences must be added in the templates.
\end{definition}


\begin{example}
\label{example:DProjectionDefinitional}
Definitional projections w.r.t. the clauses shown in Example \ref{example:PlusTimesDefinitionalNormalization} are:

\begin{multicols}{3}{
$\projectionDefinitional{\varphi}{D \seq \hE{C}}$:
\begin{scriptsize}
\begin{prooftree}
\AXC{$\hE{C} \seq \hE{C}$} \RightLabel{$w_l$}
\UIC{$\hE{C}, \neg \hF{C} \seq \hE{C}$} \RightLabel{$\wedge_l$}
\UIC{$\hE{C} \wedge \neg \hF{C} \seq \hE{C}$} \RightLabel{$d_l$}
\UIC{$D \seq \hE{C}$} 
\end{prooftree}
\end{scriptsize}
\vfill\columnbreak


$\projectionDefinitional{\varphi}{D, \hF{C} \seq }$:
\begin{scriptsize}
\begin{prooftree}
\AXC{$\hF{C} \seq \hF{C}$} \RightLabel{$w_l$}
\UIC{$\hE{C}, \hF{C} \seq \hF{C}$} \RightLabel{$\neg_l$}
\UIC{$\hE{C}, \neg \hF{C}, \hF{C} \seq $} \RightLabel{$\wedge_l$}
\UIC{$\hE{C} \wedge \neg \hF{C}, \hF{C} \seq $} \RightLabel{$d_l$}
\UIC{$D, \hF{C} \seq $} 
\end{prooftree}
\end{scriptsize}

$\projectionDefinitional{\varphi}{\hE{C} \seq D, \hF{C}}$:
\begin{scriptsize}
\begin{prooftree}
\AXC{$\hE{C} \seq \hE{C}$}
		\AXC{$\hF{C} \seq \hF{C}$} \RightLabel{$\neg_r$}
		\UIC{$ \seq  \hF{C}, \neg \hF{C}$} \RightLabel{$\wedge_r$}
	\BIC{$\hE{C} \seq \hF{C}, \hE{C} \wedge \neg \hF{C}$} \RightLabel{$d_r$}
	\UIC{$\hE{C} \seq \hF{C}, D$} 
\end{prooftree}
\end{scriptsize}
}
\end{multicols}


\begin{multicols}{3}{
$\projectionDefinitional{\varphi}{\seq E, \hD{A}}$:
\begin{scriptsize}
\begin{prooftree}
\AXC{$\hD{A} \seq \hD{A}$} \RightLabel{$\neg_r$}
\UIC{$\seq \neg \hD{A}, \hD{A}$} \RightLabel{$w_r$}
\UIC{$\seq \neg \hC{B}, \neg \hD{A}, \hD{A}$} \RightLabel{$\vee_r$}
\UIC{$\seq \neg \hC{B} \vee \neg \hD{A}, \hD{A}$} \RightLabel{$d_r$}
\UIC{$\seq E, \hD{A}$} 
\end{prooftree}
\end{scriptsize}

$\projectionDefinitional{\varphi}{\seq E, \hC{B}}$:
\begin{scriptsize}
\begin{prooftree}
\AXC{$\hC{B} \seq \hC{B}$} \RightLabel{$\neg_r$}
\UIC{$\seq \neg \hC{B}, \hC{B}$} \RightLabel{$w_r$}
\UIC{$\seq \neg \hC{B}, \neg \hD{A}, \hC{B}$} \RightLabel{$\vee_r$}
\UIC{$\seq \neg \hC{B} \vee \neg \hD{A}, \hC{B}$} \RightLabel{$d_r$}
\UIC{$\seq E, \hC{B}$} 
\end{prooftree}
\end{scriptsize}

$\projectionDefinitional{\varphi}{E, \hD{A}, \hC{B} \seq }$:
\begin{tiny}
\begin{prooftree}
\AXC{$\hC{B} \seq \hC{B}$} \RightLabel{$\neg_l$}
\UIC{$\neg \hC{B}, \hC{B} \seq $}
		\AXC{$\hD{A} \seq \hD{A}$} \RightLabel{$\neg_l$}
		\UIC{$\neg \hD{A}, \hD{A} \seq $} \RightLabel{$\vee_l$}
	\BIC{$\neg \hC{B} \vee \neg \hD{A}, \hC{B}, \hD{A} \seq $} \RightLabel{$d_l$}
	\UIC{$E, \hC{B}, \hD{A} \seq $} 
\end{prooftree}
\end{tiny}
}
\end{multicols}




\begin{multicols}{3}{
$\projectionDefinitional{\varphi}{F \seq \hA{A}}$:
\begin{scriptsize}
\begin{prooftree}
\AXC{$\hA{A} \seq \hA{A}$} \RightLabel{$w_l$}
\UIC{$\hA{A}, \hB{B} \seq \hA{A}$} \RightLabel{$\wedge_l$}
\UIC{$\hA{A} \wedge \hB{B} \seq \hA{A}$} \RightLabel{$d_l$}
\UIC{$F \seq \hA{A}$} 
\end{prooftree}
\end{scriptsize}

$\projectionDefinitional{\varphi}{F \seq \hB{B}}$:
\begin{scriptsize}
\begin{prooftree}
\AXC{$\hB{B} \seq \hB{B}$} \RightLabel{$w_l$}
\UIC{$\hA{A}, \hB{B} \seq \hB{B}$} \RightLabel{$\wedge_l$}
\UIC{$\hA{A} \wedge \hB{B} \seq \hB{B}$} \RightLabel{$d_l$}
\UIC{$F \seq \hB{B}$} 
\end{prooftree}
\end{scriptsize}

$\projectionDefinitional{\varphi}{\hA{A}, \hB{B} \seq F}$:
\begin{scriptsize}
\begin{prooftree}
\AXC{$\hA{A} \seq \hA{A}$}
		\AXC{$\hB{B} \seq \hB{B}$} \RightLabel{$\wedge_r$}
	\BIC{$\hA{A}, \hB{B} \seq \hA{A} \wedge \hB{B}$} \RightLabel{$d_r$}
	\UIC{$\hA{A}, \hB{B} \seq F$} 
\end{prooftree}
\end{scriptsize}
}
\end{multicols}





\begin{multicols}{3}{
$\projectionDefinitional{\varphi}{G \seq F}$:
\begin{scriptsize}
\begin{prooftree}
\AXC{$F \seq F$} \RightLabel{$w_l$}
\UIC{$F, E \seq F$} \RightLabel{$\wedge_l$}
\UIC{$F \wedge E \seq F$} \RightLabel{$d_l$}
\UIC{$G \seq F$} 
\end{prooftree}
\end{scriptsize}

$\projectionDefinitional{\varphi}{G \seq E}$:
\begin{scriptsize}
\begin{prooftree}
\AXC{$E \seq E$} \RightLabel{$w_l$}
\UIC{$F, E \seq E$} \RightLabel{$\wedge_l$}
\UIC{$F \wedge E \seq E$} \RightLabel{$d_l$}
\UIC{$G \seq E$} 
\end{prooftree}
\end{scriptsize}

$\projectionDefinitional{\varphi}{F, E \seq G}$:
\begin{scriptsize}
\begin{prooftree}
\AXC{$F \seq F$}
		\AXC{$E \seq E$} \RightLabel{$\wedge_r$}
	\BIC{$F, E \seq F \wedge E$} \RightLabel{$d_r$}
	\UIC{$F, E \seq G$} 
\end{prooftree}
\end{scriptsize}
}
\end{multicols}


\begin{multicols}{3}{
$\projectionDefinitional{\varphi}{D \seq H}$:
\begin{scriptsize}
\begin{prooftree}
\AXC{$D \seq D$} \RightLabel{$w_r$}
\UIC{$D \seq G, D$} \RightLabel{$\vee_r$}
\UIC{$D \seq G\vee D$} \RightLabel{$d_r$}
\UIC{$D \seq H$} 
\end{prooftree}
\end{scriptsize}

$\projectionDefinitional{\varphi}{G \seq H}$:
\begin{scriptsize}
\begin{prooftree}
\AXC{$G \seq G$} \RightLabel{$w_r$}
\UIC{$G \seq G, D$} \RightLabel{$\vee_r$}
\UIC{$G \seq G\vee D$} \RightLabel{$d_r$}
\UIC{$G \seq H$} 
\end{prooftree}
\end{scriptsize}

$\projectionDefinitional{\varphi}{H \seq G, D}$:
\begin{scriptsize}
\begin{prooftree}
\AXC{$G \seq G$}
		\AXC{$D \seq D$} \RightLabel{$\vee_l$}
	\BIC{$G \vee D \seq G, D$} \RightLabel{$d_l$}
	\UIC{$H \seq G, D$} 
\end{prooftree}
\end{scriptsize}
}
\end{multicols}
%\hfill\QED
\end{example}


\noindent
While projections w.r.t. definitional clauses can always be easily constructed 
using the four templates given in Definition \ref{definition:DProjectionDefinitional},
the proper clause requires a more sophisticated inductively constructed projection 
that actually depends on the input proof.


\begin{definition}[Proper Projection]
\label{definition:DProjectionProper}
The \emph{proper projection} $\projectionProper{\varphi}{\seq n(\struct{\varphi})}$ of a
proof $\varphi$ w.r.t. its proper clause $\seq n(\struct{\varphi})$ is constructed 
inductively. Let $\varphi'$ be a subproof of $\varphi$ and $\rho$ be its last inference.
Let $S'$ be the subformula of $\struct{\varphi}$ at $\varphi'$. The following cases can be distinguished:

\begin{itemize}
\item $\rho$ is an axiom inference: Then $\varphi'$ is of the form:

\begin{small}
\begin{prooftree}
\AXC{$ $} \RightLabel{$\rho$}
\UIC{$A \seq A$}
\end{prooftree}
\end{small}

	\begin{itemize}
	\item If both occurrences of $A$ are cut-ancestors, then let $\varphi''$ be:

	\begin{small}
	\begin{prooftree}
	\AXC{$ $} \RightLabel{$\rho$}
	\UIC{$A \seq A$} \RightLabel{$\neg_r$}
	\UIC{$\seq \neg A, A$} \RightLabel{$\vee_r$}
	\UIC{$\seq \neg A \vee A$} \RightLabel{$d_r$}
	\UIC{$\seq n(S')$}
	\end{prooftree}
	\end{small}

	\item If only the antecedent's $A$ is a cut-ancestor, then let $\varphi''$ be:

	\begin{small}
	\begin{prooftree}
	\AXC{$ $} \RightLabel{$\rho$}
	\UIC{$A \seq A$} \RightLabel{$\neg_r$}
	\UIC{$\seq \neg A, A$}
	\end{prooftree}
	\end{small}

 
	\item Otherwise, let $\varphi'' \defEq \varphi'$
	\end{itemize}

\item $\rho$ is a n-ary inference (with $n \geq 2$): Then $\varphi'$ is of the form:

\begin{small}
\begin{prooftree}
\AXC{$\psi'_1 $} \noLine
\UIC{$\Gamma'_1 \seq \Delta'_1$}
	\AXC{$\ldots$}
		\AXC{$\psi'_n $} \noLine
		\UIC{$\Gamma'_n \seq \Delta'_n$}		\RightLabel{$\rho$}
	\TIC{$\Gamma' \seq \Delta'$}
\end{prooftree}
\end{small}

By induction, $\psi''_k$ is of the form:

\begin{small}
\begin{prooftree}
\AXC{$\psi''_k $} \noLine
\UIC{$\Gamma''_1 \seq \Delta''_1, n(S'_{\psi'_k})$}
\end{prooftree}
\end{small}

where $S'_{\psi'_k}$ is the substruct of $S'$ corresponding to $\psi'_k$.

	\begin{itemize}
	\item $\rho$ operates on end-sequent-ancestors: Then let $\varphi''$ be:

\begin{small}
\begin{prooftree}
\AXC{$\psi''_1 $} \noLine
\UIC{$\Gamma''_1 \seq \Delta''_1, n(S'_{\psi'_1})$}
	\AXC{$\ldots$}
		\AXC{$\psi''_n $} \noLine
		\UIC{$\Gamma''_n \seq \Delta''_n, n(S'_{\psi'_n})$}		\RightLabel{$\rho$}
	\TIC{$\Gamma'' \seq \Delta'', n(S'_{\psi'_1}), \ldots, n(S'_{\psi'_n})$} \RightLabel{$\vee_r$}
	\UIC{$\Gamma'' \seq \Delta'', n(S'_{\psi'_1}) \vee \ldots \vee n(S'_{\psi'_n})$}\RightLabel{$d_r$}
	\UIC{$\Gamma'' \seq \Delta'', n(S')$}
\end{prooftree}
\end{small}

	\item $\rho$ operates on cut-ancestors: Then let $\varphi''$ be:

\begin{small}
\begin{prooftree}
\AXC{$\psi''_1 $} \noLine
\UIC{$\Gamma''_1 \seq \Delta''_1, n(S'_{\psi'_1})$}
	\AXC{$\ldots$}
		\AXC{$\psi''_n $} \noLine
		\UIC{$\Gamma''_n \seq \Delta''_n, n(S'_{\psi'_n})$}		\RightLabel{$\wedge_r$}
	\TIC{$\Gamma'' \seq \Delta'', n(S'_{\psi'_1}) \wedge \ldots \wedge n(S'_{\psi'_n})$} \RightLabel{$d_r$}
	\UIC{$\Gamma'' \seq \Delta'', n(S')$}
\end{prooftree}
\end{small}

	\end{itemize}


\item $\rho$ is a unary inference: Then $\varphi'$ is of the form:

\begin{small}
\begin{prooftree}
\AXC{$\psi' $} \RightLabel{$\rho$}
\UIC{$\Gamma' \seq \Delta'$}
\end{prooftree}
\end{small}

	\begin{itemize}

	\item $\rho$ operates on cut-ancestors: then $\rho$ is simply skipped and thus $\varphi'' = \psi''$.

	\item $\rho$ operates on end-sequent-ancestors: then $\rho$ is performed on the transformed premise and thus $\varphi''$ is of the form:

\begin{small}
\begin{prooftree}
\AXC{$\psi'' $} \RightLabel{$\rho$}
\UIC{$\Gamma'' \seq \Delta''$}
\end{prooftree}
\end{small}

	\end{itemize}

\end{itemize}

\noindent
The \emph{proper projection} $\projectionProper{\varphi}{\seq n(\struct{\varphi})}$ is the final result of this inductive construction. $\projectionProper{\varphi}{\seq n(\struct{\varphi})} \defEq \varphi''$ when $\varphi' = \varphi$.
\end{definition}


\begin{example}
\label{example:DProjectionProper}
The proper projection $\projectionProper{\varphi}{\seq H }$ for the proof $\varphi$ from the previous example is:
\begin{scriptsize}
\begin{prooftree}
\AXC{$\hA{A} \seq \hG{\hA{A}} $}
		\AXC{$\hB{B} \seq \hG{\hB{B}}$} \RightLabel{$\hG{\wedge_r}$}
	\BIC{$\hA{A}, \hB{B} \seq \hG{\hA{A}} \wedge \hG{\hB{B}}$} \RightLabel{$\hG{d_r}$}
	\UIC{$\hA{A}, \hB{B} \seq \hG{F}$} \RightLabel{$\wedge_l$}
	\UIC{$\hA{A} \wedge \hB{B} \seq \hG{F}$}
				\AXC{$\hG{\hC{B}} \seq \hC{B} $} \RightLabel{$\hG{\neg_r}$}
				\UIC{$ \seq \neg \hG{\hC{B}}, \hC{B} $}
						\AXC{$\hG{\hD{A}} \seq \hD{A}$} \RightLabel{$\hG{\neg_r}$}
						\UIC{$ \seq \neg \hG{\hD{A}}, \hD{A}$} \RightLabel{$\wedge_r$}
					\BIC{$ \seq \neg \hG{\hC{B}}, \neg \hG{\hD{A}}, \hC{B} \wedge \hD{A}$} \RightLabel{$\hG{\vee_r}$}
					\UIC{$ \seq \neg \hG{\hC{B}} \vee \neg \hG{\hD{A}}, \hC{B} \wedge \hD{A}$} \RightLabel{$\hG{d_r}$}
					\UIC{$ \seq \hG{E}, \hC{B} \wedge \hD{A}$} \RightLabel{$\hG{\wedge_r}$}
			\BIC{$\hA{A} \wedge \hB{B} \seq \hG{F} \wedge \hG{E}, \hC{B} \wedge \hD{A}$} \RightLabel{$\hG{d_r}$}
			\UIC{$\hA{A} \wedge \hB{B} \seq \hG{G}, \hC{B} \wedge \hD{A}$}
							\AXC{$\hE{C} \seq \hG{\hE{C}}$} 
									\AXC{$\hG{\hF{C}} \seq \hF{C}$} \RightLabel{$\hG{\neg_r}$}
									\UIC{$\seq \neg \hG{\hF{C}}, \hF{C}$}	 \RightLabel{$\hG{\wedge_r}$}
								\BIC{$\hE{C} \seq \hE{C} \wedge \neg \hG{\hF{C}}, \hF{C}$} \RightLabel{$\hG{d_r}$} 
								\UIC{$\hE{C} \seq \hG{D}, \hF{C}$} \RightLabel{$\vee_l$}
					\BIC{$(\hA{A} \wedge \hB{B}) \vee \hE{C} \seq \hG{G}, \hG{D}, \hC{B} \wedge \hD{A}, \hF{C}$} \RightLabel{$\hG{\vee_r}$} 
					\UIC{$(\hA{A} \wedge \hB{B}) \vee \hE{C} \seq \hG{G} \vee \hG{D}, \hC{B} \wedge \hD{A}, \hF{C}$} \RightLabel{$\hG{d_r}$}
					\UIC{$(\hA{A} \wedge \hB{B}) \vee \hE{C} \seq \hG{H}, \hC{B} \wedge \hD{A}, \hF{C}$}
\end{prooftree}
\end{scriptsize}

\noindent
A step-by-step construction of this projection can be found in \cite{Woltzenlogel-Paleo2009A-General-Analysis-of-Cut-Elimination-by-CERes}.
\hfill\QED
\end{example}

\begin{theorem}[Size of Definitional Clause Set]
\label{theorem:SizeOfDefinitionalClauseSets}
There exists a positive constant $k$ such that $|\clausesetDefinitional{\varphi}| \leq k \proofsizeSymbol{\varphi}$, for any proof $\varphi$.
\end{theorem}
\begin{proof}
The number of subformulas in $\struct{\varphi}$ is linearly proportional to the size of $\struct{\varphi}$, and structural clause form transformation adds three definitional clauses for each subformula. The definitional clauses have at most $k_a + 1$ literals each, where $k_a$ is the maximum arity of disjunctions and conjunctions in $\struct{\varphi}$. Therefore, $|\clausesetDefinitional{\varphi}| \leq k |\struct{\varphi}| \leq k \proofsizeSymbol{\varphi}$, for some positive $k$.
\hfill\QED
\end{proof}

\begin{theorem}[Size of Proper and Definitional Projections]
\label{theorem:SizeOfCEResDProjections}
There exist positive constants $k$ and $k'$ such that, for any proof $\varphi$:
\begin{itemize}
\item $\proofsizeSymbol{\projectionProper{\varphi}{c}} \leq k \proofsizeSymbol{\varphi_n}$
\item $\proofsizeSymbol{\projectionDefinitional{\varphi}{c'}} \leq k'$
\end{itemize}
\end{theorem}
\begin{proof}
Definitional projections have a size that is bounded by a constant proportional to the maximum arity of disjunctions and conjunctions. The proper projection $\projectionProper{\varphi}{c}$ is just $\varphi$ with inferences operating on cut-ancestors replaced by a bounded number of $\neg_r$, $\vee_r$, $\wedge_r$ and $d_r$ inferences.
\hfill\QED
\end{proof}

\begin{landscape}
\begin{definition}[$\CEResD$-normal-form]
$\CEResNFD{\varphi}{\delta}$ denotes the \emph{$\CEResD$-normal-form} of the proof $\varphi$ w.r.t. a resolution refutation $\delta$ of the definitional clause set $\clausesetDefinitional{\varphi}$. It is obtained in the same way as a $\CERes$-normal-form, but using the definitional clause set $\clausesetDefinitional{\varphi}$ instead of $\clauseset{\varphi}$ and using definitional and proper projections instead of the usual projections.
\end{definition}


\begin{example}[$\CEResD$-Normal-Form]
By combining the proper projection shown in Example \ref{example:DProjectionProper} 
and the definitional projections shown in Example \ref{example:DProjectionDefinitional}
with a resolution refutation $\delta$ of $\clausesetDefinitional{\varphi}$, 
the resulting normal form $\CEResNFD{\varphi}{\delta}$ is:

\begin{tiny}
\begin{prooftree}
					\AXC{$ \projectionProper{\varphi}{\seq H} $} \noLine
					\UIC{$(\hA{A} \wedge \hB{B}) \vee \hE{C} \seq \hG{H}, \hC{B} \wedge \hD{A}, \hF{C}$}
		\AXC{$\projectionDefinitional{\varphi}{H \seq G, D}$}
				\AXC{$\projectionDefinitional{\varphi}{G\seq F}$}
						\AXC{$\projectionDefinitional{\varphi}{G\seq E}$}
								\AXC{$\projectionDefinitional{\varphi}{F \seq \hA{A}}$}
										\AXC{$\projectionDefinitional{\varphi}{F \seq \hB{B}}$}
												\AXC{$\projectionDefinitional{\varphi}{E, \hC{B}, \hD{A} \seq }$} \RightLabel{$ $}
											\BIC{$F, E, \hD{A} \seq $} \RightLabel{$cut$}
									\BIC{$F, F, E \seq $} \RightLabel{$c_l$}
									\UIC{$F, E \seq$} \RightLabel{$cut$}
							\BIC{$G, F \seq$} \RightLabel{$cut$}
					\BIC{$G, G \seq$} \RightLabel{$c_l$}
					\UIC{$G \seq$} \RightLabel{$cut$}
			\BIC{$H \seq D$}
												\AXC{$\projectionDefinitional{\varphi}{D \seq \hE{C}}$}
														\AXC{$\projectionDefinitional{\varphi}{D, \hF{C} \seq}$} \RightLabel{$cut$}
													\BIC{$D, D \seq$} \RightLabel{$c_l$}
													\UIC{$D \seq$} \RightLabel{$cut$}
								\BIC{$H \seq$} \RightLabel{$cut$}
	\BIC{$(\hA{A} \wedge \hB{B}) \vee \hE{C} \seq \hC{B} \wedge \hD{A}, \hF{C}$}
\end{prooftree}
\end{tiny}
\hfill\QED
\end{example}

\noindent
Although in a $\CEResD$-normal-form the cuts appear to be atomic, 
the atomic cut-formulas are actually defined predicate symbols that stand
for complex propositional formulas. Therefore, it can be argued that 
$\CEResD$ does not eliminate cuts as much as $\CERes$ and $\CEResSwap$ do. 
Although this is a reasonable objection to $\CEResD$, it should be noted 
that as long as the cuts are quantifier-free, as is the case for 
$\CEResD$-normal-forms, summarized mathematical information in the form 
of Herbrand sequents can still be obtained from the proof \cite{Paleo2007Herbrand-Sequent-Extraction,Paleo2008Herbrand-Sequent-Extraction,HetzlLeitschWellerPaleo2008Herbrand-Sequent-Extraction}.
\end{landscape}
