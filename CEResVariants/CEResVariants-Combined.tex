\section{A Combined Approach}
\label{sec:Combination}

Although the number of defined symbols introduced by the construction of
definitional clause sets is bounded linearly with respect to the size of the
characteristic formula, it still creates a new symbol for every subformula of
the characteristic formula. The number of new symbols can be reduced with a
technique that combines ideas from swapped and definitional clause sets. The
idea is to use $\normalizePlusTimesS$ as long as no duplications occur and
then use $\normalizePlusTimesD$ only for the subformulas that cannot be
normalized with $\normalizePlusTimesS$ without duplications.


\begin{definition}[$\normalizePlusTimesSD$]
\label{definition:NormalizationPlusTimesDefinitionalSwap}
$\normalizePlusTimesSDs$ denotes a restricted form of $\normalizePlusTimesS$ where the distribution of disjunction over conjunction cannot lead to duplications. The first rewriting rule from Definition \ref{definition:NormalizationPlusTimesSwap} is replaced by the rewriting rule below, where $\hF{S}$ is distributed to at most one conjunct $\hF{S_k}$:
$$
\hF{S} \hF{\structtimes} (S_1 \structplus \ldots \structplus \hF{S_k} \structplus \ldots \structplus S_n) \normalizePlusTimesSDs  S_1 \structplus \ldots \structplus (\hF{S} \hF{\structtimes} \hF{S_{k}}) \structplus \ldots \structplus S_n
$$

\noindent
$\normalizePlusTimesSDd$ denotes a restricted form of $\normalizePlusTimesD$, defined by the following rewriting rule, which can be applied only if $S \vee (S_1 \structplus \ldots \structplus S_n)$ is already in $\normalizePlusTimesSDs$-normal-form:
\begin{small}
$$
C[S \vee (S_1 \structplus \ldots \structplus S_n)]   
\normalizePlusTimesSDd 
C[S \vee N(x_1,\ldots, x_m)] \structplus (N(x_1,\ldots, x_m) \biimp S_1 \structplus \ldots \structplus S_n)
$$
\end{small}
where $N$ is a new symbol and $x_1, \ldots, x_m$ are free-variables of $(S_1 \structplus \ldots \structplus S_n)$.

\medskip
\noindent
The relation $\normalizePlusTimesSD$ is the union of $\normalizePlusTimesSDs$ and $\normalizePlusTimesSDd$.
\end{definition}

\begin{definition}[Definitional Swapped Clause Set]
\label{definition:CutPertinentClauseSetSwappeDefinitional}
A \emph{definitional swapped clause set} $\clausesetSwapDef{\varphi}{S}$ of a proof $\varphi$ w.r.t. to a $\normalizePlusTimesSD$-normal-form $S$ of $\struct{\varphi}$ is $S$ written in sequent notation.
Clauses originating from defining equations introduced by $\normalizePlusTimesSDd$ are \emph{definitional clauses}. Non-definitional clauses not containing new symbols are \emph{pure clauses}. All other clauses are \emph{mixed clauses}.
\end{definition}

\begin{remark}
In cases where $\clausesetSwapDef{\varphi}{S_1} = \clausesetSwapDef{\varphi}{S_2}$ for any $S_1$ and $S_2$, the unique definitional swapped clause set is denoted simply as $\clausesetSwapDefUnique{\varphi}$.
\end{remark}



\begin{example}
\label{example:PlusTimesSwappeDefinitionalNormalization}
Let $\varphi$ be the proof shown in Example \ref{example:PlusTimesSwapNormalization}.
Its characteristic formula can be normalized as follows:
$$
\begin{array}{rcl}
\struct{\varphi} 
& \equiv &
((\hA{A} \structplus^1 \hB{B}) \structplus^3 (\hC{\structdual{B}} \structtimes^2 \hD{\structdual{A}}))
\structtimes^5
(\hE{C} \structplus^4 \hF{\structdual{C}}) \\
%
& \normalizePlusTimesSDs &
	((\hA{A} \structplus^1 \hB{B})\structtimes^5 (\hE{C} \structplus^4 \hF{\structdual{C}}))
\structplus^3 
	(\hC{\structdual{B}} \structtimes^2 \hD{\structdual{A}}) \\
%
& \normalizePlusTimesSDs &
	((((\hA{A} \structplus^1 \hB{B})\structtimes^5 \hE{C}) \structplus^4 \hF{\structdual{C}}))
\structplus^3 
	(\hC{\structdual{B}} \structtimes^2 \hD{\structdual{A}}) \\
%
& \normalizePlusTimesSDd &
	(((D_{\hA{A} \structplus \hB{B}} \structtimes^5 \hE{C}) \structplus^4 \hF{\structdual{C}}))
\structplus^3 
	(\hC{\structdual{B}} \structtimes^2 \hD{\structdual{A}}) 
\structplus
(D_{\hA{A} \structplus \hB{B}} \biimp (\hA{A} \structplus^1 \hB{B})) \\
%
& \equiv &
	(D_{\hA{A} \structplus \hB{B}} \structtimes^5 \hE{C}) 
\structplus^4 
	\hF{\structdual{C}}
\structplus^3 
	(\hC{\structdual{B}} \structtimes^2 \hD{\structdual{A}}) 
\structplus \\
&		   &	
	(\neg D_{\hA{A} \structplus \hB{B}} \structtimes \hA{A} ) 
\structplus
	(\neg D_{\hA{A} \structplus \hB{B}} \structtimes \hB{B} )
\structplus
	(D_{\hA{A} \structplus \hB{B}} \structtimes (\neg \hA{A} \structtimes \neg \hB{B})) \\
\end{array}
$$

\noindent
And the corresponding definitional swapped clause set is:
$$
\clausesetSwapDefUnique{\varphi} \equiv  \left\{ \begin{array}{l}
	\seq D_{\hA{A} \structplus \hB{B}} , \hE{C} 
\ \ \ ; \\
	\hF{C} \seq 
\ \ \ ; \\
	\hC{B} , \hD{A} \seq
\ \ \ ; \\
	D_{\hA{A} \structplus \hB{B}} \seq \hA{A}  
\ \ \ ; \\
	D_{\hA{A} \structplus \hB{B}} \seq \hB{B} 
\ \ \ ; \\
	\hA{A}, \hB{B} \seq D_{\hA{A} \structplus \hB{B}} 
\end{array} \right\}
$$

\noindent
$D_{\hA{A} \structplus \hB{B}} \seq \hA{A}$, $D_{\hA{A} \structplus \hB{B}} \seq \hB{B}$ 
and $\hA{A}, \hB{B} \seq D_{\hA{A} \structplus \hB{B}}$ are definitional clauses. 
$\hF{C} \seq $ and 
$\hC{B} , \hD{\structdual{A}} \seq$ are pure clauses. 
And $\seq D_{\hA{A} \structplus \hB{B}} , \hE{C}$ is a mixed clause.
\hfill\QED
\end{example}


\noindent
While construction of definitional swapped clause sets is reasonably
straightforward, the construction of projections
presents some difficulties. As in the case of definitional clause sets, some
clauses in definitional swapped clause sets are definitional, and their projections can be easily constructed according to Definition \ref{ToDo}. Other
clauses are pure in the sense that they do not contain any defined predicate
symbol, and hence their projections can be constructed in the standard way explained in Definition \ref{ToDo}. However, for mixed clauses, which contain a mix of defined and undefined predicate symbols, it is necessary to construct a \emph{mixed projection}, which combines the construction methods of standard and proper projections (from Definition \ref{ToDo}). 



\begin{definition}[Encapsulated Formulas]
\label{definition:EncapsulatedFormulaOccurrences}
Let $S$ be a characteristic formula and $S'$ be a subformula of $S$ having $N$ as the new predicate symbol for $S'$ created during the $\normalizePlusTimesSD$-normalization of $S$. Then, the \emph{encapsulated formulas} of $N$ are all the atomic formulas of $S'$.
\end{definition}

\begin{example}
\label{example:EncapsulatedFormulaOccurrences}
The formulas encapsulated by the new predicate symbol $D_{\hA{A} \structplus \hB{B}}$ of the $\normalizePlusTimesSD$-normal-form of the struct $\struct{\varphi}$ shown in Example \ref{example:PlusTimesSwappeDefinitionalNormalization} are: $\hA{A}$ and $\hB{B}$.
\hfill\QED
\end{example}


% \begin{definition}[Encapsulated Inferences]
% \label{definition:EncapsulatedInferences}
% Let $S$ be a cut-pertinent struct of a proof $\varphi$ and $S'$ be a substruct of $S$. Let $N_{S'}$ be the defined predicate for $S'$. Then, every inference $\rho$ of $\varphi$ which corresponds to a connective $\structplus_{\rho}$ or $\structtimes_{\rho}$ in $S'$ or that is an axiom inference having a formula occurrence of $S'$ in its conclusion sequent is an \emph{encapsulated inference} of $N_{S'}$.
% \end{definition}

% \begin{example}[Encapsulated Inferences]
% \label{example:EncapsulatedInferences}


% The encapsulated inferences of the defined predicate $D_{\hA{A} \structplus \hB{B}}$ of the $\normalizePlusTimesSD$-normal-form of the struct $\struct{\varphi}$ shown in Example \ref{example:PlusTimesSwappeDefinitionalNormalization} are: $\wedge^1_r$ and the axiom inferences having $\hA{A} \seq \hA{A}$ and $\hB{B} \seq \hB{B}$ as conclusion sequents.
% \end{example}


Roughly, constructing a mixed DW-projection is initially similar to constructing an O-projection, taking care to include encapsulated formula occurrences in the slice. Later cut-pertinent inferences are replaced by $\wedge_r$ and $d_r$ inferences, similarly to what is done during the construction of proper D-projections, in order to re-encapsulate the encapsulated formula occurrences into the defined predicate symbol.


\begin{definition}[Mixed DW-Projection]
\label{definition:DWProjectionMixed}
\index{Projection!Mixed DW-Projection}
Let $\varphi$ be a proof and $c$ a mixed clause in $\clausesetSwapDef{\varphi}{S}$. Let $\Omega_E$ and $\Upsilon_E$ be the sets of, respectively, encapsulated formula occurrences and encapsulated inferences of defined predicates occurring in $c$. Let $\Omega_c$ be the set of undefined formula occurrences in $c$. Then the \emph{mixed DW-projection} of $\varphi$ with respect to $c$ can be computed according to the following steps:

\begin{enumerate}
%\item Construct $\varphi^1 \defEq \slice{\varphi}{\Omega_E \cup \Omega_c}$.
\item Replace the inferences of $\Upsilon_E$ in $\varphi^1$ by $\neg_r$, $\wedge_r$, $\vee_r$ and $d_r$ (analogously to what is done in the construction of proper D-projections). Let $\varphi^2$ be the resulting proofoid.
\item Construct $\varphi^3 \defEq \replacePert{\varphi^2}{\occCutPert{\varphi^2}}$ by replacing the cut-pertinent inferences of $\varphi^2$ by $Y$-inferences.
\item Construct $\varphi^4 \defEq \WFix{\varphi^3}$ by fixing broken inferences with weakening.
%\item Finally, construct the mixed DW-projection $\projectionDWMixed{\varphi}{c} \defEq \EliminateY{\varphi^4}$ by eliminating the $Y$-inferences from $\varphi^4$.
\end{enumerate}
\end{definition}


\begin{example}[Mixed DW-Projection]
\label{example:DWProjectionMixed}



Let ${\varphi}$ be the proof shown in Example \ref{example:PlusTimesSwappeDefinitionalNormalization}, which is displayed again for convenience below:

\begin{prooftree}
\AXC{$\hA{A} \seq \hA{A} $}
		\AXC{$\hB{B} \seq \hB{B}$} \RightLabel{$\wedge_r^1$}
	\BIC{$\hA{A}, \hB{B} \seq \hA{A} \wedge \hB{B}$} \RightLabel{$\wedge_l$}
	\UIC{$\hA{A} \wedge \hB{B} \seq \hA{A} \wedge \hB{B}$}
				\AXC{$\hC{B} \seq \hC{B} $}
						\AXC{$\hD{A} \seq \hD{A}$} \RightLabel{$\wedge_r^2$}
					\BIC{$\hD{A}, \hC{B} \seq \hC{B} \wedge \hD{A}$} \RightLabel{$\wedge_l$}
					\UIC{$\hD{A} \wedge \hC{B} \seq \hC{B} \wedge \hD{A}$} \RightLabel{$cut^3$}
			\BIC{$\hA{A} \wedge \hB{B} \seq \hC{B} \wedge \hD{A}$}
							\AXC{$\hE{C} \seq \hE{C}$} 
									\AXC{$\hF{C} \seq \hF{C}$} \RightLabel{$cut^4$}
								\BIC{$\hE{C} \seq \hF{C}$} \RightLabel{$\vee_l^5$}
					\BIC{$(\hA{A} \wedge \hB{B}) \vee \hE{C} \seq \hC{B} \wedge \hD{A}, \hF{C} $} 
\end{prooftree}

The first step in the construction of the mixed DW-projection $\projectionDWMixed{\varphi}{\seq D_{\hA{A} \structplus \hB{B}} , \hE{C}}$ is the slicing with respect to $\Omega_E \cup \Omega_c$ where $\Omega_E = \{\hA{A}, \hB{B}\}$ and $\Omega_c = \{\hE{C}\}$:

\renewcommand{\hC}[1]{\phantom{#1}}
\renewcommand{\hD}[1]{\phantom{#1}}
\renewcommand{\hF}[1]{\phantom{#1}}

\begin{prooftree}
\AXC{$\hA{A} \seq \hA{A} $}
		\AXC{$\hB{B} \seq \hB{B}$} \RightLabel{$\wedge_r^1$}
	\BIC{$\hA{A}, \hB{B} \seq \hA{A} \wedge \hB{B}$} \RightLabel{$\wedge_l$}
	\UIC{$\hA{A} \wedge \hB{B} \seq \hA{A} \wedge \hB{B}$}
				\AXC{$\hC{B} \seq \hC{B} $}
						\AXC{$\hD{A} \seq \hD{A}$} \RightLabel{$Y$}
					\BIC{$\hD{A}\phantom{,} \hC{B} \seq \hC{B} \phantom{\wedge} \hD{A}$} \RightLabel{$Y$}
					\UIC{$\hD{A} \phantom{\wedge} \hC{B} \seq \hC{B} \phantom{\wedge} \hD{A}$} \RightLabel{$Y$}
			\BIC{$\hA{A} \wedge \hB{B} \seq \hA{A} \wedge \hB{B}$}
							\AXC{$\hE{C} \seq \hE{C}$} 
									\AXC{$\hF{C} \seq \hF{C}$} \RightLabel{$Y$}
								\BIC{$\hE{C} \seq \hE{C}$} \RightLabel{$\vee_l^5$}
					\BIC{$(\hA{A} \wedge \hB{B}) \vee \hE{C} \seq \hA{A} \wedge \hB{B}, \hC{B} \phantom{\wedge} \hD{A}\phantom{,} \hE{C} $} 
\end{prooftree}


The second step is the introduction of definition inferences, resulting in the proofoid $\varphi^2$ below:

\begin{prooftree}
\AXC{$\hA{A} \seq \hA{A} $}
		\AXC{$\hB{B} \seq \hB{B}$} \RightLabel{$\wedge_r$}
	\BIC{$\hA{A}, \hB{B} \seq \hA{A} \wedge \hB{B}$} \RightLabel{$d_r$}
	\UIC{$\hA{A}, \hB{B} \seq D_{\hA{A} \structplus \hB{B}}$} \RightLabel{$\wedge_l$}
	\UIC{$\hA{A} \wedge \hB{B} \seq D_{\hA{A} \structplus \hB{B}}$}
				\AXC{$\hC{B} \seq \hC{B} $}
						\AXC{$\hD{A} \seq \hD{A}$} \RightLabel{$Y$}
					\BIC{$\hD{A}\phantom{,} \hC{B} \seq \hC{B} \phantom{\wedge} \hD{A}$} \RightLabel{$Y$}
					\UIC{$\hD{A} \phantom{\wedge} \hC{B} \seq \hC{B} \phantom{\wedge} \hD{A}$} \RightLabel{$Y$}
			\BIC{$\hA{A} \wedge \hB{B} \seq D_{\hA{A} \structplus \hB{B}}$}
							\AXC{$\hE{C} \seq \hE{C}$} 
									\AXC{$\hF{C} \seq \hF{C}$} \RightLabel{$Y$}
								\BIC{$\hE{C} \seq \hE{C}$} \RightLabel{$\vee_l^5$}
					\BIC{$(\hA{A} \wedge \hB{B}) \vee \hE{C} \seq D_{\hA{A} \structplus \hB{B}},  \hC{B} \phantom{\wedge} \hD{A}\phantom{,} \hE{C} $} 
\end{prooftree}

Subsequently, cut-pertinent inferences of $\varphi^2$ should be replaced by $Y$-inferences. However, since $\varphi^2$ has no cuts, there is nothing to be replaced, and hence $\varphi^3 = \varphi^2$. Subsequently, broken inferences of $\varphi^3$ should be W-fixed. However, there are no broken inferences in $\varphi^3$. Therefore, only the last step of eliminating $Y$-inferences remains and its result is the mixed DW-projection $\projectionDWMixed{\varphi}{\seq D_{\hA{A} \structplus \hB{B}} , \hE{C}}$ shown below:

\begin{prooftree}
\AXC{$\hA{A} \seq \hA{A} $}
		\AXC{$\hB{B} \seq \hB{B}$} \RightLabel{$\wedge_r$}
	\BIC{$\hA{A}, \hB{B} \seq \hA{A} \wedge \hB{B}$} \RightLabel{$d_r$}
	\UIC{$\hA{A}, \hB{B} \seq D_{\hA{A} \structplus \hB{B}}$} \RightLabel{$\wedge_l$}
	\UIC{$\hA{A} \wedge \hB{B} \seq D_{\hA{A} \structplus \hB{B}}$}
							\AXC{$\hE{C} \seq \hE{C}$} \RightLabel{$\vee_l^5$}
					\BIC{$(\hA{A} \wedge \hB{B}) \vee \hE{C} \seq D_{\hA{A} \structplus \hB{B}}, \hE{C} $} 
\end{prooftree}

\end{example}


\begin{example}[$\CEResDSwapDSwap$-Normal-Form]

Consider again the swapped definitional clause set of the proof $\varphi$ shown in Example \ref{example:SwappeDefinitionalClauseSet}:

$$
\clausesetSwapDefUnique{\varphi} \equiv  \left\{
	\seq D_{\hA{A} \structplus \hB{B}} , \hE{C} 
\ \ \ ; \ \
	\hF{C} \seq 
\ \ \ ; \ \
	\hC{B} , \hD{A} \seq
\ \ \ ; \ \
	D_{\hA{A} \structplus \hB{B}} \seq \hA{A}  
\ \ \ ; \ \
	D_{\hA{A} \structplus \hB{B}} \seq \hB{B} 
\ \ \ ; \ \
	\hA{A}, \hB{B} \seq D_{\hA{A} \structplus \hB{B}} 
\right\}
$$

The shortest refutation $\delta$ of $\clausesetSwapDefUnique{\varphi}$ is shown below:

\begin{prooftree}
\AXC{$\seq D_{\hA{A} \structplus \hB{B}} , \hE{C}$}
		\AXC{$\hF{C} \seq$} \RightLabel{$r$}
	\BIC{$\seq D_{\hA{A} \structplus \hB{B}}$}
				\AXC{$D_{\hA{A} \structplus \hB{B}} \seq \hA{A}$}
						\AXC{$D_{\hA{A} \structplus \hB{B}} \seq \hB{B}$}
								\AXC{$\hC{B} , \hD{A} \seq$} \RightLabel{$r$}
							\BIC{$D_{\hA{A} \structplus \hB{B}}, \hD{A} \seq $} \RightLabel{$r$}
					\BIC{$D_{\hA{A} \structplus \hB{B}}, D_{\hA{A} \structplus \hB{B}} \seq $} \RightLabel{$f_l$}
					\UIC{$D_{\hA{A} \structplus \hB{B}} \seq $} \RightLabel{$r$}
			\BIC{$ \seq $}
\end{prooftree}

By using the mixed DW-projection shown in Example \ref{example:DWProjectionMixed}, pure DW-projections shown in Example \ref{example:OProjections} and definitional DW-projections shown in Example \ref{example:DProjectionDefinitional}, $\CEResNFDSwapDSwap{\varphi}{\delta}$ is:


%\begin{small}
\begin{prooftree}
\AXC{$\hA{A} \seq \hA{A} $}
		\AXC{$\hB{B} \seq \hB{B}$} \RightLabel{$\wedge_r$}
	\BIC{$\hA{A}, \hB{B} \seq \hA{A} \wedge \hB{B}$} \RightLabel{$d_r$}
	\UIC{$\hA{A}, \hB{B} \seq D_{\hA{A} \structplus \hB{B}}$} \RightLabel{$\wedge_l$}
	\UIC{$\hA{A} \wedge \hB{B} \seq D_{\hA{A} \structplus \hB{B}}$}
							\AXC{$\hE{C} \seq \hE{C}$} \RightLabel{$\vee_l^5$}
					\BIC{$(\hA{A} \wedge \hB{B}) \vee \hE{C} \seq D_{\hA{A} \structplus \hB{B}}, \hE{C} $} 
               		\AXC{$\hF{C} \seq \hF{C}$} \RightLabel{$cut$}
               	\BIC{$(\hA{A} \wedge \hB{B}) \vee \hE{C} \seq D_{\hA{A} \structplus \hB{B}}, \hF{C}$}
	            				\AXC{$\hA{A} \seq \hA{A}$} \RightLabel{$w_l$}
               				\UIC{$\hA{A}, \hB{B} \seq \hA{A}$} \RightLabel{$\wedge_l$}
               				\UIC{$\hA{A} \wedge \hB{B} \seq \hA{A}$} \RightLabel{$d_l$}
               				\UIC{$D_{\hA{A} \structplus \hB{B}} \seq \hA{A}$}
               						\AXC{$\hB{B} \seq \hB{B}$} \RightLabel{$w_l$}
               						\UIC{$\hA{A}, \hB{B} \seq \hB{B}$} \RightLabel{$\wedge_l$}
               						\UIC{$\hA{A} \wedge \hB{B} \seq \hB{B}$} \RightLabel{$d_l$}
               						\UIC{$D_{\hA{A} \structplus \hB{B}} \seq \hB{B}$}
               								\AXC{$\hC{B} \seq \hC{B} $}
															\AXC{$\hD{A} \seq \hD{A}$} \RightLabel{$\wedge_r$}
														\BIC{$ \hC{B}, \hD{A} \seq \hC{B} \wedge \hD{A}$} \RightLabel{$cut$}
               							\BIC{$D_{\hA{A} \structplus \hB{B}}, \hD{A} \seq \hC{B} \wedge \hD{A}$} \RightLabel{$cut$}
               					\BIC{$D_{\hA{A} \structplus \hB{B}}, D_{\hA{A} \structplus \hB{B}} \seq \hC{B} \wedge \hD{A}$} \RightLabel{$c_l$}
               					\UIC{$D_{\hA{A} \structplus \hB{B}} \seq \hC{B} \wedge \hD{A}$} \RightLabel{$cut$}
               			\BIC{$(\hA{A} \wedge \hB{B}) \vee \hE{C} \seq \hC{B} \wedge \hD{A}, \hF{C}$}
\end{prooftree}
%\end{small}


\end{example}


\section{Ignoring Atomic and Quantifier-Free Cuts}
\label{sec:CutEliminationByResolution:CEResIgnoringAtomicCuts}

If $\CERes$ is applied to a proof containing only atomic cuts, $\CERes$ still transforms the proof into a new proof containing only atomic cuts, but with additional structural inferences and with the atomic cuts located in the bottom of the proof. This is clearly non-ideal, because the proof could be simply left unchanged. More generally, if $\CERes$ is applied to a proof containing complex cuts and atomic cuts, $\CERes$ unnecessarily includes the atomic cuts in the process of reduction, even though atomic cuts cannot be reduced further. The inclusion of atomic cuts results in larger clause sets that are more costly to refute, and in normal forms with possibly additional structural inferences. This indicates that there is a very simple and evident improvement of the $\CERes$ method that has been thoroughly overlooked so far: instead of distinguishing between cut-pertinent and cut-impertinent formula occurrences (i.e. between ancestors and non-ancestors of \emph{all} cut formula occurrences) and cut-pertinent and cut-impertinent inferences (i.e inferences that operate on the ancestors and on the non-ancestors of cut formula occurrences), it suffices to distinguish between ancestors of \emph{complex} cut formula occurrences and ancestors of either occurrences in the end-sequent or of atomic cut-formula occurrences.


