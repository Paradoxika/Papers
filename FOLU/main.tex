\documentclass{llncs}
\usepackage{etex}

\usepackage{xcolor}
\usepackage{enumitem,amsmath,amssymb}
\usepackage{breakurl}    % used for \url and \burl
\usepackage[linesnumbered,boxed,noline,noend]{algorithm2e}
\def\defaultHypSeparation{\hskip.1in}

\usepackage{tikz}
\usepackage{subfig}
\usepackage{array,booktabs,multirow}
\usepackage{placeins}

\usepackage{logictools}
\usepackage{prooftheory}
\usepackage{comment}
\usepackage{mathenvironments}
\usepackage{drawproof}
\usepackage{bussproofs}
\usepackage{tensor}
\usepackage{mathtools}
\usepackage{amsmath}

\usepackage{graphicx}
%\usepackage{caption}
%\usepackage{subcaption}

\renewcommand{\topfraction}{0.85}
\renewcommand{\textfraction}{0.1}
\renewcommand{\floatpagefraction}{0.75}


\newcommand{\freevar}[1]{\mathrm{FV}(#1)}

\newcommand{\Vertices}[1]{V_{#1}}
\newcommand{\Edges}[1]{E_{#1}}
\newcommand{\Conclusion}[1]{\clause_{#1}}

\newcommand{\axiom}[1]{\widehat{#1}}
\newcommand{\n}{v}
\newcommand{\raiz}[1]{\rho(#1)}

\newcommand{\pedge}[3]{\ensuremath{\raiz{#1} \xrightarrow{#2} \raiz{#3}}}


\newcommand\inlineeqno{\stepcounter{equation}\ (\theequation)}


% Contraction
\newcommand{\con}[3]{\lfloor #1 \rfloor_{#2}^{#3}}

% Resolution
%\newcommand{\res}[6]{#1 \tensor[^{#2}_{#3}]{\odot}{^{#4}_{#5}} #6}
%\newcommand{\res}[6]{#1 \prescript{#2}{#3}{\odot^{#4}_{#5}} #6}

\newcommand{\res}[4]{\mathrel{\operatorname*{\odot}_{#1 #3}^{#2 #4}}}

\title{Towards the Compression of First-Order Resolution Proofs by Lowering Unit Clauses}

\author{
  Jan Gorzny\inst{1}
  \thanks{Supported by the Google Summer of Code 2014 program.}
  \and 
  Bruno Woltzenlogel Paleo\inst{2}
  \thanks{Supported by the Austrian Science Fund, project P24300.}
}

\authorrunning{J.\~Gorzny \and B.\~Woltzenlogel Paleo}

\institute{
  University of Victoria, Canada \\
  \email{jgorzny@uvic.ca}
  \and 
  Vienna University of Technology, Austria \\
  \email{bruno@logic.at}
}




\begin{document}

\maketitle


\begin{abstract}
The recently developed {\LowerUnits} algorithm compresses
propositional resolution proofs generated by SAT- and SMT-solvers by lowering (i.e. postponing) resolution inferences involving unit clauses (i.e. clauses having exactly one literal). This paper describes a generalization of this algorithm to the case of first-order resolution proofs generated by automated theorem provers. An empirical evaluation of a simplified version of this algorithm on hundreds of proofs shows promising results.
\end{abstract}


\setcounter{footnote}{0}

\section{Introduction}

Most of the effort in automated reasoning so far has been dedicated to the design and implementation of proof systems and efficient theorem proving procedures. As a result, saturation-based first-order automated theorem provers have achieved a high degree of maturity, with resolution \cite{Robinson} and superposition \cite{todo} being among the most common underlying proof calculi. Proof production is an essential feature of modern state-of-the-art provers and proofs are crucial for applications where the user requires certification of the answer provided by the prover. Nevertheless, efficient proof production is non-trivial \cite{SchultzAPPA}, and it is to be expected that the best, most efficient, provers do not necessarily generate the best, least redundant, proofs. And while the foundational problem of simplicity of proofs can be traced back at least to Hilbert's 24th Problem \cite{Hilbert}, the maturity of automated deduction has made it particularly relevant today. Therefore, it is a timely moment to develop methods that post-process and simplify proofs. 

For proofs generated by SAT- and SMT-solvers, which use propositional resolution as the basis for the DPLL and CDCL decision procedures, there is now a wide variety of proof compression techniques. Algebraic properties of the resolution
operation that might be useful for compression were investigated in \cite{bwp10}.
Compression algorithms based on rearranging and sharing chains of resolution inferences have been
developed in \cite{Amjad07} and \cite{Sinz}.  Cotton \cite{CottonSplit} proposed an algorithm that
compresses a refutation by repeteadly splitting it into a proof of a heuristically chosen literal $\ell$
and a proof of $\dual{\ell}$, and then resolving them to form a new refutation.  The {\ReduceReconstruct} algorithm \cite{RedRec} searches for locally redundant
subproofs that can be rewritten into subproofs of stronger clauses and with fewer resolution steps.
A linear time proof compression algorithm based on partial
regularization was proposed in \cite{RP08} and improved in \cite{LURPI}. Furthermore, \cite{LURPI} also described a new linear time algorithm called {\LowerUnits}, which delays resolution with unit clauses.

In contrast, for first-order theorem provers, there has been up to now (to the best of our knowledge) no attempt to design and implement an algorithm capable of taking a first-order resolution DAG-proof and efficiently simplifying it, outputting a possibly shorter pure first-order resolution DAG-proof. There are algorithms aimed at simplifying first-order sequent calculus tree-like proofs, based on cut-introduction \cite{BrunoLPAR,Hetzl}, and while in principle resolution DAG-proofs can be translated to sequent-calculus tree-like proofs (and then back), such translations lead to undesirable efficiency overheads. There is also an algorithm \cite{LPARCzech} that looks for terms that occur often in any TSTP \cite{TPTP} proof (including first-order resolution DAG-proofs) and introduces abbreviations for these terms. However, as the definitions of the abbreviations are not part of the output proof, it cannot be checked by a pure first-order resolution proof checker.

In this paper, we initiate the process of lifting propositional proof compression techniques to the first-order case, starting with the simplest known algorithm: {\LowerUnits} (described in Section \ref{sec:PropositionalLU}). As shown in Section \ref{sec:Challenges}, even for this simple algorithm, the fact that first-order resolution makes use of unification leads to many challenges that simply do not exist in the propositional case. In Section \ref{sec:FOLU} we describe a sophisticated algorithm that overcomes these challenges. Furthermore, in Section \ref{sec:SimpleFOLU} we describe a simpler version of this algorithm, which is easier to implement and possibly more efficient, at the cost of compressing less. In Section \ref{sec:exp} we present experimental results obtained by applying the simpler algorithm on hundreds of proofs generated with the {\SPASS} theorem prover \cite{SPASS}. The next section introduces the first-order resolution calculus using notations that are more convenient for describing proof transformation operations.






\section{The Resolution Calculus}
\label{sec:res}

As usual, our language has infinitely many variable symbols (e.g. $x$, $y$, $z$, $x_1$, $x_2$, \ldots), constant symbols (e.g. $a$, $b$, $c$, $a_1$, $a_2$, \ldots), function symbols of every arity (e.g $f$, $g$, $f_1$, $f_2$, \ldots) and predicate symbols of every arity (e.g. $P$, $Q$, $P_1$, $P_2$,\ldots). A \emph{term} is any variable, constant or the application of an $n$-ary function symbol to $n$ terms.
An \emph{atomic formula} (\emph{atom}) is the application of an $n$-ary predicate symbol to $n$ terms. A \emph{literal} is an atom or the negation of an atom. The
\emph{complement} of a literal $\ell$ is denoted $\dual{\ell}$ (i.e. for any atom $P$,
$\dual{P} = \neg P$ and $\dual{\neg P} = P$). The \emph{underlying atom} of a literal $\ell$ is denoted $\abs{\ell}$ (i.e. for any atom $P$, $\abs{P} = P$ and $\abs{\neg P} = P$). A
\emph{clause} is a multiset of literals. $\bot$ denotes the \emph{empty clause}. A \emph{unit clause} is a clause with a single literal. Sequent notation is used for clauses (i.e. $P_1,\ldots,P_n \seq Q_1,\ldots, Q_m$ denotes the clause $\{ \neg P_1,\ldots, \neg P_n, Q_1, \ldots, Q_m \}$).
%$\freevar{t}$ (resp. $\freevar{\ell}$, $\freevar{\clause}$) denotes the set of variables in the term $t$ (resp. in the literal $\ell$ and in the clause $\clause$).
A \emph{substitution} $\{ x_1\backslash t_1, x_2 \backslash t_2, \ldots \}$ is a mapping from variables $\{ x_1, x_2, \ldots \}$ to, respectively, terms $\{t_1, t_2, \ldots \}$. The application of a substitution $\sigma$ to a term $t$, a literal $\ell$ or a clause $\clause$ results in, respectively, the term $t \sigma$, the literal $\ell \sigma$ or the clause $\clause \sigma$, obtained from $t$, $\ell$ and $\clause$ by replacing all occurrences of the variables in $\sigma$ by the corresponding terms in $\sigma$. A literal $\ell$ \emph{matches} another literal $\ell'$ if there is a substitution $\sigma$ such that $\ell\sigma=\ell'$. A \emph{unifier} of a set of literals is a substitution that makes all literals in the set equal. We will use $X \sqsubseteq Y$ to denote that $X$ \emph{subsumes} $Y$, when there exists a substitution $\sigma$ such that $X\sigma \subseteq Y$.

%The resolution calculus used in this paper has the following inference rules: 

% \begin{definition}[First-Order Axiom] A first-order axiom has no premises and concludes some clause $\Gamma$, as below.
% \begin{prooftree}
% \AxiomC{$~$ }
% \UnaryInfC{$\psi$: $\Gamma$}
% \end{prooftree}
% %where $\Gamma$ is a clause.
% \end{definition}

\begin{definition}[Resolution] \label{def:fores} \hfill
%An instance of first-order resolution requires two premises, as below.
\begin{prooftree}
\AxiomC{$\eta_1$: $\Gamma_L' \cup \{\ell_L\}$ }
\AxiomC{$\eta_2$: $\Gamma_R'\cup \{\ell_R\}$ }
\BinaryInfC{$\psi$: $\Gamma_L'\sigma_L \cup \Gamma_R'\sigma_R$}
\end{prooftree}
where $\sigma_L$ and $\sigma_R$ are substitutions such that $\ell_L\sigma_L=\dual{\ell_R}\sigma_R$. The literals $\ell_L$ and $\ell_R$ are \emph{resolved literals}, whereas $\ell_L \sigma_L$ and $\ell_R \sigma_R$ are its \emph{instantiated resolved literals}. The \emph{pivot} is the underlying atom of its instantiated resolved literals (i.e. $\abs{\ell_L \sigma_L}$ or, equivalently, $\abs{\ell_R \sigma_R}$).
\end{definition}

\begin{definition}[Factoring] \label{def:fofact} \hfill
%An instance of first-order factoring applies a unifier to the literals in a single premise's conclusion, as below.
\begin{prooftree}
\AxiomC{$\eta_1$: $\Gamma' \cup \{\ell_1,\ldots,\ell_n\}$ }
\UnaryInfC{$\psi$: $\Gamma'\sigma \cup \{\ell\}$}
\end{prooftree}
where $\sigma$ is a unifier of $\{\ell_1,\ldots,\ell_n\}$ and $\ell=\ell_i\sigma$ for any $i\in \{1,\ldots,n\}$.
\end{definition} %UITP

A \emph{resolution proof} is a directed acyclic graph of clauses where the edges correspond to the inference rules of resolution and factoring, as explained in detail in Definition \ref{def:proof}. A \emph{resolution refutation} is a resolution proof with root $\bot$.

\begin{definition}[First-Order Resolution Proof] 
\label{def:proof}% \hfill \\
A directed acyclic graph $\langle V, E, \clause \rangle$, where $V$ is a set of nodes and $E$ is a
set of edges labeled by literals and substitutions (i.e. $E \subset V \times 2^{\mathcal{L}} \times \mathcal{S} \times V$, where $\mathcal{L}$ is the set of all literals and $\mathcal{S}$ is the set of all substitutions, and $\n_1
\xrightarrow[\sigma]{\ell} \n_2$ denotes an edge from node $\n_1$ to node $\n_2$ labeled by the literal $\ell$ and the substitution $\sigma$), is a
proof of a clause $\clause$ iff it is inductively constructible according to the following cases:
%
\begin{itemize}
  \item \textbf{Axiom:} If $\Gamma$ is a clause, $\axiom{\Gamma}$ denotes some proof $\langle \{ \n \}, \varnothing,
    \Gamma \rangle$, where $\n$ is a new node. % (axiom) node.
  \item \textbf{Resolution\footnote{This is referred to as ``binary resolution'' elsewhere, with the understanding that ``binary'' refers to the number of resolved literals, rather than the number of premises of the inference rule.}:} If $\psi_L$ is a proof $\langle V_L, E_L, \clause_L \rangle$ and
    $\psi_R$ is a proof $\langle V_R, E_R, \clause_R \rangle$, $\sigma_L$ and $\sigma_R$ are substitutions s.t. $\ell_L\sigma_L=\dual{\ell_R}\sigma_R$,
    %and $\sigma_L$ and $\sigma_R$ are substitutions such that
    %$\ell_L \sigma_L = \dual{\ell_R} \sigma_R$ %and
    %$\freevar{\left( \clause_L \setminus \left\{ \ell_L \right\} \right) \sigma_L} \cap 
     %\freevar{\left( \clause_R
     %               \setminus \left\{ \ell_R \right\} \right) \sigma_R} = \emptyset$, 
    then
    $\psi_L \res{\ell_L}{\sigma_L}{\ell_R}{\sigma_R} \psi_R$ denotes a proof $\langle V, E, \Gamma \rangle$ s.t.
%\begin{align*}
%     V  &= V_L \cup V_R \cup \{\n \},  \hspace*{1cm}\Gamma = \clause_L' \sigma_L \cup  \clause_R' \sigma_R, \\
%     E &= E_L \cup E_R \cup  \left\{ \raiz{\psi_L} \xrightarrow[\sigma_L]{\{\ell_L\} } \n,   \raiz{\psi_R} \xrightarrow[\sigma_R]{\{\ell_R\} } \n \right\},
%\end{align*}
\begin{align*}
     V  &= V_L \cup V_R \cup \{\n \}, ~\Gamma = \clause_L' \sigma_L \cup  \clause_R' \sigma_R,~
     E = E_L \cup E_R \cup  \left\{ \raiz{\psi_L} \xrightarrow[\sigma_L]{\{\ell_L\} } \n,   \raiz{\psi_R} \xrightarrow[\sigma_R]{\{\ell_R\} } \n \right\},
\end{align*}
%UITP
%    \begin{align*}
%     \hspace{-0.6cm} V &= V_L \cup V_R \cup \{\n \}    \\
%      \hspace{-0.6cm} E &= E_L \cup E_R \cup 
%                    \left\{ \raiz{\psi_L} \xrightarrow[\sigma_L]{\{\ell_L\} } \n, 
%                            \raiz{\psi_R} \xrightarrow[\sigma_R]{\{\ell_R\} } \n \right\}    \\
%    \hspace{-0.6cm}  \Gamma &= \clause_L' \sigma_L \cup  \clause_R' \sigma_R
%    \end{align*}
    where $\n$ is a new (resolution) node and $\raiz{\varphi}$ denotes the root node of $\varphi$.  The literals $\ell_L$ and $\ell_R$ are \emph{resolved literals}, whereas $\ell_L \sigma_L$ and $\ell_R \sigma_R$ are its \emph{instantiated resolved literals}. The \emph{pivot} is the underlying atom of its instantiated resolved literals (i.e. $\abs{\ell_L \sigma_L}$ or, equivalently, $\abs{\ell_R \sigma_R}$).
  \item \textbf{Factoring:}
  %\footnote{This is often called ``Factoring'', but we prefer ``contraction'', because it is essentially the contraction rule of sequent calculus generalized with unification.} 
  If $\psi'$ is a proof $\langle V', E', \clause' \rangle$, $\sigma$ is a unifier of $\{\ell_1,\ldots,\ell_n\}$, and $\ell=\ell_i\sigma$ for any $i\in \{1,\ldots,n\}$, then $\con{\psi}{\{\ell_1, \ldots \ell_n\}}{\sigma}$ denotes a proof $\langle V, E, \Gamma \rangle$ s.t.
  %UITP
    \begin{align*}
         \hspace{-0.6cm} V &= V' \cup \{\n \},  \hspace*{1cm} \Gamma = \clause' \sigma \cup \{ \ell \}, \hspace*{1cm}  E = E' \cup \{ \raiz{\psi'} \xrightarrow[\sigma]{\{\ell_1, \ldots \ell_n\}} \n \},
%         \hspace{-0.6cm} E &= E' \cup \{ \raiz{\psi'} \xrightarrow[\sigma]{\{\ell_1, \ldots \ell_n\}} \n \} 
    \end{align*}  
%    \begin{align*}
%         \hspace{-0.6cm} V &= V' \cup \{\n \} \\
%         \hspace{-0.6cm} E &= E' \cup \{ \raiz{\psi'} \xrightarrow[\sigma]{\{\ell_1, \ldots \ell_n\}} \n \} \\
%       \hspace{-0.6cm} \Gamma &= \clause' \sigma \cup \{ \ell \}
%    \end{align*}
    where $\n$ is a new (factoring) node, and $\raiz{\varphi}$ denotes the root node of $\varphi$.
  \qed
\end{itemize}
\end{definition}





%UITP TODO: include resolution example?




%\noindent
%The resolution and contraction (factoring) rules described above are the standard rules of the resolution calculus, except for the fact that we do not require resolution to use most general unifiers. The presentation of the resolution rule here uses two substitutions, in order to explicitly handle the necessary renaming of variables, which is often left implicit in other presentations of resolution.
%\noindent
%When we write $\psi_L \res{\ell_L}{}{\ell_R}{} \psi_R$, we assume that the omitted substitutions are such that the resolved atom is most general. 
%We write $\con{\psi}{}{}$ for an arbitrary maximal contraction, and $\con{\psi}{}{\sigma}$ for a (pseudo-)contraction that does merge no literals but merely applies the substitution $\sigma$. 
%When the literals and substitutions are irrelevant or clear from the context, we may write simply $\psi_L \res{}{}{}{} \psi_R$. % instead of $\psi_L \res{\ell_L}{\sigma_L}{\ell_R}{\sigma_R} \psi_R$.
%The $\res{}{}{}{}$ operator is assumed to be left-associative. 
%In the propositional case, we omit contractions (treating clauses as sets instead of multisets) and $\psi_L \res{\ell}{\emptyset}{\dual{\ell}}{\emptyset} \psi_R$ is abbreviated by $\psi_L \odot_{\ell} \psi_R$.

%If $\psi = \varphi_L \odot \varphi_R$ or $\psi = \con{\varphi}{}{}$, then $\varphi$, $\varphi_L$ and $\varphi_R$ are \emph{direct subproofs} of $\psi$ and $\psi$ is a \emph{child} of both $\varphi_L$ and $\varphi_R$. The
%transitive closure of the direct subproof relation is the \emph{subproof} relation. A subproof which has no direct subproof is an \emph{axiom} of the proof.
%
%$\Vertices{\psi}$, $\Edges{\psi}$ and $\Conclusion{\psi}$
%denote, respectively, the nodes, edges and proved clause (conclusion) of $\psi$. If $\psi$ is a proof ending with a resolution node, then $\psi_L$ and $\psi_R$ denote, respectively, the left and right premises of $\psi$.

\section{The Propositional LowerUnits Algorithm}
\label{sec:PropositionalLU}

We denote by $\dn{\psi}{\varphi_1, \varphi_2}$ the result of deleting the subproofs $\varphi_1$ and $\varphi_2$ from the proof $\psi$ and fixing it according to Algorithm \ref{algo:del}\footnote{
  The deletion algorithm is a minor variant of the \textsc{Reconstruct-Proof} algorithm presented in \cite{RP11}.
  The basic idea is to traverse the proof in a top-down manner, replacing
  each subproof having one of its premises marked for deletion (i.e. in $D$) by its other direct subproof. For more details, we refer to \ref{Boudou}.
}. 
We say that a subproof $\varphi$ in a proof $\psi$ can be lowered 
if there exists a proof
$\psi'$ such that $\psi' = \dn{\psi}{\varphi} \odot \varphi$ and
$\Conclusion{\psi'} \subseteq \Conclusion{\psi}$. If $\varphi$ originally participated in many resolution inferences within $\psi$ (i.e. if $\varphi$ had many children in $\psi$) then lowering $\varphi$ compresses the proof (in number of resolution inferences), because $\dn{\psi}{\varphi} \odot \varphi$ contains a single resolution inference involving $\varphi$.

%
It has been noted in \cite{LURPI} that, in the propositional case, $\varphi$ can always be lowered if it is a \emph{unit} (i.e. its conclusion clause is unit). This led to the invention of {\LowerUnits} (Algorithm \ref{algo:LU}), which aims at transforming a proof $\psi$ into $(\dn{\psi}{\mu_1,\ldots,\mu_n}) \odot_{\ell_1} \mu_1 \odot \ldots \odot_{\ell_n} \mu_n$, where $(\mu_1,\ldots,\mu_n)$ are all units with more than one child. Units with only one child are ignored merely because no compression is gained by lowering them. The order in which the units are reintroduced is important:
if a unit $\varphi_2$ is a subproof of a unit
$\varphi_1$ then $\varphi_2$ has to be reintroduced later than (i.e. below) $\varphi_1$.



\SetKwFunction{Rec}{delete}
\SetKw{Let}{let}

\begin{algorithm}[bt]
  \KwIn{a proof $\varphi$}
  \KwIn{$D$ a set of subproofs}
  \KwOut{a proof $\varphi'$ obtained by deleting the subproofs in $D$ from $\varphi$}
  \BlankLine

  \newcommand{\fixL}{\ensuremath{\varphi'_L}}
  \newcommand{\fixR}{\ensuremath{\varphi'_R}}

  \lIf{$\varphi \in D$ or $\raiz{\varphi}$ has no premises}{\Return{$\varphi$}}
  \BlankLine

  \Else{
    \Let{$\varphi_L$ and $\varphi_R$} be such that
      $\varphi = \varphi_L \res{\ell_L}{\sigma_L}{\ell_R}{\sigma_R} \varphi_R$ \;
    \Let{$\varphi'_L = $ \Rec{$\varphi_L$,$D$}} \;
    \Let{$\varphi'_R = $ \Rec{$\varphi_R$,$D$}} \;
    \BlankLine

    \lIf{$\varphi'_L \in D$}{ \Return{\fixR} }
    \lElseIf{$\varphi'_R \in D$}{ \Return{\fixL} }
    \BlankLine

    \lElseIf{$\dual{\ell} \notin \Conclusion{\fixL}$}{ \Return{\fixL} }
    \lElseIf{$\ell \notin \Conclusion{\fixR}$}{ \Return{\fixR} }
    \BlankLine

    \lElse{ \Return{ \fixL~$\res{\ell_L}{\sigma_L}{\ell_R}{\sigma_R}$~\fixR} }
  }

  \caption[.]{\FuncSty{delete}}
  \label{algo:del}
\end{algorithm}





A possible presentation of {\LowerUnits} is shown in Algorithm \ref{algo:LU}. Units are collected
during a first traversal. As this traversal is bottom-up, units are stored in a queue. The traversal
could have been top-down and units stored in a stack. Units are effectively deleted during a second,
top-down traversal. The last for-loop performs the reintroduction of units.

\begin{algorithm}[bt]
  \KwIn {a proof $\psi$}
  \KwOut{a compressed proof $\psi'$}
  \BlankLine

  \SetKwData{Units}{Units}
  \Units $\leftarrow \varnothing$ \;
  \BlankLine

  \For{every subproof $\varphi$ in a bottom-up traversal}{
    \If{$\varphi$ is a unit and has more than one child}{Enqueue $\varphi$ in \Units \; }
  }
  \BlankLine

  $\psi' \leftarrow $ \Rec{$\psi$,$\Units$} \;
  \BlankLine

  \For{every unit $\varphi$ in \Units}{
    \Let{$\{\ell\} = \Conclusion{\varphi}$} \;
    \lIf{$\dual{\ell} \in \Conclusion{\psi'}$}{
    $\psi' \leftarrow \psi' \odot_\ell \varphi$}
  }

  \caption{\LowerUnits}
  \label{algo:LU}
\end{algorithm}




\section{First-Order Challenges}\label{sec:Challenges}

TODO by Jan (just writing some ideas so far--not yet final by any means)\\
{\bf Does this belong here?And is this what you had in mind for interesting examples? And are the proof formatted correctly, or should I change them? aside from the first one going over the margin right now of course}\\ 

In this section, we discuss additional requirements for lowering a unit formula in the first order case that are not required in the propositional case.

%example 1: shows requirement for pair-wise unifiability with unit
%obvious - skip?

%example 2: shows requirement for pair-wise unifiability within all aux formulas

 \begin{example} The following example shows why we must check pair-wise unifiability with the literals resolved against the unit we're trying to lower.

% \begin{tiny}
% \begin{prooftree}
% \def\e{\mbox{\ $\vdash$\ }}
% \AxiomC{$\eta_2$}
% \AxiomC{$\eta_1$: $p(a)$\e$q(Y),r(Z)$}
% \AxiomC{$\eta_2$: \e $p(X)$}
% \BinaryInfC{$\eta_3$: \e$q(Y),r(Z)$}
% \AxiomC{$\eta_4$: $r(X),p(b)$\e $s(Y)$}
% \BinaryInfC{$\eta_5$: $p(b)$\e $s(Y),q(Y)$}
% \AxiomC{$\eta_6$: $s(Y), q(Y)$\e}
% \BinaryInfC{$\eta_7$: $p(b)$\e}
% \BinaryInfC{$\psi$: $\bot$}
% \end{prooftree}
% \end{tiny}
 \end{example}


%example 3: shows requirement for contraction check

 \begin{example} The following shows why the above is not necessarily  enough (we must check the original sources of the aux formulas, and see if those can be contracted), otherwise we might not save anything.
% \begin{footnotesize}
% \begin{prooftree}
% \def\e{\mbox{\ $\vdash$\ }}
% \AxiomC{$\eta_1$: $r(Y),p(X ~q(Y~b)), p(X~Y)$\e}
% \AxiomC{$\eta_2$: \e $p(U~V)$}
% \BinaryInfC{$\eta_3$: $r(V),p(U ~q(V~b))$\e}
% \AxiomC{$\eta_4$: \e $r(W)$}
% \BinaryInfC{$\eta_5$: $p(U ~q(W~b))$\e}
% \AxiomC{$\eta_2$}
% \BinaryInfC{$\psi$: $\bot$}
% \end{prooftree}
% \end{footnotesize}
 \end{example}


%example 4: requires FOSubstitution, introduces this concept?



\section{First-Order LowerUnits} \label{sec:FOLU}


\SetKwFunction{Rec}{delete}
\SetKw{Let}{let}

\begin{algorithm}[bt]
  \KwIn{a proof $\varphi$}
  \KwIn{$D$ a set of subproofs}
  \KwOut{a proof $\varphi'$ obtained by deleting the subproofs in $D$ from $\varphi$}
  \BlankLine

  \newcommand{\fixL}{\ensuremath{\varphi'_L}}
  \newcommand{\fixR}{\ensuremath{\varphi'_R}}

  \lIf{$\varphi \in D$ or $\raiz{\varphi}$ has no premises}{\Return{$\varphi$}}
  \BlankLine

  \Else{
    \Let{$\varphi_L$ and $\varphi_R$} be such that
      $\varphi = \varphi_L \res{\ell_L}{\sigma_L}{\ell_R}{\sigma_R} \varphi_R$ \;
    \Let{$\varphi'_L = $ \Rec{$\varphi_L$,$D$}} \;
    \Let{$\varphi'_R = $ \Rec{$\varphi_R$,$D$}} \;
    \BlankLine

    \lIf{$\varphi'_L \in D$}{ \Return{\fixR} }
    \lElseIf{$\varphi'_R \in D$}{ \Return{\fixL} }
    \BlankLine

    \lElseIf{$\ell \notin \Conclusion{\fixL}$}{ \Return{\fixL} }
    \lElseIf{$\dual{\ell} \notin \Conclusion{\fixR}$}{ \Return{\fixR} }
    \BlankLine

    \lElse{ \Return{ \fixL~$\res{\ell_L}{\sigma_L}{\ell_R}{\sigma_R}$~\fixR} }
  }

  \caption[.]{\FuncSty{fo-delete}}
  \label{algo:fodel}
\end{algorithm}



\begin{proposition} \label{prop:LUniv}
Given a proof $\psi$, if 
%for an integer $n$
there is a sequence $U = (\varphi_1 \ldots \varphi_n)$
of $\psi$'s subproofs and a sequence $(\ell_1 \ldots \ell_n)$ of literals such that $\forall i \in
[1 \ldots n]$, $\ell_i$ is the univalent literal of $\varphi_i$ w.r.t. $\Delta_{i-1} =
\{\dual{\ell_1} \ldots \dual{\ell_{i-1}}\}$, then the conclusion of $$ \psi' = \dn{\psi}{U}
\odot_{\ell_n} \varphi_n \ldots \odot_{\ell_1} \varphi_1 $$ subsumes the conclusion of $\psi$.
\end{proposition}

\begin{proof}
The proposition is proven by induction on $n$, along with the fact that $\dn{\psi}{U} \notin U$.
For $n = 0$, $U = \varnothing$ and the properties trivially hold. Suppose a subproof
$\varphi_{n+1}$ of $\psi$ is univalent w.r.t. $\Delta_n$, with univalent literal $\ell_{n+1}$.
Because $\ell_{n+1} \notin \Delta_n$, there exists a subproof of $\dn{\psi}{U}$ with conclusion
containing $\dual{\ell_{n+1}}$, and therefore $\dn{\dn{\psi}{U}}{\varphi_{n+1}} \notin U \cup
\{\varphi_{n+1}\}$.  Let $\Gamma$ be the conclusion of $\dn{\psi}{U}$. The conclusion of $ \psi' =
\dn{\psi}{U \cup \{\varphi_{n+1}\}} = \dn{\dn{\psi}{U}}{\varphi_{n+1}} $ is included in $\Gamma \cup
\{\dual{\ell_{n+1}}\}$. The conclusion of $\psi' \odot_{\ell_{n+1}} \varphi_{n+1}$ is included in
$\Gamma \cup \Delta_n$. As $\Gamma \subseteq \Conclusion{\psi} \cup \Delta_n$, the conclusion of
$\psi' \odot_{\ell_{n+1}} \varphi_{n+1} \ldots \odot_{\ell_1} \varphi_1$ is included in
$\Conclusion{\psi}$. \qed
\end{proof}




% \begin{algorithm}[bt]
%   \KwIn {a proof $\psi$}
%   \KwOut{a compressed proof $\psi'$}
%   \BlankLine

%   \SetKw{Push}{push}
%   \SetKw{Pop} {pop}

%   \Units $\leftarrow \varnothing$ \;
%   $\Delta \leftarrow \varnothing$ \;
%   \BlankLine

%   \For{every subproof $\varphi$, in a top-down traversal \label{line:LUniv:step1begin} }{
%     $\psi' \leftarrow$ \Rec{$\varphi$,\Univ} \label{line:LUniv:delete} \;
%     \If{$\psi'$ is univalent w.r.t. $\Delta$ \label{line:LUniv:lunivtest} }{
%       \Let{$\ell$} be the univalent literal \;
%       \Push $\dual{\ell}$ onto $\Delta$ \label{line:LUniv:pushDelta} \;
%       \Push $\psi'$     onto \Univ \label{line:LUniv:step1end} \;
%     }
%   }
%   \BlankLine

%   \tcp{At this point, $\psi' = \dn{\psi}{\Univ}$}
%   \While{\Univ $\neq \varnothing$}{ \label{line:LUniv:reintroducebegin}
%     $\varphi \leftarrow$ \Pop from \Univ \;
%     $\ell \leftarrow$ \Pop from $\Delta$ \;
%     \lIf{$\ell \in \Conclusion{\psi'}$ \label{line:LUniv:testreintroduce} }{
%     $\psi' \leftarrow \varphi \odot_\ell \psi'$ \;}
%   }

%   \caption{Simplified \LowerUnivalents}
%   \label{algo:LUniv}
% \end{algorithm}


\begin{figure}[htb]
  \centering
  \subfloat[Original proof]{
    \centering
    \begin{tikzpicture}

      \rootnode;
      \withchildren{root} {r0}{\dual{a}}  {unit}{a};
      \withchildren{r0}   {r1}{\dual{a},c} {r2}{\dual{a},\dual{c}};
      \withchildren{r1}   {a0}{\dual{b},c} {low}{\dual{a},b};

      \proofnode[above right of=r2] {a1} {\dual{a},\dual{b},\dual{c}};
      \drawchildren {r2} {low} {a1};

    \end{tikzpicture}
  } \qquad
  \centering
  \subfloat[Compressed proof]{
    \centering
    \begin{tikzpicture}

      \rootnode;
      \withchildren{root} {r0}{\dual{a}}          {unit}{a};
      \withchildren{r0}   {r1}{\dual{a},\dual{b}} {low}{\dual{a},b};
      \withchildren{r1}   {a0}{\dual{b},c}        {a1}{\dual{a},\dual{b},\dual{c}};

    \end{tikzpicture}
  }
\caption{Example of proof compression by \LowerUnivalents} 
\label{fig:exluniv}
\end{figure}






\section{A Simpler First-Order LowerUnits}
\label{sec:SimpleFOLU}

%Recall example \ref{ex:ambig}. In order to avoid this, we introduce a proof rule that applies a substitution. So that we would get the following proof

\begin{tiny}
\begin{prooftree}
\def\e{\mbox{\ $\vdash$\ }}
\AxiomC{$\eta_1$: $p(U),r(U~V),r(V~U),q(V)$\e}
\UnaryInfC{$\eta_2$: $p(c),r(c~V),r(V~c),q(V)$\e}
\AxiomC{$\eta_3$: \e$r(X~c)$}
\BinaryInfC{$\eta_4$: $p(c),r(c~X),q(X)$\e}
\AxiomC{$\eta_5$: \e$r(W~V)$}
\BinaryInfC{$\eta_6$: $p(c),q(V)$\e}
\AxiomC{$\eta_7$: $p(Z)$\e$q(d)$}
\BinaryInfC{$\eta_8$: $p(c),p(Z)$\e}
\UnaryInfC{$\eta_9$: $p(c)$\e}
\AxiomC{$\eta_{10}$: \e$p(c)$}
\BinaryInfC{$\psi$: $\bot$}
\end{prooftree}
\end{tiny}

Now $r(V, c)$ appears in the first left resolvent, which was the left aux formula in the original proof. Thus, the implementation can find that formula, and choose it in order to resolve the ambiguous resolution, instead of guessing a formula from the left resolvent that unifies with the right resolvent, which might go wrong.\\

TODO: Explain where the sub came from.\\

TODO: define the rule formally here?\\

TODO: describe when the rule is invoked in the implementation\\

A simple way to decrease the complexity to linear with respect to the length of the proof is to return to the ideas used in the propositional case. In particular, by performing a traversal to collect the units of a proof, and then optimistically deleting units, some compression can often be achieved. By ignoring whether or not a unit satisfies Property \ref{prop:rootpair}, we can attempt to lower it, and should compression fail because deletions changed the substitutions to the point where contraction was not possible, we simply return the original proof. 

\begin{algorithm}[bt]
  \SetAlgoVlined
  \SetAlgoShortEnd
\SetKwFunction{check}{check}
  \KwIn {a proof $\psi$}
  \KwOut{a compressed proof $\psi^{\star}$}
  \KwData{a map $.'$: after line 4, it maps any $\varphi$ to \Del{$\varphi$, $D$}}
  \BlankLine

  \SetKwData{Units}{Units}

  \SetKw{Remove} {remove}
  \SetKw{Break} {break}

  \algolines{\Units $\leftarrow \varnothing$}{queue to store collected units}
  \BlankLine

  \For{every subproof $\varphi$, in a bottom-up traversal of $\psi$}{
    \lIf{$\varphi$ is a unit with more than one child and all literals of $\varphi$ are simultaneously unifiable}{enqueue $\varphi$ in \Units}
  }
  \BlankLine

    $\psi' \leftarrow $ \FuncSty{simple-fo-delete}$(\psi,\Units)$ \;
    \BlankLine

    \tcp{Reintroduce units}
    

    $\psi^{\star} \leftarrow \psi'$ \;
    \For{every unit $\varphi$ in \Units}{
        \Let{$\sigma$ be the unifier of $\rho(\psi^\star)$'s literals that contracts $\rho(\psi^\star)$ as much as possible} \;
        \Let{$c$ be the literals contracted by $\sigma$} \;
       \If{$ \con{\psi^{\star}}{c}{\sigma}$ and $\varphi'$ can be resolved} {
        $\psi^{\star} \leftarrow \con{\psi^{\star}}{c}{\sigma} \res{\ell^c}{}{\ell}{} \varphi'$ \;
        }\Else { \Return $\psi$}
      
    }
  
    

  \caption{\SFOLowerUnits}
  \label{algo:simpleFOLU}
\end{algorithm}

Algorithm \ref{algo:simpleFOLU} works similarly to the propositional algorithm.  It first performs a bottom up traversal to collect potential units and the literals that are resolved away from by those units, adding the units to a queue (line 1). As seen in Examples \ref{ex:pairwise} in Section \ref{sec:Challenges}, unification of the resolved away literals is necessary, so it performs a check to make sure these literals satisfy Property \ref{prop:pair} (line 4). If it succeeds, it attempts to re-introduce all the removed units at the bottom of the proof, where it attempts to compress the literals that would be resolved away by each unit (lines 6-15). Note that this requires the implementation to track which literals should be resolved against each unit. In order to avoid traversing the proof to find these again after the deletion of every potential unit (as is done in Algorithm \ref{algo:FOLU}), we use a modified \FuncSty{delete} function, called \FuncSty{simple-fo-delete}, which is the same as Algorithm \ref{algo:del} except with line 6 changed to the following:

   \lIf{$~\varphi'_L \in D~$}{ 
     \Return{$(\rho(\varphi'_L) \sigma_R)$} 
    }
    \lElseIf{$\varphi'_R \in D$}{ 
      \Return{$(\rho(\varphi'_R) \sigma_R)$}  
    }

\FuncSty{simple-fo-delete} is designed to reduce the complexity of tracking literals. \FuncSty{simple-fo-delete} behaves much more closely to the propositional case and requires none of the additional data structures required by \FuncSty{fo-delete}. In this function, when a unit node is returned, instead of returning the opposite node (respectively $\psi_L'$ or $\psi_R'$, line 6) in the resolution (which is done in the propositional case), or tracking the literals (which is done in \FuncSty{fo-delete}), we return the opposite node with $\sigma_L$ (respectively $\sigma_R$) applied to it. In this way, the literals not resolved with the unit will look like they would have in the original proof, and the literal which was not resolved due to the deletion looks like it is syntactically equal with the unit literal at this stage. The fact that the other literals look like they did in the original proof is key: now resolution in the compressed proof can use the old literals, which should appear as they before, and not worry about choosing the wrong literal in case of ambiguous resolution.

%A negative side-effect of this is that we may end up grounding literals, and having to carry these forms of each literal forward, which may increase the character length of the clause, though not the number of nodes in the proof.

Additionally, by modifying delete in this manner we can longer guarantee that Property \ref{prop:rootpair} is satisfied. Property \ref{prop:rootpair} so the appearance of literals that were to be resolved away from a unit clause may have changed, preventing completion of the proof. If this happens {\SFOLowerUnits} will attempt to re-introduce this node and fail, returning the original input proof (line 12). As a result, some proofs that can be compressed are returned unmodified, but those that do not require this additional property can be compressed much more quickly.



%\begin{algorithm}[bt]
  \SetAlgoVlined
  \SetAlgoShortEnd
  \KwIn{a proof $\varphi$}
  \KwIn{$D$ a set of subproofs}
  \KwOut{a proof $\varphi'$ obtained by deleting the subproofs in $D$ from $\varphi$}
  \BlankLine

  \newcommand{\fixL}{\ensuremath{\varphi'_L}}
  \newcommand{\fixR}{\ensuremath{\varphi'_R}}

  \lIf{$\varphi \in D$ or $\raiz{\varphi}$ has no premises}{\Return{$\varphi$}}
  \BlankLine

  \Else{$\varphi = \varphi_L \res{\ell_L}{\sigma_L}{\ell_R}{\sigma_R} \varphi_R$\;
    $\varphi'_L \leftarrow $ \Rec{$\varphi_L$,$D$} \;
    $\varphi'_R \leftarrow $ \Rec{$\varphi_R$,$D$} \;
    \BlankLine

    \lIf{$\varphi'_L \in D$}{ 
      \Return{$($\fixR $\sigma_R)$} 
    }
    \lElseIf{$\varphi'_R \in D$}{ 
      \Return{$($\fixL $\sigma_R)$}  
    }
    \BlankLine


    \lElse{ 
      \Return{ \fixL~$\res{\ell_L}{}{\ell_R}{}$~\fixR}
    }
  }



  \caption[.]{\FuncSty{simple-fo-delete}}
  \label{algo:sfodel}
\end{algorithm}




\section{Experiments} \label{sec:exp}

A prototype\footnote{Source code available at \url{https://github.com/jgorzny/Skeptik}} of a (two-traversal) version of {\SFOLowerUnits} has been implemented in the functional programming language Scala\footnote{\url{http://www.scala-lang.org/}} as part of the \skeptik
 library\footnote{\url{https://github.com/Paradoxika/Skeptik}}. 

Before evaluating this algorithm, we first generated several benchmark proofs. This was done by executing the {\SPASS}\footnote{\url{http://www.spass-prover.org/}} theorem prover on ToDo(numberOfProblems) problems of the ToDo categories of the TPTP Problem Library \footnote{\url{http://www.cs.miami.edu/{\textasciitilde}tptp/}}. In order to generate pure resolution proofs, most advanced inference rules used by {\SPASS}  were disabled. The Euler Cluster at the University of Victoria\footnote{\url{https://rcf.uvic.ca/euler.php}} was used and the time limit was 300 seconds per problem. Under these conditions, {\SPASS} was able to generate 308 proofs. 

The evaluation of {\SFOLowerUnits} was performed on a laptop (2.8GHz Intel Core i7 processor with 4 GB of RAM (1333MHz DDR3) available to the Java Virtual Machine). For each benchmark proof $\psi$, we measured\footnote{The raw data is available at ToDo (this link is not working) \url{https://docs.google.com/spreadsheets/d/1F1-t2OuhypmTQhLU6yTj42aiZ5CqqaZvhVvOzeFgn0k/edit\#gid=1182923972}} the time needed to compress the proof ($t(\psi)$) and the compression ratio ($(|\psi|-|\alpha(\psi)|)/|\psi|$), where $|\psi|$ is the length of $\psi$ (i.e. the number of axioms, resolution and contractions (ignoring substitutions)) and $\alpha(\psi)$ is the result of applying {\SFOLowerUnits} to $\psi$.

The proofs generated by {\SPASS} were small (with lengths from 3 to 49). These proofs are specially small in comparison with the typical proofs generated by SAT- and SMT-solvers, which usually have from a few hundred to a few million nodes. The number of proofs (compressed and uncompressed) per length is shown in Figure \ref{fig:ex} (b). Uncompressed proofs are those which had either no lowerable units to lower or for which \SFOLowerUnits failed and returned the original proof. Such failures occurred on only 14 benchmark proofs. Among the smallest of the 308 proofs, very few proofs were compressed. This is to be expected, since the likelihood that a very short proof contain a lowerable unit (or even merely a unit with more than one child) is low. The proportion of compressed proofs among longer proofs is, as expected, larger, since they have more nodes and it is more likely that some of these nodes are lowerable units. 13 out of 18 proofs with length greater than or equal to 30 were compressed. 

Figure \ref{fig:ex} (a) shows a box-whisker plot of compression ratio with proofs grouped by length and whiskers indicating minimum and maximum compression ratio achieved within the group. Besides the median compression ratio (the horizontal thick black line), the chart also shows the mean compression ratios for all proofs of that length and for all compressed proofs (the red cross and the blue circle). In the longer proofs (length greater than 34), the median and the means are in the range from 5\% to 15\%, which is satisfactory in comparison with the total compression ratio of 7.5\% that has been measured for the propositional {\LowerUnits} algorithm on much longer propositional proofs \cite{Boudou}.

Figure \ref{fig:ex} (c) shows a scatter plot comparing the length of the input proof against the length of the compressed proof. For the longer proofs (circles in the right half of the plot), it is often the case that the length of the compressed proof is significantly lesser than the length of the input proof.

Figure \ref{fig:ex} (d) plots the cumulative original and compressed lengths of all benchmark proofs (for an x-axis value of $k$, the cumulative curves show the sum of the lengths of the shortest $k$input proofs). The total cumulative length of all original proofs is ToDo:4500(put the correct number here) while the cumulative length of all proofs after compression is ToDo:4000(correct this number). This results in a total compression ratio of ToDo:12\%(compute this number), which is impressive, considering the inclusion of all the short proofs (in which the presence of lowerable units is a priori unlikely) tends to decrease the total compression ratio. For comparison, the total compression ratio considering only the 100 longest input proofs is ToDo:(compute this percentage).

Figure \ref{fig:ex} also indicates an interesting potential trend. The gap between the two cumulative curves seems to grow superlinearly. If this trend is extrapolated, progressively larger compression ratios can be expected for longer proofs. This is compatible with Theorem 10 in \cite{LURPI}, which shows that, for proofs generated by eagerly resolving units against all clauses, the propositional {\LowerUnits} algorithm can achieve quadratic assymptotic compression. SAT- and SMT-solvers based on CDCL (Conflict-Driven Clause Learning) avoid eagerly resolving unit clauses by dealing with unit clauses via boolean propagation on a conflict graph and extracting subproofs from the conflict graph with every unit being used at most once per subproof (even when it was used multiple times in the conflict graph). Saturation-based automated theorem provers, on the other hand, might be susceptible to the eager unit resolution redundancy described in Theorem 10 \cite{LURPI}. This potential trend would need to be confirmed by further experiments with more data (more proofs and longer proofs).

The total time needed by {\SPASS} to generate all 308 proofs on the Euler Cluster was ToDo. The total time for {\SFOLowerUnits} to be executed on all 308 proofs was ToDo on a simple laptop. (ToDo: make sure the total time calculation either includes or excludes parsing times for both Skeptik and SPASS. otherwise the comparison would be biased and unfair). Therefore, {\SFOLowerUnits} is a fast algorithm. For a small overhead in time (in comparison to proving time), it may simplify the proof considerably.


% \begin{figure}
% \includegraphics[scale=0.5]{images/compress_time_vs_proof_length.pdf}
% \end{figure}

% \begin{figure}
% \includegraphics[scale=0.5]{images/compress_time_vs_proof_length_res.pdf}
% \end{figure}

% \begin{figure}
% \includegraphics[scale=0.5]{images/compress_ratio_vs_proof_length.pdf}
% \end{figure}

%\begin{figure}\label{fig:compressRatioResVLength} %USED
%\includegraphics[scale=0.5]{images/compress_ratio_res_vs_proof_length.pdf}
%\end{figure}

% \begin{figure}
% \includegraphics[scale=0.5]{images/compress_ratio_res_vs_proof_length_res.pdf}
% \end{figure}

%\begin{figure}%USED
%\includegraphics[scale=0.5]{images/compress_ratio_res_vs_proof_length_all_proofs.pdf}
%\end{figure}
\begin{figure}
\centering
%    \subfloat[Average compression (only success)]{{\includegraphics[scale=0.5]{images/compress_ratio_res_vs_proof_length.pdf} }}
    \subfloat[Compression ratio]{{\includegraphics[scale=0.5]{images/compress_ratio_res_vs_proof_length_all_proofs.pdf} }}%\hfilll
    \subfloat[Number of (non-)compressed proofs]{{\includegraphics[scale=0.5]{images/num_compressed_stacked.pdf}}}\hfill
    \subfloat[Compressed length against input length]{{\includegraphics[scale=0.5]{images/compress_length_no_sub_vs_length_all_proofs.pdf} }}
%    \subfloat[Total proof nodes]{{\includegraphics[scale=0.5]{images/cumulative_res_nodes_no_subs.pdf} }}
    \subfloat[Cumulative proof lengths]{{\includegraphics[scale=0.5]{images/cumulative_res_nodes_no_subs_top100.pdf}}}
\caption{Empirical evaluation results}
\label{fig:ex}
\end{figure}


%\begin{figure}
%\centering
%    \subfloat{{\includegraphics[scale=0.5]{images/compress_ratio_res_vs_proof_length.pdf}
%}}%
%    \subfloat{{\includegraphics[scale=0.5]{images/compress_ratio_res_vs_proof_length_all_proofs.pdf} }}%
%\caption{Compression ratio versus proof length without uncompressed proofs (left) and with with uncompressed proofs (right).}
%\label{fig:ex1}
%\end{figure}



% \begin{figure}
% \includegraphics[scale=0.5]{images/num_compressed_count.pdf}
% \end{figure}

% \begin{figure}
% \includegraphics[scale=0.5]{images/num_compressed_percent.pdf}
% \end{figure}


%\begin{figure}%USED
%\includegraphics[scale=0.5]{images/num_compressed_stacked.pdf}
%\end{figure}

%\begin{figure}
%\centering
%    \subfloat{{\includegraphics[scale=0.5]{images/num_compressed_stacked.pdf}
%}}%
%    \subfloat{{\includegraphics[scale=0.5]{images/cumulative_res_nodes_no_subs.pdf} }}
%\caption{Number of proofs compressed of each length (left), and total number of nodes before and after compression (right).}
%\label{fig:ex2}
%\end{figure}


% \begin{figure}
% \includegraphics[scale=0.5]{images/res_length_vs_compress_res_length_all_proofs.pdf}
% \end{figure}
% \begin{figure}
% \includegraphics[scale=0.5]{images/res_length_vs_compress_res_length.pdf}
% \end{figure}

% \begin{figure}
% \includegraphics[scale=0.5]{images/cumulative_res_nodes.pdf}
% \end{figure}

%\begin{figure}%USED
%\includegraphics[scale=0.5]{images/cumulative_res_nodes_no_subs.pdf}
%\end{figure}

%\begin{figure} %USED
%\includegraphics[scale=0.5]{images/cumulative_res_nodes_no_subs_top100.pdf}
%\end{figure}




% \begin{figure}
% \includegraphics[scale=0.5]{images/cumulative_res_nodes_no_subs_log.pdf}
% \end{figure}
% \begin{figure}
% \includegraphics[scale=0.5]{images/compress_length_no_sub_vs_length.pdf}
% \end{figure}

%\begin{figure}%USED
%\includegraphics[scale=0.5]{images/compress_length_no_sub_vs_length_all_proofs.pdf}
%\end{figure}



\section{Conclusions and Future Work}

ToDo: by Bruno

{\LowerUnivalents}, the algorithm presented here, has been shown in the previous section to compress
more than {\LowerUnits}. This is so because, as demonstrated in Proposition \ref{prop:compression}, the
set of subproofs it lowers is always a superset of the set of subproofs lowered by {\LowerUnits}. It might
be possible to lower even more subproofs by finding a characterization of (efficiently) lowerable subproofs
broader than that of univalent subproofs considered here. This direction for future work promises to be challenging, though, as evidenced by the non-triviality of the optimizations discussed in Section \ref{sec:LUniv} for obtaining a linear-time implementation of {\LowerUnivalents}.



As discussed in Section \ref{sec:LUnivRPI}, the proposed algorithm can be embedded in the deletion traversal of other algorithms.  As
an example, it has been shown that the combination of {\LowerUnivalents} with {\RPI}, compared to
the sequential composition of {\LowerUnits} after {\RPI}, results in a better compression ratio with
only a small processing time overhead (Figure \ref{fig:LUnivRPI}). Other compression algorithms that also have a subproof
deletion or reconstruction phase (e.g. \ReduceReconstruct) could probably benefit from being
combined with {\LowerUnivalents} as well.

%\vspace{-10pt}
%\paragraph{Acknowledgments:}



\bibliographystyle{splncs}
\bibliography{biblio}


\end{document}

% vim: tw=100
