

\section{First-Order LowerUnits} \label{sec:FOLU}

ToDo by Bruno

{\LowerUnits} does not lower every lowerable subproof. In particular, it does not take into
account the already lowered subproofs. For instance, if a unit $\varphi_1$ proving $\{a\}$ has
already been lowered, a subproof $\varphi_2$ with conclusion $\{\neg a, b\}$ may be lowered as well and
reintroduced above $\varphi_1$. The posterior reintroduction of $\varphi_1$ will resolve away $\neg a$ and guarantee that it does not occur in the resulting proof's conclusion. But care must also be taken not to lower $\varphi_2$ if $\neg a$ is a valent literal of
$\varphi_2$, otherwise $a$ will undesirably occur in the resulting proof's conclusion.

\begin{definition}[Univalent subproof]
A subproof $\varphi$ in a proof $\psi$ is \emph{univalent} w.r.t. a set $\Delta$ of literals iff
$\varphi$ has exactly one valent literal $\ell$ in $\psi$, $\ell \notin \Delta$ and
$\Conclusion{\varphi} \subseteq \Delta \cup \left\{ \ell \right\}$. $\ell$ is called the \emph{univalent
literal} of $\varphi$ in $\psi$ w.r.t.  $\Delta$.
\end{definition}

The principle of {\LowerUnivalents} is to lower all univalent subproofs. Having only one valent literal makes them behave essentially like units w.r.t. the technique of lowering. $\Delta$ is
initialized to the empty set. Then the complements of the univalent literals are incrementally added to
$\Delta$. Proposition \ref{prop:LUniv} ensures that the conclusion of the resulting proof
subsumes the conclusion of the original one.

\begin{proposition} \label{prop:LUniv}
Given a proof $\psi$, if 
%for an integer $n$
there is a sequence $U = (\varphi_1 \ldots \varphi_n)$
of $\psi$'s subproofs and a sequence $(\ell_1 \ldots \ell_n)$ of literals such that $\forall i \in
[1 \ldots n]$, $\ell_i$ is the univalent literal of $\varphi_i$ w.r.t. $\Delta_{i-1} =
\{\dual{\ell_1} \ldots \dual{\ell_{i-1}}\}$, then the conclusion of $$ \psi' = \dn{\psi}{U}
\odot_{\ell_n} \varphi_n \ldots \odot_{\ell_1} \varphi_1 $$ subsumes the conclusion of $\psi$.
\end{proposition}

\begin{proof}
The proposition is proven by induction on $n$, along with the fact that $\dn{\psi}{U} \notin U$.
For $n = 0$, $U = \varnothing$ and the properties trivially hold. Suppose a subproof
$\varphi_{n+1}$ of $\psi$ is univalent w.r.t. $\Delta_n$, with univalent literal $\ell_{n+1}$.
Because $\ell_{n+1} \notin \Delta_n$, there exists a subproof of $\dn{\psi}{U}$ with conclusion
containing $\dual{\ell_{n+1}}$, and therefore $\dn{\dn{\psi}{U}}{\varphi_{n+1}} \notin U \cup
\{\varphi_{n+1}\}$.  Let $\Gamma$ be the conclusion of $\dn{\psi}{U}$. The conclusion of $ \psi' =
\dn{\psi}{U \cup \{\varphi_{n+1}\}} = \dn{\dn{\psi}{U}}{\varphi_{n+1}} $ is included in $\Gamma \cup
\{\dual{\ell_{n+1}}\}$. The conclusion of $\psi' \odot_{\ell_{n+1}} \varphi_{n+1}$ is included in
$\Gamma \cup \Delta_n$. As $\Gamma \subseteq \Conclusion{\psi} \cup \Delta_n$, the conclusion of
$\psi' \odot_{\ell_{n+1}} \varphi_{n+1} \ldots \odot_{\ell_1} \varphi_1$ is included in
$\Conclusion{\psi}$. \qed
\end{proof}

For this principle to lead to proof compression, it is important to take care
of the mutual inclusion of univalent subproofs.
%not only of the order in which subproofs are collected for lowering but also of deleting all
%already collected univalent subproofs from the next subproof $\psi_i$ before reintroducing it.
Suppose, for instance, that $\varphi_i, \varphi_j, \varphi_k \in U$, $i < j < k$, $\varphi_j$ is a
subproof of $\varphi_i$ but not a subproof of $\dn{\psi}{\varphi_i}$, and $\dual{\ell_j} \in
\Conclusion{\varphi_k}$.  In this case, $\varphi_j$ will have one more child in
$$
\dn{\psi}{U} \odot_{\ell_n} \varphi_n \ldots \odot_{\ell_k} \varphi_k \ldots \odot_{\ell_j} \varphi_j \ldots \odot_{\ell_i} \varphi_i \ldots \odot_{\ell_1} \varphi_1
$$
than in the original proof $\psi$. The additional child is created when $\varphi_j$ is reintroduced.
All the other children are reintroduced with the reintroduction of $\varphi_i$, because
$\varphi_j$ was not deleted from $\varphi_i$.

To solve this issue, {\LowerUnivalents} traverses the proof in a top-down manner and simultaneously
deletes already collected univalent subproofs, as sketched in Algorithm \ref{algo:LUniv}.  


\SetKwData{Univ}{Univalents}
\begin{algorithm}[bt]
  \KwIn {a proof $\psi$}
  \KwOut{a compressed proof $\psi'$}
  \BlankLine

  \SetKw{Push}{push}
  \SetKw{Pop} {pop}

  \Univ $\leftarrow \varnothing$ \;
  $\Delta \leftarrow \varnothing$ \;
  \BlankLine

  \For{every subproof $\varphi$, in a top-down traversal \label{line:LUniv:step1begin} }{
    $\psi' \leftarrow$ \Rec{$\varphi$,\Univ} \label{line:LUniv:delete} \;
    \If{$\psi'$ is univalent w.r.t. $\Delta$ \label{line:LUniv:lunivtest} }{
      \Let{$\ell$} be the univalent literal \;
      \Push $\dual{\ell}$ onto $\Delta$ \label{line:LUniv:pushDelta} \;
      \Push $\psi'$     onto \Univ \label{line:LUniv:step1end} \;
    }
  }
  \BlankLine

  \tcp{At this point, $\psi' = \dn{\psi}{\Univ}$}
  \While{\Univ $\neq \varnothing$}{ \label{line:LUniv:reintroducebegin}
    $\varphi \leftarrow$ \Pop from \Univ \;
    $\ell \leftarrow$ \Pop from $\Delta$ \;
    \lIf{$\ell \in \Conclusion{\psi'}$ \label{line:LUniv:testreintroduce} }{
    $\psi' \leftarrow \varphi \odot_\ell \psi'$ \;}
  }

  \caption{Simplified \LowerUnivalents}
  \label{algo:LUniv}
\end{algorithm}


Figure \ref{fig:exluniv} shows an example proof and the result of compressing it with \LowerUnivalents. The top-down traversal starts with the leaves (axioms) and only visits a child when all its parents have already been visited. Assuming the unit with conclusion $\{a\}$ is the first visited leaf, it passes the univalent test in line \ref{line:LUniv:lunivtest}, is marked for lowering (line \ref{line:LUniv:step1end}) and the complement of its univalent literal is pushed onto $\Delta$ (line \ref{line:LUniv:pushDelta}). When the subproof with
conclusion $\{\dual{a},b\}$ is considered, $\Delta = \{\dual{a}\}$. As this subproof has only one
valent literal $b \notin \Delta$ and $\{\dual{a},b\} \subseteq \Delta \cup \{b\}$, it is
marked for lowering as well. At this point, $\Delta = \{\dual{a}, \dual{b}\}$, \texttt{Univalents} contains the two subproofs marked for lowering and $\psi'$ is the subproof with conclusion $\{\dual{a}, \dual{b}\}$ shown in Subfig. (b) (i.e. the result of deleting the two marked subproofs from the original proof in Subfig. (a)). No other subproof is univalent; no other subproof is marked for lowering. The final compressed proof (Subfig. (b)) is obtained by reintroducing the two univalent subproofs that had been marked (lines \ref{line:LUniv:reintroducebegin} -- \ref{line:LUniv:testreintroduce}). It has one resolution less than the original. This is so because the subproof with conclusion $\{\dual{a},b\}$ had been used (resolved) twice in the original proof, but lowering delays its use to a point where a single use is sufficient.

\begin{figure}[htb]
  \centering
  \subfloat[Original proof]{
    \centering
    \begin{tikzpicture}

      \rootnode;
      \withchildren{root} {r0}{\dual{a}}  {unit}{a};
      \withchildren{r0}   {r1}{\dual{a},c} {r2}{\dual{a},\dual{c}};
      \withchildren{r1}   {a0}{\dual{b},c} {low}{\dual{a},b};

      \proofnode[above right of=r2] {a1} {\dual{a},\dual{b},\dual{c}};
      \drawchildren {r2} {low} {a1};

    \end{tikzpicture}
  } \qquad
  \centering
  \subfloat[Compressed proof]{
    \centering
    \begin{tikzpicture}

      \rootnode;
      \withchildren{root} {r0}{\dual{a}}          {unit}{a};
      \withchildren{r0}   {r1}{\dual{a},\dual{b}} {low}{\dual{a},b};
      \withchildren{r1}   {a0}{\dual{b},c}        {a1}{\dual{a},\dual{b},\dual{c}};

    \end{tikzpicture}
  }
\caption{Example of proof crompression by \LowerUnivalents} 
\label{fig:exluniv}
\end{figure}


% Discussion of optimizations follow

Although the
call to \FuncSty{delete} inside the first loop (line \ref{line:LUniv:step1begin} to
\ref{line:LUniv:step1end}) suggests quadratic time complexity, this loop (line
\ref{line:LUniv:step1begin} to \ref{line:LUniv:step1end}) can be (and has been) actually implemented
as a recursive function extending a recursive implementation of \FuncSty{delete}. With such an
implementation, {\LowerUnivalents} has a time complexity linear w.r.t. the size of the proof, assuming the
univalent test (at line \ref{line:LUniv:lunivtest}) is performed in constant bounded time. 


Determining whether a literal is valent is expensive. But thanks to Proposition \ref{prop:valentactive},
subproofs with one active literal which is not in $\Conclusion{\psi}$ can be considered instead
of subproofs with one valent literal.  If the active literal is not valent, the corresponding
subproof will simply not be reintroduced later (i.e. the condition in line 28 of Algorithm \ref{algo:fullLUniv} will fail).

While verifying if a subproof could be univalent, some edges might be deleted. If a
subproof $\varphi_i$ has already been collected as univalent subproof with univalent literal
$\ell_i$ and the subproof $\varphi'$ being considered now has $\ell_i$ as active literal, the
corresponding incoming edges can be removed. Even if $\ell_i$ is valent for $\varphi'$, only
$\dual{\ell_i}$ would be introduced, and it would be resolved away when reintroducing
$\varphi_i$. The \FuncSty{delete} operation can be easily modified to remove both nodes and edges.

Algorithm \ref{algo:fullLUniv} sums up the previous remarks for an efficient implementation of
{\LowerUnivalents}. As noticed above, sometimes this algorithm may consider a subproof as univalent when it
is actually not. But as care is taken when reintroducing subproofs (at line \ref{line:full:testreintroduce}),
the resulting conclusion still subsumes the original.  The test that $\ell \in \Conclusion{\varphi}$
at line \ref{line:full:testactive} is mandatory since $\ell$ might have been deleted from
$\Conclusion{\varphi}$ by the deletion of previously collected subproofs.

\begin{algorithm}[pbt]
  \SetAlgoVlined
  \SetAlgoShortEnd

  \KwData {a proof $\psi$, compressed in place}
  \KwIn {a set $D_V$ of subproofs to delete}
  \KwIn {a set $D_E$ of edges to delete}
%  \KwOut{the proof $\psi$ compressed in place}
  \BlankLine

  \SetKw{Push}{push}
  \SetKw{Pop} {pop}
  \SetKw{Add} {add}
  \SetKw{Rep} {replace}

  \SetKwData{Activ}{ActiveLiterals}

  \Univ $\leftarrow \varnothing$ \;
  $\Delta \leftarrow \varnothing$ \;
  \BlankLine

  \For{every subproof $\varphi$, in a top-down traversal of $\psi$ }{

    \tcp{The deletion part.}
    \If{$\varphi$ is not an axiom}{
      \Let{$\varphi = \varphi_L \odot_\ell \varphi_R$} \;
      \uIf{ $\varphi_L \in D_V$ or $\pedge{\varphi}{\dual{\ell}}{\varphi_L} \in D_E$ }{
        \uIf{ $\pedge{\varphi}{\ell}{\varphi_R} \in D_E$ }{
          \Add $\varphi$ to $D_V$ \;
        }
        \Else{
          \Rep $\varphi$ by $\varphi_R$ \;
        }
      }
      \ElseIf{ $\varphi_R \in D_V$ or $\pedge{\varphi}{\dual{\ell}}{\varphi_R} \in D_E$ }{
        \uIf{ $\pedge{\varphi}{\ell}{\varphi_L} \in D_E$ }{
          \Add $\varphi$ to $D_V$ \;
        }
        \Else{
          \Rep $\varphi$ by $\varphi_L$ \;
        }
      }
    }
    \BlankLine
    
    \tcp{Test whether $\varphi$ is univalent.}
    \Activ $\leftarrow \varnothing$ \;
    \For{each incoming edge $e = \n \xrightarrow{\ell} \raiz{\varphi}$, $e \notin D_E$ }{
      \uIf{$\dual{\ell} \in \Delta$}{
        \Add $e$ to $D_E$ \;
      }
      \ElseIf{$\ell \notin \Delta$, $\ell \in \Conclusion{\varphi}$ \label{line:full:testactive}
              and $\ell \notin \Conclusion{\psi}$ }{
        \Add $\ell$ to \Activ \;
      }
    }

%    \BlankLine
    \If{\Activ $= \{\ell\}$ and $\Conclusion{\varphi} \subseteq \Delta \cup \{\ell\}$ }{
      \Push $\dual{\ell}$ onto $\Delta$ \;
      \Push $\varphi$     onto \Univ  \;
    }
  }
  \BlankLine

  \tcp{Reintroduce lowered subproofs.}
  \While{\Univ $\neq \varnothing$}{
    $\varphi \leftarrow$ \Pop from \Univ \;
    $\ell \leftarrow$ \Pop from $\Delta$ \;
    \If{$\ell \in \Conclusion{\psi}$ \label{line:full:testreintroduce}  }{
      \Rep $\psi$ by $\varphi \odot_\ell \psi$ \;}
  }

  \caption{Optimized {\LowerUnivalents} as an enhanced \texttt{delete}}
  \label{algo:fullLUniv}
\end{algorithm}


Every node in a proof $\langle V, E, \Gamma \rangle$ has exactly two outgoing edges unless it is the
root of an axiom. Hence the number of axioms is $|V| - \frac{1}{2}\,|E|$ and because there is at
least one axiom, the average number of active literals per node is strictly less than two.
Therefore, if {\LowerUnivalents} is implemented as an improved recursive \FuncSty{delete}, its time
complexity remains linear, assuming membership of literals to the set $\Delta$ is computed in constant
time.

\begin{proposition} \label{prop:compression}
Given a proof $\psi$,
{\LowerUnits\unskip\FuncSty{(}$\psi$\FuncSty{)}}
has at least as many nodes as 
{\LowerUnivalents\unskip\FuncSty{(}$\psi$\FuncSty{)}}
if there are no two units in $\psi$ with the same conclusion.
\end{proposition}

\begin{proof}
A unit $\varphi$ has exactly one active literal $\ell$. Therefore $\varphi$ is collected by
{\LowerUnivalents} unless $\dual{\ell} \in \Delta$ or $\ell \in \Delta$. If $\dual{\ell} \in \Delta$
all the incoming edges to $\raiz{\varphi}$ are deleted. If $\ell \in \Delta$, every edge
$\n \xrightarrow{\dual{\ell}} \n'$ where $\n$ is on a path from $\raiz{\psi}$ to $\raiz{\varphi}$
is deleted.
%coming from a descendent of $\raiz{\varphi}$ and labeled by $\dual{\ell} are deleted.
In particular, for every edge $\n \xrightarrow{\ell} \raiz{\varphi}$ the edge $\n
\xrightarrow{\dual{\ell}} \n'$ is deleted.  Moreover, as $\ell$ is the only literal of $\varphi$'s
conclusion, $\varphi$ is propagated down the proof until the univalent subproof with valent literal
$\dual{\ell}$ is reintroduced. \qed
\end{proof}

In the case where there are at least two units with the same conclusion in $\psi$, the
compressed proof depends on the order in which the units are collected. For both algorithms, only one of these units appears in the compressed proof.
