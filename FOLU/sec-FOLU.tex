

\section{First-Order LowerUnits} \label{sec:FOLU}


\SetKwFunction{Rec}{delete}
\SetKw{Let}{let}

\begin{algorithm}[bt]
  \KwIn{a proof $\varphi$}
  \KwIn{$D$ a set of subproofs}
  \KwOut{a proof $\varphi'$ obtained by deleting the subproofs in $D$ from $\varphi$}
  \BlankLine

  \newcommand{\fixL}{\ensuremath{\varphi'_L}}
  \newcommand{\fixR}{\ensuremath{\varphi'_R}}

  \lIf{$\varphi \in D$ or $\raiz{\varphi}$ has no premises}{\Return{$\varphi$}}
  \BlankLine

  \Else{
    \Let{$\varphi_L$ and $\varphi_R$} be such that
      $\varphi = \varphi_L \res{\ell_L}{\sigma_L}{\ell_R}{\sigma_R} \varphi_R$ \;
    \Let{$\varphi'_L = $ \Rec{$\varphi_L$,$D$}} \;
    \Let{$\varphi'_R = $ \Rec{$\varphi_R$,$D$}} \;
    \BlankLine

    \lIf{$\varphi'_L \in D$}{ \Return{\fixR} }
    \lElseIf{$\varphi'_R \in D$}{ \Return{\fixL} }
    \BlankLine

    \lElseIf{$\ell \notin \Conclusion{\fixL}$}{ \Return{\fixL} }
    \lElseIf{$\dual{\ell} \notin \Conclusion{\fixR}$}{ \Return{\fixR} }
    \BlankLine

    \lElse{ \Return{ \fixL~$\res{\ell_L}{\sigma_L}{\ell_R}{\sigma_R}$~\fixR} }
  }

  \caption[.]{\FuncSty{fo-delete}}
  \label{algo:fodel}
\end{algorithm}



\begin{proposition} \label{prop:LUniv}
Given a proof $\psi$, if 
%for an integer $n$
there is a sequence $U = (\varphi_1 \ldots \varphi_n)$
of $\psi$'s subproofs and a sequence $(\ell_1 \ldots \ell_n)$ of literals such that $\forall i \in
[1 \ldots n]$, $\ell_i$ is the univalent literal of $\varphi_i$ w.r.t. $\Delta_{i-1} =
\{\dual{\ell_1} \ldots \dual{\ell_{i-1}}\}$, then the conclusion of $$ \psi' = \dn{\psi}{U}
\odot_{\ell_n} \varphi_n \ldots \odot_{\ell_1} \varphi_1 $$ subsumes the conclusion of $\psi$.
\end{proposition}

\begin{proof}
The proposition is proven by induction on $n$, along with the fact that $\dn{\psi}{U} \notin U$.
For $n = 0$, $U = \varnothing$ and the properties trivially hold. Suppose a subproof
$\varphi_{n+1}$ of $\psi$ is univalent w.r.t. $\Delta_n$, with univalent literal $\ell_{n+1}$.
Because $\ell_{n+1} \notin \Delta_n$, there exists a subproof of $\dn{\psi}{U}$ with conclusion
containing $\dual{\ell_{n+1}}$, and therefore $\dn{\dn{\psi}{U}}{\varphi_{n+1}} \notin U \cup
\{\varphi_{n+1}\}$.  Let $\Gamma$ be the conclusion of $\dn{\psi}{U}$. The conclusion of $ \psi' =
\dn{\psi}{U \cup \{\varphi_{n+1}\}} = \dn{\dn{\psi}{U}}{\varphi_{n+1}} $ is included in $\Gamma \cup
\{\dual{\ell_{n+1}}\}$. The conclusion of $\psi' \odot_{\ell_{n+1}} \varphi_{n+1}$ is included in
$\Gamma \cup \Delta_n$. As $\Gamma \subseteq \Conclusion{\psi} \cup \Delta_n$, the conclusion of
$\psi' \odot_{\ell_{n+1}} \varphi_{n+1} \ldots \odot_{\ell_1} \varphi_1$ is included in
$\Conclusion{\psi}$. \qed
\end{proof}




% \begin{algorithm}[bt]
%   \KwIn {a proof $\psi$}
%   \KwOut{a compressed proof $\psi'$}
%   \BlankLine

%   \SetKw{Push}{push}
%   \SetKw{Pop} {pop}

%   \Units $\leftarrow \varnothing$ \;
%   $\Delta \leftarrow \varnothing$ \;
%   \BlankLine

%   \For{every subproof $\varphi$, in a top-down traversal \label{line:LUniv:step1begin} }{
%     $\psi' \leftarrow$ \Rec{$\varphi$,\Univ} \label{line:LUniv:delete} \;
%     \If{$\psi'$ is univalent w.r.t. $\Delta$ \label{line:LUniv:lunivtest} }{
%       \Let{$\ell$} be the univalent literal \;
%       \Push $\dual{\ell}$ onto $\Delta$ \label{line:LUniv:pushDelta} \;
%       \Push $\psi'$     onto \Univ \label{line:LUniv:step1end} \;
%     }
%   }
%   \BlankLine

%   \tcp{At this point, $\psi' = \dn{\psi}{\Univ}$}
%   \While{\Univ $\neq \varnothing$}{ \label{line:LUniv:reintroducebegin}
%     $\varphi \leftarrow$ \Pop from \Univ \;
%     $\ell \leftarrow$ \Pop from $\Delta$ \;
%     \lIf{$\ell \in \Conclusion{\psi'}$ \label{line:LUniv:testreintroduce} }{
%     $\psi' \leftarrow \varphi \odot_\ell \psi'$ \;}
%   }

%   \caption{Simplified \LowerUnivalents}
%   \label{algo:LUniv}
% \end{algorithm}


\begin{figure}[htb]
  \centering
  \subfloat[Original proof]{
    \centering
    \begin{tikzpicture}

      \rootnode;
      \withchildren{root} {r0}{\dual{a}}  {unit}{a};
      \withchildren{r0}   {r1}{\dual{a},c} {r2}{\dual{a},\dual{c}};
      \withchildren{r1}   {a0}{\dual{b},c} {low}{\dual{a},b};

      \proofnode[above right of=r2] {a1} {\dual{a},\dual{b},\dual{c}};
      \drawchildren {r2} {low} {a1};

    \end{tikzpicture}
  } \qquad
  \centering
  \subfloat[Compressed proof]{
    \centering
    \begin{tikzpicture}

      \rootnode;
      \withchildren{root} {r0}{\dual{a}}          {unit}{a};
      \withchildren{r0}   {r1}{\dual{a},\dual{b}} {low}{\dual{a},b};
      \withchildren{r1}   {a0}{\dual{b},c}        {a1}{\dual{a},\dual{b},\dual{c}};

    \end{tikzpicture}
  }
\caption{Example of proof compression by \LowerUnivalents} 
\label{fig:exluniv}
\end{figure}




