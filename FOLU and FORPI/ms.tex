\documentclass[runningheads]{llncs}
%\documentclass{iosart2x}
%\documentclass{article}
\usepackage{etex}
\usepackage[margin=1.5in]{geometry}
\usepackage{mathtools}
\usepackage{xcolor}
\usepackage{enumitem,amsmath,amssymb}
%\usepackage{breakurl}    % used for \url and \burl
\usepackage{url}
\usepackage[linesnumbered,boxed,noline,noend]{algorithm2e}
\usepackage{todonotes}

\def\defaultHypSeparation{\hskip.1in}

\usepackage{tikz}
\usetikzlibrary{arrows}

\usepackage{subfig}
\usepackage{array,booktabs,multirow}
\usepackage{placeins}

\usepackage{logictools}
\usepackage{prooftheory}
\usepackage{comment}
\usepackage{mathenvironments}
\usepackage{drawproof}
\usepackage{bussproofs}
\usepackage{tensor}
\usepackage{mathtools}
\usepackage{amsmath}
%\usepackage{amsthm}

\usepackage{graphicx}
%\usepackage{caption}
%\usepackage{subcaption}


\renewcommand{\topfraction}{0.85}
\renewcommand{\textfraction}{0.1}
\renewcommand{\floatpagefraction}{0.75}


%\newcommand{\freevar}[1]{\mathrm{FV}(#1)}
\newcommand{\freevar}[1]{\mathrm{Var}(#1)}


\newcommand{\Vertices}[1]{V_{#1}}
\newcommand{\Edges}[1]{E_{#1}}
\newcommand{\Conclusion}[1]{\clause_{#1}}

\newcommand{\axiom}[1]{\widehat{#1}}
\newcommand{\n}{v}
\newcommand{\raiz}[1]{\rho(#1)}

\newcommand{\pedge}[3]{\ensuremath{\raiz{#1} \xrightarrow{#2} \raiz{#3}}}

\DeclarePairedDelimiter{\abs}{\lvert}{\rvert}

\newcommand\inlineeqno{\stepcounter{equation}\ (\theequation)}


% Contraction
\newcommand{\con}[3]{\lfloor #1 \rfloor_{#2}^{#3}}

% Resolution
%\newcommand{\res}[6]{#1 \tensor[^{#2}_{#3}]{\odot}{^{#4}_{#5}} #6}
%\newcommand{\res}[6]{#1 \prescript{#2}{#3}{\odot^{#4}_{#5}} #6}

\newcommand{\res}[4]{\mathrel{\operatorname*{\odot}_{#1 #3}^{#2 #4}}}

%\newtheorem{thm}{Theorem}[section]
%\newtheorem{lem}{Lemma}[section]
%\newtheorem{example}[thm]{Example}
%\theoremstyle{definition}
%\newtheorem{definition}{Definition}[section]


\usepackage{authblk}


\begin{document}
%\begin{frontmatter} 
\title{Lifting Propositional Proof Compression Algorithms to First-Order Logic}
\titlerunning{Lifting Propositional Proof Compression Algorithms}
\authorrunning{J. Gorzny, E. Postan, and B. Woltzenlogel Paleo}


%\author{Jan Gorzny \and Ezequiel Postan \and Bruno Woltzenlogel Paleo}
%\affil[1]{School of Computer Science, University of Waterloo, 200 University Ave. W., Waterloo, ON N2L 3G1, Canada}
%\affil[2]{Universidad Nacional de Rosario, Av. Pellegrini 250, S2000BTP Rosario, Santa Fe, Argentina}
%\affil[3]{Vienna University of Technology, Karlsplatz 13, 1040, Vienna, Austria}



\author{
  Jan Gorzny\inst{1}\thanks{Supported by the Google Summer of Code 2014 program.}
  \and
  Ezequiel Postan\inst{2}\thanks{Supported by the Google Summer of Code 2016 program.}
  \and 
  Bruno Woltzenlogel Paleo\inst{3,4}\thanks{Bruno ist Stipendiat der \"Osterreichischen Akademie der Wissenschaft (APART) an der TU-Wien.}
  %\thanks{Supported by the Austrian Science Fund, project P24300.}
}
\institute{
  \email{jgorzny@uwaterloo.ca}, University of Waterloo, Canada
  \and 
  \email{ezequiel@fceia.unr.edu.ar},  Universidad Nacional de Rosario, Av. Pellegrini 250, S2000BTP Rosario, Santa Fe, Argentina
  \and 
  \email{bruno@logic.at}, Vienna University of Technology, Austria
  \and 
  Australian National University
}

%\author[A]{\inits{J.}\fnms{Jan} \snm{Gorzny}\ead[label=e1]{jgorzny@uwaterloo.ca}%
%\thanks{Supported by the Google Summer of Code 2014 program. Corresponding author. \printead{e1}.}%
%\thanks{Corresponding author. \printead{e1}.}%
%}, 
%\author[B]{\inits{E.}\fnms{Ezequiel} \snm{Postan}\ead[label=e2]{ezequiel@fceia.unr.edu.ar}}, and 
%\author[C]{\inits{B.}\fnms{Bruno} \snm{Woltzenlogel Paleo}\ead[label=e3]{bruno@logic.at}}

%\address[A]{School of Computer Science, \institution{University of Waterloo}, 200 University Ave. W., Waterloo, ON N2L 3G1, \cny{Canada}\printead[presep={\\}]{e1}}
%\address[B]{\institution{Universidad Nacional de Rosario}, Av. Pellegrini 250, S2000BTP Rosario, Santa Fe, \cny{Argentina}\printead[presep={\\}]{e2}}
%\address[C]{\institution{Vienna University of Technology},
%Karlsplatz 13, 1040, Vienna, \cny{Austria}
%\printead[presep={\\}]{e3}}
%\runningauthor{J. Gorzny \and E. Postan \and B. Woltzenlogel Paleo}
%\authorrunning{J.\~Gorzny \and B.\~Woltzenlogel Paleo}

\maketitle

\begin{abstract}
Proofs are a key feature of modern propositional and first-order theorem provers. 
Proofs generated by such tools serve as explanations for unsatisfiability of statements. 
However, these explanations are complicated by proofs which are not necessarily as concise as possible.
There are a wide variety of compression techniques for propositional resolution proofs, but fewer compression techniques for first-order resolution proofs generated by automated theorem provers.
This paper describes an approach to compressing first-order logic proofs based on lifting proof compression ideas used in propositional logic to first-order logic. 

The first approach lifted from propositional logic delays resolution with \emph{unit clauses}, which are clauses that have a single literal.
The second approach is \emph{partial regularization}, which removes an inference $\eta$ when it is redundant in the sense that its pivot literal already occurs as the pivot of another inference in every path from $\eta$ to the root of the proof. 
This paper describes the generalization of the algorithms \LowerUnits and \RecyclePivotsIntersection [P. Fontain, S. Merz, and B. Woltzenlogel Paleo, Compression of Propostional Resolution Proofs via Partial Regularization, \emph{CADE-23}, 2011] from propositional logic to first-order logic. 
The generalized algorithms compresses resolution proofs containing resolution and factoring inferences with \emph{unification}.

An empirical evaluation of these approaches is included.



%This paper describes the generalization of the 
%proof compression algorithm
%\RecyclePivotsIntersection %\texttt{RecyclePivots\-WithIntersection}
%from propositional logic to first-order logic. The generalized algorithm performs partial regularization of resolution proofs containing resolution and factoring inferences with \emph{unification}. The resolution calculus is the underlying proof-theoretical foundation for many automated theorem provers. An empirical evaluation of the generalized algorithm and its combinations with \SFOLowerUnits is also presented.
\end{abstract}

%\begin{keyword}
%\kwd{proof compression}
%\kwd{first-order logic}
%\kwd{resolution}
%\kwd{unification}
%\end{keyword}



%\end{frontmatter}


%\begin{document}

%\maketitle


\sloppy  %to prevent overfull margins for e.g. \RPI

\setcounter{footnote}{0}
\newcommand{\la}{\leftarrow}

\section{Introduction} 

%Recently, there has been interest in the combination of theorem provers and machine learning techniques (e.g. \cite{irving2016deepmath,kaliszyk2018reinforcement,DBLP:conf/lpar/LoosISK17}). 
Explainable artificial intelligence is a major challenge for the artificial intelligence community \cite{bonacina2017automated}.
As artificial intelligence systems are used in a wider range of applications with greater consequences, the need to justify and verify the choices made by these systems will grow as well.
In the logical approach to artificial intelligence, theorem provers provide explanations through verifiable proofs of the decisions that they make.
On the other hand, machine learning-based approaches often fail to explain why they produced a particular answer (see e.g., \cite{miller2019explanation}). 
In order to improve the ability to explain machine learning-based systems, there have been suggestions and attempts to combine machine learning with automated reasoning tools to generate explainable results \cite{bonacina2017automated,siebert2019corg}. 
The logical approach to artificial intelligence is no longer separate from the machine learning approach.
Good proofs are therefore useful for the successful combination of these approaches, and this paper aims to improve generated proofs through proof compression.

Proof production is a key feature for modern theorem provers. 
Proofs are explanations for unsatisfiability, and are crucial for applications that require certification of a prover's answers or that extract additional information from proofs (e.g. unsat cores, interpolants, instances of quantified variables).
Mature first-order automated theorem provers, commonly based on refinements and extensions of resolution and superposition calculi \cite{Vampire,EProver,Spass,spassT,Beagle,cruanes2015extending,prover9-mace4}, support proof generation. 
However, proof production is non-trivial \cite{SchulzAPPA}, and the most efficient provers do not necessarily generate the shortest proofs.
One reason for this is that efficient resolution provers use refinements that restrict the application of inference rules.
Although fewer clauses are generated and the search space is reduced, refinements may exclude short proofs whose inferences do not satisfy the restriction.

Proof compression techniques ameliorate the difficulties that automated reasoning tools encounter during proof generation. Such techniques can be integrated into theorem provers or external tools with minimal overhead. Moreover, proof compression techniques (like those described in this paper) may result in a stronger proof which uses a strict subset of the original axioms required, which could also be considered simpler. The problem of proof compression is also closely related to Hilbert's 24th Problem \cite{Hilbert24Problem}, which asks for criteria to judge the simplicity of proofs; proof length is one possible criterion. 


%Proof certification is an important challenge for the artificial intelligence community, and longer proofs are worse explanations than shorter proofs. 
There are also technical reasons to seek smaller proofs.
Longer proofs take longer to check, consume more memory during proof-checking, occupy more storage space and are harder to exchange, may have a larger unsat core (if more input clauses are used in the proof), and have a larger Herbrand sequent if more variables are instantiated \cite{B10,B16,ResolutionHerbrand,Reis}. Recent applications of SAT solvers to mathematical problems have resulted in very large proofs; e.g., the proof of a long-standing problem in combinatorics was initially 200GB \cite{heule2016solving}. Such proofs are hard to store, let alone validate. More practically, a restriction of 100GB of disk space per benchmark per solver prevented validation of proofs in the SAT 2014 competition \cite{clausal}. 
%Although the first example is extreme, the second is not. %, especially when users may wish to instantiate parallel solvers. 
The inability to write their results to disk renders these solvers useless in some cases. Moreover, even if the only direct improvement of shorter proofs is in the communication between systems, there are indirect benefits to the end-user of a tool e.g., in terms of its responsiveness. 


For propositional resolution proofs, as those typically generated by SAT- and SMT-solvers, there is a wide variety of proof compression techniques. Algebraic properties of the resolution operation that are potentially useful for compression were investigated in \cite{bwp10}.
Compression algorithms based on rearranging and sharing chains of resolution inferences have been
developed in \cite{Amjad07} and \cite{Sinz}.  Cotton \cite{CottonSplit} proposed an algorithm that
compresses a refutation by repeatedly splitting it into a proof of a heuristically chosen literal $\ell$
and a proof of $\dual{\ell}$, and then resolving them to form a new refutation.  The {\ReduceReconstruct} algorithm \cite{RedRec} searches for locally redundant
subproofs that can be rewritten into subproofs of stronger clauses and with fewer resolution steps.
Bar-Ilan et al. \cite{RP08} and Fontaine et al. \cite{LURPI} described a linear time proof compression algorithm based on partial
regularization, which removes an inference $\eta$ when it is redundant in the sense that its pivot literal already occurs as the pivot of another inference in every path from $\eta$ to the root of the proof.

In contrast, although proof output has been a concern in first-order automated reasoning for a longer time than in propositional SAT-solving, there has been much less work on simplifying first-order proofs. For tree-like sequent calculus proofs, algorithms based on cut-introduction \cite{BrunoLPAR,Hetzl} have been proposed. However, converting a DAG-like resolution or superposition proof, as usually generated by current provers, into a tree-like sequent calculus proof may increase the size of the proof. For arbitrary proofs in the Thousands of Problems for Theorem Provers (TPTP) \cite{TPTP} format (including DAG-like first-order resolution proofs), there is an algorithm \cite{LPARCzech} that looks for terms that occur often in any Thousands of Solutions from Theorem Provers (TSTP) \cite{TPTP} proof and abbreviates them. 

The work reported in this paper lifts successful propositional proof compression algorithms to first-order logic.
We first lift the {\LowerUnits} ({\LU}) algorithm \cite{LURPI}, which delays resolution steps with unit clauses, resulting in a new algorithm that we called {\SFOLowerUnits} ({\GFOLU}). 
Following this, we lift the \texttt{Recycle\-PivotsWithIntersection} ({\RPI}) algorithm \cite{LURPI}. 
\RPI improves the \texttt{RecyclePivots} ({\RP}) algorithm \cite{RP08} by detecting nodes that can be regularized even when they have multiple children.
Earlier versions of this work appeared in \cite{GFOLU,forpigcai}.

This paper is organized as follows. 
Section \ref{sec:res} introduces the well-known first-order resolution calculus with notations that are suitable for describing and manipulating proofs as first-class objects.
Section \ref{sec:PropositionalLU} describes the propositional \LowerUnits algorithm.
Section \ref{sec:LUChallenges} demonstrates some challenges of lowering units in the context of first-order logic.
Section \ref{sec:FOLU} describes a quadratic time approach to lifting units in first-order logic while Section \ref{sec:SimpleFOLU} demonstrates a simpler, linear time approach, \GFOLU.
We then repeat this structure for {\RPI}: Section \ref{Section:RPI} summarizes the propositional {\RPI} algorithm and Section \ref{sec:FORPIChallenges} discusses the challenges that arise in the first-order case (mainly due to unification), which are not present in the propositional case, and conclude with conditions useful for first-order regularization. 
Section \ref{sec:FORPI} describes an algorithm that overcomes these challenges. 
Section \ref{sec:exp} presents experimental results obtained by applying the first-order variant of \RPI and its combinations with {\GFOLU}, on hundreds of proofs generated with the {\SPASS} theorem prover on TPTP benchmarks \cite{TPTP} and on randomly generated proofs. 
Section \ref{sec:conclusion} concludes the paper.

It is important to emphasize that this paper targets proofs in a pure first-order resolution calculus (with resolution and factoring rules only), without refinements or extensions, and without equality rules. As most state-of-the-art resolution-based provers use variations and extensions of this pure calculus and there exists no common proof format, the presented algorithm cannot be directly applied to the proofs generated by most provers, and even {\SPASS} had to be specially configured to disable {\SPASS}'s extensions in order to generate pure resolution proofs for our experiments. By targeting the pure first-order resolution calculus, we address the common theoretical basis for the calculi of various provers. In the Conclusion (Section \ref{sec:conclusion}), we briefly discuss what could be done to tackle common variations and extensions, such as splitting and equality reasoning. Nevertheless, they remain topics for future research beyond the scope of this paper.

\section{The Resolution Calculus}
\label{sec:res}

As usual, our language has infinitely many variable symbols (e.g. $x$, $y$, $z$, $x_1$, $x_2$, \ldots), constant symbols (e.g. $a$, $b$, $c$, $a_1$, $a_2$, \ldots), function symbols of every arity (e.g $f$, $g$, $f_1$, $f_2$, \ldots) and predicate symbols of every arity (e.g. $P$, $Q$, $P_1$, $P_2$,\ldots). A \emph{term} is any variable, constant or the application of an $n$-ary function symbol to $n$ terms.
An \emph{atomic formula} (\emph{atom}) is the application of an $n$-ary predicate symbol to $n$ terms. A \emph{literal} is an atom or the negation of an atom. The
\emph{complement} of a literal $\ell$ is denoted $\dual{\ell}$ (i.e. for any atom $P$,
$\dual{P} = \neg P$ and $\dual{\neg P} = P$). The \emph{underlying atom} of a literal $\ell$ is denoted $\abs{\ell}$ (i.e. for any atom $P$, $\abs{P} = P$ and $\abs{\neg P} = P$). A
\emph{clause} is a multiset of literals. $\bot$ denotes the \emph{empty clause}. A \emph{unit clause} is a clause with a single literal. Sequent notation is used for clauses (i.e. $P_1,\ldots,P_n \seq Q_1,\ldots, Q_m$ denotes the clause $\{ \neg P_1,\ldots, \neg P_n, Q_1, \ldots, Q_m \}$).
%$\freevar{t}$ (resp. $\freevar{\ell}$, $\freevar{\clause}$) denotes the set of variables in the term $t$ (resp. in the literal $\ell$ and in the clause $\clause$).
A \emph{substitution} $\{ x_1\backslash t_1, x_2 \backslash t_2, \ldots \}$ is a mapping from variables $\{ x_1, x_2, \ldots \}$ to, respectively, terms $\{t_1, t_2, \ldots \}$. The application of a substitution $\sigma$ to a term $t$, a literal $\ell$ or a clause $\clause$ results in, respectively, the term $t \sigma$, the literal $\ell \sigma$ or the clause $\clause \sigma$, obtained from $t$, $\ell$ and $\clause$ by replacing all occurrences of the variables in $\sigma$ by the corresponding terms in $\sigma$. A literal $\ell$ \emph{matches} another literal $\ell'$ if there is a substitution $\sigma$ such that $\ell\sigma=\ell'$. A \emph{unifier} of a set of literals is a substitution that makes all literals in the set equal. We will use $X \sqsubseteq Y$ to denote that $X$ \emph{subsumes} $Y$, when there exists a substitution $\sigma$ such that $X\sigma \subseteq Y$.

%The resolution calculus used in this paper has the following inference rules: 

% \begin{definition}[First-Order Axiom] A first-order axiom has no premises and concludes some clause $\Gamma$, as below.
% \begin{prooftree}
% \AxiomC{$~$ }
% \UnaryInfC{$\psi$: $\Gamma$}
% \end{prooftree}
% %where $\Gamma$ is a clause.
% \end{definition}

\begin{definition}[Resolution] \label{def:fores} \hfill
%An instance of first-order resolution requires two premises, as below.
\begin{prooftree}
\AxiomC{$\eta_1$: $\Gamma_L' \cup \{\ell_L\}$ }
\AxiomC{$\eta_2$: $\Gamma_R'\cup \{\ell_R\}$ }
\BinaryInfC{$\psi$: $\Gamma_L'\sigma_L \cup \Gamma_R'\sigma_R$}
\end{prooftree}
where $\sigma_L$ and $\sigma_R$ are substitutions such that $\ell_L\sigma_L=\dual{\ell_R}\sigma_R$. The literals $\ell_L$ and $\ell_R$ are \emph{resolved literals}, whereas $\ell_L \sigma_L$ and $\ell_R \sigma_R$ are its \emph{instantiated resolved literals}. The \emph{pivot} is the underlying atom of its instantiated resolved literals (i.e. $\abs{\ell_L \sigma_L}$ or, equivalently, $\abs{\ell_R \sigma_R}$).
\end{definition}

\begin{definition}[Factoring] \label{def:fofact} \hfill
%An instance of first-order factoring applies a unifier to the literals in a single premise's conclusion, as below.
\begin{prooftree}
\AxiomC{$\eta_1$: $\Gamma' \cup \{\ell_1,\ldots,\ell_n\}$ }
\UnaryInfC{$\psi$: $\Gamma'\sigma \cup \{\ell\}$}
\end{prooftree}
where $\sigma$ is a unifier of $\{\ell_1,\ldots,\ell_n\}$ and $\ell=\ell_i\sigma$ for any $i\in \{1,\ldots,n\}$.
\end{definition} %UITP

A \emph{resolution proof} is a directed acyclic graph of clauses where the edges correspond to the inference rules of resolution and factoring, as explained in detail in Definition \ref{def:proof}. A \emph{resolution refutation} is a resolution proof with root $\bot$.

\begin{definition}[First-Order Resolution Proof] 
\label{def:proof}% \hfill \\
A directed acyclic graph $\langle V, E, \clause \rangle$, where $V$ is a set of nodes and $E$ is a
set of edges labeled by literals and substitutions (i.e. $E \subset V \times 2^{\mathcal{L}} \times \mathcal{S} \times V$, where $\mathcal{L}$ is the set of all literals and $\mathcal{S}$ is the set of all substitutions, and $\n_1
\xrightarrow[\sigma]{\ell} \n_2$ denotes an edge from node $\n_1$ to node $\n_2$ labeled by the literal $\ell$ and the substitution $\sigma$), is a
proof of a clause $\clause$ iff it is inductively constructible according to the following cases:
%
\begin{itemize}
  \item \textbf{Axiom:} If $\Gamma$ is a clause, $\axiom{\Gamma}$ denotes some proof $\langle \{ \n \}, \varnothing,
    \Gamma \rangle$, where $\n$ is a new node. % (axiom) node.
  \item \textbf{Resolution\footnote{This is referred to as ``binary resolution'' elsewhere, with the understanding that ``binary'' refers to the number of resolved literals, rather than the number of premises of the inference rule.}:} If $\psi_L$ is a proof $\langle V_L, E_L, \clause_L \rangle$ and
    $\psi_R$ is a proof $\langle V_R, E_R, \clause_R \rangle$, $\sigma_L$ and $\sigma_R$ are substitutions s.t. $\ell_L\sigma_L=\dual{\ell_R}\sigma_R$,
    %and $\sigma_L$ and $\sigma_R$ are substitutions such that
    %$\ell_L \sigma_L = \dual{\ell_R} \sigma_R$ %and
    %$\freevar{\left( \clause_L \setminus \left\{ \ell_L \right\} \right) \sigma_L} \cap 
     %\freevar{\left( \clause_R
     %               \setminus \left\{ \ell_R \right\} \right) \sigma_R} = \emptyset$, 
    then
    $\psi_L \res{\ell_L}{\sigma_L}{\ell_R}{\sigma_R} \psi_R$ denotes a proof $\langle V, E, \Gamma \rangle$ s.t.
%\begin{align*}
%     V  &= V_L \cup V_R \cup \{\n \},  \hspace*{1cm}\Gamma = \clause_L' \sigma_L \cup  \clause_R' \sigma_R, \\
%     E &= E_L \cup E_R \cup  \left\{ \raiz{\psi_L} \xrightarrow[\sigma_L]{\{\ell_L\} } \n,   \raiz{\psi_R} \xrightarrow[\sigma_R]{\{\ell_R\} } \n \right\},
%\end{align*}
\begin{align*}
     V  &= V_L \cup V_R \cup \{\n \}, ~\Gamma = \clause_L' \sigma_L \cup  \clause_R' \sigma_R,~
     E = E_L \cup E_R \cup  \left\{ \raiz{\psi_L} \xrightarrow[\sigma_L]{\{\ell_L\} } \n,   \raiz{\psi_R} \xrightarrow[\sigma_R]{\{\ell_R\} } \n \right\},
\end{align*}
%UITP
%    \begin{align*}
%     \hspace{-0.6cm} V &= V_L \cup V_R \cup \{\n \}    \\
%      \hspace{-0.6cm} E &= E_L \cup E_R \cup 
%                    \left\{ \raiz{\psi_L} \xrightarrow[\sigma_L]{\{\ell_L\} } \n, 
%                            \raiz{\psi_R} \xrightarrow[\sigma_R]{\{\ell_R\} } \n \right\}    \\
%    \hspace{-0.6cm}  \Gamma &= \clause_L' \sigma_L \cup  \clause_R' \sigma_R
%    \end{align*}
    where $\n$ is a new (resolution) node and $\raiz{\varphi}$ denotes the root node of $\varphi$.  The literals $\ell_L$ and $\ell_R$ are \emph{resolved literals}, whereas $\ell_L \sigma_L$ and $\ell_R \sigma_R$ are its \emph{instantiated resolved literals}. The \emph{pivot} is the underlying atom of its instantiated resolved literals (i.e. $\abs{\ell_L \sigma_L}$ or, equivalently, $\abs{\ell_R \sigma_R}$).
  \item \textbf{Factoring:}
  %\footnote{This is often called ``Factoring'', but we prefer ``contraction'', because it is essentially the contraction rule of sequent calculus generalized with unification.} 
  If $\psi'$ is a proof $\langle V', E', \clause' \rangle$, $\sigma$ is a unifier of $\{\ell_1,\ldots,\ell_n\}$, and $\ell=\ell_i\sigma$ for any $i\in \{1,\ldots,n\}$, then $\con{\psi}{\{\ell_1, \ldots \ell_n\}}{\sigma}$ denotes a proof $\langle V, E, \Gamma \rangle$ s.t.
  %UITP
    \begin{align*}
         \hspace{-0.6cm} V &= V' \cup \{\n \},  \hspace*{1cm} \Gamma = \clause' \sigma \cup \{ \ell \}, \hspace*{1cm}  E = E' \cup \{ \raiz{\psi'} \xrightarrow[\sigma]{\{\ell_1, \ldots \ell_n\}} \n \},
%         \hspace{-0.6cm} E &= E' \cup \{ \raiz{\psi'} \xrightarrow[\sigma]{\{\ell_1, \ldots \ell_n\}} \n \} 
    \end{align*}  
%    \begin{align*}
%         \hspace{-0.6cm} V &= V' \cup \{\n \} \\
%         \hspace{-0.6cm} E &= E' \cup \{ \raiz{\psi'} \xrightarrow[\sigma]{\{\ell_1, \ldots \ell_n\}} \n \} \\
%       \hspace{-0.6cm} \Gamma &= \clause' \sigma \cup \{ \ell \}
%    \end{align*}
    where $\n$ is a new (factoring) node, and $\raiz{\varphi}$ denotes the root node of $\varphi$.
  \qed
\end{itemize}
\end{definition}





%UITP TODO: include resolution example?




%\noindent
%The resolution and contraction (factoring) rules described above are the standard rules of the resolution calculus, except for the fact that we do not require resolution to use most general unifiers. The presentation of the resolution rule here uses two substitutions, in order to explicitly handle the necessary renaming of variables, which is often left implicit in other presentations of resolution.
%\noindent
%When we write $\psi_L \res{\ell_L}{}{\ell_R}{} \psi_R$, we assume that the omitted substitutions are such that the resolved atom is most general. 
%We write $\con{\psi}{}{}$ for an arbitrary maximal contraction, and $\con{\psi}{}{\sigma}$ for a (pseudo-)contraction that does merge no literals but merely applies the substitution $\sigma$. 
%When the literals and substitutions are irrelevant or clear from the context, we may write simply $\psi_L \res{}{}{}{} \psi_R$. % instead of $\psi_L \res{\ell_L}{\sigma_L}{\ell_R}{\sigma_R} \psi_R$.
%The $\res{}{}{}{}$ operator is assumed to be left-associative. 
%In the propositional case, we omit contractions (treating clauses as sets instead of multisets) and $\psi_L \res{\ell}{\emptyset}{\dual{\ell}}{\emptyset} \psi_R$ is abbreviated by $\psi_L \odot_{\ell} \psi_R$.

%If $\psi = \varphi_L \odot \varphi_R$ or $\psi = \con{\varphi}{}{}$, then $\varphi$, $\varphi_L$ and $\varphi_R$ are \emph{direct subproofs} of $\psi$ and $\psi$ is a \emph{child} of both $\varphi_L$ and $\varphi_R$. The
%transitive closure of the direct subproof relation is the \emph{subproof} relation. A subproof which has no direct subproof is an \emph{axiom} of the proof.
%
%$\Vertices{\psi}$, $\Edges{\psi}$ and $\Conclusion{\psi}$
%denote, respectively, the nodes, edges and proved clause (conclusion) of $\psi$. If $\psi$ is a proof ending with a resolution node, then $\psi_L$ and $\psi_R$ denote, respectively, the left and right premises of $\psi$.

%\section{The Propositional Algorithm}

{\RPI} (formally defined in Appendix \ref{Section:RPI} and in \cite{LURPI}) removes \emph{irregularities}, which are resolution inferences with a node $\eta$ when the resolved literal occurs as the pivot of another inference located below in the path from $\eta$ to the root of the proof. In the worst case, regular resolution proofs can be exponentially bigger than irregular ones, but {\RPI} takes care of regularizing the proof only partially, removing inferences only when this does not enlarge the proof.

%ToDo: Informal textual description of the propositional algorithm, explaining what safe literals are. 
{\RPI} traverses the proof twice. On the first traversal (bottom-up), it computes and stores for each node a set of \emph{safe literals}: literals that are resolved in all paths from the node to the root of the proof or that occur in the root clause. If one of the node's resolved literals belongs to the set of safe literals, then it is possible to \emph{regularize} the node by replacing it by the parent containing the safe literal. To do this replacement efficiently, the replacement is postponed by marking the other parent as a \texttt{deletedNode}. Then, on a single second traversal (top-down), regularization is performed: any node that has a parent node marked as a \texttt{deletedNode} is replaced by its other parent.
%Refer reader to the CADE 2011 paper (where RPI is described) for a formal description of the propositional algorithm. 
% contains a formal description of {\RPI} (taken from \cite{LURPI}).
%Consider adding the formal description to an appendix in this paper, for the convenience of the reviewer.

The {\RPI} and the {\RP} algorithms differ from each other mainly in the
computation of the safe literals of a node that has many children. While the former 
returns the intersection as shown in Algorithm~\ref{algo:SetSafeLiterals}, the latter
returns the empty set. 
Moreover, while in {\RPI} the safe literals of the root node contain all the literals of the root clause, in {\RP} the root node is always assigned an empty set of literals. 
\section{The Propositional LowerUnits Algorithm}
\label{sec:PropositionalLU}

We denote by $\dn{\psi}{\varphi_1, \varphi_2}$ the result of deleting the subproofs $\varphi_1$ and $\varphi_2$ from the proof $\psi$ and fixing it according to Algorithm \ref{algo:del}\footnote{
  The deletion algorithm is a minor variant of the \textsc{Reconstruct-Proof} algorithm presented in \cite{RP11}.
  The basic idea is to traverse the proof in a top-down manner, replacing
  each subproof having one of its premises marked for deletion (i.e. in $D$) by its other direct subproof. For more details, we refer to \ref{Boudou}.
}. 
We say that a subproof $\varphi$ in a proof $\psi$ can be lowered 
if there exists a proof
$\psi'$ such that $\psi' = \dn{\psi}{\varphi} \odot \varphi$ and
$\Conclusion{\psi'} \subseteq \Conclusion{\psi}$. If $\varphi$ originally participated in many resolution inferences within $\psi$ (i.e. if $\varphi$ had many children in $\psi$) then lowering $\varphi$ compresses the proof (in number of resolution inferences), because $\dn{\psi}{\varphi} \odot \varphi$ contains a single resolution inference involving $\varphi$.

%
It has been noted in \cite{LURPI} that, in the propositional case, $\varphi$ can always be lowered if it is a \emph{unit} (i.e. its conclusion clause is unit). This led to the invention of {\LowerUnits} (Algorithm \ref{algo:LU}), which aims at transforming a proof $\psi$ into $(\dn{\psi}{\mu_1,\ldots,\mu_n}) \odot_{\ell_1} \mu_1 \odot \ldots \odot_{\ell_n} \mu_n$, where $(\mu_1,\ldots,\mu_n)$ are all units with more than one child. Units with only one child are ignored merely because no compression is gained by lowering them. The order in which the units are reintroduced is important:
if a unit $\varphi_2$ is a subproof of a unit
$\varphi_1$ then $\varphi_2$ has to be reintroduced later than (i.e. below) $\varphi_1$.



\SetKwFunction{Rec}{delete}
\SetKw{Let}{let}

\begin{algorithm}[bt]
  \KwIn{a proof $\varphi$}
  \KwIn{$D$ a set of subproofs}
  \KwOut{a proof $\varphi'$ obtained by deleting the subproofs in $D$ from $\varphi$}
  \BlankLine

  \newcommand{\fixL}{\ensuremath{\varphi'_L}}
  \newcommand{\fixR}{\ensuremath{\varphi'_R}}

  \lIf{$\varphi \in D$ or $\raiz{\varphi}$ has no premises}{\Return{$\varphi$}}
  \BlankLine

  \Else{
    \Let{$\varphi_L$ and $\varphi_R$} be such that
      $\varphi = \varphi_L \res{\ell_L}{\sigma_L}{\ell_R}{\sigma_R} \varphi_R$ \;
    \Let{$\varphi'_L = $ \Rec{$\varphi_L$,$D$}} \;
    \Let{$\varphi'_R = $ \Rec{$\varphi_R$,$D$}} \;
    \BlankLine

    \lIf{$\varphi'_L \in D$}{ \Return{\fixR} }
    \lElseIf{$\varphi'_R \in D$}{ \Return{\fixL} }
    \BlankLine

    \lElseIf{$\dual{\ell} \notin \Conclusion{\fixL}$}{ \Return{\fixL} }
    \lElseIf{$\ell \notin \Conclusion{\fixR}$}{ \Return{\fixR} }
    \BlankLine

    \lElse{ \Return{ \fixL~$\res{\ell_L}{\sigma_L}{\ell_R}{\sigma_R}$~\fixR} }
  }

  \caption[.]{\FuncSty{delete}}
  \label{algo:del}
\end{algorithm}





A possible presentation of {\LowerUnits} is shown in Algorithm \ref{algo:LU}. Units are collected
during a first traversal. As this traversal is bottom-up, units are stored in a queue. The traversal
could have been top-down and units stored in a stack. Units are effectively deleted during a second,
top-down traversal. The last for-loop performs the reintroduction of units.

\begin{algorithm}[bt]
  \KwIn {a proof $\psi$}
  \KwOut{a compressed proof $\psi'$}
  \BlankLine

  \SetKwData{Units}{Units}
  \Units $\leftarrow \varnothing$ \;
  \BlankLine

  \For{every subproof $\varphi$ in a bottom-up traversal}{
    \If{$\varphi$ is a unit and has more than one child}{Enqueue $\varphi$ in \Units \; }
  }
  \BlankLine

  $\psi' \leftarrow $ \Rec{$\psi$,$\Units$} \;
  \BlankLine

  \For{every unit $\varphi$ in \Units}{
    \Let{$\{\ell\} = \Conclusion{\varphi}$} \;
    \lIf{$\dual{\ell} \in \Conclusion{\psi'}$}{
    $\psi' \leftarrow \psi' \odot_\ell \varphi$}
  }

  \caption{\LowerUnits}
  \label{algo:LU}
\end{algorithm}

\section{First-Order Challenges for Lowering Units}\label{sec:LUChallenges}


In this section, we describe challenges that have to be overcome in order to successfully adapt {\LowerUnits} to the first-order case. The first example illustrates the need to take unification into account. The other two examples discuss complex issues that can arise when unification is naively taken into account.


\begin{example}\label{ex:uniflu} 
Consider the following proof $\psi$, noting that the unit subproof $\eta_2$ is used twice. It is resolved once with $\eta_1$ (against the literal $p(W)$ and producing the child $\eta_3$) and once with $\eta_5$ (against the literal $p(X)$ and producing $\psi$).

\begin{footnotesize}
\begin{prooftree}
\def\e{\mbox{\ $\vdash$\ }}
\AxiomC{$\eta_1$: $p(W)$\e$q(Z)$}
\AxiomC{$\eta_2$: \e$p(Y)$}
\BinaryInfC{$\eta_3$: \e$q(Z)$}
\AxiomC{$\eta_4$: $p(X),q(Z)$\e}
\BinaryInfC{$\eta_5$: $p(X)$\e}
\AxiomC{$\eta_2$}
\BinaryInfC{$\psi$: $\bot$}
\end{prooftree}
\end{footnotesize}

\noindent
The result of deleting $\eta_2$ from $\psi$ is the proof $\dn{\psi}{\eta_2}$ shown below:

\begin{footnotesize}
\begin{prooftree}
\def\e{\mbox{\ $\vdash$\ }}
\AxiomC{$\eta'_1$: $p(W)$\e$q(Z)$}
\AxiomC{$\eta'_4$: $p(X),q(Z)$\e}
\BinaryInfC{$\eta'_5$ ($\psi'$): $p(W), p(X)$\e}
\end{prooftree}
\end{footnotesize}

\noindent
Unlike in the propositional case, where the literals that had been resolved against the unit are all syntactically equal, in the first-order case, this is not necessarily the case. As illustrated above, $p(W)$ and $p(X)$ are not syntactically equal. Nevertheless, they are unifiable. Therefore, in order to reintroduce $\eta'_2$, we may first perform a contraction, as shown below:
\begin{footnotesize}
\begin{prooftree}
\def\e{\mbox{\ $\vdash$\ }}
\AxiomC{$\eta_1'$: $p(W)$\e$q(Z)$}
\AxiomC{$\eta_4'$: $p(X),q(Z)$\e}
\BinaryInfC{$\eta_5'$: $p(X),p(Y)$\e}
\UnaryInfC{$\con{\eta_5'}{}{}$: $p(U)$\e}
\AxiomC{$\eta_2'$: \e$p(Y)$}
\BinaryInfC{$\psi^{\star}$: $\bot$}
\end{prooftree}
\end{footnotesize}
 \end{example}


 \begin{example}\label{ex:pairwise}

There are cases, as shown below, when the literals that had been resolved away are not unifiable, and then a contraction is not possible.

\begin{footnotesize}
\begin{prooftree}
\def\e{\mbox{\ $\vdash$\ }}
\AxiomC{$\eta_2$}
\AxiomC{$\eta_4$: $r(X),p(b)$\e $s(Y)$}
\AxiomC{$\eta_1$: $p(a)$\e$q(Y),r(Z)$}
\AxiomC{$\eta_2$: \e $p(X)$}
\BinaryInfC{$\eta_3$: \e$q(Y),r(Z)$}
\BinaryInfC{$\eta_5$: $p(b)$\e $s(Y),q(Y)$}
\AxiomC{$\eta_6$: $s(Y)$\e}
\insertBetweenHyps{\hskip -0.5in}
\BinaryInfC{$\eta_7$: $p(b)$\e$q(Y)$}
\AxiomC{$\eta_8$: $q(Y)$\e}
\insertBetweenHyps{\hskip -0.5in}
\BinaryInfC{$\eta_9$: $p(b)$\e}
\insertBetweenHyps{\hskip -0.8in}
\BinaryInfC{$\psi$: $\bot$}
\end{prooftree}
\end{footnotesize}

\noindent
If we attempted to postpone the resolution inferences involving the unit $\eta_2$ (i.e. by deleting $\eta_2$ and reintroducing it with a single resolution inference in the bottom of the proof), a contraction of the literals $p(a)$ and $p(b)$ would be needed. 
Since these literals are not unifiable, the contraction is not possible. 
Note that, in principle, we could still lower $\eta_2$ if we resolved it not only once but twice when reintroducing it in the bottom of the proof.
However, this would lead to no compression of the proof's length.
\end{example}

\noindent
The observations above lead to the idea of requiring units to satisfy the following property before collecting them to be lowered.

\begin{definition}
\label{prop:pair}
Let $\eta$ be a unit with literal $\ell$ and let $\eta_1$, \ldots, $\eta_n$ be subproofs that are resolved with $\eta$ in a proof $\psi$, respectively, with resolved literals $\ell_1$, \ldots, $\ell_n$. 
$\eta$ is said to satisfy the \emph{pre-deletion unifiability property} in $\psi$ if $\ell_1$,\ldots,$\ell_n$, and $\dual{\ell}$ are unifiable.
\end{definition}

\begin{example}\label{ex:rootpair}
Satisfaction of the pre-deletion unifiability property is not enough. Deletion of the units from a proof $\psi$ may actually change the literals that had been resolved away by the units, because fewer substitutions are applied to them. This is exemplified below:

\begin{footnotesize}
\begin{prooftree}
\def\e{\mbox{\ $\vdash$\ }}
\AxiomC{$\eta_1$: $r(Y),p(X, q(Y, b)), p(X, Y)$\e}
\AxiomC{$\eta_2$: \e $p(U, V)$}
\BinaryInfC{$\eta_3$: $r(V),p(U, q(V, b))$\e}
\AxiomC{$\eta_4$: \e $r(W)$}
\BinaryInfC{$\eta_5$: $p(U, q(W, b))$\e}
\AxiomC{$\eta_2$}
\BinaryInfC{$\psi$: $\bot$}
\end{prooftree}
\end{footnotesize}

\noindent
If $\eta_2$ is collected for lowering and deleted from $\psi$, we obtain the proof $\dn{\psi}{\eta_2}$:

\begin{footnotesize}
\begin{prooftree}
\def\e{\mbox{\ $\vdash$\ }}
\AxiomC{$\eta'_1$: $r(Y),p(X, q(Y, b)), p(X, Y)$\e}
\AxiomC{$\eta'_4$: \e $r(W)$}
\BinaryInfC{$\eta'_5 (\psi')$: $p(X, q(W, b)), p(X, W)$\e}
\end{prooftree}
\end{footnotesize}

\noindent
Note that, even though $\eta_2$ satisfies the pre-deletion unifiability property (since $p(X, q(Y, b))$ and $p(U, q(W, b))$ are unifiable), $\eta_2$ still cannot be lowered and reintroduced by a single resolution inference, because the corresponding modified post-deletion literals $p(X, q(W, b))$ and $p(X, W)$ are actually not unifiable.
\end{example}

The observation above leads to the following stronger property:

\begin{definition}
\label{prop:rootpair}
Let $\eta$ be a unit with literal $\ell_{\eta}$ and let $\eta_1$, \ldots, $\eta_n$ be subproofs that are resolved with $\eta$ in a proof $\psi$, respectively, with resolved literals $\ell_1$, \ldots, $\ell_m$. 
$\eta$ is said to satisfy the \emph{post-deletion unifiability property} in $\psi$ if $\ell_1^{\dagger\downarrow}$,\ldots,$\ell_m^{\dagger\downarrow}$, and $\dual{\ell_{\eta}^{\dagger}}$ are unifiable, where $\ell^{\dagger}$ is the literal in $\dn{\psi}{\eta}$ corresponding to $\ell$ in $\psi$ and $\ell_k^{\dagger\downarrow}$ is the descendant of $\ell_k^{\dagger}$ in the root of $\dn{\psi}{\eta}$.
\end{definition}



\section{First-Order LowerUnits} \label{sec:FOLU}


\SetKwFunction{Rec}{delete}
\SetKw{Let}{let}

\begin{algorithm}[bt]
  \KwIn{a proof $\varphi$}
  \KwIn{$D$ a set of subproofs}
  \KwOut{a proof $\varphi'$ obtained by deleting the subproofs in $D$ from $\varphi$}
  \BlankLine

  \newcommand{\fixL}{\ensuremath{\varphi'_L}}
  \newcommand{\fixR}{\ensuremath{\varphi'_R}}

  \lIf{$\varphi \in D$ or $\raiz{\varphi}$ has no premises}{\Return{$\varphi$}}
  \BlankLine

  \Else{
    \Let{$\varphi_L$ and $\varphi_R$} be such that
      $\varphi = \varphi_L \res{\ell_L}{\sigma_L}{\ell_R}{\sigma_R} \varphi_R$ \;
    \Let{$\varphi'_L = $ \Rec{$\varphi_L$,$D$}} \;
    \Let{$\varphi'_R = $ \Rec{$\varphi_R$,$D$}} \;
    \BlankLine

    \lIf{$\varphi'_L \in D$}{ \Return{\fixR} }
    \lElseIf{$\varphi'_R \in D$}{ \Return{\fixL} }
    \BlankLine

    \lElseIf{$\ell \notin \Conclusion{\fixL}$}{ \Return{\fixL} }
    \lElseIf{$\dual{\ell} \notin \Conclusion{\fixR}$}{ \Return{\fixR} }
    \BlankLine

    \lElse{ \Return{ \fixL~$\res{\ell_L}{\sigma_L}{\ell_R}{\sigma_R}$~\fixR} }
  }

  \caption[.]{\FuncSty{fo-delete}}
  \label{algo:fodel}
\end{algorithm}



\begin{proposition} \label{prop:LUniv}
Given a proof $\psi$, if 
%for an integer $n$
there is a sequence $U = (\varphi_1 \ldots \varphi_n)$
of $\psi$'s subproofs and a sequence $(\ell_1 \ldots \ell_n)$ of literals such that $\forall i \in
[1 \ldots n]$, $\ell_i$ is the univalent literal of $\varphi_i$ w.r.t. $\Delta_{i-1} =
\{\dual{\ell_1} \ldots \dual{\ell_{i-1}}\}$, then the conclusion of $$ \psi' = \dn{\psi}{U}
\odot_{\ell_n} \varphi_n \ldots \odot_{\ell_1} \varphi_1 $$ subsumes the conclusion of $\psi$.
\end{proposition}

\begin{proof}
The proposition is proven by induction on $n$, along with the fact that $\dn{\psi}{U} \notin U$.
For $n = 0$, $U = \varnothing$ and the properties trivially hold. Suppose a subproof
$\varphi_{n+1}$ of $\psi$ is univalent w.r.t. $\Delta_n$, with univalent literal $\ell_{n+1}$.
Because $\ell_{n+1} \notin \Delta_n$, there exists a subproof of $\dn{\psi}{U}$ with conclusion
containing $\dual{\ell_{n+1}}$, and therefore $\dn{\dn{\psi}{U}}{\varphi_{n+1}} \notin U \cup
\{\varphi_{n+1}\}$.  Let $\Gamma$ be the conclusion of $\dn{\psi}{U}$. The conclusion of $ \psi' =
\dn{\psi}{U \cup \{\varphi_{n+1}\}} = \dn{\dn{\psi}{U}}{\varphi_{n+1}} $ is included in $\Gamma \cup
\{\dual{\ell_{n+1}}\}$. The conclusion of $\psi' \odot_{\ell_{n+1}} \varphi_{n+1}$ is included in
$\Gamma \cup \Delta_n$. As $\Gamma \subseteq \Conclusion{\psi} \cup \Delta_n$, the conclusion of
$\psi' \odot_{\ell_{n+1}} \varphi_{n+1} \ldots \odot_{\ell_1} \varphi_1$ is included in
$\Conclusion{\psi}$. \qed
\end{proof}




% \begin{algorithm}[bt]
%   \KwIn {a proof $\psi$}
%   \KwOut{a compressed proof $\psi'$}
%   \BlankLine

%   \SetKw{Push}{push}
%   \SetKw{Pop} {pop}

%   \Units $\leftarrow \varnothing$ \;
%   $\Delta \leftarrow \varnothing$ \;
%   \BlankLine

%   \For{every subproof $\varphi$, in a top-down traversal \label{line:LUniv:step1begin} }{
%     $\psi' \leftarrow$ \Rec{$\varphi$,\Univ} \label{line:LUniv:delete} \;
%     \If{$\psi'$ is univalent w.r.t. $\Delta$ \label{line:LUniv:lunivtest} }{
%       \Let{$\ell$} be the univalent literal \;
%       \Push $\dual{\ell}$ onto $\Delta$ \label{line:LUniv:pushDelta} \;
%       \Push $\psi'$     onto \Univ \label{line:LUniv:step1end} \;
%     }
%   }
%   \BlankLine

%   \tcp{At this point, $\psi' = \dn{\psi}{\Univ}$}
%   \While{\Univ $\neq \varnothing$}{ \label{line:LUniv:reintroducebegin}
%     $\varphi \leftarrow$ \Pop from \Univ \;
%     $\ell \leftarrow$ \Pop from $\Delta$ \;
%     \lIf{$\ell \in \Conclusion{\psi'}$ \label{line:LUniv:testreintroduce} }{
%     $\psi' \leftarrow \varphi \odot_\ell \psi'$ \;}
%   }

%   \caption{Simplified \LowerUnivalents}
%   \label{algo:LUniv}
% \end{algorithm}


\begin{figure}[htb]
  \centering
  \subfloat[Original proof]{
    \centering
    \begin{tikzpicture}

      \rootnode;
      \withchildren{root} {r0}{\dual{a}}  {unit}{a};
      \withchildren{r0}   {r1}{\dual{a},c} {r2}{\dual{a},\dual{c}};
      \withchildren{r1}   {a0}{\dual{b},c} {low}{\dual{a},b};

      \proofnode[above right of=r2] {a1} {\dual{a},\dual{b},\dual{c}};
      \drawchildren {r2} {low} {a1};

    \end{tikzpicture}
  } \qquad
  \centering
  \subfloat[Compressed proof]{
    \centering
    \begin{tikzpicture}

      \rootnode;
      \withchildren{root} {r0}{\dual{a}}          {unit}{a};
      \withchildren{r0}   {r1}{\dual{a},\dual{b}} {low}{\dual{a},b};
      \withchildren{r1}   {a0}{\dual{b},c}        {a1}{\dual{a},\dual{b},\dual{c}};

    \end{tikzpicture}
  }
\caption{Example of proof compression by \LowerUnivalents} 
\label{fig:exluniv}
\end{figure}






\section{A Simpler First-Order LowerUnits}
\label{sec:SimpleFOLU}

%Recall example \ref{ex:ambig}. In order to avoid this, we introduce a proof rule that applies a substitution. So that we would get the following proof

\begin{tiny}
\begin{prooftree}
\def\e{\mbox{\ $\vdash$\ }}
\AxiomC{$\eta_1$: $p(U),r(U~V),r(V~U),q(V)$\e}
\UnaryInfC{$\eta_2$: $p(c),r(c~V),r(V~c),q(V)$\e}
\AxiomC{$\eta_3$: \e$r(X~c)$}
\BinaryInfC{$\eta_4$: $p(c),r(c~X),q(X)$\e}
\AxiomC{$\eta_5$: \e$r(W~V)$}
\BinaryInfC{$\eta_6$: $p(c),q(V)$\e}
\AxiomC{$\eta_7$: $p(Z)$\e$q(d)$}
\BinaryInfC{$\eta_8$: $p(c),p(Z)$\e}
\UnaryInfC{$\eta_9$: $p(c)$\e}
\AxiomC{$\eta_{10}$: \e$p(c)$}
\BinaryInfC{$\psi$: $\bot$}
\end{prooftree}
\end{tiny}

Now $r(V, c)$ appears in the first left resolvent, which was the left aux formula in the original proof. Thus, the implementation can find that formula, and choose it in order to resolve the ambiguous resolution, instead of guessing a formula from the left resolvent that unifies with the right resolvent, which might go wrong.\\

TODO: Explain where the sub came from.\\

TODO: define the rule formally here?\\

TODO: describe when the rule is invoked in the implementation\\

A simple way to decrease the complexity to linear with respect to the length of the proof is to return to the ideas used in the propositional case. In particular, by performing a traversal to collect the units of a proof, and then optimistically deleting units, some compression can often be achieved. By ignoring whether or not a unit satisfies Property \ref{prop:rootpair}, we can attempt to lower it, and should compression fail because deletions changed the substitutions to the point where contraction was not possible, we simply return the original proof. 

\begin{algorithm}[bt]
  \SetAlgoVlined
  \SetAlgoShortEnd
\SetKwFunction{check}{check}
  \KwIn {a proof $\psi$}
  \KwOut{a compressed proof $\psi^{\star}$}
  \KwData{a map $.'$: after line 4, it maps any $\varphi$ to \Del{$\varphi$, $D$}}
  \BlankLine

  \SetKwData{Units}{Units}

  \SetKw{Remove} {remove}
  \SetKw{Break} {break}

  \algolines{\Units $\leftarrow \varnothing$}{queue to store collected units}
  \BlankLine

  \For{every subproof $\varphi$, in a bottom-up traversal of $\psi$}{
    \lIf{$\varphi$ is a unit with more than one child and all literals of $\varphi$ are simultaneously unifiable}{enqueue $\varphi$ in \Units}
  }
  \BlankLine

    $\psi' \leftarrow $ \FuncSty{simple-fo-delete}$(\psi,\Units)$ \;
    \BlankLine

    \tcp{Reintroduce units}
    

    $\psi^{\star} \leftarrow \psi'$ \;
    \For{every unit $\varphi$ in \Units}{
        \Let{$\sigma$ be the unifier of $\rho(\psi^\star)$'s literals that contracts $\rho(\psi^\star)$ as much as possible} \;
        \Let{$c$ be the literals contracted by $\sigma$} \;
       \If{$ \con{\psi^{\star}}{c}{\sigma}$ and $\varphi'$ can be resolved} {
        $\psi^{\star} \leftarrow \con{\psi^{\star}}{c}{\sigma} \res{\ell^c}{}{\ell}{} \varphi'$ \;
        }\Else { \Return $\psi$}
      
    }
  
    

  \caption{\SFOLowerUnits}
  \label{algo:simpleFOLU}
\end{algorithm}

Algorithm \ref{algo:simpleFOLU} works similarly to the propositional algorithm.  It first performs a bottom up traversal to collect potential units and the literals that are resolved away from by those units, adding the units to a queue (line 1). As seen in Examples \ref{ex:pairwise} in Section \ref{sec:Challenges}, unification of the resolved away literals is necessary, so it performs a check to make sure these literals satisfy Property \ref{prop:pair} (line 4). If it succeeds, it attempts to re-introduce all the removed units at the bottom of the proof, where it attempts to compress the literals that would be resolved away by each unit (lines 6-15). Note that this requires the implementation to track which literals should be resolved against each unit. In order to avoid traversing the proof to find these again after the deletion of every potential unit (as is done in Algorithm \ref{algo:FOLU}), we use a modified \FuncSty{delete} function, called \FuncSty{simple-fo-delete}, which is the same as Algorithm \ref{algo:del} except with line 6 changed to the following:

   \lIf{$~\varphi'_L \in D~$}{ 
     \Return{$(\rho(\varphi'_L) \sigma_R)$} 
    }
    \lElseIf{$\varphi'_R \in D$}{ 
      \Return{$(\rho(\varphi'_R) \sigma_R)$}  
    }

\FuncSty{simple-fo-delete} is designed to reduce the complexity of tracking literals. \FuncSty{simple-fo-delete} behaves much more closely to the propositional case and requires none of the additional data structures required by \FuncSty{fo-delete}. In this function, when a unit node is returned, instead of returning the opposite node (respectively $\psi_L'$ or $\psi_R'$, line 6) in the resolution (which is done in the propositional case), or tracking the literals (which is done in \FuncSty{fo-delete}), we return the opposite node with $\sigma_L$ (respectively $\sigma_R$) applied to it. In this way, the literals not resolved with the unit will look like they would have in the original proof, and the literal which was not resolved due to the deletion looks like it is syntactically equal with the unit literal at this stage. The fact that the other literals look like they did in the original proof is key: now resolution in the compressed proof can use the old literals, which should appear as they before, and not worry about choosing the wrong literal in case of ambiguous resolution.

%A negative side-effect of this is that we may end up grounding literals, and having to carry these forms of each literal forward, which may increase the character length of the clause, though not the number of nodes in the proof.

Additionally, by modifying delete in this manner we can longer guarantee that Property \ref{prop:rootpair} is satisfied. Property \ref{prop:rootpair} so the appearance of literals that were to be resolved away from a unit clause may have changed, preventing completion of the proof. If this happens {\SFOLowerUnits} will attempt to re-introduce this node and fail, returning the original input proof (line 12). As a result, some proofs that can be compressed are returned unmodified, but those that do not require this additional property can be compressed much more quickly.



%\begin{algorithm}[bt]
  \SetAlgoVlined
  \SetAlgoShortEnd
  \KwIn{a proof $\varphi$}
  \KwIn{$D$ a set of subproofs}
  \KwOut{a proof $\varphi'$ obtained by deleting the subproofs in $D$ from $\varphi$}
  \BlankLine

  \newcommand{\fixL}{\ensuremath{\varphi'_L}}
  \newcommand{\fixR}{\ensuremath{\varphi'_R}}

  \lIf{$\varphi \in D$ or $\raiz{\varphi}$ has no premises}{\Return{$\varphi$}}
  \BlankLine

  \Else{$\varphi = \varphi_L \res{\ell_L}{\sigma_L}{\ell_R}{\sigma_R} \varphi_R$\;
    $\varphi'_L \leftarrow $ \Rec{$\varphi_L$,$D$} \;
    $\varphi'_R \leftarrow $ \Rec{$\varphi_R$,$D$} \;
    \BlankLine

    \lIf{$\varphi'_L \in D$}{ 
      \Return{$($\fixR $\sigma_R)$} 
    }
    \lElseIf{$\varphi'_R \in D$}{ 
      \Return{$($\fixL $\sigma_R)$}  
    }
    \BlankLine


    \lElse{ 
      \Return{ \fixL~$\res{\ell_L}{}{\ell_R}{}$~\fixR}
    }
  }



  \caption[.]{\FuncSty{simple-fo-delete}}
  \label{algo:sfodel}
\end{algorithm}



\section{Algorithm {\RecyclePivotsIntersection}}
\label{Section:RPI}

\newcommand{\tRes}{\odot}
\newcommand{\tResFact}{\otimes}
\newcommand{\tResChain}{\ominus}
\newcommand{\AXC}{\AxiomC}
\newcommand{\BIC}{\BinaryInfC}
\newcommand{\RName}[1]{\RightLabel{#1}}
\newcommand{\p}[1]{\hat{#1}}
\newcommand{\ub}[2]{\underbrace{#1}_{#2}}
\newcommand{\tResStar}{\circledast}

%\textbf{Note:} for the reviewers' convenience, this appendix summarizes \cite{LURPI}.

%\bigskip

%\noindent
This section explains {\RecyclePivotsIntersection} ({\RPI}) \cite{LURPI}, which aims to compress irregular propositional proofs. It can be seen as a simple 
but significant modification of the {\RP} algorithm described in 
\cite{RP08}, 
from which it derives its name. 
Although in the worst case full regularization can increase the proof length exponentially 
\cite{Tseitin}, these algorithms show that 
many irregular proofs can have their length decreased if a careful partial regularization is performed. 

We write $\psi[\eta]$ to denote a \emph{proof-context}
$\psi[\_]$ with a single placeholder replaced by the subproof $\eta$.
We say that a proof of the form $\psi[\eta \tRes_p \psi'[\eta'\tRes_p\eta_2]]$ is \emph{irregular}.

\begin{example}
%Consider an irregular proof of the form $\psi[ \eta \tRes_p \psi'[\eta' \tRes_p \eta''] ]$, 
Consider an irregular proof and assume, without loss of generality, that $p \in \eta$ and $p \in \eta'$, as in the proof of $\psi$ below. The proof of $\psi$ can be written as $(\eta \tRes_p ( \eta_1 \tRes (\eta' \tRes_p \eta'')))$, or $(\eta \tRes_p \psi'[(\eta' \tRes_p \eta'')])$ where $\psi'[(\eta' \tRes_p \eta'')] = (\eta_1 \tRes (\eta' \tRes_p \eta''))$ is the sub-proof of $\lnot p$.
\begin{footnotesize}
\begin{prooftree}

		\AXC{$\eta$: $p$} \RName{$p$}
			\AXC{$\eta_1$: $\lnot r, \lnot p$}
		\AXC{$\eta'$: $p$}
				\AXC{$ \eta''$: $\lnot p, r$} \RName{$p$}
	\BIC{$r$}

	\BIC{$\lnot p$} \RName{$p$}

		\BIC{$\psi$: $\bot$}	
%		 \DisplayProof 
\end{prooftree}
\label{ex:rpi-example-c}
\end{footnotesize}
\noindent
Then, if $\eta' \tRes_p \eta''$ is replaced by $\eta''$ within the proof-context $\psi'[\ ]$, the clause $\eta \tRes_p \psi'[\eta'']$ subsumes the clause $\eta \tRes_p \psi'[\eta' \tRes_p \eta'']$, because even though the literal $\neg p$ of $\eta''$ is
propagated down, it gets resolved against the literal $p$ of $\eta$ later on below in the proof. More precisely, even though it might be the case that $\neg p \in \psi'[\eta'']$ while $\neg p \notin \psi'[\eta' \tRes_p \eta'']$, it is necessarily the case that $\neg p \notin \eta \tRes_p \psi'[\eta' \tRes_p \eta'']$ and $\neg p \notin \eta \tRes_p \psi'[\eta'']$. In this case, the proof can be regularized as follows.


\begin{footnotesize}
\begin{prooftree}

		\AXC{$\eta$: $p$}
			\AXC{$\eta_1$: $\lnot r, \lnot p$}

				\AXC{$ \eta''$: $\lnot p, r$}


	\BIC{$\lnot p$} \RName{$p$}

		\BIC{$\psi$: $\bot$}	

\end{prooftree}
\end{footnotesize}
\end{example}


\begin{figure*}[bt]%
\centering
\subfloat[A propositional proof before compression by {\RPI}.]{%
\begin{footnotesize}
%\begin{prooftree}
\AXC{$ \eta_1 $}
		\AXC{$ \eta_2: a, c, \neg b $}
				\AXC{$ \eta_1: \neg a$}
						\AXC{$ \eta_3:  a, b $} \RName{$a$}
					\BIC{$ \eta_4: b$} \RName{$b$}
			\BIC{$ \eta_5: a, c$}	\RName{$a$}
	\BIC{$\eta_6: c$}
		\AXC{$ \eta_4 $}
				\AXC{$ \eta_7: a, \neg b, \neg c $} \RName{$b$}
			\BIC{$ \eta_8: a, \neg c$}	\RName{$c$}
					\AXC{$ \eta_1 $}  \RName{$a$}
				\BIC{$ \eta_9: \neg c$}	\RName{$c$}
		\BIC{$\psi: \bot$}	
		 \DisplayProof 
%\end{prooftree}
\label{ex:rpi-example-a}
\end{footnotesize}}\\%
\subfloat[A propositional proof after compression by {\RPI}.]{%
\begin{small}
%\begin{prooftree}
\AXC{$ \eta_1: \neg a $}
		\AXC{$ \eta_2: a, c, \neg b $}
						\AXC{$ \eta_3:  a, b $}\RName{$ $}
			\BIC{$ \eta_5: a, c$}	\RName{$ $}
	\BIC{$\eta_6: c$}
		\AXC{$ \eta_3 $}
				\AXC{$ \eta_7: a, \neg c, \neg b $} \RName{$ $}
			\BIC{$ \eta_8: a, \neg c$}	\RName{$ $}
					\AXC{$ \eta_1 $}  \RName{$ $}
				\BIC{$ \eta_9: \neg c$}	\RName{$ $}
		\BIC{$\psi: \bot$}	
		 \DisplayProof 
%\end{prooftree}
\label{ex:rpi-example-b}
\end{small}
}%

\caption{A {\RPI} example.}
\label{ex:rpi-example}
\end{figure*}


Although the remarks above suggest that it is safe to replace $\eta' \tRes_p
\eta''$ by $\eta''$ within the proof-context $\psi'[\ ]$, this is not always the
case. If a node in $\psi'[\ ]$ has a child in $\psi[\ ]$, then the literal $\neg
p$ might be propagated down to the root of the proof, and hence, the clause
$\psi[ \eta \tRes_p \psi'[ \eta''] ]$ might not subsume the clause $\psi[ \eta
\tRes_p \psi'[\eta' \tRes_p \eta''] ]$. Therefore, it is only safe to do the
replacement if the literal $\neg p$ gets resolved in all paths from $\eta''$ to the root or if it already occurs in the root clause of the original proof $\psi[ \eta \tRes_p \psi'[\eta' \tRes_p \eta''] ]$.

\IncMargin{0.5em}
\begin{algorithm}[b]
\begin{footnotesize}
\SetKwInOut{Input}{input}\SetKwInOut{Output}{output}
\SetKwData{units}{unitsQueue}
\SetKwData{fixedUnits}{fixedUnitsQueue}

\Input{A proof $\psi$}
\Output{A possibly less-irregular proof $\psi'$}

\BlankLine

$\psi'$ $\la$ $\psi$\;
traverse $\psi'$ bottom-up and \ForEach{node $\eta$ in $\psi'$}{
   % \uIf{$\eta$ is an input node}{ 
   % 		do nothing \;
   %  }
   %  \ElseIf{$\eta$ is a resolvent node}{
   %   setSafeLiterals($\eta$) \;
   %   regularizeIfPossible($\eta$)
   % }
%% SM: original version: do we really need the "do nothing" branch?
   \If{$\eta$ is a resolvent node}{
     setSafeLiterals($\eta$) \;
     regularizeIfPossible($\eta$)
   }
  }
$\psi'$ $\la$ fix($\psi'$) \;
\Return {$\psi'$}\;
\caption{\label{algo:RPI} \texttt{\RPI}}
\end{footnotesize}
\end{algorithm}
\DecMargin{0.5em}
%\begin{code}
%  function recyclePivotsWithIntersection(p: Proof): Proof = {
%    traverseBottomUp(p)(
%      n => {
%        if (n is Input) doNothing
%        else if (n is Resolvent) {
%          setSafeLiterals(n)
%          regularizeIfPossible(n)
%          for (c in n.children) c.freeMemory
%        } 
%      }
%    )
%    fix(p)  
%  }
%\end{code}

These observations lead to the idea of traversing the proof in a bottom-up
manner, storing for every node a set of \emph{safe literals} that get resolved
in all paths below it in the proof (or that already occurred in the root clause
of the original proof). Moreover, if one of the node's resolved literals belongs
to the set of safe literals, then it is possible to regularize the node by
replacing it by one of its parents (cf.\ Algorithm~\ref{algo:RPI}). 

The regularization of a node should replace a node by one of its parents, and more precisely by the parent whose clause contains the resolved literal that is safe. After regularization, all nodes below the regularized node may have to be fixed. However, since the regularization is done with a bottom-up traversal, and only nodes below the regularized node need to be fixed, it is again possible to postpone fixing and do it with only a single traversal afterwards. 
Therefore, instead of replacing the irregular node by one of its parents immediately, 
its other parent is marked as \texttt{deletedNode}, as shown in Algorithm~\ref{algo:Regularize}. Only later during fixing, 
the irregular node is actually replaced by its surviving parent (i.e. the parent that is not marked as \texttt{deletedNode}).


\IncMargin{0.5em}
\begin{algorithm}[p]
\begin{footnotesize}
\SetKwInOut{Input}{input}\SetKwInOut{Output}{output}
\SetKwData{units}{unitsQueue}
\SetKwData{fixedUnits}{fixedUnitsQueue}

\Input{A node $\eta$}
\Output{nothing (but the proof containing $\eta$ may be changed)}

\BlankLine
    \uIf{$\eta${\upshape.rightResolvedLiteral} $\in$ $\mathcal{S}(\eta)$}{
      mark left parent of $\eta$ as \texttt{deletedNode} \;
      mark $\eta$ as regularized
    }
    \ElseIf{\textrm{$\eta${\upshape.leftResolvedLiteral} $\in$  $\mathcal{S}(\eta)$}}{
      mark right parent of $\eta$ as \texttt{deletedNode} \;
      mark $\eta$ as regularized
    }
\caption{\label{algo:Regularize} \texttt{regularizeIfPossible}}
\end{footnotesize}
\end{algorithm}
\DecMargin{0.5em}

\IncMargin{0.5em}
\begin{algorithm}[p]
\begin{footnotesize}
\SetKwInOut{Input}{input}\SetKwInOut{Output}{output}
\SetKwData{units}{unitsQueue}
\SetKwData{fixedUnits}{fixedUnitsQueue}

\Input{A node $\eta$}
\Output{nothing (but the node $\eta$ gets a set of safe literals)}

\BlankLine

    \uIf{$\eta$ is a root node with no children}{
      $\mathcal{S}(\eta)$ $\la$ $\eta$.clause  
    }
    \Else{
      \ForEach{$\eta'$ $\in$ $\eta${\upshape.children}}{
        \uIf{$\eta'$ is marked as regularized}{ 
          safeLiteralsFrom($\eta'$) $\la$  $\mathcal{S}(\eta')$ \;}
        \uElseIf{$\eta$ is left parent of $\eta'$}{ 
        	safeLiteralsFrom($\eta'$) $\la$  $\mathcal{S}(\eta')$ $\cup$ 
        	\hspace{6cm} $~~~~$ \{ $\eta'$.rightResolvedLiteral \} \;
        }
        \ElseIf{$\eta$ is right parent of $\eta'$}{ 
			safeLiteralsFrom($\eta'$) $\la$ $\mathcal{S}(\eta')$ $\cup$ 
			\hspace{6cm} $~~~~$ \{ $\eta'$.leftResolvedLiteral \} \;
        }
      }
       $\mathcal{S}(\eta)$ $\la$ $\bigcap_{\eta' \in \eta\textrm{.children}}$ safeLiteralsFrom($\eta'$)
    }
\caption{\label{algo:SetSafeLiterals} \texttt{setSafeLiterals}}
\end{footnotesize}
\end{algorithm}
\DecMargin{0.5em}
%\begin{code}
%  function setSafeLiterals(n: Resolvent) = {
%    if (n is a root node with no children) {
%      n.safeLiterals = n.clause  
%    }
%    else {
%      val safeLiteralsPerChild = for (c in n.children) yield {
%        if (c is marked as regularized) c.safeLiterals 
%        else if (c.left == n) c.safeLiterals $+cup$ {c.resolvedLiterals.right}
%        else if (c.right == n) c.safeLiterals $+cup$ {c.resolvedLiterals.left}
%      }
%      n.safeLiterals = intersection(safeLiteralsPerChild)
%    }
%  }
%\end{code}

The set of safe literals of a node $\eta$ can be computed from the set of safe literals of its children (cf.\ Algorithm~\ref{algo:SetSafeLiterals}). 
In the case when $\eta$ has a single child $\varsigma$, the safe literals of $\eta$ are simply the safe literals of $\varsigma$ together with the resolved literal $p$ of $\varsigma$ belonging to $\eta$ ($p$ is safe for $\eta$, because whenever $p$ is propagated down the proof through $\eta$, $p$ gets resolved in $\varsigma$). 
It is important to note, however, that if $\varsigma$ has been marked as regularized, it will eventually be replaced by $\eta$, and hence $p$ should not be added to the safe literals of $\eta$. 
In this case, the safe literals of $\eta$ should be exactly the same as the safe literals of $\varsigma$.
When $\eta$ has several children, the safe literals of $\eta$ w.r.t. a child $\varsigma_i$ contain literals that are safe on all paths that go from $\eta$ through $\varsigma_i$ to the root. 
For a literal to be safe for all paths from $\eta$ to the root, it should therefore be in the intersection of the sets of safe literals w.r.t. each child.

The {\RP} and the {\RPI} algorithms differ from each other mainly in the computation of the safe literals of a node that has many children. 
While {\RPI} returns the intersection as shown in Algorithm~\ref{algo:SetSafeLiterals}, {\RP} returns the empty set (cf. Algorithm~\ref{algo:SetSafeLiteralsRP}). 
Additionally, while in {\RPI} the safe literals of the root node contain all the literals of the root clause, in {\RP} the root node is always assigned an empty set of literals. 
(Of course, this makes a difference only when the proof is not a refutation.)
Note that during a traversal of the proof, 
the lines from 5 to 10 in Algorithm~\ref{algo:SetSafeLiterals} are executed as many times as the number of edges in the proof. 
Since every node has at most two parents, the number of edges is at most twice the number of nodes. 
Therefore, during a traversal of a proof with $n$ nodes, lines from 5 to 10 are executed at most $2n$ times, and the algorithm remains linear.
In our prototype implementation, the sets of safe literals are instances of Scala's \texttt{mutable.HashSet} class. 
Being mutable, new elements can be added efficiently.
And being HashSets, membership checking is done in constant time in the average case,  and set intersection (line 12) can be done in $O(k.s)$, where $k$ is the number of sets and $s$ is the size of the smallest set.

\IncMargin{0.5em}
\begin{algorithm}[p]
\begin{footnotesize}
\SetKwInOut{Input}{input}\SetKwInOut{Output}{output}
\SetKwData{units}{unitsQueue}
\SetKwData{fixedUnits}{fixedUnitsQueue}

\Input{A node $\eta$}
\Output{nothing (but the node $\eta$ gets a set of safe literals)}

\BlankLine

    \uIf{$\eta$ is a root node with no children}{
      $\mathcal{S}(\eta)$ $\la$ $\emptyset$ 
    }
    \Else{
      \uIf{$\eta$ has only one child $\eta'$}{
        \uIf{$\eta'$ is marked as regularized}{ 
          $\mathcal{S}(\eta)$ $\la$  $\mathcal{S}(\eta')$ \;}
        \uElseIf{$\eta$ is left parent of $\eta'$}{ 
        	$\mathcal{S}(\eta)$ $\la$  $\mathcal{S}(\eta')$ $\cup$ 
        	 \{ $\eta'$.rightResolvedLiteral \} \;
        }
        \ElseIf{$\eta$ is right parent of $\eta'$}{ 
			$\mathcal{S}(\eta)$ $\la$  $\mathcal{S}(\eta')$ $\cup$
			 \{ $\eta'$.leftResolvedLiteral \} \;
        }
      }
      \Else{
      	 $\mathcal{S}(\eta)$ $\la$ $\emptyset$
      }
    }
\caption{\label{algo:SetSafeLiteralsRP} \texttt{setSafeLiterals} for \RP}
\end{footnotesize}
\end{algorithm}
\DecMargin{0.5em}
%
%\begin{code}
%  function setSafeLiterals(n: Resolvent) = {
%     n.safeLiterals = 
%       if (n has only one child c) {
%         if (c is marked as regularized) c.safeLiterals 
%         else if (c.left == n) c.safeLiterals $+cup$ {c.resolvedLiterals.left}
%         else if (c.right == n) c.safeLiterals $+cup$ {c.resolvedLiterals.right}
%       }
%       else $+emptyset$
%  }
%\end{code}



%\begin{example}
When applied to the proof $\psi$ shown in Example \ref{Example:Proof}, the algorithm {\RPI} assigns $\{a,c\}$ and $\{a, \neg c\}$ as the safe literals of, respectively, $\eta_5$ and $\eta_8$. The safe literals of $\eta_4$ w.r.t. its children $\eta_5$ and $\eta_8$ are respectively $\{a,c,b\}$ and $\{a, \neg c, b\}$, and hence the safe literals of $\eta_4$ are $\{a,b\}$ (the intersection of $\{a,c,b\}$ and $\{a, \neg c, b\}$). Since the right resolved literal of $\eta_4$ ($a$) belongs to $\eta_4$'s safe literals, $\eta_4$ is correctly detected as a redundant node and hence regularized: $\eta_4$ is replaced by its right parent $\eta_3$. The resulting proof is shown below:

\begin{small}
\begin{prooftree}
\AXC{$ \eta_1: \neg a $}
		\AXC{$ \eta_2: a, c, \neg b $}
						\AXC{$ \eta_3:  a, b $}\RName{$ $}
			\BIC{$ \eta_5: a, c$}	\RName{$ $}
	\BIC{$\eta_6: c$}
		\AXC{$ \eta_3 $}
				\AXC{$ \eta_7: a, \neg c, \neg b $} \RName{$ $}
			\BIC{$ \eta_8: a, \neg c$}	\RName{$ $}
					\AXC{$ \eta_1 $}  \RName{$ $}
				\BIC{$ \eta_9: \neg c$}	\RName{$ $}
		\BIC{$\psi: \bot$}	
\end{prooftree}
\end{small}

%This proof corresponds to the proof term
$$
(\ub{\{\neg a\}}{\eta_1} \tRes (\{a, c, \neg b\} \tRes \ub{\{a, b\}}{\eta_3})) \tRes ((\eta_3 \tRes \{\neg b, \neg c, a\}) \tRes \eta_1)
$$
%
\noindent%
{\RP}, on the other hand, assigns $\emptyset$ as the set of safe literals for $\eta_4$. Therefore, it does not detect that $\eta_4$ is a redundant irregular node, and then $\RP(\varphi) = \varphi$. 
 %
\hfill\QED
\end{example}

\begin{theorem}
\label{Theorem:RPIBetterThanRP}
For any proof $\varphi$, $|\RPI(\varphi)| \leq |\RP(\varphi)|$.
\end{theorem}
\begin{proof}
  For every node $\eta$ in $\varphi$, let $S^{\eta}_{\RPI}$ (resp.,
  $S^{\eta}_{\RP}$) be the set of safe literals for $\eta$ computed by {\RPI} and
  {\RP}. It is easy to see that $S^{\eta}_{\RPI} \supseteq S^{\eta}_{\RP}$ for all
  $\eta$. Therefore, {\RPI} detects and eliminates more redundancies than {\RP}.
%
\hfill\QED
\end{proof}

The better compression of {\RPI} does not come for free, 
as computing an intersection of sets is more costly than assigning the empty set. 
For a node $\eta$ with $k$ children, $k$ sets must be intersected and the size of each set is 
in the worst case in $O(h)$, where $h$ is the length of the shortest path from $\eta$ to a root.

\begin{example} %This example from the original RPI source
\label{Example:Proof}
When applied to the proof $\psi$ shown in Figure \ref{ex:rpi-example-a}, the algorithm {\RPI} assigns $\{a,c\}$ and $\{a, \neg c\}$ as the safe literals of, respectively, $\eta_5$ and $\eta_8$. The safe literals of $\eta_4$ w.r.t. its children $\eta_5$ and $\eta_8$ are respectively $\{a,c,b\}$ and $\{a, \neg c, b\}$, and hence the safe literals of $\eta_4$ are $\{a,b\}$ (the intersection of $\{a,c,b\}$ and $\{a, \neg c, b\}$). Since the right resolved literal of $\eta_4$ ($a$) belongs to $\eta_4$'s safe literals, $\eta_4$ is correctly detected as a redundant node and hence regularized: $\eta_4$ is replaced by its right parent $\eta_3$. The resulting proof is shown in Figure \ref{ex:rpi-example-b}.

%This proof corresponds to the proof term
%$$ (\ub{\{\neg a\}}{\eta_1} \tRes (\{a, c, \neg b\} \tRes \ub{\{a, b\}}{\eta_3})) \tRes ((\eta_3 \tRes \{\neg b, \neg c, a\}) \tRes \eta_1) $$

%\noindent%
%{\RP}, on the other hand, assigns $\emptyset$ as the set of safe literals for $\eta_4$. Therefore, it does not detect that $\eta_4$ is a redundant irregular node, and then $\RP(\varphi) = \varphi$. 


%%The original example description.
%The node $\eta_4$ has pivot $a$, left (right) resolved literal $\neg a$ ($a$). Its conclusion is $\{b\}$ and its premises are the conclusions of its parents: the input nodes $\eta_1$ ($\{ \neg a \}$) and $\eta_3$ ($\{ a, b\} $). It has two children ($\eta_5$ and $\eta_8$). $\psi$ can be compactly represented by the following proof term:
%$$ (\ub{\{\neg a\}}{\eta_1} \tRes (\{a, c, \neg b\} \tRes \ub{(\eta_1 \tRes \{a, b\})}{\eta_4})) \tRes ((\eta_4 \tRes \{a, \neg b, \neg c\}) \tRes \eta_1). $$
\end{example}





\section{First-Order Challenges}\label{sec:Challenges}

TODO by Jan (just writing some ideas so far--not yet final by any means)\\
{\bf Does this belong here?And is this what you had in mind for interesting examples? And are the proof formatted correctly, or should I change them? aside from the first one going over the margin right now of course}\\ 

In this section, we discuss additional requirements for lowering a unit formula in the first order case that are not required in the propositional case.

%example 1: shows requirement for pair-wise unifiability with unit
%obvious - skip?

%example 2: shows requirement for pair-wise unifiability within all aux formulas

 \begin{example} The following example shows why we must check pair-wise unifiability with the literals resolved against the unit we're trying to lower.

% \begin{tiny}
% \begin{prooftree}
% \def\e{\mbox{\ $\vdash$\ }}
% \AxiomC{$\eta_2$}
% \AxiomC{$\eta_1$: $p(a)$\e$q(Y),r(Z)$}
% \AxiomC{$\eta_2$: \e $p(X)$}
% \BinaryInfC{$\eta_3$: \e$q(Y),r(Z)$}
% \AxiomC{$\eta_4$: $r(X),p(b)$\e $s(Y)$}
% \BinaryInfC{$\eta_5$: $p(b)$\e $s(Y),q(Y)$}
% \AxiomC{$\eta_6$: $s(Y), q(Y)$\e}
% \BinaryInfC{$\eta_7$: $p(b)$\e}
% \BinaryInfC{$\psi$: $\bot$}
% \end{prooftree}
% \end{tiny}
 \end{example}


%example 3: shows requirement for contraction check

 \begin{example} The following shows why the above is not necessarily  enough (we must check the original sources of the aux formulas, and see if those can be contracted), otherwise we might not save anything.
% \begin{footnotesize}
% \begin{prooftree}
% \def\e{\mbox{\ $\vdash$\ }}
% \AxiomC{$\eta_1$: $r(Y),p(X ~q(Y~b)), p(X~Y)$\e}
% \AxiomC{$\eta_2$: \e $p(U~V)$}
% \BinaryInfC{$\eta_3$: $r(V),p(U ~q(V~b))$\e}
% \AxiomC{$\eta_4$: \e $r(W)$}
% \BinaryInfC{$\eta_5$: $p(U ~q(W~b))$\e}
% \AxiomC{$\eta_2$}
% \BinaryInfC{$\psi$: $\bot$}
% \end{prooftree}
% \end{footnotesize}
 \end{example}


%example 4: requires FOSubstitution, introduces this concept?

\section{First-Order RecyclePivotsWithIntersection}
\label{sec:FORPI}
%TODO: this section
This section presents {\FORPI} (Algorithm \ref{algo:FORPI}), a first order generalization of {\RecyclePivotsIntersection}, which aims to compress irregular proofs. Recall that \RecyclePivotsIntersection
is a modification of the \texttt{RecyclePivots} algorithm, %described in  \cite{Bar-IlanFuhrmannHooryShachamStrichman2009Linear-time-reductions-of-resolution-proofs}, from which it derives its name. 
and \RecyclePivotsIntersection provides better compression on proofs where nodes have several children, when compared to \texttt{RecyclePivots}. Through a small modification to our algorithm (described later), a first order generalization of \texttt{RecyclePivots} is also possible.

%TODO: move to intro?
%Although in the worst case full regularization can increase the proof length exponentially \cite{Tseitin1983On-The-Complexity-of-Proofs-in-Propositional-Logics}, these algorithms show that many irregular proofs can have their length decreased if a careful partial regularization is performed. 

\newcommand{\la}{\leftarrow}


\begin{algorithm}[!b]
\begin{footnotesize}
\SetKwInOut{Input}{input}\SetKwInOut{Output}{output}
\SetKwData{units}{unitsQueue}
\SetKwData{fixedUnits}{fixedUnitsQueue}

\Input{A first-order proof $\psi$}
\Output{A possibly less-irregular first-order proof $\psi'$}

\BlankLine

$\psi'$ $\la$ $\psi$\;
traverse $\psi'$ bottom-up and \ForEach{node $\eta$ in $\psi'$}{
   \If{$\eta$ is a resolvent node}{
     setSafeLiterals($\eta$) \;
     regularizeIfPossible($\eta$)
   }
  }
$\psi'$ $\la$ fix($\psi'$) \;
\Return {$\psi'$}\;
\caption{\label{algo:FORPI} \texttt{\FORPI}}
\end{footnotesize}
\end{algorithm}



Our generalization, Algorithm~\ref{algo:FORPI}, follows the propositional idea of traversing the proof in a bottom-up manner, storing for every node a set of \emph{safe literals} that get resolved in all paths below it in the proof (or that already occurred in the root clause of the original proof). If one of the node's resolved literals can be unified to a literal in the set of safe literals, then it may be possible to regularize the node by replacing it by one of its parents. 

%TODO: re-write -- taken from FORPI paper
In the propositional case, regularization of a node replaces it by the parent whose clause contains the resolved literal that is safe. In the first order case, because unification introduces complications like those seen in Example \ref{ex:unifcheck}, we ensure that the replacement parent is (possibly after unification) contained entirely in the safe literals. This ensures that the remainder of the proof does not expect a variable to be unified to different values simultaneously. After regularization, all nodes below the regularized node may have to be fixed. 
Similar to \RecyclePivotsIntersection, instead of replacing the irregular node by one of its parents immediately, 
its other parent is replaced by \texttt{deletedNodeMarker}, as shown in Algorithm~\ref{algo:Regularize}.
As in the propositional case, fixing of the proof is postponed to another (single) traversal, as regularization proceeds bottom up and only nodes below a regularized node may require fixing.
During fixing, the irregular node is actually replaced by the parent that is not \texttt{deletedNodeMarker}.


%Unchanged from propositional case? %TODO: or should the $\in$ relation be unification?
\begin{algorithm}[t]
\begin{footnotesize}

\SetKwInOut{Input}{input}\SetKwInOut{Output}{output}
\SetKwData{units}{unitsQueue}
\SetKwData{fixedUnits}{fixedUnitsQueue}

\Input{A node $\psi=\psi_L  \res{\ell_L}{\sigma_L}{\ell_R}{\sigma_R} \psi_R$}
\Output{nothing (but the proof containing $\psi$ may be changed)}

\BlankLine
    \uIf{$\exists \sigma$  and $l \in \psi${\upshape.safeLiterals} such that $\sigma l = l_R$ or $l=\sigma l_R$}{
     \uIf{$\exists \sigma$ such that $\sigma\psi_R\subseteq\psi${\upshape.safeLiterals}} {
      replace $\psi_L$ of $\eta$ by \texttt{deletedNodeMarker} \;
      mark $\psi$ as regularized
}
    }
    \ElseIf{$\exists \sigma$  and $l \in \psi${\upshape.safeLiterals} such that $\sigma l = l_L$ or $l=\sigma l_L$}{
     \uIf{$\exists \sigma$ such that $\sigma\psi_L\subseteq\psi${\upshape.safeLiterals}} {
      replace $\psi_R$ by \texttt{deletedNodeMarker} \;
      mark $\psi$ as regularized
}
    }
\caption{\label{algo:Regularize} \texttt{regularizeIfPossible}}
\end{footnotesize}
\end{algorithm}


\begin{algorithm}[!b]
\begin{footnotesize}

\SetKwInOut{Input}{input}\SetKwInOut{Output}{output}
\SetKwData{units}{unitsQueue}
\SetKwData{fixedUnits}{fixedUnitsQueue}

\Input{A node $\eta$}
\Output{nothing (but the node $\eta$ gets a set of safe literals)}

\BlankLine

    \uIf{$\eta$ is a root node with no children}{
      $\eta$.safeLiterals $\la$ $\eta$.clause  
    }
    \Else{
      \ForEach{$\eta'$ $\in$ $\eta${\upshape.children}}{
        \uIf{$\eta'$ is marked as regularized}{ 
          safeLiteralsFrom($\eta'$) $\la$ $\eta'$.safeLiterals \;}
        \uElseIf{$\eta$ is left parent of $\eta'$}{ 
        	safeLiteralsFrom($\eta'$) $\la$ $\eta'$.safeLiterals $\cup$ \{ $\eta'$.rightResolvedLiteral \} \;
        }
        \ElseIf{$\eta$ is right parent of $\eta'$}{ 
			safeLiteralsFrom($\eta'$) $\la$ $\eta'$.safeLiterals $\cup$ \{ $\eta'$.leftResolvedLiteral \} \;
        }
      }
      $\eta$.safeLiterals $\la$ $\bigcap_{\eta' \in \eta\textrm{.children}}$ safeLiteralsFrom($\eta'$)
    }
\caption{\label{algo:SetSafeLiterals} \texttt{setSafeLiterals}}
\end{footnotesize}
\end{algorithm}

%The set of safe literals of a node $\eta$ can be computed from the set of safe literals of its children (cf.\ Algorithm~\ref{algo:SetSafeLiterals}). In the case when $\eta$ has a single child $\varsigma$, the safe literals of $\eta$ are simply the safe literals of $\varsigma$ together with the resolved literal $p$ of $\varsigma$ belonging to $\eta$ ($p$ is safe for $\eta$, because whenever $p$ is propagated down the proof through $\eta$, $p$ gets resolved in $\varsigma$). It is important to note, however, that if $\varsigma$ has been marked as regularized, it will eventually be replaced by $\eta$, and hence $p$ should not be added to the safe literals of $\eta$. In this case, the safe literals of $\eta$ should be exactly the same as the safe literals of $\varsigma$. When $\eta$ has several children, the safe literals of $\eta$ w.r.t. a child $\varsigma_i$ contain literals that are safe on all paths that go from $\eta$ through $\varsigma_i$ to the root. For a literal to be safe for all paths from $\eta$ to the root, it should therefore be in the intersection of the sets of safe literals w.r.t. each child.

The {\RecyclePivotsIntersection} and the \texttt{RecyclePivots} algorithms differ from each other mainly in the
computation of the safe literals of a node that has many children. While the former 
returns the intersection as shown in Algorithm~\ref{algo:SetSafeLiterals}, the latter
returns the empty set. 
Further, while in \RecyclePivotsIntersection the safe literals of the root node contain all the literals of the root clause, in \texttt{RecyclePivots} the root node is always assigned an empty set of literals. 
This is easy accomplished in the first order case by changing lines 11 and 2, respectively, of Algorithm~\ref{algo:SetSafeLiterals}.
This makes a difference only when the proof is not a refutation.

The set of safe literals of a node $\eta$ can be computed from the set of safe literals of its children (cf.\ Algorithm~\ref{algo:SetSafeLiterals}), in a manner identical to the propositional case.


%Note that during a traversal of the proof,  the lines from 5 to 10 in Algorithm~\ref{algo:SetSafeLiterals} are executed as many times as the number of edges in the proof.  Since every node has at most two parents, the number of edges is at most twice the number of nodes.  Therefore, during a traversal of a proof with $n$ nodes, lines from 5 to 10 are executed at most $2n$ times, and the algorithm remains linear. In our prototype implementation, the sets of safe literals are instances of Scala's  \texttt{mutable.HashSet} class. Being mutable, new elements can be added efficiently. And being HashSets, membership checking is done in constant time in the average case, and set intersection (line 12) can be done in $O(k.s)$, where $k$ is the number of sets and $s$ is the size of the smallest set.






\section{Experiments} \label{sec:exp}

A prototype\footnote{Source code available at \url{https://github.com/jgorzny/Skeptik}} of a (two-traversal) version of {\SFOLowerUnits} has been implemented in the functional programming language Scala\footnote{\url{http://www.scala-lang.org/}} as part of the \skeptik
 library\footnote{\url{https://github.com/Paradoxika/Skeptik}}. 

Before evaluating this algorithm, we first generated several benchmark proofs. This was done by executing the {\SPASS}\footnote{\url{http://www.spass-prover.org/}} theorem prover on ToDo(numberOfProblems) problems of the ToDo categories of the TPTP Problem Library \footnote{\url{http://www.cs.miami.edu/{\textasciitilde}tptp/}}. In order to generate pure resolution proofs, most advanced inference rules used by {\SPASS}  were disabled. The Euler Cluster at the University of Victoria\footnote{\url{https://rcf.uvic.ca/euler.php}} was used and the time limit was 300 seconds per problem. Under these conditions, {\SPASS} was able to generate 308 proofs. 

The evaluation of {\SFOLowerUnits} was performed on a laptop (2.8GHz Intel Core i7 processor with 4 GB of RAM (1333MHz DDR3) available to the Java Virtual Machine). For each benchmark proof $\psi$, we measured\footnote{The raw data is available at ToDo (this link is not working) \url{https://docs.google.com/spreadsheets/d/1F1-t2OuhypmTQhLU6yTj42aiZ5CqqaZvhVvOzeFgn0k/edit\#gid=1182923972}} the time needed to compress the proof ($t(\psi)$) and the compression ratio ($(|\psi|-|\alpha(\psi)|)/|\psi|$), where $|\psi|$ is the length of $\psi$ (i.e. the number of axioms, resolution and contractions (ignoring substitutions)) and $\alpha(\psi)$ is the result of applying {\SFOLowerUnits} to $\psi$.

The proofs generated by {\SPASS} were small (with lengths from 3 to 49). These proofs are specially small in comparison with the typical proofs generated by SAT- and SMT-solvers, which usually have from a few hundred to a few million nodes. The number of proofs (compressed and uncompressed) per length is shown in Figure \ref{fig:ex} (b). Uncompressed proofs are those which had either no lowerable units to lower or for which \SFOLowerUnits failed and returned the original proof. Such failures occurred on only 14 benchmark proofs. Among the smallest of the 308 proofs, very few proofs were compressed. This is to be expected, since the likelihood that a very short proof contain a lowerable unit (or even merely a unit with more than one child) is low. The proportion of compressed proofs among longer proofs is, as expected, larger, since they have more nodes and it is more likely that some of these nodes are lowerable units. 13 out of 18 proofs with length greater than or equal to 30 were compressed. 

Figure \ref{fig:ex} (a) shows a box-whisker plot of compression ratio with proofs grouped by length and whiskers indicating minimum and maximum compression ratio achieved within the group. Besides the median compression ratio (the horizontal thick black line), the chart also shows the mean compression ratios for all proofs of that length and for all compressed proofs (the red cross and the blue circle). In the longer proofs (length greater than 34), the median and the means are in the range from 5\% to 15\%, which is satisfactory in comparison with the total compression ratio of 7.5\% that has been measured for the propositional {\LowerUnits} algorithm on much longer propositional proofs \cite{Boudou}.

Figure \ref{fig:ex} (c) shows a scatter plot comparing the length of the input proof against the length of the compressed proof. For the longer proofs (circles in the right half of the plot), it is often the case that the length of the compressed proof is significantly lesser than the length of the input proof.

Figure \ref{fig:ex} (d) plots the cumulative original and compressed lengths of all benchmark proofs (for an x-axis value of $k$, the cumulative curves show the sum of the lengths of the shortest $k$input proofs). The total cumulative length of all original proofs is ToDo:4500(put the correct number here) while the cumulative length of all proofs after compression is ToDo:4000(correct this number). This results in a total compression ratio of ToDo:12\%(compute this number), which is impressive, considering the inclusion of all the short proofs (in which the presence of lowerable units is a priori unlikely) tends to decrease the total compression ratio. For comparison, the total compression ratio considering only the 100 longest input proofs is ToDo:(compute this percentage).

Figure \ref{fig:ex} also indicates an interesting potential trend. The gap between the two cumulative curves seems to grow superlinearly. If this trend is extrapolated, progressively larger compression ratios can be expected for longer proofs. This is compatible with Theorem 10 in \cite{LURPI}, which shows that, for proofs generated by eagerly resolving units against all clauses, the propositional {\LowerUnits} algorithm can achieve quadratic assymptotic compression. SAT- and SMT-solvers based on CDCL (Conflict-Driven Clause Learning) avoid eagerly resolving unit clauses by dealing with unit clauses via boolean propagation on a conflict graph and extracting subproofs from the conflict graph with every unit being used at most once per subproof (even when it was used multiple times in the conflict graph). Saturation-based automated theorem provers, on the other hand, might be susceptible to the eager unit resolution redundancy described in Theorem 10 \cite{LURPI}. This potential trend would need to be confirmed by further experiments with more data (more proofs and longer proofs).

The total time needed by {\SPASS} to generate all 308 proofs on the Euler Cluster was ToDo. The total time for {\SFOLowerUnits} to be executed on all 308 proofs was ToDo on a simple laptop. (ToDo: make sure the total time calculation either includes or excludes parsing times for both Skeptik and SPASS. otherwise the comparison would be biased and unfair). Therefore, {\SFOLowerUnits} is a fast algorithm. For a small overhead in time (in comparison to proving time), it may simplify the proof considerably.


% \begin{figure}
% \includegraphics[scale=0.5]{images/compress_time_vs_proof_length.pdf}
% \end{figure}

% \begin{figure}
% \includegraphics[scale=0.5]{images/compress_time_vs_proof_length_res.pdf}
% \end{figure}

% \begin{figure}
% \includegraphics[scale=0.5]{images/compress_ratio_vs_proof_length.pdf}
% \end{figure}

%\begin{figure}\label{fig:compressRatioResVLength} %USED
%\includegraphics[scale=0.5]{images/compress_ratio_res_vs_proof_length.pdf}
%\end{figure}

% \begin{figure}
% \includegraphics[scale=0.5]{images/compress_ratio_res_vs_proof_length_res.pdf}
% \end{figure}

%\begin{figure}%USED
%\includegraphics[scale=0.5]{images/compress_ratio_res_vs_proof_length_all_proofs.pdf}
%\end{figure}
\begin{figure}
\centering
%    \subfloat[Average compression (only success)]{{\includegraphics[scale=0.5]{images/compress_ratio_res_vs_proof_length.pdf} }}
    \subfloat[Compression ratio]{{\includegraphics[scale=0.5]{images/compress_ratio_res_vs_proof_length_all_proofs.pdf} }}%\hfilll
    \subfloat[Number of (non-)compressed proofs]{{\includegraphics[scale=0.5]{images/num_compressed_stacked.pdf}}}\hfill
    \subfloat[Compressed length against input length]{{\includegraphics[scale=0.5]{images/compress_length_no_sub_vs_length_all_proofs.pdf} }}
%    \subfloat[Total proof nodes]{{\includegraphics[scale=0.5]{images/cumulative_res_nodes_no_subs.pdf} }}
    \subfloat[Cumulative proof lengths]{{\includegraphics[scale=0.5]{images/cumulative_res_nodes_no_subs_top100.pdf}}}
\caption{Empirical evaluation results}
\label{fig:ex}
\end{figure}


%\begin{figure}
%\centering
%    \subfloat{{\includegraphics[scale=0.5]{images/compress_ratio_res_vs_proof_length.pdf}
%}}%
%    \subfloat{{\includegraphics[scale=0.5]{images/compress_ratio_res_vs_proof_length_all_proofs.pdf} }}%
%\caption{Compression ratio versus proof length without uncompressed proofs (left) and with with uncompressed proofs (right).}
%\label{fig:ex1}
%\end{figure}



% \begin{figure}
% \includegraphics[scale=0.5]{images/num_compressed_count.pdf}
% \end{figure}

% \begin{figure}
% \includegraphics[scale=0.5]{images/num_compressed_percent.pdf}
% \end{figure}


%\begin{figure}%USED
%\includegraphics[scale=0.5]{images/num_compressed_stacked.pdf}
%\end{figure}

%\begin{figure}
%\centering
%    \subfloat{{\includegraphics[scale=0.5]{images/num_compressed_stacked.pdf}
%}}%
%    \subfloat{{\includegraphics[scale=0.5]{images/cumulative_res_nodes_no_subs.pdf} }}
%\caption{Number of proofs compressed of each length (left), and total number of nodes before and after compression (right).}
%\label{fig:ex2}
%\end{figure}


% \begin{figure}
% \includegraphics[scale=0.5]{images/res_length_vs_compress_res_length_all_proofs.pdf}
% \end{figure}
% \begin{figure}
% \includegraphics[scale=0.5]{images/res_length_vs_compress_res_length.pdf}
% \end{figure}

% \begin{figure}
% \includegraphics[scale=0.5]{images/cumulative_res_nodes.pdf}
% \end{figure}

%\begin{figure}%USED
%\includegraphics[scale=0.5]{images/cumulative_res_nodes_no_subs.pdf}
%\end{figure}

%\begin{figure} %USED
%\includegraphics[scale=0.5]{images/cumulative_res_nodes_no_subs_top100.pdf}
%\end{figure}




% \begin{figure}
% \includegraphics[scale=0.5]{images/cumulative_res_nodes_no_subs_log.pdf}
% \end{figure}
% \begin{figure}
% \includegraphics[scale=0.5]{images/compress_length_no_sub_vs_length.pdf}
% \end{figure}

%\begin{figure}%USED
%\includegraphics[scale=0.5]{images/compress_length_no_sub_vs_length_all_proofs.pdf}
%\end{figure}
\vspace{-0.25cm}
\section{Conclusions and Future Work}\label{sec:conclusion}

The main contribution of this paper is the lifting of the propositional proof compression algorithm {\RPI} to the first-order case. As indicated in Section \ref{sec:Challenges}, the generalization is challenging, because unification instantiates literals and, consequently, a node may be regularizable even if its resolved literals are not syntactically equal to any safe literal. Unification must be taken into account when collecting safe literals and marking nodes for deletion.

%We first evaluated the algorithm on all 308 real proofs that the \texttt{SPASS} theorem prover (with only standard resolution enabled) was capable of generating when executed on unsatisfiable TPTP problems without equality. Although the compression achieved by the first-order {\FORPI} algorithm was not as good as the compression achieved by the propositional {\RPI} algorithm on real proofs generated by SAT and SMT solvers \cite{LURPI}, this is due to the fact that the 308 proofs were too short (less than 32 resolutions) to contain a significant amount of irregularities. In contrast, the propositional proofs used in the evaluation of the propositional {\RPI} algorithm had thousands (and sometimes hundreds of thousands) of resolutions. 

%Our second evaluation used larger, but randomly generated, proofs. The compression achieved by {\FORPI} in a short amount of time on this data set was compatible with our expectations and previous experience in the propositional level. The obtained results indicate that {\FORPI} is a promising compression technique to be reconsidered when first-order theorem provers become capable of producing larger proofs. Although we carefully selected generation probabilites in accordance with frequencies observed in real proofs, it is important to note that randomly generated proofs may still differ from real proofs in shape and may be more or less likely to contain irregularities exploitable by our algorithm. Resolution restrictions and refinements (e.g. ordered resolution %\cite{Maslov1964,KowalskiHayes1969,OrderedRes}, 
%\cite{hsiang1991proving, OrderedRes}, hyper-resolution \cite{HyperResolution,robinson1965automatic}, unit-resulting resolution \cite{UnitResultingResolution,prover9-mace4}) may result in longer chains of resolutions and, therefore, in proofs with a possibly larger height to length ratio. As the number of irregularities increases with height, such proofs could have a higher number of irregularities in relation to length.

We evaluated the algorithm on two data sets, and
%First, we evaluated the algorithm on all 308 real proofs that the \texttt{SPASS} theorem prover (with only standard resolution enabled) was capable of generating when executed on unsatisfiable TPTP problems without equality. Although the compression achieved by the first-order {\FORPI} algorithm was not as good as the compression achieved by the propositional {\RPI} algorithm on real proofs generated by SAT and SMT solvers \cite{LURPI}, this is due to the fact that the 308 proofs were too short (less than 32 resolutions) to contain a significant amount of irregularities. %In contrast, the propositional proofs used in the evaluation of the propositional {\RPI} algorithm had thousands (and sometimes hundreds of thousands) of resolutions. 
%Our second evaluation used larger, randomly generated, proofs. 
the compression achieved by {\FORPI} in a short amount of time on this data set was compatible with our expectations and previous experience in the propositional level. 
The obtained results indicate that {\FORPI} is a promising compression technique to be reconsidered when first-order theorem provers become capable of producing larger proofs. Although we carefully selected generation probabilites in accordance with frequencies observed in real proofs, it is important to note that randomly generated proofs may still differ from real proofs in shape and may be more or less likely to contain irregularities exploitable by our algorithm. 
%Resolution restrictions and refinements (e.g. ordered resolution \cite{Maslov1964,KowalskiHayes1969,OrderedRes}, 
%\cite{hsiang1991proving, OrderedRes}, hyper-resolution \cite{HyperResolution,robinson1965automatic}, unit-resulting resolution \cite{UnitResultingResolution,prover9-mace4}) may result in longer chains of resolutions and, therefore, in proofs with a possibly larger height to length ratio. As the number of irregularities increases with height, such proofs could have a higher number of irregularities in relation to length.

In this paper, for the sake of simplicity, we considered a pure resolution calculus without restrictions, refinements or extensions. However, in practice, theorem provers do use restrictions and extensions. It is conceptually easy to adapt the algorithm described here to many variations of resolution. 
%For instance, restricted forms of resolution (e.g. ordered resolution, hyper-resolution, unit-resulting resolution) can be simply regarded as (chains of) unrestricted resolutions for the purpose of proof compression. The compression process would break the chains and change the structure of the proof, but the compressed proof would still be a correct unrestricted resolution proof, albeit not necessarily satisfying the restrictions that the input proof satisfied. 
%In the case of extensions for equality reasoning using paramodulation-like inferences, it might be necessary to apply the paramodulations to the corresponding safe literals. Alternatively, equality inferences could be replaced by resolutions with instances of equality axioms, and the proof compression algorithm could be applied to the proof resulting from this replacement. 
%Another 
For instance, a common extension of resolution is the splitting technique \cite{WeidenbachSplitting}. When splitting is used, each split sub-problem is solved by a separate refutation, and {\FORPI} could be applied to each refutation independently. 

It would be interesting to determine if proof compression could be applied during proof search, in order to improve the performance theorem provers. Additionally, it would be interesting to see if similar techniques can be applied to proofs in higher-order logics.





% These algorithms are very fast, and together they may simplify the proof considerably for a relatively quick time cost.

% {\RPI} performs best when the proofs are tall; {\FORPI} will likely perform similarly. However, the proofs in this data set are relatively short, and those compressed by {\GFOLU} first are even shorter. Thus, the performance of {\FORPI} is not surprising.

%{\FORPI} continues to support the idea of listing propositional proof compression algorithms to the first-order case. The experimental results discussed in the previous continue to be encouraging, and are consistent with trends observed in the propositional case. 

%\paragraph{Acknowledgments:}


%\begin{acks}
%%GSoC support should be mentioned here instead according to journal guidelines
%We thank the Google Summer of Code 2014 and Google Summer of Code 2016 programs for financial support of this research. 
%Bruno ist Stipendiat der \"Osterreichischen Akademie der Wissenschaft (APART) an der TU-Wien.
%\end{acks}

\begin{footnotesize}
%\bibliographystyle{splncs}
\bibliographystyle{plain}
\bibliography{biblio}
\end{footnotesize}
%\appendix
%\section{Algorithm {\RecyclePivotsIntersection}}
\label{Section:RPI}

\newcommand{\tRes}{\odot}
\newcommand{\tResFact}{\otimes}
\newcommand{\tResChain}{\ominus}
\newcommand{\AXC}{\AxiomC}
\newcommand{\BIC}{\BinaryInfC}
\newcommand{\RName}[1]{\RightLabel{#1}}
\newcommand{\p}[1]{\hat{#1}}
\newcommand{\ub}[2]{\underbrace{#1}_{#2}}
\newcommand{\tResStar}{\circledast}

%\textbf{Note:} for the reviewers' convenience, this appendix summarizes \cite{LURPI}.

%\bigskip

%\noindent
This section explains {\RecyclePivotsIntersection} ({\RPI}) \cite{LURPI}, which aims to compress irregular propositional proofs. It can be seen as a simple 
but significant modification of the {\RP} algorithm described in 
\cite{RP08}, 
from which it derives its name. 
Although in the worst case full regularization can increase the proof length exponentially 
\cite{Tseitin}, these algorithms show that 
many irregular proofs can have their length decreased if a careful partial regularization is performed. 

We write $\psi[\eta]$ to denote a \emph{proof-context}
$\psi[\_]$ with a single placeholder replaced by the subproof $\eta$.
We say that a proof of the form $\psi[\eta \tRes_p \psi'[\eta'\tRes_p\eta_2]]$ is \emph{irregular}.

\begin{example}
%Consider an irregular proof of the form $\psi[ \eta \tRes_p \psi'[\eta' \tRes_p \eta''] ]$, 
Consider an irregular proof and assume, without loss of generality, that $p \in \eta$ and $p \in \eta'$, as in the proof of $\psi$ below. The proof of $\psi$ can be written as $(\eta \tRes_p ( \eta_1 \tRes (\eta' \tRes_p \eta'')))$, or $(\eta \tRes_p \psi'[(\eta' \tRes_p \eta'')])$ where $\psi'[(\eta' \tRes_p \eta'')] = (\eta_1 \tRes (\eta' \tRes_p \eta''))$ is the sub-proof of $\lnot p$.
\begin{footnotesize}
\begin{prooftree}

		\AXC{$\eta$: $p$} \RName{$p$}
			\AXC{$\eta_1$: $\lnot r, \lnot p$}
		\AXC{$\eta'$: $p$}
				\AXC{$ \eta''$: $\lnot p, r$} \RName{$p$}
	\BIC{$r$}

	\BIC{$\lnot p$} \RName{$p$}

		\BIC{$\psi$: $\bot$}	
%		 \DisplayProof 
\end{prooftree}
\label{ex:rpi-example-c}
\end{footnotesize}
\noindent
Then, if $\eta' \tRes_p \eta''$ is replaced by $\eta''$ within the proof-context $\psi'[\ ]$, the clause $\eta \tRes_p \psi'[\eta'']$ subsumes the clause $\eta \tRes_p \psi'[\eta' \tRes_p \eta'']$, because even though the literal $\neg p$ of $\eta''$ is
propagated down, it gets resolved against the literal $p$ of $\eta$ later on below in the proof. More precisely, even though it might be the case that $\neg p \in \psi'[\eta'']$ while $\neg p \notin \psi'[\eta' \tRes_p \eta'']$, it is necessarily the case that $\neg p \notin \eta \tRes_p \psi'[\eta' \tRes_p \eta'']$ and $\neg p \notin \eta \tRes_p \psi'[\eta'']$. In this case, the proof can be regularized as follows.


\begin{footnotesize}
\begin{prooftree}

		\AXC{$\eta$: $p$}
			\AXC{$\eta_1$: $\lnot r, \lnot p$}

				\AXC{$ \eta''$: $\lnot p, r$}


	\BIC{$\lnot p$} \RName{$p$}

		\BIC{$\psi$: $\bot$}	

\end{prooftree}
\end{footnotesize}
\end{example}


\begin{figure*}[bt]%
\centering
\subfloat[A propositional proof before compression by {\RPI}.]{%
\begin{footnotesize}
%\begin{prooftree}
\AXC{$ \eta_1 $}
		\AXC{$ \eta_2: a, c, \neg b $}
				\AXC{$ \eta_1: \neg a$}
						\AXC{$ \eta_3:  a, b $} \RName{$a$}
					\BIC{$ \eta_4: b$} \RName{$b$}
			\BIC{$ \eta_5: a, c$}	\RName{$a$}
	\BIC{$\eta_6: c$}
		\AXC{$ \eta_4 $}
				\AXC{$ \eta_7: a, \neg b, \neg c $} \RName{$b$}
			\BIC{$ \eta_8: a, \neg c$}	\RName{$c$}
					\AXC{$ \eta_1 $}  \RName{$a$}
				\BIC{$ \eta_9: \neg c$}	\RName{$c$}
		\BIC{$\psi: \bot$}	
		 \DisplayProof 
%\end{prooftree}
\label{ex:rpi-example-a}
\end{footnotesize}}\\%
\subfloat[A propositional proof after compression by {\RPI}.]{%
\begin{small}
%\begin{prooftree}
\AXC{$ \eta_1: \neg a $}
		\AXC{$ \eta_2: a, c, \neg b $}
						\AXC{$ \eta_3:  a, b $}\RName{$ $}
			\BIC{$ \eta_5: a, c$}	\RName{$ $}
	\BIC{$\eta_6: c$}
		\AXC{$ \eta_3 $}
				\AXC{$ \eta_7: a, \neg c, \neg b $} \RName{$ $}
			\BIC{$ \eta_8: a, \neg c$}	\RName{$ $}
					\AXC{$ \eta_1 $}  \RName{$ $}
				\BIC{$ \eta_9: \neg c$}	\RName{$ $}
		\BIC{$\psi: \bot$}	
		 \DisplayProof 
%\end{prooftree}
\label{ex:rpi-example-b}
\end{small}
}%

\caption{A {\RPI} example.}
\label{ex:rpi-example}
\end{figure*}


Although the remarks above suggest that it is safe to replace $\eta' \tRes_p
\eta''$ by $\eta''$ within the proof-context $\psi'[\ ]$, this is not always the
case. If a node in $\psi'[\ ]$ has a child in $\psi[\ ]$, then the literal $\neg
p$ might be propagated down to the root of the proof, and hence, the clause
$\psi[ \eta \tRes_p \psi'[ \eta''] ]$ might not subsume the clause $\psi[ \eta
\tRes_p \psi'[\eta' \tRes_p \eta''] ]$. Therefore, it is only safe to do the
replacement if the literal $\neg p$ gets resolved in all paths from $\eta''$ to the root or if it already occurs in the root clause of the original proof $\psi[ \eta \tRes_p \psi'[\eta' \tRes_p \eta''] ]$.

\IncMargin{0.5em}
\begin{algorithm}[b]
\begin{footnotesize}
\SetKwInOut{Input}{input}\SetKwInOut{Output}{output}
\SetKwData{units}{unitsQueue}
\SetKwData{fixedUnits}{fixedUnitsQueue}

\Input{A proof $\psi$}
\Output{A possibly less-irregular proof $\psi'$}

\BlankLine

$\psi'$ $\la$ $\psi$\;
traverse $\psi'$ bottom-up and \ForEach{node $\eta$ in $\psi'$}{
   % \uIf{$\eta$ is an input node}{ 
   % 		do nothing \;
   %  }
   %  \ElseIf{$\eta$ is a resolvent node}{
   %   setSafeLiterals($\eta$) \;
   %   regularizeIfPossible($\eta$)
   % }
%% SM: original version: do we really need the "do nothing" branch?
   \If{$\eta$ is a resolvent node}{
     setSafeLiterals($\eta$) \;
     regularizeIfPossible($\eta$)
   }
  }
$\psi'$ $\la$ fix($\psi'$) \;
\Return {$\psi'$}\;
\caption{\label{algo:RPI} \texttt{\RPI}}
\end{footnotesize}
\end{algorithm}
\DecMargin{0.5em}
%\begin{code}
%  function recyclePivotsWithIntersection(p: Proof): Proof = {
%    traverseBottomUp(p)(
%      n => {
%        if (n is Input) doNothing
%        else if (n is Resolvent) {
%          setSafeLiterals(n)
%          regularizeIfPossible(n)
%          for (c in n.children) c.freeMemory
%        } 
%      }
%    )
%    fix(p)  
%  }
%\end{code}

These observations lead to the idea of traversing the proof in a bottom-up
manner, storing for every node a set of \emph{safe literals} that get resolved
in all paths below it in the proof (or that already occurred in the root clause
of the original proof). Moreover, if one of the node's resolved literals belongs
to the set of safe literals, then it is possible to regularize the node by
replacing it by one of its parents (cf.\ Algorithm~\ref{algo:RPI}). 

The regularization of a node should replace a node by one of its parents, and more precisely by the parent whose clause contains the resolved literal that is safe. After regularization, all nodes below the regularized node may have to be fixed. However, since the regularization is done with a bottom-up traversal, and only nodes below the regularized node need to be fixed, it is again possible to postpone fixing and do it with only a single traversal afterwards. 
Therefore, instead of replacing the irregular node by one of its parents immediately, 
its other parent is marked as \texttt{deletedNode}, as shown in Algorithm~\ref{algo:Regularize}. Only later during fixing, 
the irregular node is actually replaced by its surviving parent (i.e. the parent that is not marked as \texttt{deletedNode}).


\IncMargin{0.5em}
\begin{algorithm}[p]
\begin{footnotesize}
\SetKwInOut{Input}{input}\SetKwInOut{Output}{output}
\SetKwData{units}{unitsQueue}
\SetKwData{fixedUnits}{fixedUnitsQueue}

\Input{A node $\eta$}
\Output{nothing (but the proof containing $\eta$ may be changed)}

\BlankLine
    \uIf{$\eta${\upshape.rightResolvedLiteral} $\in$ $\mathcal{S}(\eta)$}{
      mark left parent of $\eta$ as \texttt{deletedNode} \;
      mark $\eta$ as regularized
    }
    \ElseIf{\textrm{$\eta${\upshape.leftResolvedLiteral} $\in$  $\mathcal{S}(\eta)$}}{
      mark right parent of $\eta$ as \texttt{deletedNode} \;
      mark $\eta$ as regularized
    }
\caption{\label{algo:Regularize} \texttt{regularizeIfPossible}}
\end{footnotesize}
\end{algorithm}
\DecMargin{0.5em}

\IncMargin{0.5em}
\begin{algorithm}[p]
\begin{footnotesize}
\SetKwInOut{Input}{input}\SetKwInOut{Output}{output}
\SetKwData{units}{unitsQueue}
\SetKwData{fixedUnits}{fixedUnitsQueue}

\Input{A node $\eta$}
\Output{nothing (but the node $\eta$ gets a set of safe literals)}

\BlankLine

    \uIf{$\eta$ is a root node with no children}{
      $\mathcal{S}(\eta)$ $\la$ $\eta$.clause  
    }
    \Else{
      \ForEach{$\eta'$ $\in$ $\eta${\upshape.children}}{
        \uIf{$\eta'$ is marked as regularized}{ 
          safeLiteralsFrom($\eta'$) $\la$  $\mathcal{S}(\eta')$ \;}
        \uElseIf{$\eta$ is left parent of $\eta'$}{ 
        	safeLiteralsFrom($\eta'$) $\la$  $\mathcal{S}(\eta')$ $\cup$ 
        	\hspace{6cm} $~~~~$ \{ $\eta'$.rightResolvedLiteral \} \;
        }
        \ElseIf{$\eta$ is right parent of $\eta'$}{ 
			safeLiteralsFrom($\eta'$) $\la$ $\mathcal{S}(\eta')$ $\cup$ 
			\hspace{6cm} $~~~~$ \{ $\eta'$.leftResolvedLiteral \} \;
        }
      }
       $\mathcal{S}(\eta)$ $\la$ $\bigcap_{\eta' \in \eta\textrm{.children}}$ safeLiteralsFrom($\eta'$)
    }
\caption{\label{algo:SetSafeLiterals} \texttt{setSafeLiterals}}
\end{footnotesize}
\end{algorithm}
\DecMargin{0.5em}
%\begin{code}
%  function setSafeLiterals(n: Resolvent) = {
%    if (n is a root node with no children) {
%      n.safeLiterals = n.clause  
%    }
%    else {
%      val safeLiteralsPerChild = for (c in n.children) yield {
%        if (c is marked as regularized) c.safeLiterals 
%        else if (c.left == n) c.safeLiterals $+cup$ {c.resolvedLiterals.right}
%        else if (c.right == n) c.safeLiterals $+cup$ {c.resolvedLiterals.left}
%      }
%      n.safeLiterals = intersection(safeLiteralsPerChild)
%    }
%  }
%\end{code}

The set of safe literals of a node $\eta$ can be computed from the set of safe literals of its children (cf.\ Algorithm~\ref{algo:SetSafeLiterals}). 
In the case when $\eta$ has a single child $\varsigma$, the safe literals of $\eta$ are simply the safe literals of $\varsigma$ together with the resolved literal $p$ of $\varsigma$ belonging to $\eta$ ($p$ is safe for $\eta$, because whenever $p$ is propagated down the proof through $\eta$, $p$ gets resolved in $\varsigma$). 
It is important to note, however, that if $\varsigma$ has been marked as regularized, it will eventually be replaced by $\eta$, and hence $p$ should not be added to the safe literals of $\eta$. 
In this case, the safe literals of $\eta$ should be exactly the same as the safe literals of $\varsigma$.
When $\eta$ has several children, the safe literals of $\eta$ w.r.t. a child $\varsigma_i$ contain literals that are safe on all paths that go from $\eta$ through $\varsigma_i$ to the root. 
For a literal to be safe for all paths from $\eta$ to the root, it should therefore be in the intersection of the sets of safe literals w.r.t. each child.

The {\RP} and the {\RPI} algorithms differ from each other mainly in the computation of the safe literals of a node that has many children. 
While {\RPI} returns the intersection as shown in Algorithm~\ref{algo:SetSafeLiterals}, {\RP} returns the empty set (cf. Algorithm~\ref{algo:SetSafeLiteralsRP}). 
Additionally, while in {\RPI} the safe literals of the root node contain all the literals of the root clause, in {\RP} the root node is always assigned an empty set of literals. 
(Of course, this makes a difference only when the proof is not a refutation.)
Note that during a traversal of the proof, 
the lines from 5 to 10 in Algorithm~\ref{algo:SetSafeLiterals} are executed as many times as the number of edges in the proof. 
Since every node has at most two parents, the number of edges is at most twice the number of nodes. 
Therefore, during a traversal of a proof with $n$ nodes, lines from 5 to 10 are executed at most $2n$ times, and the algorithm remains linear.
In our prototype implementation, the sets of safe literals are instances of Scala's \texttt{mutable.HashSet} class. 
Being mutable, new elements can be added efficiently.
And being HashSets, membership checking is done in constant time in the average case,  and set intersection (line 12) can be done in $O(k.s)$, where $k$ is the number of sets and $s$ is the size of the smallest set.

\IncMargin{0.5em}
\begin{algorithm}[p]
\begin{footnotesize}
\SetKwInOut{Input}{input}\SetKwInOut{Output}{output}
\SetKwData{units}{unitsQueue}
\SetKwData{fixedUnits}{fixedUnitsQueue}

\Input{A node $\eta$}
\Output{nothing (but the node $\eta$ gets a set of safe literals)}

\BlankLine

    \uIf{$\eta$ is a root node with no children}{
      $\mathcal{S}(\eta)$ $\la$ $\emptyset$ 
    }
    \Else{
      \uIf{$\eta$ has only one child $\eta'$}{
        \uIf{$\eta'$ is marked as regularized}{ 
          $\mathcal{S}(\eta)$ $\la$  $\mathcal{S}(\eta')$ \;}
        \uElseIf{$\eta$ is left parent of $\eta'$}{ 
        	$\mathcal{S}(\eta)$ $\la$  $\mathcal{S}(\eta')$ $\cup$ 
        	 \{ $\eta'$.rightResolvedLiteral \} \;
        }
        \ElseIf{$\eta$ is right parent of $\eta'$}{ 
			$\mathcal{S}(\eta)$ $\la$  $\mathcal{S}(\eta')$ $\cup$
			 \{ $\eta'$.leftResolvedLiteral \} \;
        }
      }
      \Else{
      	 $\mathcal{S}(\eta)$ $\la$ $\emptyset$
      }
    }
\caption{\label{algo:SetSafeLiteralsRP} \texttt{setSafeLiterals} for \RP}
\end{footnotesize}
\end{algorithm}
\DecMargin{0.5em}
%
%\begin{code}
%  function setSafeLiterals(n: Resolvent) = {
%     n.safeLiterals = 
%       if (n has only one child c) {
%         if (c is marked as regularized) c.safeLiterals 
%         else if (c.left == n) c.safeLiterals $+cup$ {c.resolvedLiterals.left}
%         else if (c.right == n) c.safeLiterals $+cup$ {c.resolvedLiterals.right}
%       }
%       else $+emptyset$
%  }
%\end{code}



%\begin{example}
When applied to the proof $\psi$ shown in Example \ref{Example:Proof}, the algorithm {\RPI} assigns $\{a,c\}$ and $\{a, \neg c\}$ as the safe literals of, respectively, $\eta_5$ and $\eta_8$. The safe literals of $\eta_4$ w.r.t. its children $\eta_5$ and $\eta_8$ are respectively $\{a,c,b\}$ and $\{a, \neg c, b\}$, and hence the safe literals of $\eta_4$ are $\{a,b\}$ (the intersection of $\{a,c,b\}$ and $\{a, \neg c, b\}$). Since the right resolved literal of $\eta_4$ ($a$) belongs to $\eta_4$'s safe literals, $\eta_4$ is correctly detected as a redundant node and hence regularized: $\eta_4$ is replaced by its right parent $\eta_3$. The resulting proof is shown below:

\begin{small}
\begin{prooftree}
\AXC{$ \eta_1: \neg a $}
		\AXC{$ \eta_2: a, c, \neg b $}
						\AXC{$ \eta_3:  a, b $}\RName{$ $}
			\BIC{$ \eta_5: a, c$}	\RName{$ $}
	\BIC{$\eta_6: c$}
		\AXC{$ \eta_3 $}
				\AXC{$ \eta_7: a, \neg c, \neg b $} \RName{$ $}
			\BIC{$ \eta_8: a, \neg c$}	\RName{$ $}
					\AXC{$ \eta_1 $}  \RName{$ $}
				\BIC{$ \eta_9: \neg c$}	\RName{$ $}
		\BIC{$\psi: \bot$}	
\end{prooftree}
\end{small}

%This proof corresponds to the proof term
$$
(\ub{\{\neg a\}}{\eta_1} \tRes (\{a, c, \neg b\} \tRes \ub{\{a, b\}}{\eta_3})) \tRes ((\eta_3 \tRes \{\neg b, \neg c, a\}) \tRes \eta_1)
$$
%
\noindent%
{\RP}, on the other hand, assigns $\emptyset$ as the set of safe literals for $\eta_4$. Therefore, it does not detect that $\eta_4$ is a redundant irregular node, and then $\RP(\varphi) = \varphi$. 
 %
\hfill\QED
\end{example}

\begin{theorem}
\label{Theorem:RPIBetterThanRP}
For any proof $\varphi$, $|\RPI(\varphi)| \leq |\RP(\varphi)|$.
\end{theorem}
\begin{proof}
  For every node $\eta$ in $\varphi$, let $S^{\eta}_{\RPI}$ (resp.,
  $S^{\eta}_{\RP}$) be the set of safe literals for $\eta$ computed by {\RPI} and
  {\RP}. It is easy to see that $S^{\eta}_{\RPI} \supseteq S^{\eta}_{\RP}$ for all
  $\eta$. Therefore, {\RPI} detects and eliminates more redundancies than {\RP}.
%
\hfill\QED
\end{proof}

The better compression of {\RPI} does not come for free, 
as computing an intersection of sets is more costly than assigning the empty set. 
For a node $\eta$ with $k$ children, $k$ sets must be intersected and the size of each set is 
in the worst case in $O(h)$, where $h$ is the length of the shortest path from $\eta$ to a root.

\begin{example} %This example from the original RPI source
\label{Example:Proof}
When applied to the proof $\psi$ shown in Figure \ref{ex:rpi-example-a}, the algorithm {\RPI} assigns $\{a,c\}$ and $\{a, \neg c\}$ as the safe literals of, respectively, $\eta_5$ and $\eta_8$. The safe literals of $\eta_4$ w.r.t. its children $\eta_5$ and $\eta_8$ are respectively $\{a,c,b\}$ and $\{a, \neg c, b\}$, and hence the safe literals of $\eta_4$ are $\{a,b\}$ (the intersection of $\{a,c,b\}$ and $\{a, \neg c, b\}$). Since the right resolved literal of $\eta_4$ ($a$) belongs to $\eta_4$'s safe literals, $\eta_4$ is correctly detected as a redundant node and hence regularized: $\eta_4$ is replaced by its right parent $\eta_3$. The resulting proof is shown in Figure \ref{ex:rpi-example-b}.

%This proof corresponds to the proof term
%$$ (\ub{\{\neg a\}}{\eta_1} \tRes (\{a, c, \neg b\} \tRes \ub{\{a, b\}}{\eta_3})) \tRes ((\eta_3 \tRes \{\neg b, \neg c, a\}) \tRes \eta_1) $$

%\noindent%
%{\RP}, on the other hand, assigns $\emptyset$ as the set of safe literals for $\eta_4$. Therefore, it does not detect that $\eta_4$ is a redundant irregular node, and then $\RP(\varphi) = \varphi$. 


%%The original example description.
%The node $\eta_4$ has pivot $a$, left (right) resolved literal $\neg a$ ($a$). Its conclusion is $\{b\}$ and its premises are the conclusions of its parents: the input nodes $\eta_1$ ($\{ \neg a \}$) and $\eta_3$ ($\{ a, b\} $). It has two children ($\eta_5$ and $\eta_8$). $\psi$ can be compactly represented by the following proof term:
%$$ (\ub{\{\neg a\}}{\eta_1} \tRes (\{a, c, \neg b\} \tRes \ub{(\eta_1 \tRes \{a, b\})}{\eta_4})) \tRes ((\eta_4 \tRes \{a, \neg b, \neg c\}) \tRes \eta_1). $$
\end{example}


\end{document}

% vim: tw=100
