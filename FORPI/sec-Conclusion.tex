\section{Conclusions and Future Work}\label{sec:conclusion}

The main contribution of this paper is the generalization of the propositional proof compression algorithm {\RPI} to the first-order case. As indicated in Section \ref{sec:Challenges}, the generalization is challenging, because unification changes the pivots and, consequently, must be taken into account when collecting safe literals and marking nodes for deletion.

Every computational experiment evaluates not only the algorithm but also the data on which it is executed. Although the experimental results are not as promissing as expected, this is due to the fact that the 308 proofs currently available are too short to contain a significant amount of irregularities. This is a valuable piece of information, allowing us to conclude that it is not worth applying {\FORPI} to pure resolution proofs which current state-of-the-art first-order theorem provers seem capable of producing. Nevertheless, based on our positive results for {\RPI} on much longer proofs generated by SAT and SMT solvers \cite{LURPI}, {\FORPI} remains a promising option to be revisited in the future, when the performance of first-order theorem provers catch up with advances in SAT and SMT and taller first-order benchmark proofs become available.

% These algorithms are very fast, and together they may simplify the proof considerably for a relatively quick time cost.

% {\RPI} performs best when the proofs are tall; {\FORPI} will likely perform similarly. However, the proofs in this data set are relatively short, and those compressed by {\GFOLU} first are even shorter. Thus, the performance of {\FORPI} is not surprising.

%{\FORPI} continues to support the idea of listing propositional proof compression algorithms to the first-order case. The experimental results discussed in the previous continue to be encouraging, and are consistent with trends observed in the propositional case. 

%\paragraph{Acknowledgments:}
