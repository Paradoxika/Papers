\documentclass{llncs}

\usepackage{fancyhdr}
 \pagestyle{plain} 
%\pagestyle{plain}
%\renewcommand{\sectionmark}[1]{\today }
%\lhead{}\chead{}\rhead{\today}
%\lfoot{}\cfoot{\thepage \today}\rfoot{\date{\today}}

\usepackage{etex}

\usepackage{xcolor}
\usepackage{enumitem,amsmath,amssymb}
%\usepackage{breakurl}    % used for \url and \burl
\usepackage{url}
\usepackage[linesnumbered,boxed,noline,noend]{algorithm2e}
\def\defaultHypSeparation{\hskip.1in}
\usepackage[normalem]{ulem}

\usepackage{tikz}
\usetikzlibrary{arrows}
\usepackage{subfig}
\usepackage{array,booktabs,multirow}
\usepackage{placeins}

\usepackage{logictools}
\usepackage{prooftheory}
\usepackage{comment}
\usepackage{mathenvironments}
\usepackage{drawproof}
\usepackage{bussproofs}
\usepackage{tensor}
\usepackage{mathtools}
\usepackage{amsmath}

\usepackage{graphicx}
%\usepackage{caption}
%\usepackage{subcaption}

\renewcommand{\topfraction}{0.85}
\renewcommand{\textfraction}{0.1}
\renewcommand{\floatpagefraction}{0.75}


\newcommand{\freevar}[1]{\mathrm{FV}(#1)}

\newcommand{\Vertices}[1]{V_{#1}}
\newcommand{\Edges}[1]{E_{#1}}
\newcommand{\Conclusion}[1]{\clause_{#1}}

\newcommand{\axiom}[1]{\widehat{#1}}
\newcommand{\n}{v}
\newcommand{\raiz}[1]{\rho(#1)}

\newcommand{\pedge}[3]{\ensuremath{\raiz{#1} \xrightarrow{#2} \raiz{#3}}}


\newcommand\inlineeqno{\stepcounter{equation}\ (\theequation)}


% Contraction
\newcommand{\con}[3]{\lfloor #1 \rfloor_{#2}^{#3}}

% Resolution
%\newcommand{\res}[6]{#1 \tensor[^{#2}_{#3}]{\odot}{^{#4}_{#5}} #6}
%\newcommand{\res}[6]{#1 \prescript{#2}{#3}{\odot^{#4}_{#5}} #6}

\newcommand{\res}[4]{\mathrel{\operatorname*{\odot}_{#1 #3}^{#2 #4}}}


\title{FORPI Notes}

\author{
  Jan Gorzny}






\begin{document}

%\maketitle
\noindent
\hfill\today

\section{FORPI Correctness}

The following are the definitions from the submitted version of the paper. Corrections are in {\color{red}red}.

\begin{definition}
The set of \emph{safe literals} for a node $\eta$ in a proof $\psi$ with root clause $\Gamma$, denoted $\mathcal{S}(\eta)$, is such that $\ell \in \mathcal{S}(\eta)$ if and only if $\ell \in \Gamma$ or for all paths from $\eta$ to the root of $\psi$ there is an edge $\n_1
\xrightarrow[\sigma]{\ell'} \n_2$ with $\ell' \sigma = \ell$.
\end{definition}

\begin{definition}
\label{prop:pair}
Let $\eta$ be a node with safe literals $\mathcal{S}(\eta)$ and parents $\eta_1$ and $\eta_2$, assuming without loss of generality, $\eta_1 \xrightarrow[\sigma_1]{\{\ell_1\} } \eta$.
%{\color{red}\sout{such that}}
{\color{red} The node $\eta$ is said to be \emph{pre-regularizable} in the proof $\psi$ if} $\ell_1${\color{red}$\sigma_1$} is unifiable with a safe literal $\ell^* \in \mathcal{S}(\eta)$.
%{\color{red}\sout{
%Let $\mathcal{R}(\eta)$ be the set of all resolved literals $\ell_2$ such that $\eta_1 \xrightarrow[\sigma_1']{\{\ell_1\} } \eta'$, $\eta_2' \xrightarrow[\sigma_2']{\{\ell_2\} } \eta'$, and $\ell_1\sigma_1'=\dual{\ell_2}\sigma_2'$, for some nodes $\eta_2'$ and $\eta'$ and unifiers $\sigma_1'$ and $\sigma_2'$.
%The node $\eta$ is said to be \emph{pre-regularizable} in the proof $\psi$ if all literals in $\mathcal{R}(\eta) \cup \{ \dual{\ell_1}\}$ are unifiable. 
%}}
%{\color{red}\sigma_1}
\end{definition}

\begin{definition}
\label{prop:extracheck}
Let $\eta$ be pre-regularizable, with safe literals $\mathcal{S}(\eta)$ and parents $\eta_1$ and $\eta_2$, with clauses $\Gamma_1$ and $\Gamma_2$ respectively, assuming without loss of generality that {\color{red} $\eta_1 \xrightarrow[\sigma_1]{\{\ell_1\} } \eta$}
%{\color{red} \sout{, 
%$\eta_2 \xrightarrow[\sigma_2]{\{\ell_2\} }\eta$ and $\dual{\ell_2}$ is unifiable with some $\ell^* \in \mathcal{S}(\eta)$}}
{\color{red} such that $\ell_1\sigma_1$ is unifiable with a safe literal $\ell^*\in \mathcal{S}(\eta)$}. 
%$\eta_2 \xrightarrow[\sigma_2]{\{\ell_2\} }\eta$ and $\dual{\ell_2}$ is unifiable with some $\ell^* \in \mathcal{S}(\eta)$. 
The node $\eta$ is said to be \emph{strongly regularizable} in $\psi$ if $\Gamma_1 \sigma_{\color{red}1} \subseteq \mathcal{S}(\eta)$.
\end{definition}

\noindent
The notion of \emph{pre-regulariziability} can be thought of as a \emph{necessary} condition for recycling the node $\eta$, while the notion of \emph{strongly regularizable} can be thought of as a \emph{sufficient} condition. Note that these updated definitions more closely resemble their use, e.g. in Example 4.3 of the paper, when we say that $\eta_3$ is pre-regularizable, we only say that its pivot is unifiable with a safe literal; we don't look at $\eta_2$ at all, which would be required using the definition of pre-regularizable in the paper. Moreover, since $\eta_1$ replaces a strongly regularizable node $\eta$, $\eta_1$ remains in the proof - thus for any nodes $\eta_2'$ used in the old definition of pre-regularizable, it shouldn't matter that $\mathcal{R}(\eta)\cup\{\dual{\ell_1}\}$ is unifiable - all of those nodes $\eta_2'$ remain in the proof as well. \\

\noindent
The following theorem is what the reviewer is looking for. We require the additional notion of subsumption. We will use $X \sqsubseteq Y$ to denote the following for clauses $X$ and $Y$: there exists a substitution $\sigma$ such that $X\sigma \subseteq Y$. We say that $X$ \emph{subsumes} $Y$.

%We also make the following observations, for a first-order resolution node $\eta$ such that $\eta_1 \xrightarrow[\sigma_1]{\{\ell_1\} } \eta$, and $\eta_2 \xrightarrow[\sigma_2]{\{\ell_2\} } \eta$:
%$$\mathcal{S}(\eta_1) \sqsubseteq \mathcal{S}(\eta) \cup \{ \ell_1\sigma_1 \}  \eqno(1)$$
%$$\mathcal{S}(\eta_2) \sqsubseteq \mathcal{S}(\eta) \cup \{ \ell_2\sigma_2 \}  \eqno(2)$$


\begin{theorem}\label{thm:correct}
Let $\psi$ be a proof with root clause $\Gamma$, and $\eta\in \psi$ a node. Let $\psi' = \psi\setminus \{\eta\}$ and $\Gamma'$ be the root of $\psi'$. If $\eta$ is strongly regularizable, then $\Gamma' \sqsubseteq \Gamma$. %$\Gamma'=\Gamma$.
\end{theorem}

\begin{lemma}\label{lem:cor}
Let $\eta_1$ be a node and $\rho(\eta_1)$ be a path from $\eta_1$ to the root of the proof. Suppose that $\eta \in \rho(\eta_1)$ is a node such that $\eta_1 \sqsubseteq \mathcal{S}(\eta)$. If $\eta$ is replaced by $\eta_1$ in some proof $\psi$ to obtain $\psi'$, every literal $\ell_s \in \eta_1$ is either used as a pivot below $\eta_1$ in $\psi'$ or is contained in the root clause $\Gamma(\psi')$. \\
\end{lemma}

\begin{proof}
For a pair of nodes $\eta_1$, $\eta$ that satisfy the conditions of the lemma, let $\sigma_1$ be the substitution such that $\eta_1\sigma_1 \subseteq \mathcal{S}(\eta)$. Assume that $\eta_1 \xrightarrow[\sigma]{\{\ell_1\} } \eta$ in $\psi$.

We proceed by induction $h(\eta)$, the height of $\eta$ in $\psi$, which is the length of a longest path from the root to $\eta$. For the base case $h(\eta)=0$, when deleting $\eta$, $\eta$ is replaced by $\eta_1$ and by assumption there exists a $\sigma_1$ such that $\Gamma(\eta_1)\sigma_1 \subseteq \mathcal{S}(\eta) = \Gamma(\eta) \implies \Gamma(\eta_1) \sqsubseteq \Gamma(\eta)$. This concludes the base case; assume the result holds for any node $\eta_I$ with height $h(\eta_I) > 0$ and consider a node $\eta$ at height $h(\eta)=h(\eta_I)+1$.

For the inductive step, %deleting $\eta$ results in the replacement of $\eta$ by $\eta_1$ in every first-order resolution $\eta'$ that had $\eta$ as a parent. I
consider any path $\rho(\eta')$ from $\eta'$ to the root of the proof, and let $\eta''$ be the node which is resolved against $\eta$ in $\psi$. The deletion of $\eta$ from $\psi$ attempts to replace the resolution $\eta'=\eta \odot \eta''$ with $\eta' = \eta_1 \odot \eta''$.
%Thus, that for any first-order resolution node $\eta'$ such that $\eta \xrightarrow[\sigma_1']{\{\ell_1'\} } \eta'$, and $\eta'' \xrightarrow[\sigma_2']{\{\ell_2'\} } \eta'$, either $\eta_1$ and $\eta''$ are unfiable or they are not. 
%Let $P(\eta')$ be any path from $\eta'$ to the root of the proof. 
%Let $\rho(P)$ be the set of instantiated literals that are used for some unifying resolution node on $P(\eta_1)$. Since $\mathcal{S}(\eta)$ is the intersection of all paths from $\eta$ to the root, and there is a path that uses $\eta'$, $\mathcal{S}(\eta)\subseteq \rho(P(\eta'))$ for $P(\eta')$. 
For each path $\rho(\eta')$, there are two cases: either there exists an $\ell_1''\in \eta_1$ such that $\ell_1''\sigma_1$ can be used as the instantiated resolved literal between $\eta_1$ and $\eta'''$, or no such $\ell_1''$ exists.\\

\noindent
\emph{Case 1:}  $\eta_1 \xrightarrow[\sigma_1''=\sigma_1]{\{\ell_1''\} } \eta'$ and $\eta'' \xrightarrow[\sigma_2'']{\{\ell_2''\} } \eta'$ for some $\ell_1''$, $\ell_2''$, and $\sigma_2''$.
% there exists $\ell_1'' \in \eta_1$, $\ell_2'' \in \eta''$, $\sigma_1''$,  and $\sigma_2''$ so that $\eta_1$ and $\eta''$ are unifiable with $\sigma_1'' = \sigma_1$; i.e. $\eta_1 \xrightarrow[\sigma_1''=\sigma_1]{\{\ell_1''\} } \eta'$,
%and $\eta'' \xrightarrow[\sigma_2'']{\{\ell_2''\} } \eta'$. % If it is the case that $\sigma_1''=\sigma_1$, then $\ell_1''\sigma_1 \in \mathcal{S}(\eta)$, and
Since all instantiated literals of $\eta_1\sigma_1$ are safe, for each of the remaining literals $\ell_s \sigma_1 \in \Gamma(\eta_1)\sigma_1 \cap \Gamma(\eta')$ such that $\ell_s\neq \ell_1''$, there is a node $\eta_{\ell_s}\in \rho(\eta')$ that uses $\ell_s\sigma_1$ as a resolved literal or $\ell_s\sigma_1$ is contained in the root clause $\Gamma$; i.e. every remaining literal $\ell \in \eta_1$ that is not contained in $\Gamma$ will eventually be used as a resolved literal. The nodes using $\ell_{\eta''}\sigma_2'' \in (\Gamma(\eta'')\sigma_2''\cap\Gamma(\eta'))\setminus (\Gamma(\eta_1)\sigma_1)$ are unchanged, so these literals will still be used as a resolved literal for some node below $\eta'$. 
It remains to be shown that $\ell_1$ is still used as a resolved literal. To see this, recall that clauses are sets and that $\ell_1\sigma_1$ is safe. Therefore the resolution on $\rho(\eta')$ which uses $\ell_1\sigma_1$ as a resolved literal removes all copies\footnote{Note that the desired result can be obtained by inserting a contraction before performing resolution with $\eta'$ if clauses are defined as multi-sets.} of $\ell_1\sigma_1$. \\
%The instantiated literal $\ell_1\sigma_1$ is the only literal left; since $\ell_1\sigma_1$ is safe, it must be used as the instantiated resolved literal for some node $\eta_{\ell_1} \in P(\eta')$ below $\eta$; but since clauses are sets, this will remove all copies\footnote{Note that the desired result can be obtained by as inserting a contraction before performing resolution where $\ell_1\sigma_1$ is the instantiated resolved literal if clauses are defined as multi-sets.} of $\ell_1\sigma_1$. \\

\noindent
\emph{Case 2:} $\sigma_1$ cannot be used as a unifier for literals of $\eta_1$ and $\eta''$; i.e. resolution between $\eta_1$ and $\eta''$ is not possible for any $\ell_1''\in \eta_1$ %and $\ell_2''\in \eta''$ 
with the instantiated resolved literal $\ell_1''\sigma_1$. In this case, replace $\eta'$ by $\eta_1$; since $\ell_1''\sigma_1'' \notin \Gamma(\eta_1)\sigma_1$, every $\ell_s\sigma_1 \in \Gamma(\eta_1)\sigma_1$ must still be used as a resolved literal below $\eta'$, i.e. $\eta_1\sigma_1 \subseteq \mathcal{S}(\eta') \implies \eta_1 \sqsubseteq \mathcal{S}(\eta')$. Since $h(\eta') < h(\eta)=h(\eta_I)+1$, we are done by the induction hypothesis. 

%Finally, note that since $\eta$ is strongly regularizable, it is also pre-regularizable. Therefore, in either case if there exists an $\eta_2^*$ and $\eta^*\neq \eta$ such that  $\eta_1 \xrightarrow[\sigma_1^*]{\{\ell_1\} } \eta^*$,and $\eta_2* \xrightarrow[\sigma_2^*]{\{\ell_2^*\} } \eta^*$ for some $\sigma_1^*$, $\sigma_2^*$, $\ell_1$, and $\ell_2^*$, where $\ell_1\sigma_1$ is unifiable with $\ell^* \in \mathcal{S}(\eta)$, it must be the case that $\ell_2^* \in \mathcal{R}(\eta)$. By pre-regularizability, it must be that $\ell_2^*$ is unifiable with $\dual{\ell_1}\sigma_1$ by some unifier $\hat{\sigma}$. 
\hfill$\blacksquare$
\end{proof}

\begin{proof}[of Theorem \ref{thm:correct}]
Let $\psi$ be a proof with root clause $\Gamma$, and let $\eta_S \in \psi$ be a strongly regularizable node.  Let $\psi' = \psi\setminus \{\eta_S\}$ with root clause $\Gamma'$. To prove the theorem, it suffices to observe that any strongly regularizable node $\eta_S$ satisfies Lemma \ref{lem:cor}'s hypothesis for some $\rho(\eta_1)$.
%Without loss of generality, assume $\eta_1$ is the parent of $\eta$ that replaces $\eta$ in $\psi'$.  % 


\begin{figure}[bt]
\begin{centering}

\begin{tikzpicture}
  \tikzstyle{vertex}=[circle,minimum size=10pt,inner sep=0pt]
\tikzset{edge/.style = {->,> = latex'}}

    \node[vertex] (n1) at (-2,1) {$\eta_1$};
    \node[vertex] (n2) at (0,1) {$\eta_2$};
    \node[vertex] (n5) at (1,0.5) {$\eta''$};
    \node[vertex] (n3) at (-1,0.5) {$\eta$};
    \node[vertex] (r) at (0,0) {$\eta'$};

\draw[edge] (r) -- (n5);
\draw[edge] (r) -- (n3);
\draw[edge] (n3) -- (n1);
\draw[edge] (n3) -- (n2);


    \node[vertex] (m1) at (2,1) {$ $};
        \node[vertex] (m2) at (3,1) {$ $};
\draw[edge] (m1)  -- (m2);

    \node[vertex] (ndp) at (6,0.5) {$\eta''$};
    \node[vertex] (n) at (4,0.5) {$\eta_1$};
    \node[vertex] (rp) at (5,0) {$\eta'$};
\draw[edge] (rp) -- (ndp);
\draw[edge] (rp) -- (n);
\end{tikzpicture}

\end{centering}
\caption{The a layout of $\eta_1$ and $\eta$ in proofs $\psi$ (left) and $\psi\setminus\{\eta\}$ (right), as used in the proof of Lemma \ref{lem:cor}.}
\label{fig:dagex}
\end{figure}



\hfill$\blacksquare$
\end{proof}


\section{Other Corrections}

The set to which $\sigma$ is applied in Example 4.4 of the paper is wrong; it should read ``$\{\lnot p(X),\lnot q(X),\lnot r(X)\}$''.\\

\noindent
When the definitions are final, we'll need to check the pseudo-code in Algorithm 2 again. Note that the correction already discussed in the previous email also needs to be reflected in this algorithm.

\begin{footnotesize}
%\bibliographystyle{splncs}
\bibliographystyle{plain}
\bibliography{biblio}
\end{footnotesize}

\end{document}

% vim: tw=100
