
\section{First-Order Challenges}\label{sec:Challenges}


In this section, we describe challenges that have to be overcome in order to successfully adapt {\RecyclePivotsIntersection} to the first-order case. The first example illustrates the need to take unification into account. The other two examples discuss complex issues that can arise when unification is take into account in a naive way.

%straightforward example
\begin{example}\label{ex:unif} 
Consider the following proof $\psi$, noting that the proof is largely redundant. Naively computed, the safe literals for $\eta_3$ are $\{ \vdash q(c), ~ p(a,X)\}$. $\eta_1$ and $\eta_5$ and these two literals are unifiable. Further, the safe literals for $\eta_1$ includes $\eta_5$. Thus the proof can be regularized by recycling $\eta_1$.

\begin{footnotesize}
\begin{prooftree}
\def\e{\mbox{\ $\vdash$\ }}
\AxiomC{$\eta_1$: \e $p(W,X)$}
\AxiomC{$\eta_2$: $p(W,X)$ \e $q(c)$}
\BinaryInfC{$\eta_3$: \e $q(c)$}
\AxiomC{$\eta_4$: $q(c)$ \e $p(a,X)$}
\BinaryInfC{$\eta_5$: \e $p(a,X)$}
\AxiomC{$\eta_6$: $p(Y,b)$ \e }
\BinaryInfC{$\psi$: $\bot$}
\end{prooftree}
\end{footnotesize}

\noindent
Regularization of the proof by recycling $\eta_1$ results in deleting the edge between $\eta_2$ and $\eta_3$, which in turn replaces $\eta_3$ by $\eta_1$. Since $\eta_1$ cannot be resolved against $\eta_4$, and $\eta_1$ contained safe literals, $\eta_5$ is replaced by $\eta_1$. The result is the much shorter proof below.

\begin{footnotesize}
\begin{prooftree}
\def\e{\mbox{\ $\vdash$\ }}
\AxiomC{$\eta_1$: \e$p(W,X)$}
\AxiomC{$\eta_2$: $p(Y,b)$\e}
\BinaryInfC{$\psi'$: $\bot$}
\end{prooftree}
\end{footnotesize}

\noindent
Unlike in the propositional case, where the pivots and their corresponding safe literal list are all syntactically equal, in the first-order case, this is not necessarily the case. As illustrated above, $p(W,X)$ and $p(a,X)$ are not syntactically equal. Nevertheless, they are unifiable, and the proof can be regularized.
\end{example}

%unification necessary example
\begin{example}\label{ex:pairwise}

There are cases, as shown below, that require more careful care when attempting to regularize. Again, naively computed the safe literals for $\eta_3$ are $\{ \vdash q(c), ~ p(a,X)\}$, and so $\eta_1$ and $\eta_2$ appear to be candidates for regularization. 

\begin{footnotesize}
\begin{prooftree}
\def\e{\mbox{\ $\vdash$\ }}
\AxiomC{$\eta_1$: \e $p(a,c)$}
\AxiomC{$\eta_2$: $p(a,c)$ \e $q(c)$}
\BinaryInfC{$\eta_3$: \e $q(c)$}
\AxiomC{$\eta_4$: $q(c)$ \e $p(a,X)$}
\BinaryInfC{$\eta_5$: \e $p(a,X)$}
\AxiomC{$\eta_6$: $p(Y,b)$ \e }
\BinaryInfC{$\psi$: $\bot$}
\end{prooftree}
\end{footnotesize}


\noindent
However, if we attempt to regularize the proof, the same series of actions as in Example \ref{ex:unif} would result in the following resolution, which cannot be completed.

\begin{footnotesize}
\begin{prooftree}
\def\e{\mbox{\ $\vdash$\ }}
\AxiomC{$\eta_1$: \e$p(a,c)$}
\AxiomC{$\eta_2$: $p(Y,b)$\e}
\BinaryInfC{$\psi'$: ??}
\end{prooftree}
\end{footnotesize}

\end{example}

\noindent
The observations above lead to the idea of requiring pivots to satisfy the following property before collecting them to be reused.

\begin{definition}
\label{prop:pair}
Let $\eta$ be a clause with literal $\ell'$ with corresponding safe literal $\ell$ which is resolved against literals $\ell_1$, \ldots, $\ell_n$ in a proof $\psi$. $\eta$ is said to satisfy the \emph{pre-regularization unifiability property} in $\psi$ if $\ell_1$,\ldots,$\ell_n$, and $\dual{\ell'}$ are unifiable.
\end{definition}

\noindent
One technique to ensure this property is met is to apply the unifier of a resolution to each resolvent before computing the safe literals. In the case of Example \ref{ex:pairwise}, this would result in $\eta_1$ having the safe literals $\{ \vdash q(c),~p(a,b)\}$, and now it is clear that the literal in $\eta_1$ is not safe.

%TODO: continue from here
%extra check example
\begin{example}\label{ex:rootpair}
Satisfaction of the pre-deletion unifiability property is not enough. Deletion of the units from a proof $\psi$ may actually change the literals that had been resolved away by the units, because fewer substitutions are applied to them. This is exemplified below:

\begin{footnotesize}
\begin{prooftree}
\def\e{\mbox{\ $\vdash$\ }}
\AxiomC{$\eta_1$: $r(Y),p(X, q(Y, b)), p(X, Y)$\e}
\AxiomC{$\eta_2$: \e $p(U, V)$}
\BinaryInfC{$\eta_3$: $r(V),p(U, q(V, b))$\e}
\AxiomC{$\eta_4$: \e $r(W)$}
\BinaryInfC{$\eta_5$: $p(U, q(W, b))$\e}
\AxiomC{$\eta_2$}
\BinaryInfC{$\psi$: $\bot$}
\end{prooftree}
\end{footnotesize}

\noindent
If $\eta_2$ is collected for lowering and deleted from $\psi$, we obtain the proof $\dn{\psi}{\eta_2}$:

\begin{footnotesize}
\begin{prooftree}
\def\e{\mbox{\ $\vdash$\ }}
\AxiomC{$\eta'_1$: $r(Y),p(X, q(Y, b)), p(X, Y)$\e}
\AxiomC{$\eta'_4$: \e $r(W)$}
\BinaryInfC{$\eta'_5 (\psi')$: $p(X, q(W, b)), p(X, W)$\e}
\end{prooftree}
\end{footnotesize}

\noindent
Note that, even though $\eta_2$ satisfies the pre-deletion unifiability property (since $p(X, q(Y, b))$ and $p(U, q(W, b))$ are unifiable), $\eta_2$ still cannot be lowered and reintroduced by a single resolution inference, because the corresponding modified post-deletion literals $p(X, q(W, b))$ and $p(X, W)$ are actually not unifiable.
\end{example}

The observation above leads to the following stronger property:

\begin{definition}
\label{prop:rootpair}
Let $\eta$ be a unit with literal $\ell_{\eta}$ and let $\eta_1$, \ldots, $\eta_n$ be subproofs that are resolved with $\eta$ in a proof $\psi$, respectively, with resolved literals $\ell_1$, \ldots, $\ell_m$. 
$\eta$ is said to satisfy the \emph{post-deletion unifiability property} in $\psi$ if $\ell_1^{\dagger\downarrow}$,\ldots,$\ell_m^{\dagger\downarrow}$, and $\dual{\ell_{\eta}^{\dagger}}$ are unifiable, where $\ell^{\dagger}$ is the literal in $\dn{\psi}{\eta}$ corresponding to $\ell$ in $\psi$ and $\ell_k^{\dagger\downarrow}$ is the descendant of $\ell_k^{\dagger}$ in the root of $\dn{\psi}{\eta}$.
\end{definition}

%intersection example?