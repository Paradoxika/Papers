
\section{First-Order Challenges}\label{sec:Challenges}


In this section, we describe challenges that have to be overcome in order to successfully adapt {\RecyclePivotsIntersection} to the first-order case. The first example illustrates the need to take unification into account. The other two examples discuss complex issues that can arise when unification is take into account in a naive way.

%straightforward example
\begin{example}\label{ex:unif} 
Consider the following proof $\psi$, noting that the proof is largely redundant. Naively computed, the safe literals for $\eta_3$ are $\{ \vdash q(c), ~ p(a,X)\}$. $\eta_1$ and $\eta_5$ and these two literals are unifiable. Further, the safe literals for $\eta_1$ includes $\eta_5$. Thus the proof can be regularized by recycling $\eta_1$.

\begin{footnotesize}
\begin{prooftree}
\def\e{\mbox{\ $\vdash$\ }}
\AxiomC{$\eta_1$: \e $p(W,X)$}
\AxiomC{$\eta_2$: $p(W,X)$ \e $q(c)$}
\BinaryInfC{$\eta_3$: \e $q(c)$}
\AxiomC{$\eta_4$: $q(c)$ \e $p(a,X)$}
\BinaryInfC{$\eta_5$: \e $p(a,X)$}
\AxiomC{$\eta_6$: $p(Y,b)$ \e }
\BinaryInfC{$\psi$: $\bot$}
\end{prooftree}
\end{footnotesize}

\noindent
Regularization of the proof by recycling $\eta_1$ results in deleting the edge between $\eta_2$ and $\eta_3$, which in turn replaces $\eta_3$ by $\eta_1$. Since $\eta_1$ cannot be resolved against $\eta_4$, and $\eta_1$ contained safe literals, $\eta_5$ is replaced by $\eta_1$. The result is the much shorter proof below.

\begin{footnotesize}
\begin{prooftree}
\def\e{\mbox{\ $\vdash$\ }}
\AxiomC{$\eta_1$: \e$p(W,X)$}
\AxiomC{$\eta_2$: $p(Y,b)$\e}
\BinaryInfC{$\psi'$: $\bot$}
\end{prooftree}
\end{footnotesize}

\noindent
Unlike in the propositional case, where the pivots and their corresponding safe literal list are all syntactically equal, in the first-order case, this is not necessarily the case. As illustrated above, $p(W,X)$ and $p(a,X)$ are not syntactically equal. Nevertheless, they are unifiable, and the proof can be regularized.
\end{example}

%unification necessary example
\begin{example}\label{ex:pairwise}

There are cases, as shown below, that require more careful care when attempting to regularize. Again, naively computed the safe literals for $\eta_3$ are $\{ \vdash q(c), ~ p(a,X)\}$, and so $\eta_1$ and $\eta_2$ appear to be candidates for regularization. 

\begin{footnotesize}
\begin{prooftree}
\def\e{\mbox{\ $\vdash$\ }}
\AxiomC{$\eta_1$: \e $p(a,c)$}
\AxiomC{$\eta_2$: $p(a,c)$ \e $q(c)$}
\BinaryInfC{$\eta_3$: \e $q(c)$}
\AxiomC{$\eta_4$: $q(c)$ \e $p(a,X)$}
\BinaryInfC{$\eta_5$: \e $p(a,X)$}
\AxiomC{$\eta_6$: $p(Y,b)$ \e }
\BinaryInfC{$\psi$: $\bot$}
\end{prooftree}
\end{footnotesize}


\noindent
However, if we attempt to regularize the proof, the same series of actions as in Example \ref{ex:unif} would result in the following resolution, which cannot be completed.

\begin{footnotesize}
\begin{prooftree}
\def\e{\mbox{\ $\vdash$\ }}
\AxiomC{$\eta_1$: \e$p(a,c)$}
\AxiomC{$\eta_2$: $p(Y,b)$\e}
\BinaryInfC{$\psi'$: ??}
\end{prooftree}
\end{footnotesize}

\end{example}

\noindent
The observations above lead to the idea of requiring pivots to satisfy the following property before collecting them to be reused.

\begin{definition}
\label{prop:pair}
Let $\eta$ be a clause with literal $\ell'$ with corresponding safe literal $\ell$ which is resolved against literals $\ell_1$, \ldots, $\ell_n$ in a proof $\psi$. $\eta$ is said to satisfy the \emph{pre-regularization unifiability property} in $\psi$ if $\ell_1$,\ldots,$\ell_n$, and $\dual{\ell'}$ are unifiable.
\end{definition}

\noindent
One technique to ensure this property is met is to apply the unifier of a resolution to each resolvent before computing the safe literals. In the case of Example \ref{ex:pairwise}, this would result in $\eta_1$ having the safe literals $\{ \vdash q(c),~p(a,b)\}$, and now it is clear that the literal in $\eta_1$ is not safe.

%TODO: continue from here
%extra check example
\begin{example}\label{ex:unifcheck}
Due to complications with unification, there are also cases where satisfying the pre-regularization unifiability property is not sufficient to attempt regularization. Consider the proof of $\psi$ below. After collecting the safe literals, $\eta_3$'s safe literals are $\{q(B,V),p(c,d)\vdash q(f(a,e),c)\}$.

\begin{footnotesize}
\begin{prooftree}
\def\e{\mbox{\ $\vdash$\ }}
\AxiomC{$\eta_8$: $Q(f(ae)c)\e$\hspace{-1cm}}
\AxiomC{$\eta_6$: $\e P(cd)$\hspace{-4.5cm}}
\AxiomC{$\eta_1$: $P(UV)\e Q(f(aV)U)$}
\AxiomC{$\eta_2$: $Q(f(aX)Y)Q(BX)\e Q(f(aZ)Y)$}
\BinaryInfC{$\eta_3$: $P(UV)Q(BV)\e Q(f(aZ)U)$\hspace{-3.5cm}}
\AxiomC{\hspace{-1.5cm} $\eta_4$: $\e Q(RS)$}
\BinaryInfC{$\eta_5$: $P(UV)\e Q(f(aZ)U)$}
\BinaryInfC{$\eta_7$: $\e Q(f(aZ)c)$}
\BinaryInfC{$\psi$: $\bot$}
\end{prooftree}
\end{footnotesize}

\noindent
Since $q(f(a,X),Y) \in \eta_2$ and $q(B,V)$ (in $\eta_3$'s safe literals) are unifiable, regularization would be attempted. However, in this case, we would mark the edge between $\eta_2$ and $\eta_3$ for deletion, and as a result, $\eta_3$ will be replaced with $\eta_1$. $\eta_1$ does not contain the required pivot for $\eta_5$, and so $\eta_5$ is also replaced with $\eta_1$, and resolution is attempted before $\eta_1$ and $\eta_6$, which results in $\eta_7'$, and an inability to complete the proof, as shown below.

\begin{footnotesize}
\begin{prooftree}
\def\e{\mbox{\ $\vdash$\ }}
\AxiomC{$\eta_8$: $Q(f(ae)c)\e$}
\AxiomC{$\eta_6$: $\e P(cd)$}
\AxiomC{$\eta_1$: $P(UV)\e Q(f(aV)U)$}
\BinaryInfC{$\eta_7'$: $\e Q(f(ad)c)$}
\BinaryInfC{$\psi'$: ??}
\end{prooftree}
\end{footnotesize}

In order to avoid these scenarios, we perform an additional check during edge deletion. The node $\eta*$ which will replace a resolution $\eta$ (because $\eta$ would have a deleted parent), must be entirely contained, via unification which modifies only $\eta^*$'s variables, in the safe literals of $\eta$. In the case of this example, $\eta_1$ would not satisfy this property: in order to unify with $\eta_3$'s safe literals, it would be necessary to send $V\rightarrow Z$ due to $\eta_1$'s second literal, but leave $V$ unchanged due $\eta_1$'s first literal, which is not possible. Note that this check is not necessary in the propositional case, as the replacement node would be contained exactly in the set of safe literals.

\end{example}


%intersection example?