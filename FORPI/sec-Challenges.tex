
\section{First-Order Challenges}\label{sec:Challenges}


In this section, we describe challenges that have to be overcome in order to successfully adapt {\RPI} to the first-order case. The first example illustrates the need to take unification into account. The other two examples discuss complex issues that can arise when unification is taken into account in a naive way.

%straightforward example
\begin{example}\label{ex:unif} 
Consider the following proof $\psi$. When computed as in the propositional case, the safe literals for $\eta_3$ are $\{ q(c), ~ p(a,X)\}$.
%$\eta_1$'s literal is unifiable with $p(a,X)$, which is inherited from $\eta_3$'s safe literals. 
%Thus the proof can be regularized by recycling $\eta_1$.
%$\eta_1$ and $\eta_5$ and these two literals are unifiable. Further, the safe literals for $\eta_1$ includes $\eta_5$. Thus the proof can be regularized by recycling $\eta_1$.

\begin{scriptsize}
\begin{prooftree}
\def\e{\mbox{\ $\vdash$\ }}
\AxiomC{$\eta_6$: $p(Y,b)$ \e \hspace{-2cm}}
\AxiomC{$\eta_1$: \e $p(W,X)$}
\AxiomC{$\eta_2$: $p(W,X)$ \e $q(c)$}
\BinaryInfC{$\eta_3$: \e $q(c)$  \hspace{-1.5cm}}
\AxiomC{$\eta_4$: $q(c)$ \e $p(a,X)$}
\BinaryInfC{$\eta_5$: \e $p(a,X)$}
\BinaryInfC{$\psi$: $\bot$}
\end{prooftree}
\end{scriptsize}

\noindent
As neither of $\eta_3$'s pivots is syntactically equal to a safe literal, the propositional {\RPI} algorithm would not change $\psi$. However, $\eta_3$'s left pivot $p(W,X)\in \eta_1$ is unifiable with the safe literal $p(a,X)$. Regularizing $\eta_3$, by deleting the edge between $\eta_2$ and $\eta_3$ and replacing $\eta_3$ by $\eta_1$, leads to further deletion of $\eta_4$ (because it is not resolvable with $\eta_1$) and finally to the much shorter proof below.

\begin{footnotesize}
\begin{prooftree}
\def\e{\mbox{\ $\vdash$\ }}
\AxiomC{$\eta_1$: \e$p(W,X)$}
\AxiomC{$\eta_6$: $p(Y,b)$\e}
\BinaryInfC{$\psi'$: $\bot$}
\end{prooftree}
\end{footnotesize}

\end{example}

\noindent
Unlike in the propositional case, where a pivot must be syntactically equal to a safe literal for regularization to be possible, the example above suggests that, in the first-order case, it might suffice that a pivot be unifiable with a safe literal. However, there are cases, as shown in the example below, where mere unifiability is not enough and greater care is needed.

%unification necessary example
\begin{example}\label{ex:pairwise}

Again, the safe literals for $\eta_3$, when computed as in the propositional case, are $\{q(c), ~ p(a,X)\}$, and as the pivot $p(a,c)$ is unifiable with the safe literal $p(a,X)$, $\eta_3$ appears to be a candidate for regularization. 

\begin{scriptsize}
\begin{prooftree}
\def\e{\mbox{\ $\vdash$\ }}
\AxiomC{$\eta_6$: $p(Y,b)$ \e \hspace{-2cm}}
\AxiomC{$\eta_1$: \e $p(a,c)$}
\AxiomC{$\eta_2$: $p(a,c)$ \e $q(c)$}
\BinaryInfC{$\eta_3$: \e $q(c)$}
\AxiomC{$\eta_4$: $q(c)$ \e $p(a,X)$}
\BinaryInfC{$\eta_5$: \e $p(a,X)$}
\BinaryInfC{$\psi$: $\bot$}
\end{prooftree}
\end{scriptsize}

\begin{figure*}
\begin{small}
\begin{prooftree}
\def\e{\mbox{\ $\vdash$\ }}
\AxiomC{$\eta_8$: $q(f(a,e),c)\e$}
\AxiomC{$\eta_6$: $\e p(c,d)$ \hspace{-2cm}}
\AxiomC{$\eta_1$: $p(U,V)\e q(f(a,V),U)$}
\AxiomC{$\eta_2$: $q(f(a,X),Y),q(T,X)\e q(f(a,Z),Y)$}
\BinaryInfC{$\eta_3$: $p(U,V),q(T,V)\e q(f(a,Z),U)$}
\AxiomC{\hspace{-1cm} $\eta_4$: $\e q(R,S)$}
\BinaryInfC{$\eta_5$: $p(U,V)\e q(f(a,Z),U)$}
\BinaryInfC{$\eta_7$: $\e q(f(a,Z),c)$}
\BinaryInfC{$\psi$: $\bot$}
\end{prooftree}
\end{small}
\caption{An example where pre-regularizability is not sufficient.}
\label{fig:ex-unifcheck}
\end{figure*}

\noindent
However, if we attempt to regularize the proof, the same series of actions as in Example \ref{ex:unif} would 
require resolution between $\eta_1$ and $\eta_6$, which is not possible.
%result in the following resolution, which cannot be completed.
%\begin{footnotesize}
%\begin{prooftree}
%\def\e{\mbox{\ $\vdash$\ }}
%\AxiomC{$\eta_1$: \e$p(a,c)$}
%\AxiomC{$\eta_6$: $p(Y,b)$\e}
%\BinaryInfC{$\psi'$: ??}
%\end{prooftree}
%\end{footnotesize}

\end{example}
One way to prevent the problem depicted above would be to require the pivot to be not only unifiable but in fact more general than a safe literal. A weaker (and better) requirement is possible, and requires a slight modification to the concept of safe literals. 

%\begin{definition}
%\emph{First-order safe literals} for a node $\eta$, denoted $\mathcal{S}(\eta)$, are a set of instantiated resolved literals used as pivots in all paths below $\eta$ in the proof, or that occur in the root clause of the proof.
%\end{definition}

\begin{definition}
\emph{First-order safe literals} for a node $\eta$ in a proof $\psi$, denoted $\mathcal{S}(\eta)$, are a set of instantiated resolved literals $\ell\sigma$ such that $\abs{\ell\sigma}$ is the pivot for some node $\eta'$ below $\eta$ on all paths below $\eta$, or $\ell\sigma \in \rho(\psi)$ where $\rho(\psi)$ denotes the root node of $\psi$.
\end{definition}

First-order safe literals are initially set to the root of the proof, and further safe literals are obtained by applying the unifier of the resolution step to the each pivot before adding it to the set of safe literals (cf. algorithm 3, lines 8 and 10). Note that a safe literal $\ell\sigma$ may be such that $\sigma=\emptyset$, e.g. as in those instantiated literals contained in the root of the proof. First-order safe literals will be referred to simply as safe literals when context implies a first-order setting.


 In the case of Example \ref{ex:pairwise}, computing safe literals as defined above would result in $\mathcal{S}(\eta_3)=\{q(c),~p(a,b)\}$, where clearly the pivot $p(a,c)$ in $\eta_1$ is not safe. This requirement is formalized below.

%\noindent
%Another way to prevent the problem depicted above would be to require the pivot to be not only unifiable but in fact more general than a safe literal. A weaker (and better) requirement is possible, however, as defined below.


%old - kept for reference/readability.
%\begin{definition}
%\label{prop:pair}
%Let $\eta$ be a node with safe literals $\mathcal{S}(\eta)$ such that resolved literal $\ell'$ is unifiable with a safe literal $\ell \in \mathcal{S}(\eta)$ where $\ell'$ is resolved against literals $\ell_1$, \ldots, $\ell_n$ in a proof $\psi$. The node $\eta$ is said to satisfy the \emph{pre-regularization unifiability property} in $\psi$ if $\ell_1$,\ldots,$\ell_n$, and $\dual{\ell'}$ are unifiable.
%\end{definition}
%Let $\mathcal{P}(\eta)$ (resp. $\mathcal{R}(\eta)$) be the set of all nodes $\eta_1$ (resp. resolved literals $\ell_1$) such that $\eta_2 \xrightarrow[\sigma_2]{\{\ell_2\} } \eta'$ and $\eta_1 \xrightarrow[\sigma_1]{\{\ell_1\} } \eta'$ for some $\eta'$.

\begin{definition}
\label{prop:pair}
Let $\eta$ be a node with safe literals $\mathcal{S}(\eta)$ and parents $\eta_1$ and $\eta_2$, assuming without loss of generality, $\eta_1 \xrightarrow[\sigma_1]{\{\ell_1\} } \eta$ such that $\ell_1$ is unifiable with a safe literal $\ell^* \in \mathcal{S}(\eta)$. 
Let $\mathcal{R}(\eta)$ be the set of all resolved literals $\ell_2$ such that $\eta_1 \xrightarrow[\sigma_1']{\{\ell_1\} } \eta'$, $\eta_2' \xrightarrow[\sigma_2']{\{\ell_2\} } \eta'$, and $\ell_1\sigma_1'=\dual{\ell_2}\sigma_2'$, for some nodes $\eta_2'$ and $\eta'$ and unifiers $\sigma_1'$ and $\sigma_2'$.
The node $\eta$ is said to be \emph{pre-regularizable} in the proof $\psi$ if all elements in $\mathcal{R}(\eta) \cup \{ \dual{\ell_1}\}$ are unifiable.
\end{definition}

\noindent
This property states that a node is pre-regularizable if, for a safe resolved literal $\ell'$ which is resolved against literals $\ell_1$, \ldots, $\ell_n$ in a proof $\psi$, $\ell_1$,\ldots,$\ell_n$, and $\dual{\ell'}$ are unifiable.




%extra check example
\begin{example}\label{ex:unifcheck}


Satisfying the pre-regularizability is not sufficient. Consider the proof $\psi$ in Figure \ref{fig:ex-unifcheck}. After collecting the safe literals, $\mathcal{S}(\eta_3) = \{\lnot q(R,S),\lnot p(c,d), q(f(a,e),c)\}$.
%\noindent
$\eta_3$'s pivot $q(f(a,V),U)$ is unifiable to (and even more general than) the safe literal $q(f(a,e),c)$. Attempting to regularize $\eta_3$ would lead to the removal of $\eta_2$, the replacement of $\eta_3$ by $\eta_1$ and the removal of $\eta_4$ (because $\eta_1$ does not contain the pivot required by $\eta_5$), with $\eta_5$ also being replaced by $\eta_1$. Then resolution between $\eta_1$ and $\eta_6$ results in $\eta_7'$, which cannot be resolved with $\eta_8$, as shown below.


\begin{scriptsize}
\begin{prooftree}
\def\e{\mbox{\ $\vdash$\ }}
\AxiomC{$\eta_8$: $q(f(a,e),c)\e$ \hspace{-0.5cm}}
\AxiomC{$\eta_6$: $\e p(c,d)$}
\AxiomC{$\eta_1$: $p(U,V)\e q(f(a,V),U)$}
\BinaryInfC{$\eta_7'$: $\e q(f(a,d),c)$}
\BinaryInfC{$\psi'$: ??}
\end{prooftree}
\end{scriptsize}

\noindent
$\eta_1$'s literal $q(f(a, V), U)$, which would be resolved with $\eta_8$'s literal, was changed to $q(f(a,d),c)$ due to the resolution between $\eta_1$ and $\eta_6$.
\end{example}



%ToDo: The following paragraph is not understandable. It must be made clearer. It must be made more formal (as a definition) and more mathematically precise (with the same level of detail as definition 2). It must explain \textbf{why} this additional check suffices.

\noindent
Thus we additionally require that the following property be satisfied.
\begin{definition}
\label{prop:extracheck}
Let $\eta$ be pre-regularizable, with safe literals $\mathcal{S}(\eta)$ and parents $\eta_1$ and $\eta_2$, assuming without loss of generality that
%without loss generality 
$\eta_2 \xrightarrow[\sigma_2]{\{\ell_2\} } \eta$ and $\dual{\ell_2}$ is unifiable with some $\ell^* \in \mathcal{S}(\eta)$. 
%TODO \marginpar{explain, mathematically, what is assumed about $\eta_2$ and $\ell_2$ here}  %Done? This used to say that $\eta_2$ was a deleted node, so that must mean that $\eta_1$ contains the resolved literal that is safe, i.e. $\eta_1$ contains $\ell_1 \in \mathcal{S}(\eta)$, and we have that $\ell_1 = \dual{\ell_2}$
%is marked as a \texttt{deletedNode}
%in a proof $\psi$
The node $\eta$ is said to be \emph{strongly regularizable} in $\psi$ if there exists a substitution $\sigma$ such that $\eta_1\sigma \subseteq \mathcal{S}(\eta)$.
\end{definition}
%we ensure that the replacement parent is (possibly after unification) contained entirely in the safe literals. 
This property ensures that the remainder of the proof does not expect a variable in $\eta_1$ to be unified to different values simultaneously. This property is not necessary in the propositional case, as the replacement node would not change lower in the proof. 


%In order to avoid these scenarios, we perform an additional check during inference removal. The node $\eta*$ which will replace a resolution $\eta$ (because $\eta$ would have a deleted parent), must be entirely contained, via unification which modifies only $\eta^*$'s variables, in the safe literals of $\eta$. 
%%In this example, $\eta_1$ does not satisfy this property: in order to unify with $\eta_3$'s safe literals, it would be necessary to send $V\rightarrow Z$ due to $\eta_1$'s second literal, but leave $V$ unchanged due $\eta_1$'s first literal, which is not possible. This check is not necessary in the propositional case, as the replacement node would be contained exactly in the set of safe literals, and would not change lower in the proof.
%In this example, $\eta_1$ does not satisfy this property. This check is not necessary in the propositional case, as the replacement node would be contained exactly in the set of safe literals, and would not change lower in the proof.



An alternative to the strong regularizability may also be useful for proof compression in some cases. This alternative relies on knowledge of how literals are changed after the deletion of a node in a proof (similar to the \emph{post-deletion unifiability property} observed for {\FOLowerUnits} in \cite{GFOLU}).  
\begin{definition}\label{def:postdelprop}
Let $\eta$ be a pre-regularizable node with parents $\eta_1$ and $\eta_2$, assuming without loss of generality that $\eta_1 \xrightarrow[\sigma_1]{\{\ell_1\} } \eta$ 
%and $\eta_2 \xrightarrow[\sigma_2]{\{\ell_2\} } \eta$ 
such that $\ell_1$ is unifiable with some $\ell^* \in \mathcal{S}(\eta)$.
For each safe literal $\ell = \ell_s\sigma_s \in \mathcal{S}(\eta_1)$, let $\eta_\ell$ be a node on the path from $\eta$ to the root of the proof such that $\abs{\ell}$ is the pivot of $\eta_\ell$.
Let $\mathcal{R}(\eta_\ell)$ be the set of all resolved literals $\ell_s'$ such that $\eta_2' \xrightarrow[\sigma_s]{\{\ell_s\} } \eta_\ell$, $\eta_1' \xrightarrow[\sigma_s']{\{\ell_s'\} } \eta_\ell$, and $\ell_s\sigma_s=\dual{\ell_s'}\sigma_s'$, for some nodes $\eta_2'$ and $\eta_1'$ and unifier $\sigma_s'$; if no such node $\eta_\ell$ exists, define $\mathcal{R}(\eta_\ell)=\emptyset$.
% and $\sigma_2'$.
The node $\eta$ is said to be \emph{weakly regularizable} in $\psi$ if, for all $\ell \in \mathcal{S}(\eta_1)$, all elements in $\mathcal{R}^{\dagger}(\eta_\ell) \cup \{ \dual{\ell}^\dagger \}$ are unifiable, where $\dual{\ell}^{\dagger}$ is the literal in $\dn{\psi}{\eta_2}$ corresponding to $\dual{\ell}$ in $\psi$ and $\mathcal{R}^{\dagger}(\eta_\ell)$ is the set of literals in $\dn{\psi}{\eta_2}$ corresponding to literals of $\mathcal{R}(\eta_\ell)$ in $\psi$.
\end{definition}


%old
%Let $\mathcal{R}(\eta)$ be the set of resolved literals $\dual{\ell_2}$ contained in some conclusion of a node in $\mathcal{P}(\eta)$.
%Let $\eta$ be a node with safe literals $\mathcal{S}(\eta)$. Consider $p\in \mathcal{S}(\eta)$ and let $\eta_1$, \ldots, $\eta_n$ be subproofs that are resolved using $p$ in a proof $\psi$, respectively, with resolved literals $\ell_1$, \ldots, $\ell_m$. 
%The node $\eta$ is said to satisfy the \emph{post-deletion unifiability property} in $\psi$ if, for all $p\in \mathcal{S}(\eta)$, $\ell_1^{\dagger}$,\ldots,$\ell_m^{\dagger}$, and $\dual{p^{\dagger}}$ are unifiable, where $\ell^{\dagger}$ is the literal in $\dn{\psi}{\eta}$ corresponding to $\ell$ in $\psi$.


Informally, a node $\eta$ is weakly regularizable in a proof if it can be replaced by one of its parents $\eta_1$, such that for each $\ell \in \mathcal{S}(\eta_1)$, $\abs{\ell}$ can still be used as pivots in order to complete the proof. Weakly regularizable nodes differ from strongly regularizable nodes by not requiring the entire parent $\eta_1$ replacing the resolution $\eta$ to be simultaneously matched to a subset of $\mathcal{S}(\eta)$, and requires knowledge of how literals will be instantiated after the removal of $\eta_2$ and $\eta$ from the proof.

Note that there is always at least one node $\eta_\ell$ as used in the definition for any safe literal which was not contained in the root clause of the proof: the node which resulted in $\ell$ being a safe literal for the path from $\eta$ to the root of the proof. Furthermore, it does not matter which node $\eta_\ell$ is used. To see this, consider some $\eta_\ell' \neq \eta_\ell$ for some $\ell$ with the same pivot. Consider arbitrary nodes $\eta_1$ and $\eta_2$ such that  $\eta_2 \xrightarrow[ ]{\{\ell\} } \eta_\ell$ and $\eta_1 \xrightarrow[\sigma_1]{\{\ell_1\} } \eta_\ell$ where $\ell=\dual{\ell_1}\sigma_1$. Now consider arbitrary nodes $\eta_1'$ and $\eta_2'$ such that  $\eta_2' \xrightarrow[ ]{\{\ell\} } \eta_\ell'$ and $\eta_1' \xrightarrow[\sigma_1']{\{\ell_1'\} } \eta_\ell'$ where $\ell=\dual{\ell_1'}\sigma_1'$. Since the pivots for $\eta_\ell$ and $\eta_\ell'$ are equal, we must have that %$\abs{\ell\sigma_2}=\abs{\ell\sigma_2'}$ and furthermore that 
$\abs{\ell}=\abs{\ell_1\sigma_1}$ and $\abs{\ell}=\abs{\ell_1'\sigma_1'}$, and thus $\abs{\ell_1\sigma_1}=\abs{\ell_1'\sigma_1'}$. This shows that it does not matter which $\eta_\ell$ we use; the instantiated resolved literals will always be equal implying that both of the resolved literals $\ell_1$ and $\ell_1'$ will be contained in both $\mathcal{R}(\eta_\ell)$ and $\mathcal{R}(\eta_\ell')$.

%wrong:
%since $\eta$ is assumed to be pre-regularizable, we have that one literal $\ell' \in \mathcal{R}(\eta)$ of $\eta_\ell$ (recall that the resolved literals of $\eta$ and $\eta_\ell$ are the same) is unifiable with $\dual{\ell}$.

\begin{example}
This example illustrates the usefulness of weak regularizability as an alternative to the strong regularizability. 
In the proof below, note that $\eta_6$ is not satisfy the strongly regularizable: there is no unifier $\sigma$ such that $\eta_6\sigma \subseteq \mathcal{S}(\eta_6)$ (where $\mathcal{S}(\eta_6) =\{ \lnot p(W), \lnot r(Z), \lnot q(Y)\}$).
\begin{scriptsize}
\begin{prooftree}
\def\e{\mbox{\ $\vdash$\ }}
\AxiomC{$\eta_1$: $\e p(U)$ \hspace{-2cm}}
\AxiomC{$\eta_5$: $p(Z) \e q(Z)$ \hspace{-0.5cm}}
\AxiomC{$\eta_8$: $p(X),q(X),r(X)\e$}
\AxiomC{$\eta_7$: $\e p(Y)$  \hspace{-1cm}}
\BinaryInfC{$\eta_6$: $q(Y),r(Y)\e$ }
\BinaryInfC{$\eta_4$: $p(Z),r(Z)\e$ \hspace{-2cm} }
\AxiomC{ \hspace{-1cm} $\eta_3$: $\e r(W)$}
\BinaryInfC{ $\eta_2$: $p(W)\e$}

\BinaryInfC{$\psi$: $\bot$}
\end{prooftree}

\end{scriptsize}
\noindent
We show that $\eta_6$ is weakly regularizable, and that $\eta_7$ can be removed. To do this, first observe that $\eta_6$ is pre-regularizable, since $\lnot p(X)$ is unifiable with $\lnot p(W)\in \mathcal{S}(\eta_6)$ and $\mathcal{R}(\eta_6) \cup \{\dual{\lnot p(W)}\}$ is unifiable (where $\mathcal{R}(\eta_6)=\{p(U), p(Y)\}$).
Consider the following proof of $\psi \setminus \{\eta_7\}$:
\begin{scriptsize}
\begin{prooftree}
\def\e{\mbox{\ $\vdash$\ }}
\AxiomC{$\eta_1$: $\e p(U)$ \hspace{-1.75cm}}
\AxiomC{$\eta_8$: $p(X),q(X),r(X)\e$}
\AxiomC{$\eta_5$: $p(Z) \e q(Z)$}
\BinaryInfC{$\eta_4'$: $p(Z), p(Z),r(Z)\e$}
\UnaryInfC{$\eta_4$: $p(Z),r(Z)\e$}
\AxiomC{$\eta_3$: $\e r(W)$}
\BinaryInfC{$\eta_2$: $p(W)\e$}
\BinaryInfC{$\psi$: $\bot$}
\end{prooftree}
\end{scriptsize}
Now observe that for each $\ell \in \mathcal{S}(\eta_8)=\{\lnot  q(Y), \lnot r(Z), \lnot p(W)\}$ we have the following, so that $\eta_6$ is weakly regularizable:
\begin{itemize}
\item $\ell=\lnot  q(Y)$: $\ell^\dagger = \lnot q(X)$ which is unifiable with $\dual{\ell}^\dagger=q(Z)$
\item $\ell=\lnot r(Z)$: $\ell^\dagger = \lnot r(Z)$ which is unifiable with $\dual{\ell}^\dagger=r(W)$
\item $\ell=\lnot p(W)$: $\ell^\dagger = \lnot p(W)$ which is unifiable with $\dual{\ell}^\dagger=p(U)$
\end{itemize}
\end{example}

If a node is pre-regularizable, then it is also weakly regularizable. Thus strongly regularizable nodes are also weakly regularizable nodes.

\begin{thm}
Let $\eta$ be a node that is pre-regularizable in some proof. Then $\eta$ is weakly regularizable.
%Let $\eta$ be a node that is strongly regularizable in some proof. Then $\eta$ is weakly regularizable.
\end{thm}

\begin{proof}
%Let $\eta$ be a strongly regularizable node with parents $\eta_1$ and $\eta_2$. By definition, $\eta$ is also pre-regularizable.
%Assume without loss of generality that $\eta_1\sigma \subseteq \mathcal{S}(\eta)$.

Let $\eta$ be a pre-regularizable node with parents $\eta_1$ and $\eta_2$. Let $\mathcal{R}(\eta_\ell)$ and $\mathcal{R}^\dagger(\eta_\ell)$ be defined as in Definition \ref{def:postdelprop} for a safe literal literal $\ell \in \mathcal{S}(\eta_1)$.

Let $\ell \in \mathcal{S}(\eta_1)$ be a safe literal of $\eta_1$ that is contained in the root clause of the proof such that there does not exists a node $\eta_\ell$ below $\eta$ whose pivot is $\abs{\ell}$. Then $\mathcal{R}^\dagger(\eta_\ell)\cup\{\dual{\ell}\}=\emptyset\cup\{\dual{\ell}\}=\{\dual{\ell}\}$ is trivially unifiable. Thus we may assume that for all $\ell\in\mathcal{S}(\eta_1)$, such a node $\eta_\ell$ exists.

%Let $\ell \in \mathcal{S}(\eta_1)$ be a safe literal of $\eta_1$ that is not contained in the root clause of the proof, and let $\eta_\ell$ be a node on the path from $\eta_1$ to the root of the proof such that $\abs{\ell}$ is used as the pivot. We claim that at least one such node $\eta_\ell$ exists. To see this, recall that since $\eta$ is strongly regularizable, the resolved literal of $\eta$ was unifiable with some $\ell^*\in \mathcal{S}(\eta)$, so $\eta$ could not have provided a literal to $\mathcal{S}(\eta_1)$ that does not also appear below $\eta$. Thus $\mathcal{S}(\eta_1)\subseteq \mathcal{S}(\eta)$. Note that we cannot have $\mathcal{S}(\eta)\subsetneq \mathcal{S}(\eta_1)$, as otherwise a literal $\ell \in \mathcal{S}(\eta_1)$ have been added from a path from $\eta_1$ to the root avoiding $\eta$, contradicting $\ell$ appearing in all paths below $\eta_1$, or $\ell$ appearing in the root clause (as if this were the case, then we would also have $\ell \in \mathcal{S}(\eta)$). Since $\mathcal{S}(\eta_1)\subseteq \mathcal{S}(\eta)$, there must be a node below $\eta$ that used $\abs{\ell}$ as a pivot in order to have $\ell \in \mathcal{S}(\eta_1)$. 


If $\ell \notin \eta_1$, then $\ell^\dagger =\ell$ and $\dual{\ell}^\dagger=\dual{\ell}$, there is nothing to prove (neither $\ell$ or $\dual{\ell}$ have changed in $\psi\setminus\{\eta_2\}$). So we may assume $\ell\in \eta_1$.

Consider $\dual{\ell} \in \mathcal{R}(\eta_\ell)$: $\ell$ and $\dual{\ell}$ are unifiable in $\psi$ by definition of $\mathcal{R}(\eta_\ell)$. We will show that $\ell^\dagger$ and $\dual{\ell}^\dagger$ are unifiable in $\psi\setminus\{\eta_2\}$, where $\dual{\ell}^\dagger\in \mathcal{R}^\dagger(\eta_\ell)$.

%Let $\ell_s = \ell_s'\sigma_s' \in \mathcal{S}(\eta)$ be the safe literal such that $\ell\sigma=\ell_s$, which exists since $\eta$ is strongly regularizable.
Since $\eta_\ell$ exists (with $\abs{\ell}$ as a pivot by definition), there exists nodes $\eta_L$ and $\eta_R$ such that $\eta_L \xrightarrow[\sigma_L]{\{\ell_L\} } \eta_\ell$ and $\eta_R \xrightarrow[\sigma_R]{\{\ell_R\} } \eta_\ell$ for some $\ell_L,\ell_R,\sigma_L$, and $\sigma_R$.
Since $\abs{\ell}$ was the pivot, we have that $\abs{\ell}=\abs{\ell_L\sigma_L}$ or $\abs{\ell}=\abs{\ell_R\sigma_R}$. Without loss of generality, assume that  $\abs{\ell}=\abs{\ell_L\sigma_L}$.
Thus we can write
$$\ell = \ell_L\sigma_L =\dual{\ell_R}\sigma_R$$
Note that $\dual{\ell}^\dagger=\dual{\ell_R}$ as $\dual{\ell_R}$ is unchanged in $\psi \setminus\{\eta_2\}$.
Since $\eta_1$ replaces $\eta$ in $\psi\setminus\{\eta_2\}$, we have that $\ell^\dagger=\ell$. 


\end{proof}


%intersection example?