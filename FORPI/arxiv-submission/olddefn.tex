\begin{definition}[First-Order Resolution Proof] 
\label{def:proof} \hfill \\
A directed acyclic graph $\langle V, E, \clause \rangle$, where $V$ is a set of nodes and $E$ is a
set of edges labeled by literals and substitutions (i.e. $E \subset V \times 2^{\mathcal{L}} \times \mathcal{S} \times V$, where $\mathcal{L}$ is the set of all literals and $\mathcal{S}$ is the set of all substitutions, and $\n_1
\xrightarrow[\sigma]{\ell} \n_2$ denotes an edge from node $\n_1$ to node $\n_2$ labeled by the literal $\ell$ and the substitution $\sigma$), is a
proof of a clause $\clause$ iff it is inductively constructible according to the following cases:
%
\begin{itemize}
  \item \textbf{Axiom:} If $\Gamma$ is a clause, $\axiom{\Gamma}$ denotes some proof $\langle \{ \n \}, \varnothing,
    \Gamma \rangle$, where $\n$ is a new (axiom) node.
  \item \textbf{Resolution\footnote{This is referred to as ``binary resolution'' elsewhere, with the understanding that ``binary'' refers to the number of resolved literals, rather than the number of premises of the inference rule.}:} If $\psi_L$ is a proof $\langle V_L, E_L, \clause_L \rangle$ with $\ell_L \in \clause_L$ and
    $\psi_R$ is a proof $\langle V_R, E_R, \clause_R \rangle$ with $\ell_R \in \clause_R$, and 
    $\sigma_L$ and $\sigma_R$ are substitutions such that
    $\ell_L \sigma_L = \dual{\ell_R} \sigma_R$ %and
    %$\freevar{\left( \clause_L \setminus \left\{ \ell_L \right\} \right) \sigma_L} \cap 
     %\freevar{\left( \clause_R
     %               \setminus \left\{ \ell_R \right\} \right) \sigma_R} = \emptyset$, 
    then
    $\psi_L \res{\ell_L}{\sigma_L}{\ell_R}{\sigma_R} \psi_R$ denotes a proof $\langle V, E, \Gamma \rangle$ s.t.
    \begin{align*}
     \hspace{-0.6cm} V &= V_L \cup V_R \cup \{\n \}    \\
      \hspace{-0.6cm} E &= E_L \cup E_R \cup 
                    \left\{ \raiz{\psi_L} \xrightarrow[\sigma_L]{\{\ell_L\} } \n, 
                            \raiz{\psi_R} \xrightarrow[\sigma_R]{\{\ell_R\} } \n \right\}    \\
    \hspace{-0.6cm}  \Gamma &= \left( \clause_L \setminus \left\{ \ell_L \right\} \right) \sigma_L \cup \left( \clause_R
                    \setminus \left\{ \ell_R \right\} \right) \sigma_R
    \end{align*}
    where $\n$ is a new (resolution) node and $\raiz{\varphi}$ denotes the root node of $\varphi$. $\ell_L$ and $\ell_R$ are $\n$'s \emph{resolved literals}, whereas $\ell_L \sigma_L$ and $\ell_R \sigma_R$ are its \emph{instantiated resolved literals}. The \emph{pivot} of $\n$ is the underlying atom of its instantiated resolved literals (i.e. $\abs{\ell_L \sigma_L}$ or, equivalently, $\abs{\ell_R \sigma_R}$).
  \item \textbf{Factoring:}
  %\footnote{This is often called ``Factoring'', but we prefer ``contraction'', because it is essentially the contraction rule of sequent calculus generalized with unification.} 
  If $\psi'$ is a proof $\langle V', E', \clause' \rangle$ and $\sigma$ is a unifier of $\{\ell_1, \ldots \ell_n\}$ with $\{\ell_1, \ldots \ell_n\} \subseteq \clause'$, then $\con{\psi}{\{\ell_1, \ldots \ell_n\}}{\sigma}$ denotes a proof $\langle V, E, \Gamma \rangle$ s.t.
    \begin{align*}
         \hspace{-0.6cm} V &= V' \cup \{\n \} \\
         \hspace{-0.6cm} E &= E' \cup \{ \raiz{\psi'} \xrightarrow[\sigma]{\{\ell_1, \ldots \ell_n\}} \n \} \\
       \hspace{-0.6cm} \Gamma &= (\clause' \setminus \{ \ell_1, \ldots \ell_n \} ) \sigma \cup \{ \ell \}
    \end{align*}
    where $\n$ is a new (factoring) node, $\ell = \ell_k \sigma$ (for any $k \in \{1,\ldots, n\}$) and $\raiz{\varphi}$ denotes the root node of $\varphi$.
  \qed
\end{itemize}
\end{definition}