\section{Lifting \RPI to First-Order Logic}
\label{sec:FORPI}
%\vspace{-0.5cm}
%TODO: this section
{\FirstOrderRPI} ({\FORPI}) (cf. Algorithm \ref{algo:FORPI}) is a first-order generalization of the propositional {\RPI}.
%\footnote{A generalization of {\RP} could be obtained by setting the safe literals to the empty set in lines 11 and 2 of Algorithm~\ref{algo:foSetSafeLiterals}.}. 
{\FORPI} traverses the proof in a bottom-up manner, storing for every node a set of safe literals. The set of safe literals for a node $\psi$ is computed from the set of safe literals of its children (cf.\ Algorithm~\ref{algo:foSetSafeLiterals}), similarly to the propositional case, but additionally applying unifiers to the resolved literals (cf. Example \ref{ex:pairwise}).
If one of the node's resolved literals matches a literal in the set of safe literals, then it may be possible to regularize the node by replacing it by one of its parents.  



\IncMargin{0.5em}
\begin{algorithm}[bt]
\begin{footnotesize}
\SetKwInOut{Input}{input}\SetKwInOut{Output}{output}
\SetKwData{units}{unitsQueue}
\SetKwData{fixedUnits}{fixedUnitsQueue}

\Input{A first-order proof $\psi$}
\Output{An equivalent possibly less-irregular first-order proof $\psi'$}

\BlankLine

$\psi'$ $\la$ $\psi$\;
traverse $\psi'$ bottom-up and \ForEach{node $\eta$ in $\psi'$}{
   \If{$\eta$ is a resolvent node}{
     setSafeLiterals($\eta$) \;
     regularizeIfPossible($\eta$)
   }
  }
$\psi'$ $\la$ fix($\psi'$) \;
\Return {$\psi'$}\;
\caption{\label{algo:FORPI} \texttt{\FORPI}}
\end{footnotesize}
\end{algorithm}
\DecMargin{0.5em}    

%NOTE: FORPPI desciption has been moved to challenges file to optimize whitespace use

In the first-order case, we additionally check for strong regularizability (cf. lines 2 and 6 of Algorithm~\ref{algo:foRegularize}).
%because unification introduces complications like those seen in Example \ref{ex:unifcheck},
%After regularization, all nodes below the regularized node may have to be fixed. 
Similarly to {\RPI}, instead of replacing the irregular node by one of its parents immediately, 
its other parent is marked as a \texttt{deletedNode}, as shown in Algorithm~\ref{algo:foRegularize}.
As in the propositional case, fixing of the proof is postponed to another (single) traversal, as regularization proceeds top-down and only nodes below a regularized node may require fixing.
During fixing, the irregular node is actually replaced by the parent that is not marked as \texttt{deletedNode}. During proof fixing, factoring inferences can be applied, in order to compress the proof further.

Note that, in order to reduce notation clutter in the pseudocodes, we slightly abuse notation and do not explicitly distinguish proofs, their root nodes and the clauses stored in their root nodes. It is clear from the context whether $\psi$ refers to a proof, to its root node or to its root clause.


\IncMargin{0.5em}
\begin{algorithm}[bt]
\begin{footnotesize}

\SetKwInOut{Input}{input}\SetKwInOut{Output}{result}
\SetKwData{units}{unitsQueue}
\SetKwData{fixedUnits}{fixedUnitsQueue}

\Input{A node $\psi=\psi_L  \res{\ell_L}{\sigma_L}{\ell_R}{\sigma_R} \psi_R$}
\Output{The proof containing $\psi$ may be changed}

\BlankLine
    \uIf{$\exists \sigma$  and $\ell \in \mathcal{S}(\psi)$ such that $\ell = \ell_R\sigma_R\sigma$}{
     %\uIf{$\exists \sigma'$ such that $\psi_R\sigma' \subseteq \mathcal{S}(\psi)$} {
     \uIf{$\psi_R\sigma_R\sigma \subseteq \mathcal{S}(\psi)$} {
      mark $\psi_L$ as \texttt{deletedNode} \;
      mark $\psi$ as regularized
}
    }
    \ElseIf{$\exists \sigma$  and $\ell \in \mathcal{S}(\psi)$ such that $\ell = \ell_L\sigma_L\sigma$ }{
    % \uIf{$\exists \sigma'$ such that $\psi_L\sigma' \subseteq\mathcal{S}(\psi)$} {
     \uIf{$\psi_L\sigma_L\sigma \subseteq\mathcal{S}(\psi)$} {
      mark $\psi_R$ as \texttt{deletedNode} \;
      mark $\psi$ as regularized
}
    }
\caption{\label{algo:foRegularize} \texttt{regularizeIfPossible} for \texttt{FORPI}}
\end{footnotesize}
\end{algorithm}
\DecMargin{0.5em}    

\IncMargin{0.5em}
\begin{algorithm}[bt]
\begin{footnotesize}


\SetKwInOut{Input}{input}\SetKwInOut{Output}{result}
\SetKwData{units}{unitsQueue}
\SetKwData{fixedUnits}{fixedUnitsQueue}

\Input{A first-order resolution node $\psi$}
\Output{The node $\psi$ gets a set of safe literals}

\BlankLine

    \uIf{$\psi$ is a root node with no children}{
      $\mathcal{S}(\psi) \la$ $\psi$.clause  
    }
    \Else{
      \ForEach{$\psi'$ $\in$ $\psi${\upshape.children}}{
        \uIf{$\psi'$ is marked as regularized}{ 
          safeLiteralsFrom($\psi'$) $\la$ $\mathcal{S}(\psi')$ \;}
%        \uElseIf{$\eta$ is left parent of $\eta'$}{ 
          \uElseIf{$\psi' = \psi  \res{\ell_L}{\sigma_L}{\ell_R}{\sigma_R} \psi_R$ for some $\psi_R$}{ 
        	safeLiteralsFrom($\psi'$) $\la$ $\mathcal{S}(\psi')~ \cup $ $\{\ell_R\sigma_R  \}$ %\;
        }
        \ElseIf{$\psi' = \psi_L  \res{\ell_L}{\sigma_L}{\ell_R}{\sigma_R} \psi$ for some $\psi_L$}{ 
	safeLiteralsFrom($\psi'$) $\la$ $\mathcal{S}(\psi') ~\cup $ $\{ \ell_L\sigma_L \}$%\;
        }
      }
      $\mathcal{S}(\psi)$ $\la$ $\bigcap_{\psi' \in \psi\textrm{.children}}$ safeLiteralsFrom($\psi'$)
    }

\caption{\label{algo:foSetSafeLiterals} \texttt{setSafeLiterals} for \texttt{FORPI}}
\end{footnotesize}
\end{algorithm}
\DecMargin{0.5em}    
