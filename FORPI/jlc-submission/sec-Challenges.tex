\section{First-Order Challenges for Partial Regularization}\label{sec:FORPIChallenges}


In this section, we describe challenges that have to be overcome in order to successfully adapt {\RPI} to the first-order case. The first example illustrates the need to take unification into account. The other two examples discuss complex issues that can arise when unification is taken into account in a naive way.

%straightforward example
\begin{example}\label{ex:unif} 
Consider the following proof $\psi$. When computed as in the propositional case, the safe literals for $\eta_3$ are $\{ Q(c), ~ P(a,x)\}$.


\begin{scriptsize}
\begin{prooftree}
\def\e{\mbox{\ $\vdash$\ }}
\AxiomC{$\eta_6$: $P(y,b)$ \e \hspace{-2cm}}
\AxiomC{$\eta_1$: \e $P(w,x)$}
\AxiomC{$\eta_2$: $P(w,x)$ \e $Q(c)$}
\BinaryInfC{$\eta_3$: \e $Q(c)$  \hspace{-1.5cm}}
\AxiomC{$\eta_4$: $Q(c)$ \e $P(a,x)$}
\BinaryInfC{$\eta_5$: \e $P(a,x)$}
\BinaryInfC{$\psi$: $\bot$}
\end{prooftree}
\end{scriptsize}

\noindent
As neither of $\eta_3$'s resolved literals is syntactically equal to a safe literal, the propositional {\RPI} algorithm would not change $\psi$. However, $\eta_3$'s left resolved literal $P(w,x)\in \eta_1$ is unifiable with the safe literal $P(a,x)$. Regularizing $\eta_3$, by deleting the edge between $\eta_2$ and $\eta_3$ and replacing $\eta_3$ by $\eta_1$, leads to further deletion of $\eta_4$ (because it is not resolvable with $\eta_1$) and finally to the much shorter proof below.

\begin{footnotesize}
\begin{prooftree}
\def\e{\mbox{\ $\vdash$\ }}
\AxiomC{$\eta_1$: \e$P(w,x)$}
\AxiomC{$\eta_6$: $P(y,b)$\e}
\BinaryInfC{$\psi'$: $\bot$}
\end{prooftree}
\end{footnotesize}

\end{example}

\noindent
Unlike in the propositional case, where a resolved literal must be syntactically equal to a safe literal for regularization to be possible, the example above suggests that, in the first-order case, it might suffice that the resolved literal be unifiable with a safe literal. However, there are cases, as shown in the example below, where mere unifiability is not enough and greater care is needed.

%unification necessary example
\begin{example}\label{ex:pairwise-rpi}

The node $\eta_3$ appears to be a candidate for regularization when the safe literals are computed as in the propositional case and unification is considered na\"{i}vely. Note that $\mathcal{S}(\eta_3)=\{Q(c), ~ P(a,x)\}$, and the resolved literal $P(a,c)$ is unifiable with the safe literal $P(a,x)$,

\begin{scriptsize}
\begin{prooftree}
\def\e{\mbox{\ $\vdash$\ }}
\AxiomC{$\eta_6$: $P(y,b)$ \e \hspace{-2cm}}
\AxiomC{$\eta_1$: \e $P(a,c)$}
\AxiomC{$\eta_2$: $P(a,c)$ \e $Q(c)$}
\BinaryInfC{$\eta_3$: \e $Q(c)$}
\AxiomC{$\eta_4$: $Q(c)$ \e $P(a,x)$}
\BinaryInfC{$\eta_5$: \e $P(a,x)$}
\BinaryInfC{$\psi$: $\bot$}
\end{prooftree}
\end{scriptsize}

\begin{figure*}[bt]
\begin{scriptsize}
\begin{prooftree}
\def\e{\mbox{\ $\vdash$\ }}
\AxiomC{$\eta_8$: $Q(f(a,e),c)\e$}
\AxiomC{$\eta_6$: $\e P(c,d)$ \hspace{-2cm}}
\AxiomC{$\eta_1$: $P(u,v)\e Q(f(a,v),u)$}
\AxiomC{$\eta_2$: $Q(f(a,x),y),Q(t,x)\e Q(f(a,z),y)$}
\BinaryInfC{$\eta_3$: $P(u,v),Q(t,v)\e Q(f(a,z),u)$}
\AxiomC{\hspace{-2cm} $\eta_4$: $\e Q(r,s)$}
\BinaryInfC{\hspace{-2cm} $\eta_5$: $P(u,v)\e Q(f(a,z),u)$}
\BinaryInfC{\hspace{-5cm} $\eta_7$: $\e Q(f(a,z),c)$}
\BinaryInfC{$\psi$: $\bot$}
\end{prooftree}
\end{scriptsize}
\caption{An example where pre-regularizability is not sufficient.}
\label{fig:ex-unifcheck}
\end{figure*}

\noindent
However, if we attempt to regularize the proof, the same series of actions as in Example \ref{ex:unif} would 
require resolution between $\eta_1$ and $\eta_6$, which is not possible.

\end{example}
One way to prevent the problem depicted above would be to require the resolved literal to be not only unifiable but subsume a safe literal. A weaker (and better) requirement is possible, and requires a slight modification of the concept of safe literals, taking into account the unifications that occur on the paths from a node to the root. 

\begin{definition}
The set of \emph{safe literals} for a node $\eta$ in a proof $\psi$ with root clause $\Gamma$, denoted $\mathcal{S}(\eta)$, is such that $\ell \in \mathcal{S}(\eta)$ if and only if $\ell \in \Gamma$ or for all paths from $\eta$ to the root of $\psi$ there is an edge $\n_1
\xrightarrow[\sigma]{\ell'} \n_2$ with $\ell' \sigma = \ell$.
\end{definition}

As in the propositional case, safe literals can be computed in a bottom-up traversal of the proof. Initially, at the root, the safe literals are exactly the literals that occur in the root clause. As we go up, the safe literals $\mathcal{S}(\eta')$ of a parent node $\eta'$ of $\eta$ where $\eta'
\xrightarrow[\sigma]{\ell} \eta$ is set to $\mathcal{S}(\eta) \cup \{ \ell \sigma \}$. Note that we apply the substitution to the resolved literal before adding it to the set of safe literals (cf. algorithm 3, lines 8 and 10). In other words, in the first-order case, the set of safe literals has to be a set of \emph{instantiated} resolved literals.

In the case of Example \ref{ex:pairwise-rpi}, computing safe literals as defined above would result in $\mathcal{S}(\eta_3)=\{Q(c),~P(a,b)\}$, where clearly the pivot $P(a,c)$ in $\eta_1$ is not safe. A generalization of this requirement is formalized below.

\begin{definition}
\label{prop:pair-rpi}
Let $\eta$ be a node with safe literals $\mathcal{S}(\eta)$ and parents $\eta_1$ and $\eta_2$, assuming without loss of generality, $\eta_1 \xrightarrow[\sigma_1]{\{\ell_1\} } \eta$.
The node $\eta$ is said to be \emph{pre-regularizable} in the proof $\psi$ if $\ell_1\sigma_1$ matches a safe literal $\ell^* \in \mathcal{S}(\eta)$.
\end{definition}

\noindent
%This property states that a node is pre-regularizable if, for a resolved literal $\ell'$ unifiable with a safe literal, which is resolved against literals $\ell_1$, \ldots, $\ell_n$ in a proof $\psi$, $\ell_1$,\ldots,$\ell_n$, and $\dual{\ell'}$ are unifiable.
This property states that a node is pre-regularizable if an instantiated resolved literal $\ell'$ matches a safe literal. The notion of \emph{pre-regulariziability} can be thought of as a \emph{necessary} condition for recycling the node $\eta$.

%extra check example
\begin{example}\label{ex:unifcheck}

Satisfying the pre-regularizability is not sufficient. Consider the proof $\psi$ in Figure \ref{fig:ex-unifcheck}. After collecting the safe literals, $\mathcal{S}(\eta_3) = \{\lnot Q(r,v),\lnot P(c,d), Q(f(a,e),c)\}$.
%\noindent
$\eta_3$'s pivot $Q(f(a,v),u)$ matches the safe literal $Q(f(a,e),c)$. Attempting to regularize $\eta_3$ would lead to the removal of $\eta_2$, the replacement of $\eta_3$ by $\eta_1$ and the removal of $\eta_4$ (because $\eta_1$ does not contain the pivot required by $\eta_5$), with $\eta_5$ also being replaced by $\eta_1$. Then resolution between $\eta_1$ and $\eta_6$ results in $\eta_7'$, which cannot be resolved with $\eta_8$, as shown below.


\begin{scriptsize}
\begin{prooftree}
\def\e{\mbox{\ $\vdash$\ }}
\AxiomC{$\eta_8$: $Q(f(a,e),c)\e$ \hspace{-0.5cm}}
\AxiomC{$\eta_6$: $\e P(c,d)$}
\AxiomC{$\eta_1$: $P(u,v)\e Q(f(a,v),u)$}
\BinaryInfC{$\eta_7'$: $\e Q(f(a,d),c)$}
\BinaryInfC{$\psi'$: ??}
\end{prooftree}
\end{scriptsize}

\noindent
$\eta_1$'s literal $Q(f(a, v), u)$, which would be resolved with $\eta_8$'s literal, was changed to $Q(f(a,d),c)$ due to the resolution between $\eta_1$ and $\eta_6$.
\end{example}


\noindent
Thus we additionally require that the following condition be satisfied.


\begin{definition} %This is the new definition
\label{prop:extracheck}
Let $\eta$ be pre-regularizable, with safe literals $\mathcal{S}(\eta)$ and parents $\eta_1$ and $\eta_2$, with clauses $\Gamma_1$ and $\Gamma_2$ respectively, assuming without loss of generality that $\eta_1 \xrightarrow[\sigma_1]{\{\ell_1\} } \eta$
such that $\ell_1\sigma_1$ matches a safe literal $\ell^*\in \mathcal{S}(\eta)$. 
The node $\eta$ is said to be \emph{strongly regularizable} in $\psi$ if $\Gamma_1 \sigma_{1} \sqsubseteq \mathcal{S}(\eta)$.
\end{definition}

This condition ensures that the remainder of the proof does not expect a variable in $\eta_1$ to be unified to different values simultaneously. This property is not necessary in the propositional case, as the literals of the replacement node would not change lower in the proof. 


The notion of \emph{strongly regularizable} can be thought of as a \emph{sufficient} condition. 

\begin{theorem}\label{thm:correct}
Let $\psi$ be a proof with root clause $\Gamma$ and $\eta$ be a node in $\psi$. Let $\psi^{\dagger} = \psi\setminus \{\eta\}$ and $\Gamma^{\dagger}$ be the root of $\psi^{\dagger}$. If $\eta$ is strongly regularizable, then $\Gamma^{\dagger} \sqsubseteq \Gamma$.
\end{theorem}

\begin{proof} 
By definition of strong regularizability, $\eta$ is such
that there is a node $\eta'$ with clause $\Gamma'$ and such that
$\eta' \xrightarrow[\sigma']{\{\ell'\} } \eta$ and $\ell'\sigma'$
matches a safe literal $\ell^*\in \mathcal{S}(\eta)$ and
$\Gamma' \sigma' \sqsubseteq \mathcal{S}(\eta)$.

Firstly, in $\psi^{\dagger}$, $\eta$ has been replaced by $\eta'$. Since
$\Gamma' \sigma' \sqsubseteq \mathcal{S}(\eta)$, by definition of
$\mathcal{S}(\eta)$, every literal $\ell$ in $\Gamma'$ either subsumes %more general than
a single literal that occurs as a pivot on every path
from $\eta$ to the root in $\psi$ (and hence on every new path from
$\eta'$ to the root in $\psi^{\dagger}$) or subsumes literals %more general than
$\ell \sigma_1$,\ldots,$\ell\sigma_n$ in $\Gamma$. In the former case,
$\ell$ is resolved away in the construction of $\psi^{\dagger}$ (by
contracting the descendants of $\ell$ with the pivots in each path).
In the latter case, the literal $\ell \sigma_k$ ($1 \leq k \leq n$) in
$\Gamma$ is a descendant of $\ell$ through a path $k$ and the
substitution $\sigma_k$ is the composition of all substitutions on
this path. When $\eta$ is replaced by $\eta'$, two things may happen
to $\ell \sigma_k$. If the path $k$ does not go through $\eta$, 
$\ell \sigma_k$ remains unchanged (i.e. $\ell \sigma_k \in \Gamma^{\dagger}$
unless the path $k$ ceases to exist in $\psi^{\dagger}$). If the path
$k$ goes through $\eta$, the literal is changed to 
$\ell\sigma^{\dagger}_k$, where $\sigma^{\dagger}_k$ is such that 
$\sigma_k = \sigma' \sigma^{\dagger}_k$.

Secondly, when $\eta$ is replaced by $\eta'$, the edge from
$\eta$'s other parent $\eta''$ to $\eta$ ceases to exist in
$\psi^{\dagger}$. Consequently, any literal $\ell$ in $\Gamma$ that is a
descendant of a literal $\ell''$ in the clause of $\eta''$ through a
path via $\eta$ will not belong to $\Gamma^{\dagger}$.


Thirdly, a literal from $\Gamma$ that descends neither from $\eta'$ nor from $\eta''$ either remains unchanged in $\Gamma^{\dagger}$ or, if the path to the node from which it descends ceases to exist in the construction of $\psi^{\dagger}$, does not belong to $\Gamma^{\dagger}$ at all.

Therefore, by the three facts above, $\Gamma^{\dagger} \sigma' \sqsubseteq \Gamma$, and hence $\Gamma^{\dagger} \sqsubseteq \Gamma$. \qed
\end{proof}


As the name suggests, strong regularizability is stronger than necessary. In some cases, nodes may be regularizable even if they are not strongly regularizable. A weaker condition (conjectured to be sufficient) is presented below. This alternative relies on knowledge of how literals are changed after the deletion of a node in a proof (and it is inspired by the \emph{post-deletion unifiability condition} described for {\FOLowerUnits}). However, since weak regularizability is more complicated to check, it is not as suitable for implementation as strong regularizability. 
\begin{definition}\label{def:postdelprop}
Let $\eta$ be a pre-regularizable node with parents $\eta_1$ and $\eta_2$, assuming without loss of generality that $\eta_1 \xrightarrow[\sigma_1]{\{\ell_1\} } \eta$ 
%and $\eta_2 \xrightarrow[\sigma_2]{\{\ell_2\} } \eta$ 
such that $\ell_1$ is unifiable with some $\ell^* \in \mathcal{S}(\eta)$.
For each safe literal $\ell = \ell_s\sigma_s \in \mathcal{S}(\eta_1)$, let $\eta_\ell$ be a node on the path from $\eta$ to the root of the proof such that $\abs{\ell}$ is the pivot of $\eta_\ell$.
Let $\mathcal{R}(\eta_\ell)$ be the set of all resolved literals $\ell_s'$ such that $\eta_2' \xrightarrow[\sigma_s]{\{\ell_s\} } \eta_\ell$, $\eta_1' \xrightarrow[\sigma_s']{\{\ell_s'\} } \eta_\ell$, and $\ell_s\sigma_s=\dual{\ell_s'}\sigma_s'$, for some nodes $\eta_2'$ and $\eta_1'$ and unifier $\sigma_s'$; if no such node $\eta_\ell$ exists, define $\mathcal{R}(\eta_\ell)=\emptyset$.
% and $\sigma_2'$.
The node $\eta$ is said to be \emph{weakly regularizable} in $\psi$ if, for all $\ell \in \mathcal{S}(\eta_1)$, all elements in $\mathcal{R}^{\dagger}(\eta_\ell) \cup \{ \dual{\ell}^\dagger \}$ are unifiable, where $\dual{\ell}^{\dagger}$ is the literal in $\dn{\psi}{\eta_2}$ that used to be\footnote{Because of the removal of $\eta_2$, $\dual{\ell}^{\dagger}$ may differ from $\dual{\ell}$.} $\dual{\ell}$ in $\psi$ and $\mathcal{R}^{\dagger}(\eta_\ell)$ is the set of literals in $\dn{\psi}{\eta_2}$ that used to be the literals of $\mathcal{R}(\eta_\ell)$ in $\psi$.
\end{definition}


This condition requires the ability to determine the underlying (uninstantiated) literal for each safe literal of a weakly regularizable node $\eta$. To achieve this, one could store safe literals as a pair $(\ell_s,\sigma_s)$, rather than as an instantiated literal $\ell_s\sigma_s$, although this is not necessary for the previous conditions.

Note further that there is always at least one node $\eta_\ell$ as assumed in the definition for any safe literal which was not contained in the root clause of the proof: the node which resulted in $\ell = \ell_s\sigma_s \in \mathcal{S}(\eta)$ being a safe literal for the path from $\eta$ to the root of the proof. Furthermore, it does not matter which node $\eta_\ell$ is used. To see this, consider some node $\eta_\ell' \neq \eta_\ell$ with the same pivot $\abs{\ell}=\abs{\ell_s\sigma_s}$. Consider arbitrary nodes $\eta_1$ and $\eta_2$ such that  $\eta_2 \xrightarrow[\sigma_s]{\{\ell_s\} } \eta_\ell$ and $\eta_1 \xrightarrow[\sigma_1]{\{\ell_1\} } \eta_\ell$ where $\ell_s\sigma_s=\dual{\ell_1}\sigma_1$. Now consider arbitrary nodes $\eta_1'$ and $\eta_2'$ such that  $\eta_2' \xrightarrow[\sigma_s]{\{\ell_s\} } \eta_\ell'$ and $\eta_1' \xrightarrow[\sigma_1']{\{\ell_1'\} } \eta_\ell'$ where $\ell_s\sigma_s=\dual{\ell_1'}\sigma_1'$. Since the pivots for $\eta_\ell$ and $\eta_\ell'$ are equal, we must have that %$\abs{\ell\sigma_2}=\abs{\ell\sigma_2'}$ and furthermore that 
$\abs{\ell_s\sigma_s}=\abs{\ell_1\sigma_1}$ and $\abs{\ell_s\sigma_s}=\abs{\ell_1'\sigma_1'}$, and thus $\abs{\ell_1\sigma_1}=\abs{\ell_1'\sigma_1'}$. This shows that it does not matter which $\eta_\ell$ we use; the instantiated resolved literals will always be equal implying that both of the resolved literals $\ell_1$ and $\ell_1'$ will be contained in both $\mathcal{R}(\eta_\ell)$ and $\mathcal{R}(\eta_\ell')$.


Informally, a node $\eta$ is weakly regularizable in a proof if it can be replaced by one of its parents $\eta_1$, such that for each $\ell \in \mathcal{S}(\eta_1)$, $\abs{\ell}$ can still be used as a pivot in order to complete the proof. Weakly regularizable nodes differ from strongly regularizable nodes by not requiring the entire parent $\eta_1$ replacing the resolution $\eta$ to be simultaneously matched to a subset of $\mathcal{S}(\eta)$, and requires knowledge of how literals will be instantiated after the removal of $\eta_2$ and $\eta$ from the proof.


\begin{table}[bt]
\centering
\begin{tabular}{| c | c | c | c | }
\hline
$\eta$ & $\mathcal{S}(\eta)$ & $\mathcal{R}(\eta)$ & $\mathcal{R}^\dagger(\eta)$ \\ \hline \hline
$\eta_1$ &  $\{P(w)\}$ & $\emptyset$  & $\emptyset$\\ \hline 
$\eta_2$ &  $\{\lnot P(w)\}$ & $\emptyset$  & $\emptyset$\\ \hline 
$\eta_3$ &  $\{R(a),\lnot P(w)\}$ & $\emptyset$  & $\emptyset$\\ \hline 
$\eta_4$ &  $\{\lnot R(a),\lnot P(w)\}$& $\emptyset$& $\emptyset$ \\ \hline 
$\eta_5$ &  $\{Q(z),\lnot R(a), \lnot P(w)\}$ & $\emptyset$ & $\emptyset$\\ \hline 
$\eta_6$ &  $\{\lnot P(w), \lnot Q(z), \lnot R(a) \}$ & $\{P(u),P(y)\}$& $\{P(u)\}$\\ \hline 
$\eta_7$ &  $\{P(y), \lnot P(w), \lnot Q(z), \lnot R(a) \}$ & $\emptyset$ & $\emptyset$ \\ \hline 
$\eta_8$ &   $\{\lnot P(y), \lnot P(w), \lnot Q(z), \lnot R(a) \}$ & $\emptyset$ & $\emptyset$\\ \hline 
\end{tabular}
\hfill
\caption{The sets $\mathcal{S}(\eta)$ and $\mathcal{R}(\eta)$ for each node $\eta$ in the first proof of Example \ref{ex:weak}.}
\label{tab:exweakreg}
\end{table}


\begin{example}\label{ex:weak}
This example illustrates a case where a node is weakly regularizable but not strongly regularizable. Table \ref{tab:exweakreg} shows the sets $\mathcal{S}(\eta)$, $\mathcal{R}(\eta)$ and $\mathcal{R}^\dagger(\eta)$ for the nodes $\eta$ in the proof below. Observe that $\eta_6$ is pre-regularizable, since $\lnot P(x)$ is unifiable with $\lnot P(w)\in \mathcal{S}(\eta_6)$. In fact, $\eta_6$ is the only pre-regularizable node in the proof, and thus the sets $\mathcal{R}(\eta) = \emptyset$ for all $\eta \neq \eta_6$.
In the proof below, note that $\eta_6$ is not strongly regularizable: there is no unifier $\sigma$ such that $\{\lnot P(x),\lnot Q(x),\lnot R(x)\} \sigma \subseteq \mathcal{S}(\eta_6)$.
\begin{scriptsize}
\begin{prooftree}
\def\e{\mbox{\ $\vdash$\ }}
\AxiomC{$\eta_1$: $\e P(u)$ \hspace{-2cm}}
\AxiomC{$\eta_5$: $P(z) \e Q(z)$ \hspace{-0.5cm}}
\AxiomC{$\eta_8$: $P(x),Q(x),R(a)\e$}
\AxiomC{$\eta_7$: $\e P(y)$  \hspace{-1cm}}
\BinaryInfC{$\eta_6$: $Q(y),R(a)\e$ }
\BinaryInfC{$\eta_4$: $P(z),R(a)\e$ \hspace{-2cm} }
\AxiomC{ \hspace{-1cm} $\eta_3$: $\e R(a)$}
\BinaryInfC{ $\eta_2$: $P(z)\e$}

\BinaryInfC{$\psi$: $\bot$}
\end{prooftree}

\end{scriptsize}
\noindent
We show that $\eta_6$ is weakly regularizable, and that $\eta_7$ can be removed. Recalling that $\eta_6$ is pre-regularizable, observe that $\mathcal{R}^\dagger(\eta_6) \cup \{\dual{\lnot P(w)}\}$ is unifiable.
Consider the following proof of $\psi \setminus \{\eta_7\}$:
\begin{scriptsize}
\begin{prooftree}
\def\e{\mbox{\ $\vdash$\ }}
\AxiomC{$\eta_1$: $\e P(u)$ \hspace{-1.75cm}}
\AxiomC{$\eta_8$: $P(x),Q(x),R(a)\e$}
\AxiomC{$\eta_5$: $P(z) \e Q(z)$}
\BinaryInfC{$\eta_4'$: $P(z), P(z),R(a)\e$}
\UnaryInfC{$\eta_4$: $P(z),R(a)\e$}
\AxiomC{$\eta_3$: $\e R(a)$}
\BinaryInfC{$\eta_2$: $P(z)\e$}
\BinaryInfC{$\psi$: $\bot$}
\end{prooftree}
\end{scriptsize}
Now observe that for each $\ell \in \mathcal{S}(\eta_8)$ we have the following, showing that $\eta_6$ is weakly regularizable:
\begin{itemize}
\item $\ell=\lnot  Q(y)$: $\ell^\dagger = \lnot Q(x)$ which is unifiable with $\dual{\ell}^\dagger=Q(z)$
\item $\ell=\lnot R(a)$: $\ell^\dagger = \lnot R(a)$ which is (trivially) unifiable with $\dual{\ell}^\dagger=R(a)$
\item $\ell=\lnot P(w)$: $\ell^\dagger = \lnot P(z)$ which is unifiable with $\dual{\ell}^\dagger=P(u)$
\item $\ell=\lnot P(y)$: $\ell^\dagger = \lnot P(z)$ which is unifiable with $\dual{\ell}^\dagger=P(u)$
\end{itemize}
\end{example}

If a node $\eta$ with parents $\eta_1$ and $\eta_2$ is pre-regularizable and strongly regularizable in $\psi$, then $\eta$ is also weakly regularizable in $\psi$.
