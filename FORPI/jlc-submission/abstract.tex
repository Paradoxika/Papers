Proofs are a key feature of modern propositional and first-order theorem provers. 
Proofs generated by such tools serve as explanations for unsatisfiability of statements. 
However, these explanations are complicated by proofs which are not necessarily as concise as possible.
There are a wide variety of compression techniques for propositional resolution proofs, but fewer compression techniques for first-order resolution proofs generated by automated theorem provers.
This paper describes an approach to compressing first-order logic proofs based on lifting proof compression ideas used in propositional logic to first-order logic. 

The first approach lifted from propositional logic delays resolution with \emph{unit clauses}, which are clauses that have a single literal.
The second approach is \emph{partial regularization}, which removes an inference $\eta$ when it is redundant in the sense that its pivot literal already occurs as the pivot of another inference in every path from $\eta$ to the root of the proof. 
This paper describes the generalization of the algorithms \LowerUnits and \RecyclePivotsIntersection \cite{LURPI} %[P. Fontain, S. Merz, and B. Woltzenlogel Paleo, Compression of Propositional Resolution Proofs via Partial Regularization, \emph{CADE-23}, 2011] 
from propositional logic to first-order logic. 
The generalized algorithms compresses resolution proofs containing resolution and factoring inferences with \emph{unification}.

An empirical evaluation of these approaches is included.


