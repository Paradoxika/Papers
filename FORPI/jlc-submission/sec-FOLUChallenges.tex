\section{First-Order Challenges for Lowering Units} \label{sec:LUChallenges}

In this section, we describe challenges that have to be overcome in order to successfully adapt {\LowerUnits} to the first-order case. 
The first example illustrates the need to take unification into account. The other two examples discuss complex issues that can arise when unification is naively taken into account.

\begin{example}\label{ex:uniflu} 
Consider the following proof $\psi$, noting that the unit subproof $\eta_2$ is used twice. It is resolved once with $\eta_1$ (against the literal $p(W)$ and producing the child $\eta_3$) and once with $\eta_5$ (against the literal $p(X)$ and producing $\psi$).

\begin{footnotesize}
\begin{prooftree}
\def\e{\mbox{\ $\vdash$\ }}
\AxiomC{$\eta_1$: $p(W)$\e$q(Z)$}
\AxiomC{$\eta_2$: \e$p(Y)$}
\BinaryInfC{$\eta_3$: \e$q(Z)$}
\AxiomC{$\eta_4$: $p(X),q(Z)$\e}
\BinaryInfC{$\eta_5$: $p(X)$\e}
\AxiomC{$\eta_2$}
\BinaryInfC{$\psi$: $\bot$}
\end{prooftree}
\end{footnotesize}

\noindent
The result of deleting $\eta_2$ from $\psi$ is the proof $\dn{\psi}{\eta_2}$ shown below:

\begin{footnotesize}
\begin{prooftree}
\def\e{\mbox{\ $\vdash$\ }}
\AxiomC{$\eta'_1$: $p(W)$\e$q(Z)$}
\AxiomC{$\eta'_4$: $p(X),q(Z)$\e}
\BinaryInfC{$\eta'_5$ ($\psi'$): $p(W), p(X)$\e}
\end{prooftree}
\end{footnotesize}

\noindent
Unlike in the propositional case, where the literals that had been resolved against the unit are all syntactically equal, in the first-order case, this is not necessarily the case. As illustrated above, $p(W)$ and $p(X)$ are not syntactically equal. Nevertheless, they are unifiable. Therefore, in order to reintroduce $\eta'_2$, we may first perform a contraction, as shown below:
\begin{footnotesize}
\begin{prooftree}
\def\e{\mbox{\ $\vdash$\ }}
\AxiomC{$\eta_1'$: $p(W)$\e$q(Z)$}
\AxiomC{$\eta_4'$: $p(X),q(Z)$\e}
\BinaryInfC{$\eta_5'$: $p(X),p(Y)$\e}
\UnaryInfC{$\con{\eta_5'}{}{}$: $p(U)$\e}
\AxiomC{$\eta_2'$: \e$p(Y)$}
\BinaryInfC{$\psi^{\star}$: $\bot$}
\end{prooftree}
\end{footnotesize}
\end{example}

\begin{example}\label{ex:pairwise}

There are cases, as shown below, when the literals that had been resolved away are not unifiable, and then a contraction is not possible.

\begin{footnotesize}
\begin{prooftree}
\def\e{\mbox{\ $\vdash$\ }}
\AxiomC{$\eta_2$}
\AxiomC{$\eta_4$: $r(X),p(b)$\e $s(Y)$}
\AxiomC{$\eta_1$: $p(a)$\e$q(Y),r(Z)$}
\AxiomC{$\eta_2$: \e $p(X)$}
\BinaryInfC{$\eta_3$: \e$q(Y),r(Z)$}
\BinaryInfC{$\eta_5$: $p(b)$\e $s(Y),q(Y)$}
\AxiomC{$\eta_6$: $s(Y)$\e}
\insertBetweenHyps{\hskip -0.5in}
\BinaryInfC{$\eta_7$: $p(b)$\e$q(Y)$}
\AxiomC{$\eta_8$: $q(Y)$\e}
\insertBetweenHyps{\hskip -0.5in}
\BinaryInfC{$\eta_9$: $p(b)$\e}
\insertBetweenHyps{\hskip -0.8in}
\BinaryInfC{$\psi$: $\bot$}
\end{prooftree}
\end{footnotesize}

\noindent
If we attempted to postpone the resolution inferences involving the unit $\eta_2$ (i.e. by deleting $\eta_2$ and reintroducing it with a single resolution inference in the bottom of the proof), a contraction of the literals $p(a)$ and $p(b)$ would be needed. 
Since these literals are not unifiable, the contraction is not possible. 
Note that, in principle, we could still lower $\eta_2$ if we resolved it not only once but twice when reintroducing it in the bottom of the proof.
However, this would lead to no compression of the proof's length.
\end{example}

\noindent
The observations above lead to the idea of requiring units to satisfy the following property before collecting them to be lowered.

\begin{definition}
\label{prop:pair}
Let $\eta$ be a unit with literal $\ell$ and let $\eta_1$, \ldots, $\eta_n$ be subproofs that are resolved with $\eta$ in a proof $\psi$, respectively, with resolved literals $\ell_1$, \ldots, $\ell_n$. 
$\eta$ is said to satisfy the \emph{pre-deletion unifiability property} in $\psi$ if $\ell_1$,\ldots,$\ell_n$, and $\dual{\ell}$ are unifiable.
\end{definition}

\begin{example}\label{ex:rootpair}
Satisfaction of the pre-deletion unifiability property is not enough. Deletion of the units from a proof $\psi$ may actually change the literals that had been resolved away by the units, because fewer substitutions are applied to them. This is exemplified below:

\begin{footnotesize}
\begin{prooftree}
\def\e{\mbox{\ $\vdash$\ }}
\AxiomC{$\eta_1$: $r(Y),p(X, q(Y, b)), p(X, Y)$\e}
\AxiomC{$\eta_2$: \e $p(U, V)$}
\BinaryInfC{$\eta_3$: $r(V),p(U, q(V, b))$\e}
\AxiomC{$\eta_4$: \e $r(W)$}
\BinaryInfC{$\eta_5$: $p(U, q(W, b))$\e}
\AxiomC{$\eta_2$}
\BinaryInfC{$\psi$: $\bot$}
\end{prooftree}
\end{footnotesize}

\noindent
If $\eta_2$ is collected for lowering and deleted from $\psi$, we obtain the proof $\dn{\psi}{\eta_2}$:

\begin{footnotesize}
\begin{prooftree}
\def\e{\mbox{\ $\vdash$\ }}
\AxiomC{$\eta'_1$: $r(Y),p(X, q(Y, b)), p(X, Y)$\e}
\AxiomC{$\eta'_4$: \e $r(W)$}
\BinaryInfC{$\eta'_5 (\psi')$: $p(X, q(W, b)), p(X, W)$\e}
\end{prooftree}
\end{footnotesize}

\noindent
Note that, even though $\eta_2$ satisfies the pre-deletion unifiability property (since $p(X, q(Y, b))$ and $p(U, q(W, b))$ are unifiable), $\eta_2$ still cannot be lowered and reintroduced by a single resolution inference, because the corresponding modified post-deletion literals $p(X, q(W, b))$ and $p(X, W)$ are actually not unifiable.
\end{example}

The observation above leads to the following stronger property:

\begin{definition}
\label{prop:rootpair}
Let $\eta$ be a unit with literal $\ell_{\eta}$ and let $\eta_1$, \ldots, $\eta_n$ be subproofs that are resolved with $\eta$ in a proof $\psi$, respectively, with resolved literals $\ell_1$, \ldots, $\ell_m$. 
$\eta$ is said to satisfy the \emph{post-deletion unifiability property} in $\psi$ if $\ell_1^{\dagger\downarrow}$,\ldots,$\ell_m^{\dagger\downarrow}$, and $\dual{\ell_{\eta}^{\dagger}}$ are unifiable, where $\ell^{\dagger}$ is the literal in $\dn{\psi}{\eta}$ corresponding to $\ell$ in $\psi$ and $\ell_k^{\dagger\downarrow}$ is the descendant of $\ell_k^{\dagger}$ in the root of $\dn{\psi}{\eta}$.
\end{definition}
