\begin{example}\label{ex:resolutionproof}
An example first-order resolution proof is shown below.


\begin{scriptsize}
\begin{prooftree}
\def\e{\mbox{\ $\vdash$\ }}
\AxiomC{$\eta_1$: $Q(x),Q(a)\e P(b)$ } %a
\AxiomC{$\eta_2$: $P(b)\e$ } %common
\BinaryInfC{$\eta_3$: $Q(x),Q(a) \e$}
\UnaryInfC{$\eta_3'$: $Q(a) \e$ \hspace{-1cm}}
\AxiomC{$\eta_2$} %common
\AxiomC{\hspace{-0.5cm}$\eta_4$: $\e P(b), Q(y)$ } %b
\BinaryInfC{\hspace{-2cm} $\eta_5$: $\e Q(y)$}
\BinaryInfC{$\psi$: $\bot$}
\end{prooftree}
\end{scriptsize}

\noindent
The nodes $\eta_1$, $\eta_2$, and $\eta_4$ are axioms. Node $\eta_3$ is obtained by resolution on $\eta_1$ and $\eta_2$ where $\ell_L = P(b)$, $\ell_R = \neg P(b)$, and $\sigma_L = \sigma_R = \emptyset$. The node $\eta_3'$ is obtained by a factoring on $\eta_3$ with $\sigma=\{x\setminus a\}$. The node $\eta_5$ is the result of resolution on $\eta_2$ and $\eta_4$ with $\ell_L = \neg P(b)$, $\ell_R = P(b)$, $\sigma_L=\sigma_R = \emptyset$. Lastly, the conclusion node $\psi$ is the result of a resolution of $\eta_3'$ and $\eta_5$, where $\ell_L = \neg Q(a)$, $\ell_R = Q(y)$, $\sigma_L = \emptyset$, and $\sigma_R = \{ y \setminus a\}$. The directed acyclic graph representation of the proof (with edge labels omitted) is shown in Figure \ref{fig:dagex}.\\

\begin{figure}[bt]
\begin{centering}
\begin{tikzpicture}
  \tikzstyle{vertex}=[circle,minimum size=10pt,inner sep=0pt]
\tikzset{edge/.style = {->,> = latex'}}

    \node[vertex] (n1) at (-2,1.75) {$\eta_1$};
    \node[vertex] (n2) at (0,1.75) {$\eta_2$};
    \node[vertex] (n4) at (2,1.75) {$\eta_4$};
    \node[vertex] (n5) at (1,0.5) {$\eta_5$};
    \node[vertex] (n3) at (-1,1.25) {$\eta_3$};
    \node[vertex] (n3p) at (-1,0.5) {$\eta_3'$};
    \node[vertex] (r) at (0,0) {$\psi$};

\draw[edge] (r) -- (n5);
\draw[edge] (r) -- (n3p);

\draw[edge] (n3p) -- (n3);

\draw[edge] (n3) -- (n1);
\draw[edge] (n3) -- (n2);

\draw[edge] (n5) -- (n2);
\draw[edge] (n5) -- (n4);
\end{tikzpicture}

\end{centering}
\caption{The proof in Example \ref{ex:resolutionproof}.}
\label{fig:dagex}
\end{figure}
\end{example}