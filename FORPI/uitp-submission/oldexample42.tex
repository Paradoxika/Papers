%unification necessary example
\begin{example}\label{ex:pairwise}

The node $\eta_3$ appears to be a candidate for regularization when the safe literals are computed as in the propositional case and unification is considered na\"{i}vely. Note that $\mathcal{S}(\eta_3)=\{Q(c), ~ P(a,x)\}$, and the resolved literal $P(a,c)$ is unifiable with the safe literal $P(a,x)$,

\begin{scriptsize}
\begin{prooftree}
\def\e{\mbox{\ $\vdash$\ }}
\AxiomC{$\eta_6$: $P(y,b)$ \e \hspace{-2cm}}
\AxiomC{$\eta_1$: \e $P(a,c)$}
\AxiomC{$\eta_2$: $P(a,c)$ \e $Q(c)$}
\BinaryInfC{$\eta_3$: \e $Q(c)$}
\AxiomC{$\eta_4$: $Q(c)$ \e $P(a,x)$}
\BinaryInfC{$\eta_5$: \e $P(a,x)$}
\BinaryInfC{$\psi$: $\bot$}
\end{prooftree}
\end{scriptsize}



\noindent
However, if we attempt to regularize the proof, the same series of actions as in Example \ref{ex:unif} would 
require resolution between $\eta_1$ and $\eta_6$, which is not possible.
%result in the following resolution, which cannot be completed.
%\begin{footnotesize}
%\begin{prooftree}
%\def\e{\mbox{\ $\vdash$\ }}
%\AxiomC{$\eta_1$: \e$p(a,c)$}
%\AxiomC{$\eta_6$: $p(Y,b)$\e}
%\BinaryInfC{$\psi'$: ??}
%\end{prooftree}
%\end{footnotesize}

\end{example}