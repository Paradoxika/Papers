\section{Introduction} 

First-order automated theorem provers, commonly based on refinements and extensions of resolution and superposition calculi \cite{Vampire,EProver,Spass,spassT,Beagle,cruanes2015extending,prover9-mace4}, have recently achieved a high degree of maturity. Proof production is a key feature that has been gaining importance, as proofs are crucial for applications that require certification of a prover's answers or that extract additional information from proofs (e.g. unsat cores, interpolants, instances of quantified variables). Nevertheless, proof production is non-trivial \cite{SchulzAPPA}, and the best, most efficient provers do not necessarily generate the best, least redundant proofs.

For propositional resolution proofs, as those typically generated by SAT- and SMT-solvers, there is a wide variety of proof compression techniques. Algebraic properties of the resolution operation that are potentially useful for compression were investigated in \cite{bwp10}.
Compression algorithms based on rearranging and sharing chains of resolution inferences have been
developed in \cite{Amjad07} and \cite{Sinz}.  Cotton \cite{CottonSplit} proposed an algorithm that
compresses a refutation by repeatedly splitting it into a proof of a heuristically chosen literal $\ell$
and a proof of $\dual{\ell}$, and then resolving them to form a new refutation.  The {\ReduceReconstruct} algorithm \cite{RedRec} searches for locally redundant
subproofs that can be rewritten into subproofs of stronger clauses and with fewer resolution steps.
A linear time proof compression algorithm based on partial
regularization was proposed in \cite{RP08} and improved in \cite{LURPI}.

In contrast, there has been much less work on simplifying first-order proofs. For tree-like sequent calculus proofs, algorithms based on cut-introduction \cite{BrunoLPAR,Hetzl} have been proposed. However, converting a DAG-like resolution or superposition proof, as usually generated by current provers, into a tree-like sequent calculus proof may increase the size of the proof. For arbitrary proofs in the Thousands of Problems for Theorem Provers (TPTP) \cite{TPTP} format (including DAG-like first-order resolution proofs), there is an algorithm \cite{LPARCzech} that looks for terms that occur often in any Thousands of Solutions from Theorem Provers (TSTP) \cite{TPTP} proof and abbreviates them. 


The work reported in this paper is part of a new trend that aims at lifting successful propositional proof compression algorithms to first-order logic. Our first target was the propositional {\LowerUnits} ({\LU}) algorithm, which delays resolution steps with unit clauses, and we lifted it to a new algorithm that we called
{\SFOLowerUnits} 
({\GFOLU}) algorithm \cite{GFOLU}. Here we continue this line of research by lifting the 
%\RecyclePivotsIntersection 
\texttt{Recycle\-PivotsWithIntersection}
({\RPI}) algorithm \cite{LURPI}, which improves the \texttt{RecyclePivots} ({\RP}) algorithm \cite{RP08} by detecting nodes that can be regularized even when they have multiple children. 

%TODO: edit this again
Section \ref{sec:res} introduces the well-known first-order resolution calculus with notations that are suitable for describing and manipulating proofs as first-class objects. Section \ref{sec:Challenges} discusses the challenges that arise in the first-order case (mainly due to unification), which are not present in the propositional case, and conclude with conditions useful for first-order regularization. Section \ref{sec:FORPI} describes an algorithm that overcomes these challenges. Section \ref{sec:exp} presents experimental results obtained by applying this algorithm, and its combinations with {\GFOLU}, on hundreds of proofs generated with the {\SPASS} theorem prover on TPTP benchmarks \cite{TPTP} and on randomly generated proofs. Section \ref{sec:conclusion} concludes the paper.
