%\documentclass{llncs}
%\documentclass{iosart2x}
%\documentclass{article}
%\documentclass[submission,copyright,creativecommons]{eptcs}
\documentclass{easychair}
\providecommand{\event}{UITP 2018}

\usepackage{etex}
%\usepackage[margin=1.5in]{geometry}
\usepackage{mathtools}
\usepackage{xcolor}
\usepackage{enumitem,amsmath,amssymb}
%\usepackage{breakurl}    % used for \url and \burl
\usepackage{url}
\usepackage[linesnumbered,boxed,noline,noend]{algorithm2e}


\def\defaultHypSeparation{\hskip.1in}

\usepackage{tikz}
\usetikzlibrary{arrows}

\usepackage{subfig}
\usepackage{array,booktabs,multirow}
\usepackage{placeins}

\usepackage{logictools}
\usepackage{prooftheory}
\usepackage{comment}
\usepackage{mathenvironments}
\usepackage{drawproof}
\usepackage{bussproofs}
\usepackage{tensor}
%\usepackage{mathtools}
\usepackage{amsmath}
\usepackage{amsthm}

\usepackage{graphicx}
%\usepackage{caption}
%\usepackage{subcaption}


\newenvironment{bprooftree}
  {\leavevmode\hbox\bgroup}
  {\DisplayProof\egroup}

\renewcommand{\topfraction}{0.85}
\renewcommand{\textfraction}{0.1}
\renewcommand{\floatpagefraction}{0.75}


%\newcommand{\freevar}[1]{\mathrm{FV}(#1)}
\newcommand{\freevar}[1]{\mathrm{Var}(#1)}


\newcommand{\Vertices}[1]{V_{#1}}
\newcommand{\Edges}[1]{E_{#1}}
\newcommand{\Conclusion}[1]{\clause_{#1}}

\newcommand{\axiom}[1]{\widehat{#1}}
\newcommand{\n}{v}
\newcommand{\raiz}[1]{\rho(#1)}

\newcommand{\pedge}[3]{\ensuremath{\raiz{#1} \xrightarrow{#2} \raiz{#3}}}

\DeclarePairedDelimiter{\abs}{\lvert}{\rvert}

\newcommand\inlineeqno{\stepcounter{equation}\ (\theequation)}


% Contraction
\newcommand{\con}[3]{\lfloor #1 \rfloor_{#2}^{#3}}

% Resolution
%\newcommand{\res}[6]{#1 \tensor[^{#2}_{#3}]{\odot}{^{#4}_{#5}} #6}
%\newcommand{\res}[6]{#1 \prescript{#2}{#3}{\odot^{#4}_{#5}} #6}

\newcommand{\res}[4]{\mathrel{\operatorname*{\odot}_{#1 #3}^{#2 #4}}}

\newtheorem{thm}{Theorem}[section]
\newtheorem{lem}{Lemma}[section]

\newtheorem{example}[thm]{Example}

\theoremstyle{definition}
\newtheorem{definition}{Definition}[section]


%\usepackage{authblk}
\usepackage{underscore}  
\title{Partial Regularization of First-Order Resolution Proofs}
\author{Jan Gorzny\inst{1}\thanks{Supported by the Google Summer of Code 2014 and Google Summer of Code 2016 programs}
\and
Ezequiel Postan\inst{2}$^{*}$
\and
Bruno Woltzenlogel Paleo\inst{3}\thanks{Bruno ist Stipendiat der \"Osterreichischen Akademie der Wissenschaft (APART) an der TU-Wien}
}


\institute{%School of Computer Science\\
University of Waterloo, Waterloo, ON, Canada\\
\email{jgorzny@uwaterloo.ca}
\and
Universidad Nacional de Rosario, Rosario, Santa Fe, Argentina\\
\email{ezequiel@fceia.unr.edu.ar}
\and
Vienna University of Technology, Vienna, Austria\\
\email{bruno@logic.at}
}


\titlerunning{Partial Regularization of First-Order Resolution Proofs}
\authorrunning{Gorzny, Postan \& Woltzenlogel Paleo}

\begin{document}

  \maketitle


\begin{abstract}
%Resolution and superposition are common techniques which have seen widespread use with propositional and first-order logic in modern theorem provers. In these cases, resolution proof production is a key feature of such tools; however, the proofs that they produce are not necessarily as concise as possible.
Proofs are a key interface of modern propositional and first-order theorem provers. However, this interface is complicated by proofs which are not necessarily as concise as possible.
%For propositional resolution proofs, there are a wide variety of proof compression techniques. There are fewer techniques for compressing first-order resolution proofs generated by automated theorem provers.
There are a wide variety of compression techniques for propositional resolution proofs, but fewer compression techniques for first-order resolution proofs generated by automated theorem provers.
This paper describes an approach to compressing first-order logic proofs based on lifting proof compression ideas used in propositional logic to first-order logic. 
An empirical evaluation of the approach is included.

%One method for propositional proof compression is \emph{partial regularization}, which removes an inference $\eta$ when it is redundant in the sense that its pivot literal already occurs as the pivot of another inference in every path from $\eta$ to the root of the proof. 
%This paper describes the generalization of the partial-regularization algorithm
%\RecyclePivotsIntersection \cite{LURPI}
%%[P. Fontain, S. Merz, and B. Woltzenlogel Paleo, Compression of Propostional Resolution Proofs via Partial Regularization, In \emph{Automated Deduction -- CADE-23, Proceedings} LNCS vol. 6803, 237-351. 2011]
%from propositional logic to first-order logic. The generalized algorithm performs partial regularization of resolution proofs containing resolution and factoring inferences with \emph{unification}. %The resolution calculus is the underlying proof-theoretical foundation for many automated theorem provers. 
%An empirical evaluation of the generalized algorithm and its combinations with the previously lifted \SFOLowerUnits algorithm
%\cite{GFOLU} is also presented.
%%[J. Gorzny and B. Woltzenlogel Paleo, Towards the Compression of First-Order Resolution Proofs by Lowering Unit Clauses. In \emph{Automated Deduction -- CADE-25, Proceedings}, LNCS vol. 9195, 356-366. 2015], 




%\texttt{RecyclePivots\-WithIntersection} 
\end{abstract}


\sloppy  %to prevent overfull margins for e.g. \RPI

\setcounter{footnote}{0}
\newcommand{\la}{\leftarrow}

\section{Introduction} 

%Recently, there has been interest in the combination of theorem provers and machine learning techniques (e.g. \cite{irving2016deepmath,kaliszyk2018reinforcement,DBLP:conf/lpar/LoosISK17}). 
Explainable artificial intelligence is a major challenge for the artificial intelligence community \cite{bonacina2017automated}.
As artificial intelligence systems are used in a wider range of applications with greater consequences, the need to justify and verify the choices made by these systems will grow as well.
In the logical approach to artificial intelligence, theorem provers provide explanations through verifiable proofs of the decisions that they make.
On the other hand, machine learning-based approaches often fail to explain why they produced a particular answer (see e.g., \cite{miller2019explanation}). 
In order to improve the ability to explain machine learning-based systems, there have been suggestions and attempts to combine machine learning with automated reasoning tools to generate explainable results \cite{bonacina2017automated,siebert2019corg}. 
The logical approach to artificial intelligence is no longer separate from the machine learning approach.
Good proofs are therefore useful for the successful combination of these approaches, and this paper aims to improve generated proofs through proof compression.

Proof production is a key feature for modern theorem provers. 
Proofs are explanations for unsatisfiability, and are crucial for applications that require certification of a prover's answers or that extract additional information from proofs (e.g. unsat cores, interpolants, instances of quantified variables).
Mature first-order automated theorem provers, commonly based on refinements and extensions of resolution and superposition calculi \cite{Vampire,EProver,Spass,spassT,Beagle,cruanes2015extending,prover9-mace4}, support proof generation. 
However, proof production is non-trivial \cite{SchulzAPPA}, and the most efficient provers do not necessarily generate the shortest proofs.
One reason for this is that efficient resolution provers use refinements that restrict the application of inference rules.
Although fewer clauses are generated and the search space is reduced, refinements may exclude short proofs whose inferences do not satisfy the restriction.

Proof compression techniques ameliorate the difficulties that automated reasoning tools encounter during proof generation. Such techniques can be integrated into theorem provers or external tools with minimal overhead. Moreover, proof compression techniques (like those described in this paper) may result in a stronger proof which uses a strict subset of the original axioms required, which could also be considered simpler. The problem of proof compression is also closely related to Hilbert's 24th Problem \cite{Hilbert24Problem}, which asks for criteria to judge the simplicity of proofs; proof length is one possible criterion. 


%Proof certification is an important challenge for the artificial intelligence community, and longer proofs are worse explanations than shorter proofs. 
There are also technical reasons to seek smaller proofs.
Longer proofs take longer to check, consume more memory during proof-checking, occupy more storage space and are harder to exchange, may have a larger unsat core (if more input clauses are used in the proof), and have a larger Herbrand sequent if more variables are instantiated \cite{B10,B16,ResolutionHerbrand,Reis}. Recent applications of SAT solvers to mathematical problems have resulted in very large proofs; e.g., the proof of a long-standing problem in combinatorics was initially 200GB \cite{heule2016solving}. Such proofs are hard to store, let alone validate. More practically, a restriction of 100GB of disk space per benchmark per solver prevented validation of proofs in the SAT 2014 competition \cite{clausal}. 
%Although the first example is extreme, the second is not. %, especially when users may wish to instantiate parallel solvers. 
The inability to write their results to disk renders these solvers useless in some cases. Moreover, even if the only direct improvement of shorter proofs is in the communication between systems, there are indirect benefits to the end-user of a tool e.g., in terms of its responsiveness. 


For propositional resolution proofs, as those typically generated by SAT- and SMT-solvers, there is a wide variety of proof compression techniques. Algebraic properties of the resolution operation that are potentially useful for compression were investigated in \cite{bwp10}.
Compression algorithms based on rearranging and sharing chains of resolution inferences have been
developed in \cite{Amjad07} and \cite{Sinz}.  Cotton \cite{CottonSplit} proposed an algorithm that
compresses a refutation by repeatedly splitting it into a proof of a heuristically chosen literal $\ell$
and a proof of $\dual{\ell}$, and then resolving them to form a new refutation.  The {\ReduceReconstruct} algorithm \cite{RedRec} searches for locally redundant
subproofs that can be rewritten into subproofs of stronger clauses and with fewer resolution steps.
Bar-Ilan et al. \cite{RP08} and Fontaine et al. \cite{LURPI} described a linear time proof compression algorithm based on partial
regularization, which removes an inference $\eta$ when it is redundant in the sense that its pivot literal already occurs as the pivot of another inference in every path from $\eta$ to the root of the proof.

In contrast, although proof output has been a concern in first-order automated reasoning for a longer time than in propositional SAT-solving, there has been much less work on simplifying first-order proofs. For tree-like sequent calculus proofs, algorithms based on cut-introduction \cite{BrunoLPAR,Hetzl} have been proposed. However, converting a DAG-like resolution or superposition proof, as usually generated by current provers, into a tree-like sequent calculus proof may increase the size of the proof. For arbitrary proofs in the Thousands of Problems for Theorem Provers (TPTP) \cite{TPTP} format (including DAG-like first-order resolution proofs), there is an algorithm \cite{LPARCzech} that looks for terms that occur often in any Thousands of Solutions from Theorem Provers (TSTP) \cite{TPTP} proof and abbreviates them. 

The work reported in this paper lifts successful propositional proof compression algorithms to first-order logic.
We first lift the {\LowerUnits} ({\LU}) algorithm \cite{LURPI}, which delays resolution steps with unit clauses, resulting in a new algorithm that we called {\SFOLowerUnits} ({\GFOLU}). 
Following this, we lift the \texttt{Recycle\-PivotsWithIntersection} ({\RPI}) algorithm \cite{LURPI}. 
\RPI improves the \texttt{RecyclePivots} ({\RP}) algorithm \cite{RP08} by detecting nodes that can be regularized even when they have multiple children.
Earlier versions of this work appeared in \cite{GFOLU,forpigcai}.

This paper is organized as follows. 
Section \ref{sec:res} introduces the well-known first-order resolution calculus with notations that are suitable for describing and manipulating proofs as first-class objects.
Section \ref{sec:PropositionalLU} describes the propositional \LowerUnits algorithm.
Section \ref{sec:LUChallenges} demonstrates some challenges of lowering units in the context of first-order logic.
Section \ref{sec:FOLU} describes a quadratic time approach to lifting units in first-order logic while Section \ref{sec:SimpleFOLU} demonstrates a simpler, linear time approach, \GFOLU.
We then repeat this structure for {\RPI}: Section \ref{Section:RPI} summarizes the propositional {\RPI} algorithm and Section \ref{sec:FORPIChallenges} discusses the challenges that arise in the first-order case (mainly due to unification), which are not present in the propositional case, and conclude with conditions useful for first-order regularization. 
Section \ref{sec:FORPI} describes an algorithm that overcomes these challenges. 
Section \ref{sec:exp} presents experimental results obtained by applying the first-order variant of \RPI and its combinations with {\GFOLU}, on hundreds of proofs generated with the {\SPASS} theorem prover on TPTP benchmarks \cite{TPTP} and on randomly generated proofs. 
Section \ref{sec:conclusion} concludes the paper.

It is important to emphasize that this paper targets proofs in a pure first-order resolution calculus (with resolution and factoring rules only), without refinements or extensions, and without equality rules. As most state-of-the-art resolution-based provers use variations and extensions of this pure calculus and there exists no common proof format, the presented algorithm cannot be directly applied to the proofs generated by most provers, and even {\SPASS} had to be specially configured to disable {\SPASS}'s extensions in order to generate pure resolution proofs for our experiments. By targeting the pure first-order resolution calculus, we address the common theoretical basis for the calculi of various provers. In the Conclusion (Section \ref{sec:conclusion}), we briefly discuss what could be done to tackle common variations and extensions, such as splitting and equality reasoning. Nevertheless, they remain topics for future research beyond the scope of this paper.

\section{The Resolution Calculus}
\label{sec:res}

As usual, our language has infinitely many variable symbols (e.g. $x$, $y$, $z$, $x_1$, $x_2$, \ldots), constant symbols (e.g. $a$, $b$, $c$, $a_1$, $a_2$, \ldots), function symbols of every arity (e.g $f$, $g$, $f_1$, $f_2$, \ldots) and predicate symbols of every arity (e.g. $P$, $Q$, $P_1$, $P_2$,\ldots). A \emph{term} is any variable, constant or the application of an $n$-ary function symbol to $n$ terms.
An \emph{atomic formula} (\emph{atom}) is the application of an $n$-ary predicate symbol to $n$ terms. A \emph{literal} is an atom or the negation of an atom. The
\emph{complement} of a literal $\ell$ is denoted $\dual{\ell}$ (i.e. for any atom $P$,
$\dual{P} = \neg P$ and $\dual{\neg P} = P$). The \emph{underlying atom} of a literal $\ell$ is denoted $\abs{\ell}$ (i.e. for any atom $P$, $\abs{P} = P$ and $\abs{\neg P} = P$). A
\emph{clause} is a multiset of literals. $\bot$ denotes the \emph{empty clause}. A \emph{unit clause} is a clause with a single literal. Sequent notation is used for clauses (i.e. $P_1,\ldots,P_n \seq Q_1,\ldots, Q_m$ denotes the clause $\{ \neg P_1,\ldots, \neg P_n, Q_1, \ldots, Q_m \}$).
%$\freevar{t}$ (resp. $\freevar{\ell}$, $\freevar{\clause}$) denotes the set of variables in the term $t$ (resp. in the literal $\ell$ and in the clause $\clause$).
A \emph{substitution} $\{ x_1\backslash t_1, x_2 \backslash t_2, \ldots \}$ is a mapping from variables $\{ x_1, x_2, \ldots \}$ to, respectively, terms $\{t_1, t_2, \ldots \}$. The application of a substitution $\sigma$ to a term $t$, a literal $\ell$ or a clause $\clause$ results in, respectively, the term $t \sigma$, the literal $\ell \sigma$ or the clause $\clause \sigma$, obtained from $t$, $\ell$ and $\clause$ by replacing all occurrences of the variables in $\sigma$ by the corresponding terms in $\sigma$. A literal $\ell$ \emph{matches} another literal $\ell'$ if there is a substitution $\sigma$ such that $\ell\sigma=\ell'$. A \emph{unifier} of a set of literals is a substitution that makes all literals in the set equal. We will use $X \sqsubseteq Y$ to denote that $X$ \emph{subsumes} $Y$, when there exists a substitution $\sigma$ such that $X\sigma \subseteq Y$.

%The resolution calculus used in this paper has the following inference rules: 

% \begin{definition}[First-Order Axiom] A first-order axiom has no premises and concludes some clause $\Gamma$, as below.
% \begin{prooftree}
% \AxiomC{$~$ }
% \UnaryInfC{$\psi$: $\Gamma$}
% \end{prooftree}
% %where $\Gamma$ is a clause.
% \end{definition}

\begin{definition}[Resolution] \label{def:fores} \hfill
%An instance of first-order resolution requires two premises, as below.
\begin{prooftree}
\AxiomC{$\eta_1$: $\Gamma_L' \cup \{\ell_L\}$ }
\AxiomC{$\eta_2$: $\Gamma_R'\cup \{\ell_R\}$ }
\BinaryInfC{$\psi$: $\Gamma_L'\sigma_L \cup \Gamma_R'\sigma_R$}
\end{prooftree}
where $\sigma_L$ and $\sigma_R$ are substitutions such that $\ell_L\sigma_L=\dual{\ell_R}\sigma_R$. The literals $\ell_L$ and $\ell_R$ are \emph{resolved literals}, whereas $\ell_L \sigma_L$ and $\ell_R \sigma_R$ are its \emph{instantiated resolved literals}. The \emph{pivot} is the underlying atom of its instantiated resolved literals (i.e. $\abs{\ell_L \sigma_L}$ or, equivalently, $\abs{\ell_R \sigma_R}$).
\end{definition}

\begin{definition}[Factoring] \label{def:fofact} \hfill
%An instance of first-order factoring applies a unifier to the literals in a single premise's conclusion, as below.
\begin{prooftree}
\AxiomC{$\eta_1$: $\Gamma' \cup \{\ell_1,\ldots,\ell_n\}$ }
\UnaryInfC{$\psi$: $\Gamma'\sigma \cup \{\ell\}$}
\end{prooftree}
where $\sigma$ is a unifier of $\{\ell_1,\ldots,\ell_n\}$ and $\ell=\ell_i\sigma$ for any $i\in \{1,\ldots,n\}$.
\end{definition} %UITP

A \emph{resolution proof} is a directed acyclic graph of clauses where the edges correspond to the inference rules of resolution and factoring, as explained in detail in Definition \ref{def:proof}. A \emph{resolution refutation} is a resolution proof with root $\bot$.

\begin{definition}[First-Order Resolution Proof] 
\label{def:proof}% \hfill \\
A directed acyclic graph $\langle V, E, \clause \rangle$, where $V$ is a set of nodes and $E$ is a
set of edges labeled by literals and substitutions (i.e. $E \subset V \times 2^{\mathcal{L}} \times \mathcal{S} \times V$, where $\mathcal{L}$ is the set of all literals and $\mathcal{S}$ is the set of all substitutions, and $\n_1
\xrightarrow[\sigma]{\ell} \n_2$ denotes an edge from node $\n_1$ to node $\n_2$ labeled by the literal $\ell$ and the substitution $\sigma$), is a
proof of a clause $\clause$ iff it is inductively constructible according to the following cases:
%
\begin{itemize}
  \item \textbf{Axiom:} If $\Gamma$ is a clause, $\axiom{\Gamma}$ denotes some proof $\langle \{ \n \}, \varnothing,
    \Gamma \rangle$, where $\n$ is a new node. % (axiom) node.
  \item \textbf{Resolution\footnote{This is referred to as ``binary resolution'' elsewhere, with the understanding that ``binary'' refers to the number of resolved literals, rather than the number of premises of the inference rule.}:} If $\psi_L$ is a proof $\langle V_L, E_L, \clause_L \rangle$ and
    $\psi_R$ is a proof $\langle V_R, E_R, \clause_R \rangle$, $\sigma_L$ and $\sigma_R$ are substitutions s.t. $\ell_L\sigma_L=\dual{\ell_R}\sigma_R$,
    %and $\sigma_L$ and $\sigma_R$ are substitutions such that
    %$\ell_L \sigma_L = \dual{\ell_R} \sigma_R$ %and
    %$\freevar{\left( \clause_L \setminus \left\{ \ell_L \right\} \right) \sigma_L} \cap 
     %\freevar{\left( \clause_R
     %               \setminus \left\{ \ell_R \right\} \right) \sigma_R} = \emptyset$, 
    then
    $\psi_L \res{\ell_L}{\sigma_L}{\ell_R}{\sigma_R} \psi_R$ denotes a proof $\langle V, E, \Gamma \rangle$ s.t.
%\begin{align*}
%     V  &= V_L \cup V_R \cup \{\n \},  \hspace*{1cm}\Gamma = \clause_L' \sigma_L \cup  \clause_R' \sigma_R, \\
%     E &= E_L \cup E_R \cup  \left\{ \raiz{\psi_L} \xrightarrow[\sigma_L]{\{\ell_L\} } \n,   \raiz{\psi_R} \xrightarrow[\sigma_R]{\{\ell_R\} } \n \right\},
%\end{align*}
\begin{align*}
     V  &= V_L \cup V_R \cup \{\n \}, ~\Gamma = \clause_L' \sigma_L \cup  \clause_R' \sigma_R,~
     E = E_L \cup E_R \cup  \left\{ \raiz{\psi_L} \xrightarrow[\sigma_L]{\{\ell_L\} } \n,   \raiz{\psi_R} \xrightarrow[\sigma_R]{\{\ell_R\} } \n \right\},
\end{align*}
%UITP
%    \begin{align*}
%     \hspace{-0.6cm} V &= V_L \cup V_R \cup \{\n \}    \\
%      \hspace{-0.6cm} E &= E_L \cup E_R \cup 
%                    \left\{ \raiz{\psi_L} \xrightarrow[\sigma_L]{\{\ell_L\} } \n, 
%                            \raiz{\psi_R} \xrightarrow[\sigma_R]{\{\ell_R\} } \n \right\}    \\
%    \hspace{-0.6cm}  \Gamma &= \clause_L' \sigma_L \cup  \clause_R' \sigma_R
%    \end{align*}
    where $\n$ is a new (resolution) node and $\raiz{\varphi}$ denotes the root node of $\varphi$.  The literals $\ell_L$ and $\ell_R$ are \emph{resolved literals}, whereas $\ell_L \sigma_L$ and $\ell_R \sigma_R$ are its \emph{instantiated resolved literals}. The \emph{pivot} is the underlying atom of its instantiated resolved literals (i.e. $\abs{\ell_L \sigma_L}$ or, equivalently, $\abs{\ell_R \sigma_R}$).
  \item \textbf{Factoring:}
  %\footnote{This is often called ``Factoring'', but we prefer ``contraction'', because it is essentially the contraction rule of sequent calculus generalized with unification.} 
  If $\psi'$ is a proof $\langle V', E', \clause' \rangle$, $\sigma$ is a unifier of $\{\ell_1,\ldots,\ell_n\}$, and $\ell=\ell_i\sigma$ for any $i\in \{1,\ldots,n\}$, then $\con{\psi}{\{\ell_1, \ldots \ell_n\}}{\sigma}$ denotes a proof $\langle V, E, \Gamma \rangle$ s.t.
  %UITP
    \begin{align*}
         \hspace{-0.6cm} V &= V' \cup \{\n \},  \hspace*{1cm} \Gamma = \clause' \sigma \cup \{ \ell \}, \hspace*{1cm}  E = E' \cup \{ \raiz{\psi'} \xrightarrow[\sigma]{\{\ell_1, \ldots \ell_n\}} \n \},
%         \hspace{-0.6cm} E &= E' \cup \{ \raiz{\psi'} \xrightarrow[\sigma]{\{\ell_1, \ldots \ell_n\}} \n \} 
    \end{align*}  
%    \begin{align*}
%         \hspace{-0.6cm} V &= V' \cup \{\n \} \\
%         \hspace{-0.6cm} E &= E' \cup \{ \raiz{\psi'} \xrightarrow[\sigma]{\{\ell_1, \ldots \ell_n\}} \n \} \\
%       \hspace{-0.6cm} \Gamma &= \clause' \sigma \cup \{ \ell \}
%    \end{align*}
    where $\n$ is a new (factoring) node, and $\raiz{\varphi}$ denotes the root node of $\varphi$.
  \qed
\end{itemize}
\end{definition}





%UITP TODO: include resolution example?




%\noindent
%The resolution and contraction (factoring) rules described above are the standard rules of the resolution calculus, except for the fact that we do not require resolution to use most general unifiers. The presentation of the resolution rule here uses two substitutions, in order to explicitly handle the necessary renaming of variables, which is often left implicit in other presentations of resolution.
%\noindent
%When we write $\psi_L \res{\ell_L}{}{\ell_R}{} \psi_R$, we assume that the omitted substitutions are such that the resolved atom is most general. 
%We write $\con{\psi}{}{}$ for an arbitrary maximal contraction, and $\con{\psi}{}{\sigma}$ for a (pseudo-)contraction that does merge no literals but merely applies the substitution $\sigma$. 
%When the literals and substitutions are irrelevant or clear from the context, we may write simply $\psi_L \res{}{}{}{} \psi_R$. % instead of $\psi_L \res{\ell_L}{\sigma_L}{\ell_R}{\sigma_R} \psi_R$.
%The $\res{}{}{}{}$ operator is assumed to be left-associative. 
%In the propositional case, we omit contractions (treating clauses as sets instead of multisets) and $\psi_L \res{\ell}{\emptyset}{\dual{\ell}}{\emptyset} \psi_R$ is abbreviated by $\psi_L \odot_{\ell} \psi_R$.

%If $\psi = \varphi_L \odot \varphi_R$ or $\psi = \con{\varphi}{}{}$, then $\varphi$, $\varphi_L$ and $\varphi_R$ are \emph{direct subproofs} of $\psi$ and $\psi$ is a \emph{child} of both $\varphi_L$ and $\varphi_R$. The
%transitive closure of the direct subproof relation is the \emph{subproof} relation. A subproof which has no direct subproof is an \emph{axiom} of the proof.
%
%$\Vertices{\psi}$, $\Edges{\psi}$ and $\Conclusion{\psi}$
%denote, respectively, the nodes, edges and proved clause (conclusion) of $\psi$. If $\psi$ is a proof ending with a resolution node, then $\psi_L$ and $\psi_R$ denote, respectively, the left and right premises of $\psi$.


\section{The Propositional Algorithm}

{\RPI} (formally defined in Appendix \ref{Section:RPI} and in \cite{LURPI}) removes \emph{irregularities}, which are resolution inferences with a node $\eta$ when the resolved literal occurs as the pivot of another inference located below in the path from $\eta$ to the root of the proof. In the worst case, regular resolution proofs can be exponentially bigger than irregular ones, but {\RPI} takes care of regularizing the proof only partially, removing inferences only when this does not enlarge the proof.

%ToDo: Informal textual description of the propositional algorithm, explaining what safe literals are. 
{\RPI} traverses the proof twice. On the first traversal (bottom-up), it computes and stores for each node a set of \emph{safe literals}: literals that are resolved in all paths from the node to the root of the proof or that occur in the root clause. If one of the node's resolved literals belongs to the set of safe literals, then it is possible to \emph{regularize} the node by replacing it by the parent containing the safe literal. To do this replacement efficiently, the replacement is postponed by marking the other parent as a \texttt{deletedNode}. Then, on a single second traversal (top-down), regularization is performed: any node that has a parent node marked as a \texttt{deletedNode} is replaced by its other parent.
%Refer reader to the CADE 2011 paper (where RPI is described) for a formal description of the propositional algorithm. 
% contains a formal description of {\RPI} (taken from \cite{LURPI}).
%Consider adding the formal description to an appendix in this paper, for the convenience of the reviewer.

The {\RPI} and the {\RP} algorithms differ from each other mainly in the
computation of the safe literals of a node that has many children. While the former 
returns the intersection as shown in Algorithm~\ref{algo:SetSafeLiterals}, the latter
returns the empty set. 
Moreover, while in {\RPI} the safe literals of the root node contain all the literals of the root clause, in {\RP} the root node is always assigned an empty set of literals.  






\section{First-Order Challenges}\label{sec:Challenges}

TODO by Jan (just writing some ideas so far--not yet final by any means)\\
{\bf Does this belong here?And is this what you had in mind for interesting examples? And are the proof formatted correctly, or should I change them? aside from the first one going over the margin right now of course}\\ 

In this section, we discuss additional requirements for lowering a unit formula in the first order case that are not required in the propositional case.

%example 1: shows requirement for pair-wise unifiability with unit
%obvious - skip?

%example 2: shows requirement for pair-wise unifiability within all aux formulas

 \begin{example} The following example shows why we must check pair-wise unifiability with the literals resolved against the unit we're trying to lower.

% \begin{tiny}
% \begin{prooftree}
% \def\e{\mbox{\ $\vdash$\ }}
% \AxiomC{$\eta_2$}
% \AxiomC{$\eta_1$: $p(a)$\e$q(Y),r(Z)$}
% \AxiomC{$\eta_2$: \e $p(X)$}
% \BinaryInfC{$\eta_3$: \e$q(Y),r(Z)$}
% \AxiomC{$\eta_4$: $r(X),p(b)$\e $s(Y)$}
% \BinaryInfC{$\eta_5$: $p(b)$\e $s(Y),q(Y)$}
% \AxiomC{$\eta_6$: $s(Y), q(Y)$\e}
% \BinaryInfC{$\eta_7$: $p(b)$\e}
% \BinaryInfC{$\psi$: $\bot$}
% \end{prooftree}
% \end{tiny}
 \end{example}


%example 3: shows requirement for contraction check

 \begin{example} The following shows why the above is not necessarily  enough (we must check the original sources of the aux formulas, and see if those can be contracted), otherwise we might not save anything.
% \begin{footnotesize}
% \begin{prooftree}
% \def\e{\mbox{\ $\vdash$\ }}
% \AxiomC{$\eta_1$: $r(Y),p(X ~q(Y~b)), p(X~Y)$\e}
% \AxiomC{$\eta_2$: \e $p(U~V)$}
% \BinaryInfC{$\eta_3$: $r(V),p(U ~q(V~b))$\e}
% \AxiomC{$\eta_4$: \e $r(W)$}
% \BinaryInfC{$\eta_5$: $p(U ~q(W~b))$\e}
% \AxiomC{$\eta_2$}
% \BinaryInfC{$\psi$: $\bot$}
% \end{prooftree}
% \end{footnotesize}
 \end{example}


%example 4: requires FOSubstitution, introduces this concept?

\section{First-Order RecyclePivotsWithIntersection}
\label{sec:FORPI}
%TODO: this section
This section presents {\FORPI} (Algorithm \ref{algo:FORPI}), a first order generalization of {\RecyclePivotsIntersection}, which aims to compress irregular proofs. Recall that \RecyclePivotsIntersection
is a modification of the \texttt{RecyclePivots} algorithm, %described in  \cite{Bar-IlanFuhrmannHooryShachamStrichman2009Linear-time-reductions-of-resolution-proofs}, from which it derives its name. 
and \RecyclePivotsIntersection provides better compression on proofs where nodes have several children, when compared to \texttt{RecyclePivots}. Through a small modification to our algorithm (described later), a first order generalization of \texttt{RecyclePivots} is also possible.

%TODO: move to intro?
%Although in the worst case full regularization can increase the proof length exponentially \cite{Tseitin1983On-The-Complexity-of-Proofs-in-Propositional-Logics}, these algorithms show that many irregular proofs can have their length decreased if a careful partial regularization is performed. 

\newcommand{\la}{\leftarrow}


\begin{algorithm}[!b]
\begin{footnotesize}
\SetKwInOut{Input}{input}\SetKwInOut{Output}{output}
\SetKwData{units}{unitsQueue}
\SetKwData{fixedUnits}{fixedUnitsQueue}

\Input{A first-order proof $\psi$}
\Output{A possibly less-irregular first-order proof $\psi'$}

\BlankLine

$\psi'$ $\la$ $\psi$\;
traverse $\psi'$ bottom-up and \ForEach{node $\eta$ in $\psi'$}{
   \If{$\eta$ is a resolvent node}{
     setSafeLiterals($\eta$) \;
     regularizeIfPossible($\eta$)
   }
  }
$\psi'$ $\la$ fix($\psi'$) \;
\Return {$\psi'$}\;
\caption{\label{algo:FORPI} \texttt{\FORPI}}
\end{footnotesize}
\end{algorithm}



Our generalization, Algorithm~\ref{algo:FORPI}, follows the propositional idea of traversing the proof in a bottom-up manner, storing for every node a set of \emph{safe literals} that get resolved in all paths below it in the proof (or that already occurred in the root clause of the original proof). If one of the node's resolved literals can be unified to a literal in the set of safe literals, then it may be possible to regularize the node by replacing it by one of its parents. 

%TODO: re-write -- taken from FORPI paper
In the propositional case, regularization of a node replaces it by the parent whose clause contains the resolved literal that is safe. In the first order case, because unification introduces complications like those seen in Example \ref{ex:unifcheck}, we ensure that the replacement parent is (possibly after unification) contained entirely in the safe literals. This ensures that the remainder of the proof does not expect a variable to be unified to different values simultaneously. After regularization, all nodes below the regularized node may have to be fixed. 
Similar to \RecyclePivotsIntersection, instead of replacing the irregular node by one of its parents immediately, 
its other parent is replaced by \texttt{deletedNodeMarker}, as shown in Algorithm~\ref{algo:Regularize}.
As in the propositional case, fixing of the proof is postponed to another (single) traversal, as regularization proceeds bottom up and only nodes below a regularized node may require fixing.
During fixing, the irregular node is actually replaced by the parent that is not \texttt{deletedNodeMarker}.


%Unchanged from propositional case? %TODO: or should the $\in$ relation be unification?
\begin{algorithm}[t]
\begin{footnotesize}

\SetKwInOut{Input}{input}\SetKwInOut{Output}{output}
\SetKwData{units}{unitsQueue}
\SetKwData{fixedUnits}{fixedUnitsQueue}

\Input{A node $\psi=\psi_L  \res{\ell_L}{\sigma_L}{\ell_R}{\sigma_R} \psi_R$}
\Output{nothing (but the proof containing $\psi$ may be changed)}

\BlankLine
    \uIf{$\exists \sigma$  and $l \in \psi${\upshape.safeLiterals} such that $\sigma l = l_R$ or $l=\sigma l_R$}{
     \uIf{$\exists \sigma$ such that $\sigma\psi_R\subseteq\psi${\upshape.safeLiterals}} {
      replace $\psi_L$ of $\eta$ by \texttt{deletedNodeMarker} \;
      mark $\psi$ as regularized
}
    }
    \ElseIf{$\exists \sigma$  and $l \in \psi${\upshape.safeLiterals} such that $\sigma l = l_L$ or $l=\sigma l_L$}{
     \uIf{$\exists \sigma$ such that $\sigma\psi_L\subseteq\psi${\upshape.safeLiterals}} {
      replace $\psi_R$ by \texttt{deletedNodeMarker} \;
      mark $\psi$ as regularized
}
    }
\caption{\label{algo:Regularize} \texttt{regularizeIfPossible}}
\end{footnotesize}
\end{algorithm}


\begin{algorithm}[!b]
\begin{footnotesize}

\SetKwInOut{Input}{input}\SetKwInOut{Output}{output}
\SetKwData{units}{unitsQueue}
\SetKwData{fixedUnits}{fixedUnitsQueue}

\Input{A node $\eta$}
\Output{nothing (but the node $\eta$ gets a set of safe literals)}

\BlankLine

    \uIf{$\eta$ is a root node with no children}{
      $\eta$.safeLiterals $\la$ $\eta$.clause  
    }
    \Else{
      \ForEach{$\eta'$ $\in$ $\eta${\upshape.children}}{
        \uIf{$\eta'$ is marked as regularized}{ 
          safeLiteralsFrom($\eta'$) $\la$ $\eta'$.safeLiterals \;}
        \uElseIf{$\eta$ is left parent of $\eta'$}{ 
        	safeLiteralsFrom($\eta'$) $\la$ $\eta'$.safeLiterals $\cup$ \{ $\eta'$.rightResolvedLiteral \} \;
        }
        \ElseIf{$\eta$ is right parent of $\eta'$}{ 
			safeLiteralsFrom($\eta'$) $\la$ $\eta'$.safeLiterals $\cup$ \{ $\eta'$.leftResolvedLiteral \} \;
        }
      }
      $\eta$.safeLiterals $\la$ $\bigcap_{\eta' \in \eta\textrm{.children}}$ safeLiteralsFrom($\eta'$)
    }
\caption{\label{algo:SetSafeLiterals} \texttt{setSafeLiterals}}
\end{footnotesize}
\end{algorithm}

%The set of safe literals of a node $\eta$ can be computed from the set of safe literals of its children (cf.\ Algorithm~\ref{algo:SetSafeLiterals}). In the case when $\eta$ has a single child $\varsigma$, the safe literals of $\eta$ are simply the safe literals of $\varsigma$ together with the resolved literal $p$ of $\varsigma$ belonging to $\eta$ ($p$ is safe for $\eta$, because whenever $p$ is propagated down the proof through $\eta$, $p$ gets resolved in $\varsigma$). It is important to note, however, that if $\varsigma$ has been marked as regularized, it will eventually be replaced by $\eta$, and hence $p$ should not be added to the safe literals of $\eta$. In this case, the safe literals of $\eta$ should be exactly the same as the safe literals of $\varsigma$. When $\eta$ has several children, the safe literals of $\eta$ w.r.t. a child $\varsigma_i$ contain literals that are safe on all paths that go from $\eta$ through $\varsigma_i$ to the root. For a literal to be safe for all paths from $\eta$ to the root, it should therefore be in the intersection of the sets of safe literals w.r.t. each child.

The {\RecyclePivotsIntersection} and the \texttt{RecyclePivots} algorithms differ from each other mainly in the
computation of the safe literals of a node that has many children. While the former 
returns the intersection as shown in Algorithm~\ref{algo:SetSafeLiterals}, the latter
returns the empty set. 
Further, while in \RecyclePivotsIntersection the safe literals of the root node contain all the literals of the root clause, in \texttt{RecyclePivots} the root node is always assigned an empty set of literals. 
This is easy accomplished in the first order case by changing lines 11 and 2, respectively, of Algorithm~\ref{algo:SetSafeLiterals}.
This makes a difference only when the proof is not a refutation.

The set of safe literals of a node $\eta$ can be computed from the set of safe literals of its children (cf.\ Algorithm~\ref{algo:SetSafeLiterals}), in a manner identical to the propositional case.


%Note that during a traversal of the proof,  the lines from 5 to 10 in Algorithm~\ref{algo:SetSafeLiterals} are executed as many times as the number of edges in the proof.  Since every node has at most two parents, the number of edges is at most twice the number of nodes.  Therefore, during a traversal of a proof with $n$ nodes, lines from 5 to 10 are executed at most $2n$ times, and the algorithm remains linear. In our prototype implementation, the sets of safe literals are instances of Scala's  \texttt{mutable.HashSet} class. Being mutable, new elements can be added efficiently. And being HashSets, membership checking is done in constant time in the average case, and set intersection (line 12) can be done in $O(k.s)$, where $k$ is the number of sets and $s$ is the size of the smallest set.






\section{Experiments} \label{sec:exp}

A prototype\footnote{Source code available at \url{https://github.com/jgorzny/Skeptik}} of a (two-traversal) version of {\SFOLowerUnits} has been implemented in the functional programming language Scala\footnote{\url{http://www.scala-lang.org/}} as part of the \skeptik
 library\footnote{\url{https://github.com/Paradoxika/Skeptik}}. 

Before evaluating this algorithm, we first generated several benchmark proofs. This was done by executing the {\SPASS}\footnote{\url{http://www.spass-prover.org/}} theorem prover on ToDo(numberOfProblems) problems of the ToDo categories of the TPTP Problem Library \footnote{\url{http://www.cs.miami.edu/{\textasciitilde}tptp/}}. In order to generate pure resolution proofs, most advanced inference rules used by {\SPASS}  were disabled. The Euler Cluster at the University of Victoria\footnote{\url{https://rcf.uvic.ca/euler.php}} was used and the time limit was 300 seconds per problem. Under these conditions, {\SPASS} was able to generate 308 proofs. 

The evaluation of {\SFOLowerUnits} was performed on a laptop (2.8GHz Intel Core i7 processor with 4 GB of RAM (1333MHz DDR3) available to the Java Virtual Machine). For each benchmark proof $\psi$, we measured\footnote{The raw data is available at ToDo (this link is not working) \url{https://docs.google.com/spreadsheets/d/1F1-t2OuhypmTQhLU6yTj42aiZ5CqqaZvhVvOzeFgn0k/edit\#gid=1182923972}} the time needed to compress the proof ($t(\psi)$) and the compression ratio ($(|\psi|-|\alpha(\psi)|)/|\psi|$), where $|\psi|$ is the length of $\psi$ (i.e. the number of axioms, resolution and contractions (ignoring substitutions)) and $\alpha(\psi)$ is the result of applying {\SFOLowerUnits} to $\psi$.

The proofs generated by {\SPASS} were small (with lengths from 3 to 49). These proofs are specially small in comparison with the typical proofs generated by SAT- and SMT-solvers, which usually have from a few hundred to a few million nodes. The number of proofs (compressed and uncompressed) per length is shown in Figure \ref{fig:ex} (b). Uncompressed proofs are those which had either no lowerable units to lower or for which \SFOLowerUnits failed and returned the original proof. Such failures occurred on only 14 benchmark proofs. Among the smallest of the 308 proofs, very few proofs were compressed. This is to be expected, since the likelihood that a very short proof contain a lowerable unit (or even merely a unit with more than one child) is low. The proportion of compressed proofs among longer proofs is, as expected, larger, since they have more nodes and it is more likely that some of these nodes are lowerable units. 13 out of 18 proofs with length greater than or equal to 30 were compressed. 

Figure \ref{fig:ex} (a) shows a box-whisker plot of compression ratio with proofs grouped by length and whiskers indicating minimum and maximum compression ratio achieved within the group. Besides the median compression ratio (the horizontal thick black line), the chart also shows the mean compression ratios for all proofs of that length and for all compressed proofs (the red cross and the blue circle). In the longer proofs (length greater than 34), the median and the means are in the range from 5\% to 15\%, which is satisfactory in comparison with the total compression ratio of 7.5\% that has been measured for the propositional {\LowerUnits} algorithm on much longer propositional proofs \cite{Boudou}.

Figure \ref{fig:ex} (c) shows a scatter plot comparing the length of the input proof against the length of the compressed proof. For the longer proofs (circles in the right half of the plot), it is often the case that the length of the compressed proof is significantly lesser than the length of the input proof.

Figure \ref{fig:ex} (d) plots the cumulative original and compressed lengths of all benchmark proofs (for an x-axis value of $k$, the cumulative curves show the sum of the lengths of the shortest $k$input proofs). The total cumulative length of all original proofs is ToDo:4500(put the correct number here) while the cumulative length of all proofs after compression is ToDo:4000(correct this number). This results in a total compression ratio of ToDo:12\%(compute this number), which is impressive, considering the inclusion of all the short proofs (in which the presence of lowerable units is a priori unlikely) tends to decrease the total compression ratio. For comparison, the total compression ratio considering only the 100 longest input proofs is ToDo:(compute this percentage).

Figure \ref{fig:ex} also indicates an interesting potential trend. The gap between the two cumulative curves seems to grow superlinearly. If this trend is extrapolated, progressively larger compression ratios can be expected for longer proofs. This is compatible with Theorem 10 in \cite{LURPI}, which shows that, for proofs generated by eagerly resolving units against all clauses, the propositional {\LowerUnits} algorithm can achieve quadratic assymptotic compression. SAT- and SMT-solvers based on CDCL (Conflict-Driven Clause Learning) avoid eagerly resolving unit clauses by dealing with unit clauses via boolean propagation on a conflict graph and extracting subproofs from the conflict graph with every unit being used at most once per subproof (even when it was used multiple times in the conflict graph). Saturation-based automated theorem provers, on the other hand, might be susceptible to the eager unit resolution redundancy described in Theorem 10 \cite{LURPI}. This potential trend would need to be confirmed by further experiments with more data (more proofs and longer proofs).

The total time needed by {\SPASS} to generate all 308 proofs on the Euler Cluster was ToDo. The total time for {\SFOLowerUnits} to be executed on all 308 proofs was ToDo on a simple laptop. (ToDo: make sure the total time calculation either includes or excludes parsing times for both Skeptik and SPASS. otherwise the comparison would be biased and unfair). Therefore, {\SFOLowerUnits} is a fast algorithm. For a small overhead in time (in comparison to proving time), it may simplify the proof considerably.


% \begin{figure}
% \includegraphics[scale=0.5]{images/compress_time_vs_proof_length.pdf}
% \end{figure}

% \begin{figure}
% \includegraphics[scale=0.5]{images/compress_time_vs_proof_length_res.pdf}
% \end{figure}

% \begin{figure}
% \includegraphics[scale=0.5]{images/compress_ratio_vs_proof_length.pdf}
% \end{figure}

%\begin{figure}\label{fig:compressRatioResVLength} %USED
%\includegraphics[scale=0.5]{images/compress_ratio_res_vs_proof_length.pdf}
%\end{figure}

% \begin{figure}
% \includegraphics[scale=0.5]{images/compress_ratio_res_vs_proof_length_res.pdf}
% \end{figure}

%\begin{figure}%USED
%\includegraphics[scale=0.5]{images/compress_ratio_res_vs_proof_length_all_proofs.pdf}
%\end{figure}
\begin{figure}
\centering
%    \subfloat[Average compression (only success)]{{\includegraphics[scale=0.5]{images/compress_ratio_res_vs_proof_length.pdf} }}
    \subfloat[Compression ratio]{{\includegraphics[scale=0.5]{images/compress_ratio_res_vs_proof_length_all_proofs.pdf} }}%\hfilll
    \subfloat[Number of (non-)compressed proofs]{{\includegraphics[scale=0.5]{images/num_compressed_stacked.pdf}}}\hfill
    \subfloat[Compressed length against input length]{{\includegraphics[scale=0.5]{images/compress_length_no_sub_vs_length_all_proofs.pdf} }}
%    \subfloat[Total proof nodes]{{\includegraphics[scale=0.5]{images/cumulative_res_nodes_no_subs.pdf} }}
    \subfloat[Cumulative proof lengths]{{\includegraphics[scale=0.5]{images/cumulative_res_nodes_no_subs_top100.pdf}}}
\caption{Empirical evaluation results}
\label{fig:ex}
\end{figure}


%\begin{figure}
%\centering
%    \subfloat{{\includegraphics[scale=0.5]{images/compress_ratio_res_vs_proof_length.pdf}
%}}%
%    \subfloat{{\includegraphics[scale=0.5]{images/compress_ratio_res_vs_proof_length_all_proofs.pdf} }}%
%\caption{Compression ratio versus proof length without uncompressed proofs (left) and with with uncompressed proofs (right).}
%\label{fig:ex1}
%\end{figure}



% \begin{figure}
% \includegraphics[scale=0.5]{images/num_compressed_count.pdf}
% \end{figure}

% \begin{figure}
% \includegraphics[scale=0.5]{images/num_compressed_percent.pdf}
% \end{figure}


%\begin{figure}%USED
%\includegraphics[scale=0.5]{images/num_compressed_stacked.pdf}
%\end{figure}

%\begin{figure}
%\centering
%    \subfloat{{\includegraphics[scale=0.5]{images/num_compressed_stacked.pdf}
%}}%
%    \subfloat{{\includegraphics[scale=0.5]{images/cumulative_res_nodes_no_subs.pdf} }}
%\caption{Number of proofs compressed of each length (left), and total number of nodes before and after compression (right).}
%\label{fig:ex2}
%\end{figure}


% \begin{figure}
% \includegraphics[scale=0.5]{images/res_length_vs_compress_res_length_all_proofs.pdf}
% \end{figure}
% \begin{figure}
% \includegraphics[scale=0.5]{images/res_length_vs_compress_res_length.pdf}
% \end{figure}

% \begin{figure}
% \includegraphics[scale=0.5]{images/cumulative_res_nodes.pdf}
% \end{figure}

%\begin{figure}%USED
%\includegraphics[scale=0.5]{images/cumulative_res_nodes_no_subs.pdf}
%\end{figure}

%\begin{figure} %USED
%\includegraphics[scale=0.5]{images/cumulative_res_nodes_no_subs_top100.pdf}
%\end{figure}




% \begin{figure}
% \includegraphics[scale=0.5]{images/cumulative_res_nodes_no_subs_log.pdf}
% \end{figure}
% \begin{figure}
% \includegraphics[scale=0.5]{images/compress_length_no_sub_vs_length.pdf}
% \end{figure}

%\begin{figure}%USED
%\includegraphics[scale=0.5]{images/compress_length_no_sub_vs_length_all_proofs.pdf}
%\end{figure}
\vspace{-0.25cm}
\section{Conclusions and Future Work}\label{sec:conclusion}

The main contribution of this paper is the lifting of the propositional proof compression algorithm {\RPI} to the first-order case. As indicated in Section \ref{sec:Challenges}, the generalization is challenging, because unification instantiates literals and, consequently, a node may be regularizable even if its resolved literals are not syntactically equal to any safe literal. Unification must be taken into account when collecting safe literals and marking nodes for deletion.

%We first evaluated the algorithm on all 308 real proofs that the \texttt{SPASS} theorem prover (with only standard resolution enabled) was capable of generating when executed on unsatisfiable TPTP problems without equality. Although the compression achieved by the first-order {\FORPI} algorithm was not as good as the compression achieved by the propositional {\RPI} algorithm on real proofs generated by SAT and SMT solvers \cite{LURPI}, this is due to the fact that the 308 proofs were too short (less than 32 resolutions) to contain a significant amount of irregularities. In contrast, the propositional proofs used in the evaluation of the propositional {\RPI} algorithm had thousands (and sometimes hundreds of thousands) of resolutions. 

%Our second evaluation used larger, but randomly generated, proofs. The compression achieved by {\FORPI} in a short amount of time on this data set was compatible with our expectations and previous experience in the propositional level. The obtained results indicate that {\FORPI} is a promising compression technique to be reconsidered when first-order theorem provers become capable of producing larger proofs. Although we carefully selected generation probabilites in accordance with frequencies observed in real proofs, it is important to note that randomly generated proofs may still differ from real proofs in shape and may be more or less likely to contain irregularities exploitable by our algorithm. Resolution restrictions and refinements (e.g. ordered resolution %\cite{Maslov1964,KowalskiHayes1969,OrderedRes}, 
%\cite{hsiang1991proving, OrderedRes}, hyper-resolution \cite{HyperResolution,robinson1965automatic}, unit-resulting resolution \cite{UnitResultingResolution,prover9-mace4}) may result in longer chains of resolutions and, therefore, in proofs with a possibly larger height to length ratio. As the number of irregularities increases with height, such proofs could have a higher number of irregularities in relation to length.

We evaluated the algorithm on two data sets, and
%First, we evaluated the algorithm on all 308 real proofs that the \texttt{SPASS} theorem prover (with only standard resolution enabled) was capable of generating when executed on unsatisfiable TPTP problems without equality. Although the compression achieved by the first-order {\FORPI} algorithm was not as good as the compression achieved by the propositional {\RPI} algorithm on real proofs generated by SAT and SMT solvers \cite{LURPI}, this is due to the fact that the 308 proofs were too short (less than 32 resolutions) to contain a significant amount of irregularities. %In contrast, the propositional proofs used in the evaluation of the propositional {\RPI} algorithm had thousands (and sometimes hundreds of thousands) of resolutions. 
%Our second evaluation used larger, randomly generated, proofs. 
the compression achieved by {\FORPI} in a short amount of time on this data set was compatible with our expectations and previous experience in the propositional level. 
The obtained results indicate that {\FORPI} is a promising compression technique to be reconsidered when first-order theorem provers become capable of producing larger proofs. Although we carefully selected generation probabilites in accordance with frequencies observed in real proofs, it is important to note that randomly generated proofs may still differ from real proofs in shape and may be more or less likely to contain irregularities exploitable by our algorithm. 
%Resolution restrictions and refinements (e.g. ordered resolution \cite{Maslov1964,KowalskiHayes1969,OrderedRes}, 
%\cite{hsiang1991proving, OrderedRes}, hyper-resolution \cite{HyperResolution,robinson1965automatic}, unit-resulting resolution \cite{UnitResultingResolution,prover9-mace4}) may result in longer chains of resolutions and, therefore, in proofs with a possibly larger height to length ratio. As the number of irregularities increases with height, such proofs could have a higher number of irregularities in relation to length.

In this paper, for the sake of simplicity, we considered a pure resolution calculus without restrictions, refinements or extensions. However, in practice, theorem provers do use restrictions and extensions. It is conceptually easy to adapt the algorithm described here to many variations of resolution. 
%For instance, restricted forms of resolution (e.g. ordered resolution, hyper-resolution, unit-resulting resolution) can be simply regarded as (chains of) unrestricted resolutions for the purpose of proof compression. The compression process would break the chains and change the structure of the proof, but the compressed proof would still be a correct unrestricted resolution proof, albeit not necessarily satisfying the restrictions that the input proof satisfied. 
%In the case of extensions for equality reasoning using paramodulation-like inferences, it might be necessary to apply the paramodulations to the corresponding safe literals. Alternatively, equality inferences could be replaced by resolutions with instances of equality axioms, and the proof compression algorithm could be applied to the proof resulting from this replacement. 
%Another 
For instance, a common extension of resolution is the splitting technique \cite{WeidenbachSplitting}. When splitting is used, each split sub-problem is solved by a separate refutation, and {\FORPI} could be applied to each refutation independently. 

It would be interesting to determine if proof compression could be applied during proof search, in order to improve the performance theorem provers. Additionally, it would be interesting to see if similar techniques can be applied to proofs in higher-order logics.





% These algorithms are very fast, and together they may simplify the proof considerably for a relatively quick time cost.

% {\RPI} performs best when the proofs are tall; {\FORPI} will likely perform similarly. However, the proofs in this data set are relatively short, and those compressed by {\GFOLU} first are even shorter. Thus, the performance of {\FORPI} is not surprising.

%{\FORPI} continues to support the idea of listing propositional proof compression algorithms to the first-order case. The experimental results discussed in the previous continue to be encouraging, and are consistent with trends observed in the propositional case. 

%\paragraph{Acknowledgments:}



\begin{footnotesize}
\bibliographystyle{eptcs}
\bibliography{biblio}
\end{footnotesize}
\end{document}

% vim: tw=100
