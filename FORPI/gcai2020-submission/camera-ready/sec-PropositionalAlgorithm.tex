\section{The Propositional Algorithm}\label{sec:rpi}

{\RPI} (formally defined in \cite{LURPI}) removes \emph{irregularities}, which are resolution inferences deriving a node $\eta$ when the resolved literal occurs as the pivot of another inference located below in the path from $\eta$ to the root of the proof. In the worst case, regular resolution proofs can be exponentially bigger than irregular ones, but {\RPI} takes care of regularizing the proof only partially, removing inferences only when this does not enlarge the proof.


{\RPI} traverses the proof twice. On the first traversal (bottom-up), it computes and stores for each node a set of \emph{safe literals}: literals that are resolved in all paths from the node to the root of the proof or that occur in the root clause. If one of the node's resolved literals belongs to the set of safe literals, then it is possible to \emph{regularize} the node by replacing it by the parent containing the safe literal. To do this replacement efficiently, the replacement is postponed by marking the other parent as a \texttt{deletedNode}. Then, on a single second traversal (top-down), regularization is performed: any node that has a parent node marked as a \texttt{deletedNode} is replaced by its other parent.


The {\RPI} and the {\RP} algorithms differ from each other mainly in the
computation of the safe literals of a node that has many children. While the former 
returns the intersection, 
the latter returns the empty set. 
Moreover, while in {\RPI} the safe literals of the root node contain all the literals of the root clause, in {\RP} the root node's set of safe literals is always empty. 