
\section{Lifting to First-Order}\label{sec:Challenges}




\def\e{\mbox{\ $\vdash$\ }}


 
%straightforward example
\begin{example}\label{ex:unif} 
Consider the left proof $\psi$ in Figure \ref{ex1joined}. When computed as in the propositional case, the safe literals for $\eta_3$ are $\{ Q(c), ~ P(a,x)\}$.
As neither of $\eta_3$'s resolved literals is syntactically equal to a safe literal, the propositional {\RPI} algorithm would not change $\psi$. However, $\eta_3$'s left resolved literal $P(w,x)\in \eta_1$ is unifiable with the safe literal $P(a,x)$. Regularizing $\eta_3$, by deleting the edge between $\eta_2$ and $\eta_3$ and replacing $\eta_3$ by $\eta_1$, leads to further deletion of $\eta_4$ (because it is not resolvable with $\eta_1$) and finally to the much shorter proof $\psi'$ in Figure \ref{ex1joined}.



\begin{figure}[bt]%
    \centering
    \begin{scriptsize}
\begin{bprooftree}
\AxiomC{$\eta_6$: $P(y,b)$ \e \hspace{-2cm}}
\AxiomC{$\eta_1$: \e $P(w,x)$}
\AxiomC{$\eta_2$: $P(w,x)$ \e $Q(c)$}
\BinaryInfC{$\eta_3$: \e $Q(c)$  \hspace{-1.5cm}}
\AxiomC{$\eta_4$: $Q(c)$ \e $P(a,x)$}
\BinaryInfC{$\eta_5$: \e $P(a,x)$}
\BinaryInfC{$\psi$: $\bot$}
\end{bprooftree}
%\qquad
\begin{bprooftree}
\def\e{\mbox{\ $\vdash$\ }}
\AxiomC{$\eta_1$: \e$P(w,x)$}
\AxiomC{$\eta_6$: $P(y,b)$\e}
\BinaryInfC{$\psi'$: $\bot$}
\end{bprooftree}
\end{scriptsize}

\caption{A proof $\psi$ (left), and a regularized proof $\psi'$ (right).}
\label{ex1joined}
    \end{figure}





\end{example}

\noindent
Unlike in the propositional case, where a resolved literal must be syntactically equal to a safe literal for regularization to be possible, Example \ref{ex:unif} suggests that, in the first-order case, it might suffice that the resolved literal be unifiable with a safe literal. However, there are cases where mere unifiability is not enough and greater care is needed: e.g., when $\eta_1:~ \vdash P(a,c)$ and $\eta_2: P(a,c) \vdash Q(c)$ in Example \ref{ex:unif}. One way to prevent these cases is to require the resolved literal to be not only unifiable but subsume a safe literal. 
A slight modification to the concept of safe literals, which takes into account the unifications that occur on the paths from a node to the root, results in a weaker (and better) requirement.



\begin{definition}\label{def:safefirst}
The set of \emph{safe literals} for a node $\eta$ in a proof $\psi$ with root clause $\Gamma$, denoted $\mathcal{S}(\eta)$, is such that $\ell \in \mathcal{S}(\eta)$ if and only if $\ell \in \Gamma$ or for all paths from $\eta$ to the root of $\psi$ there is an edge $\n_1
\xrightarrow[\sigma]{\ell'} \n_2$ with $\ell' \sigma = \ell$.
\end{definition}

\noindent
As in the propositional case, safe literals can be computed in a bottom-up traversal of the proof. Initially, at the root, the safe literals are exactly the literals that occur in the root clause. As we go up, the safe literals $\mathcal{S}(\eta')$ of a parent node $\eta'$ of $\eta$ where $\eta'
\xrightarrow[\sigma]{\ell} \eta$ is set to $\mathcal{S}(\eta) \cup \{ \ell \sigma \}$. Note that we apply the substitution to the resolved literal before adding it to the set of safe literals (cf. Algorithm \ref{algo:foSetSafeLiterals}, lines 8 and 10). In other words, in the first-order case, the set of safe literals has to be a set of \emph{instantiated} resolved literals.


In the modified case of Example \ref{ex:unif}, computing safe literals as in Definition \ref{def:safefirst} would result in $\mathcal{S}(\eta_3)=\{Q(c),~P(a,b)\}$, where clearly the pivot $P(a,c)$ in $\eta_1$ is not safe. A generalization of this requirement, which can be thought of a \emph{necessary} condition, is Definition \ref{prop:pair}.



\begin{definition}
\label{prop:pair}
Let $\eta$ be a node with safe literals $\mathcal{S}(\eta)$ and parents $\eta_1$ and $\eta_2$, assuming without loss of generality, $\eta_1 \xrightarrow[\sigma_1]{\{\ell_1\} } \eta$.
The node $\eta$ is said to be \emph{pre-regularizable} in the proof $\psi$ if $\ell_1\sigma_1$ matches a safe literal $\ell^* \in \mathcal{S}(\eta)$.
\end{definition}

\noindent



\begin{figure*}
\begin{scriptsize}
\begin{prooftree}
\def\e{\mbox{\ $\vdash$\ }}
\AxiomC{$\eta_8$: $Q(f(a,e),c)\e$}
\AxiomC{$\eta_6$: $\e P(c,d)$ \hspace{-2cm}}
\AxiomC{$\eta_1$: $P(u,v)\e Q(f(a,v),u)$}
\AxiomC{$\eta_2$: $Q(f(a,x),y),Q(t,x)\e Q(f(a,z),y)$}
\BinaryInfC{$\eta_3$: $P(u,v),Q(t,v)\e Q(f(a,z),u)$}
\AxiomC{\hspace{-1.25cm} $\eta_4$: $\e Q(r,s)$}
\BinaryInfC{$\eta_5$: $P(u,v)\e Q(f(a,z),u)$}
\BinaryInfC{$\eta_7$: $\e Q(f(a,z),c)$}
\BinaryInfC{$\psi$: $\bot$}
\end{prooftree}
\end{scriptsize}
\caption{An example where pre-regularizability is not sufficient.}
\label{fig:ex-unifcheck}
\end{figure*}
%extra check example
\begin{example}\label{ex:unifcheck}


Satisfying the pre-regularizability is not sufficient. Consider the proof $\psi$ in Figure \ref{fig:ex-unifcheck}. After collecting the safe literals, $\mathcal{S}(\eta_3) = \{\lnot Q(r,v),\lnot P(c,d), Q(f(a,e),c)\}$.
%\noindent
$\eta_3$'s pivot $Q(f(a,v),u)$ matches the safe literal $Q(f(a,e),c)$. Attempting to regularize $\eta_3$ would lead to the removal of $\eta_2$, the replacement of $\eta_3$ by $\eta_1$ and the removal of $\eta_4$ (because $\eta_1$ does not contain the pivot required by $\eta_5$), with $\eta_5$ also being replaced by $\eta_1$. Then resolution between $\eta_1$ and $\eta_6$ results in $\eta_7'$, which cannot be resolved with $\eta_8$, as shown below.


\begin{scriptsize}
\begin{prooftree}
\def\e{\mbox{\ $\vdash$\ }}
\AxiomC{$\eta_8$: $Q(f(a,e),c)\e$ \hspace{-0.5cm}}
\AxiomC{$\eta_6$: $\e P(c,d)$}
\AxiomC{$\eta_1$: $P(u,v)\e Q(f(a,v),u)$}
\BinaryInfC{$\eta_7'$: $\e Q(f(a,d),c)$}
\BinaryInfC{$\psi'$: ??}
\end{prooftree}
\end{scriptsize}

\noindent
$\eta_1$'s literal $Q(f(a, v), u)$, which would be resolved with $\eta_8$'s literal, was changed to $Q(f(a,d),c)$ due to the resolution between $\eta_1$ and $\eta_6$.
\end{example}





\noindent
Thus we additionally require that the following condition be satisfied, which ensures that the remainder of the proof does not expect a variable in $\eta_1$ to be unified to different values simultaneously. This property is not necessary in the propositional case, as the literals of the replacement node do not change lower in the proof. 



\begin{definition} %This is the new definition
\label{prop:extracheck}
Let $\eta$ be pre-regularizable, with safe literals $\mathcal{S}(\eta)$ and parents $\eta_1$ and $\eta_2$, with clauses $\Gamma_1$ and $\Gamma_2$ respectively, assuming without loss of generality that $\eta_1 \xrightarrow[\sigma_1]{\{\ell_1\} } \eta$
such that $\ell_1\sigma_1$ matches a safe literal $\ell^*\in \mathcal{S}(\eta)$. 
The node $\eta$ is said to be \emph{strongly regularizable} in $\psi$ if $\Gamma_1 \sigma_{1} \sqsubseteq \mathcal{S}(\eta)$.
\end{definition}



\noindent
The notion of \emph{strongly regularizable} can be thought of as a \emph{sufficient} condition.
The longer version of this paper (available on the ArXiv \cite{longversion}) discusses a conjectured weaker condition. 



\begin{thm}\label{thm:correct}
Let $\psi$ be a proof with root clause $\Gamma$ and $\eta$ be a node in $\psi$. Let $\psi^{\dagger} = \psi\setminus \{\eta\}$ and $\Gamma^{\dagger}$ be the root of $\psi^{\dagger}$. If $\eta$ is strongly regularizable, then $\Gamma^{\dagger} \sqsubseteq \Gamma$.
\end{thm}


\begin{proof} 
By definition of strong regularizability, $\eta$ is such
that there is a node $\eta'$ with clause $\Gamma'$ and such that
$\eta' \xrightarrow[\sigma']{\{\ell'\} } \eta$ and $\ell'\sigma'$
matches a safe literal $\ell^*\in \mathcal{S}(\eta)$ and
$\Gamma' \sigma' \sqsubseteq \mathcal{S}(\eta)$.

Firstly, in $\psi^{\dagger}$, $\eta$ has been replaced by $\eta'$. Since
$\Gamma' \sigma' \sqsubseteq \mathcal{S}(\eta)$, by definition of
$\mathcal{S}(\eta)$, every literal $\ell$ in $\Gamma'$ either subsumes %more general than
a single literal that occurs as a pivot on every path
from $\eta$ to the root in $\psi$ (and hence on every new path from
$\eta'$ to the root in $\psi^{\dagger}$) or subsumes literals %more general than
$\ell \sigma_1$,\ldots,$\ell\sigma_n$ in $\Gamma$. In the former case,
$\ell$ is resolved away in the construction of $\psi^{\dagger}$ (by
contracting the descendants of $\ell$ with the pivots in each path).
In the latter case, the literal $\ell \sigma_k$ ($1 \leq k \leq n$) in
$\Gamma$ is a descendant of $\ell$ through a path $k$ and the
substitution $\sigma_k$ is the composition of all substitutions on
this path. When $\eta$ is replaced by $\eta'$, two things may happen
to $\ell \sigma_k$. If the path $k$ does not go through $\eta$, 
$\ell \sigma_k$ remains unchanged (i.e. $\ell \sigma_k \in \Gamma^{\dagger}$
unless the path $k$ ceases to exist in $\psi^{\dagger}$). If the path
$k$ goes through $\eta$, the literal is changed to 
$\ell\sigma^{\dagger}_k$, where $\sigma^{\dagger}_k$ is such that 
$\sigma_k = \sigma' \sigma^{\dagger}_k$.

Secondly, when $\eta$ is replaced by $\eta'$, the edge from
$\eta$'s other parent $\eta''$ to $\eta$ ceases to exist in
$\psi^{\dagger}$. Consequently, any literal $\ell$ in $\Gamma$ that is a
descendant of a literal $\ell''$ in the clause of $\eta''$ through a
path via $\eta$ will not belong to $\Gamma^{\dagger}$.

% In summary, $\Gamma^{\dagger}$ will contain more general versions ($\ell\sigma^{\dagger}_i$) of some of the literals ($\ell\sigma_i$) and unchanged versions of other literals that occurred in $\Gamma$ and other literals that occurred in $\Gamma$ will not occur in $\Gamma^{\dagger}$. 

Thirdly, a literal from $\Gamma$ that descends neither from $\eta'$ nor from $\eta''$ either remains unchanged in $\Gamma^{\dagger}$ or, if the path to the node from which it descends ceases to exist in the construction of $\psi^{\dagger}$, does not belong to $\Gamma^{\dagger}$ at all.

Therefore, by the three facts above, $\Gamma^{\dagger} \sigma' \sqsubseteq \Gamma$, and hence $\Gamma^{\dagger} \sqsubseteq \Gamma$.
\end{proof} 

