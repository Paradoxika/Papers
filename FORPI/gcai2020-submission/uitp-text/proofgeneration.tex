\subsection{Proof Generation}
Additional proofs were generated by the following procedure: start with a root node whose conclusion is $\bot$, and make two premises $\eta_1$ and $\eta_2$ using a randomly generated literal such that the desired conclusion is the result of resolving $\eta_1$ and $\eta_2$. For each node $\eta_i$, determine the inference rule used to make its conclusion: with probability $p=0.9$, $\eta_i$ is the result of a resolution, otherwise it is the result of  factoring. 

Literals are generated by uniformly choosing a number from $\{1,\dots,k,k+1\}$ where $k$ is the number of predicates generated so far; if the chosen number $j$ is between $1$ and $k$, the $j$-th predicate is used; otherwise, if the chosen number is $k+1$, a new predicate with a new random arity (at most four) is generated and used. Each argument is a constant with probability $p=0.7$ and a complex term (i.e. a function applied to other terms) otherwise; functions are generated similarly to predicates. 

If a node $\eta$ should be the result of a resolution, then with probability $p=0.2$ we generate a left parent $\eta_\ell$ and a right parent $\eta_r$ for $\eta$ (i.e. $\eta = \eta_\ell \odot \eta_r$) having a common parent $\eta_c$ (i.e. $\eta_l = (\eta_\ell)_\ell \odot \eta_c$ and $\eta_r = \eta_c \odot (\eta_r)_r$, for some newly generated nodes $(\eta_\ell)_\ell$ and $(\eta_r)_r$ ). The common parent ensures that also non-tree-like DAG proofs are generated. 

This procedure is recursively applied to the generated parent nodes. 
Each parent of a resolution has each of its terms not contained in the pivot replaced by a fresh variable with probability $p=0.7$.
At each recursive call, the additional minimum height required for the remainder of the branch is decreased by one with probability $p=0.5$. Thus if each branch always decreases the additional required height, the proof has height equal to the initial minimum value. The process stops when every branch is required to add a subproof of height zero or after a timeout is reached. In any case, the topmost generated node for each branch is generated as an axiom node. 

The minimum height was set to 7 (which is the minimum number of nodes in an irregular proof plus one) and the timeout was set to 300 seconds (the same timeout allowed for {\SPASS}). The probability values used in the random generation were carefully chosen to produce random proofs similar in shape to the real proofs obtained by {\SPASS}. For instance, the probability of a new node being a resolution (respectively, factoring) is approximately the same as the frequency of resolutions (respectively, factorings) observed in the real proofs produced by {\SPASS}.