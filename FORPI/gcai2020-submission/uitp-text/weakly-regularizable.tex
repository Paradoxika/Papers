As the name suggests, strong regularizability is stronger than necessary. In some cases, nodes may be regularizable even if they are not strongly regularizable. A weaker condition (conjectured to be sufficient) is presented below. This alternative relies on knowledge of how literals are changed after the deletion of a node in a proof (and it is inspired by the \emph{post-deletion unifiability condition} described for {\FOLowerUnits} in \cite{GFOLU}). However, since weak regularizability is more complicated to check, it is not as suitable for implementation as strong regularizability. 
\begin{definition}\label{def:postdelprop}
Let $\eta$ be a pre-regularizable node with parents $\eta_1$ and $\eta_2$, assuming without loss of generality that $\eta_1 \xrightarrow[\sigma_1]{\{\ell_1\} } \eta$ 
%and $\eta_2 \xrightarrow[\sigma_2]{\{\ell_2\} } \eta$ 
such that $\ell_1$ is unifiable with some $\ell^* \in \mathcal{S}(\eta)$.
For each safe literal $\ell = \ell_s\sigma_s \in \mathcal{S}(\eta_1)$, let $\eta_\ell$ be a node on the path from $\eta$ to the root of the proof such that $\abs{\ell}$ is the pivot of $\eta_\ell$.
Let $\mathcal{R}(\eta_\ell)$ be the set of all resolved literals $\ell_s'$ such that $\eta_2' \xrightarrow[\sigma_s]{\{\ell_s\} } \eta_\ell$, $\eta_1' \xrightarrow[\sigma_s']{\{\ell_s'\} } \eta_\ell$, and $\ell_s\sigma_s=\dual{\ell_s'}\sigma_s'$, for some nodes $\eta_2'$ and $\eta_1'$ and unifier $\sigma_s'$; if no such node $\eta_\ell$ exists, define $\mathcal{R}(\eta_\ell)=\emptyset$.
% and $\sigma_2'$.
The node $\eta$ is said to be \emph{weakly regularizable} in $\psi$ if, for all $\ell \in \mathcal{S}(\eta_1)$, all elements in $\mathcal{R}^{\dagger}(\eta_\ell) \cup \{ \dual{\ell}^\dagger \}$ are unifiable, where $\dual{\ell}^{\dagger}$ is the literal in $\dn{\psi}{\eta_2}$ that used to be\footnote{Because of the removal of $\eta_2$, $\dual{\ell}^{\dagger}$ may differ from $\dual{\ell}$.} $\dual{\ell}$ in $\psi$ and $\mathcal{R}^{\dagger}(\eta_\ell)$ is the set of literals in $\dn{\psi}{\eta_2}$ that used to be the literals of $\mathcal{R}(\eta_\ell)$ in $\psi$.
\end{definition}


%old
%Let $\mathcal{R}(\eta)$ be the set of resolved literals $\dual{\ell_2}$ contained in some conclusion of a node in $\mathcal{P}(\eta)$.
%Let $\eta$ be a node with safe literals $\mathcal{S}(\eta)$. Consider $p\in \mathcal{S}(\eta)$ and let $\eta_1$, \ldots, $\eta_n$ be subproofs that are resolved using $p$ in a proof $\psi$, respectively, with resolved literals $\ell_1$, \ldots, $\ell_m$. 
%The node $\eta$ is said to satisfy the \emph{post-deletion unifiability property} in $\psi$ if, for all $p\in \mathcal{S}(\eta)$, $\ell_1^{\dagger}$,\ldots,$\ell_m^{\dagger}$, and $\dual{p^{\dagger}}$ are unifiable, where $\ell^{\dagger}$ is the literal in $\dn{\psi}{\eta}$ corresponding to $\ell$ in $\psi$.

This condition requires the ability to determine the underlying (uninstantiated) literal for each safe literal of a weakly regularizable node $\eta$. To achieve this, one could store safe literals as a pair $(\ell_s,\sigma_s)$, rather than as an instantiated literal $\ell_s\sigma_s$, although this is not necessary for the previous conditions.

Note further that there is always at least one node $\eta_\ell$ as assumed in the definition for any safe literal which was not contained in the root clause of the proof: the node which resulted in $\ell = \ell_s\sigma_s \in \mathcal{S}(\eta)$ being a safe literal for the path from $\eta$ to the root of the proof. Furthermore, it does not matter which node $\eta_\ell$ is used. To see this, consider some node $\eta_\ell' \neq \eta_\ell$ with the same pivot $\abs{\ell}=\abs{\ell_s\sigma_s}$. Consider arbitrary nodes $\eta_1$ and $\eta_2$ such that  $\eta_2 \xrightarrow[\sigma_s]{\{\ell_s\} } \eta_\ell$ and $\eta_1 \xrightarrow[\sigma_1]{\{\ell_1\} } \eta_\ell$ where $\ell_s\sigma_s=\dual{\ell_1}\sigma_1$. Now consider arbitrary nodes $\eta_1'$ and $\eta_2'$ such that  $\eta_2' \xrightarrow[\sigma_s]{\{\ell_s\} } \eta_\ell'$ and $\eta_1' \xrightarrow[\sigma_1']{\{\ell_1'\} } \eta_\ell'$ where $\ell_s\sigma_s=\dual{\ell_1'}\sigma_1'$. Since the pivots for $\eta_\ell$ and $\eta_\ell'$ are equal, we must have that %$\abs{\ell\sigma_2}=\abs{\ell\sigma_2'}$ and furthermore that 
$\abs{\ell_s\sigma_s}=\abs{\ell_1\sigma_1}$ and $\abs{\ell_s\sigma_s}=\abs{\ell_1'\sigma_1'}$, and thus $\abs{\ell_1\sigma_1}=\abs{\ell_1'\sigma_1'}$. This shows that it does not matter which $\eta_\ell$ we use; the instantiated resolved literals will always be equal implying that both of the resolved literals $\ell_1$ and $\ell_1'$ will be contained in both $\mathcal{R}(\eta_\ell)$ and $\mathcal{R}(\eta_\ell')$.


Informally, a node $\eta$ is weakly regularizable in a proof if it can be replaced by one of its parents $\eta_1$, such that for each $\ell \in \mathcal{S}(\eta_1)$, $\abs{\ell}$ can still be used as a pivot in order to complete the proof. Weakly regularizable nodes differ from strongly regularizable nodes by not requiring the entire parent $\eta_1$ replacing the resolution $\eta$ to be simultaneously matched to a subset of $\mathcal{S}(\eta)$, and requires knowledge of how literals will be instantiated after the removal of $\eta_2$ and $\eta$ from the proof.


%wrong:
%since $\eta$ is assumed to be pre-regularizable, we have that one literal $\ell' \in \mathcal{R}(\eta)$ of $\eta_\ell$ (recall that the resolved literals of $\eta$ and $\eta_\ell$ are the same) is unifiable with $\dual{\ell}$.

%\begin{table}
\centering
\begin{tabular}{| c | c | c | c | }
\hline
$\eta$ & $\mathcal{S}(\eta)$ & $\mathcal{R}(\eta)$ & $\mathcal{R}^\dagger(\eta)$ \\ \hline \hline
$\eta_1$ &  $\{P(w)\}$ & $\emptyset$  & $\emptyset$\\ \hline 
$\eta_2$ &  $\{\lnot P(w)\}$ & $\emptyset$  & $\emptyset$\\ \hline 
$\eta_3$ &  $\{R(a),\lnot P(w)\}$ & $\emptyset$  & $\emptyset$\\ \hline 
$\eta_4$ &  $\{\lnot R(a),\lnot P(w)\}$& $\emptyset$& $\emptyset$ \\ \hline 
$\eta_5$ &  $\{Q(z),\lnot R(a), \lnot P(w)\}$ & $\emptyset$ & $\emptyset$\\ \hline 
$\eta_6$ &  $\{\lnot P(w), \lnot Q(z), \lnot R(a) \}$ & $\{P(u),P(y)\}$& $\{P(u)\}$\\ \hline 
$\eta_7$ &  $\{P(y), \lnot P(w), \lnot Q(z), \lnot R(a) \}$ & $\emptyset$ & $\emptyset$ \\ \hline 
$\eta_8$ &   $\{\lnot P(y), \lnot P(w), \lnot Q(z), \lnot R(a) \}$ & $\emptyset$ & $\emptyset$\\ \hline 
\end{tabular}
\hfill
\caption{The sets $\mathcal{S}(\eta)$ and $\mathcal{R}(\eta)$ for each node $\eta$ in the first proof of Example \ref{ex:weak}.}
\label{tab:exweakreg}
\end{table}%UITP


\begin{example}\label{ex:weak}
This example illustrates a case where a node is weakly regularizable but not strongly regularizable. %Table \ref{tab:exweakreg} shows the sets $\mathcal{S}(\eta)$, $\mathcal{R}(\eta)$ and $\mathcal{R}^\dagger(\eta)$ for the nodes $\eta$ in the proof below.  %UITP
Observe that $\eta_6$ is pre-regularizable, since $\lnot P(x)$ is unifiable with $\lnot P(w)\in \mathcal{S}(\eta_6)$. In fact, $\eta_6$ is the only pre-regularizable node in the proof, and thus the sets $\mathcal{R}(\eta) = \emptyset$ for all $\eta \neq \eta_6$.
In the proof below, note that $\eta_6$ is not strongly regularizable: there is no unifier $\sigma$ such that $\{\lnot P(x),\lnot Q(x),\lnot R(x)\} \sigma \subseteq \mathcal{S}(\eta_6)$.
\begin{scriptsize}
\begin{prooftree}
\def\e{\mbox{\ $\vdash$\ }}
\AxiomC{$\eta_1$: $\e P(u)$ \hspace{-2cm}}
\AxiomC{$\eta_5$: $P(z) \e Q(z)$ \hspace{-0.5cm}}
\AxiomC{$\eta_8$: $P(x),Q(x),R(a)\e$}
\AxiomC{$\eta_7$: $\e P(y)$  \hspace{-1cm}}
\BinaryInfC{$\eta_6$: $Q(y),R(a)\e$ }
\BinaryInfC{$\eta_4$: $P(z),R(a)\e$ \hspace{-2cm} }
\AxiomC{ \hspace{-1cm} $\eta_3$: $\e R(a)$}
\BinaryInfC{ $\eta_2$: $P(z)\e$}

\BinaryInfC{$\psi$: $\bot$}
\end{prooftree}

\end{scriptsize}
\noindent
We show that $\eta_6$ is weakly regularizable, and that $\eta_7$ can be removed. Recalling that $\eta_6$ is pre-regularizable, observe that $\mathcal{R}^\dagger(\eta_6) \cup \{\dual{\lnot P(w)}\}$ is unifiable.
Consider the following proof of $\psi \setminus \{\eta_7\}$:
\begin{scriptsize}
\begin{prooftree}
\def\e{\mbox{\ $\vdash$\ }}
\AxiomC{$\eta_1$: $\e P(u)$ \hspace{-1.75cm}}
\AxiomC{$\eta_8$: $P(x),Q(x),R(a)\e$}
\AxiomC{$\eta_5$: $P(z) \e Q(z)$}
\BinaryInfC{$\eta_4'$: $P(z), P(z),R(a)\e$}
\UnaryInfC{$\eta_4$: $P(z),R(a)\e$}
\AxiomC{$\eta_3$: $\e R(a)$}
\BinaryInfC{$\eta_2$: $P(z)\e$}
\BinaryInfC{$\psi$: $\bot$}
\end{prooftree}
\end{scriptsize}
It is straightforward to check that $\eta_6$ is weakly regularizable. %UITP
%Now observe that for each $\ell \in \mathcal{S}(\eta_8)$ we have the following, showing that $\eta_6$ is weakly regularizable:
%\begin{itemize}
%\item $\ell=\lnot  Q(y)$: $\ell^\dagger = \lnot Q(x)$ which is unifiable with $\dual{\ell}^\dagger=Q(z)$
%\item $\ell=\lnot R(a)$: $\ell^\dagger = \lnot R(a)$ which is (trivially) unifiable with $\dual{\ell}^\dagger=R(a)$
%\item $\ell=\lnot P(w)$: $\ell^\dagger = \lnot P(z)$ which is unifiable with $\dual{\ell}^\dagger=P(u)$
%\item $\ell=\lnot P(y)$: $\ell^\dagger = \lnot P(z)$ which is unifiable with $\dual{\ell}^\dagger=P(u)$
%\end{itemize}
\end{example}

If a node $\eta$ with parents $\eta_1$ and $\eta_2$ is pre-regularizable and strongly regularizable in $\psi$, then $\eta$ is also weakly regularizable in $\psi$.

%If a node is pre-regularizable and $\eta$ can be removed, then it is also weakly regularizable. Thus strongly regularizable nodes are also weakly regularizable nodes.

%\begin{thm}
%Let $\eta$ be a node that is pre-regularizable in some proof such that $\psi\setminus\{\eta_2\}$ is a proof with the same conclusion clause. Then $\eta$ is weakly regularizable.
%Let $\eta$ be a node that is strongly regularizable in some proof. Then $\eta$ is weakly regularizable.
%\end{thm}

%\begin{proof}
%Let $\eta$ be a strongly regularizable node with parents $\eta_1$ and $\eta_2$. By definition, $\eta$ is also pre-regularizable.
%Assume without loss of generality that $\eta_1\sigma \subseteq \mathcal{S}(\eta)$.

%Let $\eta$ be a pre-regularizable node with parents $\eta_1$ and $\eta_2$. Let $\mathcal{R}(\eta_\ell)$ and $\mathcal{R}^\dagger(\eta_\ell)$ be defined as in Definition \ref{def:postdelprop} for a safe literal literal $\ell \in \mathcal{S}(\eta_1)$.

%Let $\ell \in \mathcal{S}(\eta_1)$ be a safe literal of $\eta_1$ that is contained in the root clause of the proof such that there does not exists a node $\eta_\ell$ below $\eta$ whose pivot is $\abs{\ell}$. Then $\mathcal{R}^\dagger(\eta_\ell)\cup\{\dual{\ell}\}=\emptyset\cup\{\dual{\ell}\}=\{\dual{\ell}\}$ is trivially unifiable. Thus we may assume that for all $\ell\in\mathcal{S}(\eta_1)$, such a node $\eta_\ell$ exists.

%old
%Let $\ell \in \mathcal{S}(\eta_1)$ be a safe literal of $\eta_1$ that is not contained in the root clause of the proof, and let $\eta_\ell$ be a node on the path from $\eta_1$ to the root of the proof such that $\abs{\ell}$ is used as the pivot. We claim that at least one such node $\eta_\ell$ exists. To see this, recall that since $\eta$ is strongly regularizable, the resolved literal of $\eta$ was unifiable with some $\ell^*\in \mathcal{S}(\eta)$, so $\eta$ could not have provided a literal to $\mathcal{S}(\eta_1)$ that does not also appear below $\eta$. Thus $\mathcal{S}(\eta_1)\subseteq \mathcal{S}(\eta)$. Note that we cannot have $\mathcal{S}(\eta)\subsetneq \mathcal{S}(\eta_1)$, as otherwise a literal $\ell \in \mathcal{S}(\eta_1)$ have been added from a path from $\eta_1$ to the root avoiding $\eta$, contradicting $\ell$ appearing in all paths below $\eta_1$, or $\ell$ appearing in the root clause (as if this were the case, then we would also have $\ell \in \mathcal{S}(\eta)$). Since $\mathcal{S}(\eta_1)\subseteq \mathcal{S}(\eta)$, there must be a node below $\eta$ that used $\abs{\ell}$ as a pivot in order to have $\ell \in \mathcal{S}(\eta_1)$. 


%If $\ell \notin \eta_1$, then $\ell^\dagger =\ell$ and $\dual{\ell}^\dagger=\dual{\ell}$, there is nothing to prove (neither $\ell$ or $\dual{\ell}$ have changed in $\psi\setminus\{\eta_2\}$). So we may assume $\ell\in \eta_1$.

%Consider $\dual{\ell} \in \mathcal{R}(\eta_\ell)$: $\ell$ and $\dual{\ell}$ are unifiable in $\psi$ by definition of $\mathcal{R}(\eta_\ell)$. We will show that $\ell^\dagger$ and $\dual{\ell}^\dagger$ are unifiable in $\psi\setminus\{\eta_2\}$, where $\dual{\ell}^\dagger\in \mathcal{R}^\dagger(\eta_\ell)$.

%Since $\eta_\ell$ exists (with $\abs{\ell}$ as a pivot by definition), there exists nodes $\eta_L$ and $\eta_R$ such that $\eta_L \xrightarrow[\sigma_L]{\{\ell_L\} } \eta_\ell$ and $\eta_R \xrightarrow[\sigma_R]{\{\ell_R\} } \eta_\ell$ for some $\ell_L,\ell_R,\sigma_L$, and $\sigma_R$.
%Since $\abs{\ell}$ was the pivot, we have that $\abs{\ell}=\abs{\ell_L\sigma_L}$ or $\abs{\ell}=\abs{\ell_R\sigma_R}$. Without loss of generality, assume that  $\abs{\ell}=\abs{\ell_L\sigma_L}$.
%Thus we can write
%$$\ell = \ell_L\sigma_L =\dual{\ell_R}\sigma_R$$
%Note that $\dual{\ell}^\dagger=\dual{\ell_R}$ as $\dual{\ell_R}$ is unchanged in $\psi \setminus\{\eta_2\}$.
%Since $\eta_1$ replaces $\eta$ in $\psi\setminus\{\eta_2\}$, we have that $\ell^\dagger=\ell$. 

%\end{proof}


%intersection example?