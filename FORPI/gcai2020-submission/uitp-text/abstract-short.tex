%Resolution and superposition are common techniques which have seen widespread use with propositional and first-order logic in modern theorem provers. In these cases, resolution proof production is a key feature of such tools; however, the proofs that they produce are not necessarily as concise as possible.
Proofs are a key interface of modern propositional and first-order theorem provers. However, this interface is complicated by proofs which are not necessarily as concise as possible.
%For propositional resolution proofs, there are a wide variety of proof compression techniques. There are fewer techniques for compressing first-order resolution proofs generated by automated theorem provers.
There are a wide variety of compression techniques for propositional resolution proofs, but fewer compression techniques for first-order resolution proofs generated by automated theorem provers.
This paper describes an approach to compressing first-order logic proofs based on lifting proof compression ideas used in propositional logic to first-order logic. 
An empirical evaluation of the approach is included.

%One method for propositional proof compression is \emph{partial regularization}, which removes an inference $\eta$ when it is redundant in the sense that its pivot literal already occurs as the pivot of another inference in every path from $\eta$ to the root of the proof. 
%This paper describes the generalization of the partial-regularization algorithm
%\RecyclePivotsIntersection \cite{LURPI}
%%[P. Fontain, S. Merz, and B. Woltzenlogel Paleo, Compression of Propostional Resolution Proofs via Partial Regularization, In \emph{Automated Deduction -- CADE-23, Proceedings} LNCS vol. 6803, 237-351. 2011]
%from propositional logic to first-order logic. The generalized algorithm performs partial regularization of resolution proofs containing resolution and factoring inferences with \emph{unification}. %The resolution calculus is the underlying proof-theoretical foundation for many automated theorem provers. 
%An empirical evaluation of the generalized algorithm and its combinations with the previously lifted \SFOLowerUnits algorithm
%\cite{GFOLU} is also presented.
%%[J. Gorzny and B. Woltzenlogel Paleo, Towards the Compression of First-Order Resolution Proofs by Lowering Unit Clauses. In \emph{Automated Deduction -- CADE-25, Proceedings}, LNCS vol. 9195, 356-366. 2015], 




%\texttt{RecyclePivots\-WithIntersection}