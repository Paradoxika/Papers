The resolution calculus used in this paper has the following inference rules: 

% \begin{definition}[First-Order Axiom] A first-order axiom has no premises and concludes some clause $\Gamma$, as below.
% \begin{prooftree}
% \AxiomC{$~$ }
% \UnaryInfC{$\psi$: $\Gamma$}
% \end{prooftree}
% %where $\Gamma$ is a clause.
% \end{definition}

\begin{definition}[Resolution] \label{def:fores} \hfill
%An instance of first-order resolution requires two premises, as below.
\begin{prooftree}
\AxiomC{$\eta_1$: $\Gamma_L' \cup \{\ell_L\}$ }
\AxiomC{$\eta_2$: $\Gamma_R'\cup \{\ell_R\}$ }
\BinaryInfC{$\psi$: $\Gamma_L'\sigma_L \cup \Gamma_R'\sigma_R$}
\end{prooftree}
where $\sigma_L$ and $\sigma_R$ are substitutions such that $\ell_L\sigma_L=\dual{\ell_R}\sigma_R$. The literals $\ell_L$ and $\ell_R$ are \emph{resolved literals}, whereas $\ell_L \sigma_L$ and $\ell_R \sigma_R$ are its \emph{instantiated resolved literals}. The \emph{pivot} is the underlying atom of its instantiated resolved literals (i.e. $\abs{\ell_L \sigma_L}$ or, equivalently, $\abs{\ell_R \sigma_R}$).
\end{definition}

\begin{definition}[Factoring] \label{def:fofact} \hfill
%An instance of first-order factoring applies a unifier to the literals in a single premise's conclusion, as below.
\begin{prooftree}
\AxiomC{$\eta_1$: $\Gamma' \cup \{\ell_1,\ldots,\ell_n\}$ }
\UnaryInfC{$\psi$: $\Gamma'\sigma \cup \{\ell\}$}
\end{prooftree}
where $\sigma$ is a unifier of $\{\ell_1,\ldots,\ell_n\}$ and $\ell=\ell_i\sigma$ for any $i\in \{1,\ldots,n\}$.
\end{definition}